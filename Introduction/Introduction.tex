\chapter*{{\fontfamily{lmss}\selectfont Introduction générale}}
\label{chap:intro_géné}
 
\addstarredchapter{Introduction générale}
\markboth{\uppercase{Introduction générale}}{}

L' eau liquide est une ressource qui peut sembler abondante par son omniprésence à la surface de la Terre mais qui est à la fois rare et unique. Sa présence est aussi importante pour l'émergence et le maintien de la vie que pour les paysages qu'elle façonne, les territoires qu'elle délimite et les écosystèmes, peu connus mais riches, qu'elle héberge.\\
Pourtant, sa préservation et son accès sont de plus en plus mis en péril sous l'effet conjugué de la croissance démographique, du développement socio-économique et des dérèglements climatiques. Alors que l'accès à des conditions suffisantes d'alimentation en eau potable est un droit fondamental, le dernier rapport des Nations Unies sur la mise en valeur des ressources en eau fait état d'une situation alarmante. Au 21\ieme{} siècle, pas moins de deux milliards de personnes vivent dans des pays en situation de stress hydrique permanent et ce chiffre atteint quatre milliards si les pénuries saisonnières sont prises en compte \citep{WDDR2019}. Quasiment tous les pays sont touchés par des pénuries saisonnières même si les pays faisant face aux pénuries les plus élevées sont situés en Afrique du Nord et au Moyen/Proche-Orient (Figure \ref{stress_hydrique}).\\ 
Cette pression anthropique tend à croître puisque la consommation d'eau augmente de 1\% par an depuis 1980 et devrait continuer à augmenter jusqu'à 2050 pour être 20\% à 30\% plus élevée qu'aujourd'hui \citep{burek2016}.\\

\begin{figure}[h!]
\includegraphics[width=1.\textwidth]{stress_hydrique}
\caption{Carte des niveaux de stress hydrique pour l'année 2019. Source: \citet{WDDR2019}.}
\label{stress_hydrique}
\end{figure}


Lorsqu'on y regarde de plus près, seule une infime proportion, 2.5\%, de l'eau liquide est douce, dont les trois quarts sous forme de glace, ce qui rend cette ressource encore plus précieuse et limitée. Le tableau \ref{repartition_eau} représente la fraction des réserves totales des eaux douces et salées à l'échelle du globe répartie par réservoir et les temps de rétention associés. Ces données montrent la prépondérance de l'eau salée par rapport à l'eau douce mais aussi que les stocks peuvent être impactés de différentes manières par des perturbations externes, à cause de la variabilité de leurs temps de rétention. Ainsi les eaux souterraines, malgré un stock conséquent, sont, d'une part, difficilement accessibles et, d'autre part, possèdent des temps de rétention longs. Ces réserves d'eau sont particulièrement sensibles aux moindres perturbations telles que les abaissements de nappes qui peuvent avoir des conséquences quasi-irréversibles. D'un autre côté, les rivières ont des temps de rétention plus courts et donc un taux de renouvellement élevé. En contrepartie, les quantités stockées en surface sont moindres et la ressource est moins pérenne et plus sujette aux variations saisonnières. Dans un contexte où l'homme prélève un volume d'eau douce deux fois supérieur à toute la quantité qui ruisselle sur le globe il apparaît donc nécessaire de repenser notre mode de consommation. \\

\begin{table}[h!]
 \caption{Répartition globale des stocks d'eau dans les principaux compartiments en millions de $m^{3}$ et les temps de rétention associés.}
 \label{repartition_eau}
 \begin{tabularx}{\textwidth}{ccXXXX}
 \hline
 &\footnotesize{Nom}&\footnotesize{Volume (milliers de km$^3$)}& \footnotesize{Pourcentage des réserves totales d'eau}& \footnotesize{Pourcentage des réserves totales d'eau douce}& \footnotesize{Temps de rétention}\\
 \hline
 \multirow{3}{1.5cm}{Réserves salées}&Océans&1338000&96.6&-&3,100 ans\\
  &Aquifères&12570&0.91&-&300 ans\\
  &Lacs salés&85&0.006&-&10-1,000 ans\\
  \hline
 \multirow{8}{1.5cm}{Réserves eau douce} &Glaciers/Pergélisols&24364&\multirow{8}{1.5cm}{2.5}&69.5&16,000ans\\
  &Aquifères&10530&&30&300 ans\\
  &Lacs d'eau douce&91&&0.26&1-100ans\\
  &Humidité du sol&16.5&&0.05&280 jours\\
  &Atmosphère&12.9&&0.04&9 jours\\
  &Zones humides&11.5&&0.03&/\\
  &Rivières, fleuves&2.12&&0.006&12-20 jours\\ 
  &Biosphère&1.12&&0.003&/\\  
  \hline
 \end{tabularx}
\end{table}

La variabilité de la distribution spatio-temporelle de l'eau liquide est un moteur dans l'émergence d'enjeux à court et long terme. Les régions où la ressource se raréfie doivent faire face à des problèmes de pénuries, d'aridifications et de désertifications amenant les populations à se déplacer. À l'inverse, l'abondance provoque une augmentation de la récurrence d'évènements climatiques exceptionnels qui mettent en péril la sécurité de pays entiers. C'est par exemple le cas du Bangladesh où la montée des eaux océaniques accentue les conséquences liées aux inondations en période cyclonique.\\
Par ailleurs, l'accès à l'eau potable n'est pas seulement défini par des critères quantitatifs mais aussi qualitatifs. Ainsi, 65\% des eaux fluviales sont considérées comme menacées \citep{vorosmarty2010} et cette dégradation des écosystèmes aquatiques devient un enjeu majeur dans un contexte où nos sociétés ont besoin des services écosystémiques existant et où les conflits autour de la gestion de l'eau se multiplient. \\

Le changement climatique exacerbe ces difficultés et ces inégalités en provoquant des risques majeurs pour l'équilibre des écosystèmes globaux mais aussi pour nos sociétés. Alors qu'un milliard de personnes vit dans des zones inondables \citep{di2013}, le changement climatique contribue à une augmentation significative de l'occurence des crues à l'échelle du globe. Les conséquences de cette augmentation sont multiples et variées. Elles devraient aussi engendrer une augmentation de 580\% du nombre de personnes affectées par les crues \citep{alfieri2015} et altérer les régimes d'écoulements et les périodes d'enneigement à l'échelle continentale \citep{schneider2013, forzieri2014, ribes2019} et globale \citep{rodell2018}.\\


L'hydrologie, qui s'intéresse à tous les aspects du mouvement de l'eau sur Terre, ses conséquences sur l'environnement et nos sociétés, permet d'étudier ces phénomènes, à la confluence de domaines comme la glaciologie, la météorologie, la chimie ou encore la géographie. \'Etymologiquement, le mot "hydrologie" fait référence à l'étude de l'eau dans sa globalité et traduit plus précisément l'étude de son cycle. C'est pourquoi il incombe à l'hydrologue, au-delà de s'intéresser aux équations de bilan d'eau aux différentes échelles, de comprendre et d'illustrer ce système complexe, hétérogène et en constant renouvellement. \\
Les champs d'applications de l'hydrologie sont - donc - à la fois vastes et parfaitement définis par le cycle de l'eau. Celui-ci correspond au mouvement et renouvellement perpétuel de l'eau sur Terre que ce soit sous la forme de glace, de liquide ou de vapeur. Il décrit ainsi les connections entre les processus, plus ou moins distants, qui le composent (évaporation, infiltration, ruissellement). Il sert aussi de support à la représentation des principes physiques clés comme la conservation de la masse et illustre les changements globaux. \\
Les études dans le domaine de l'hydrologie continentale se portent plus particulièrement sur les échanges d'eau se produisant au niveau des terres émergées et répond aux enjeux liés à la ressource en eau et à la protection des biens et des personnes par la prévision du risque inondation ou de sécheresse. Sur des temps plus longs, l'hydrologie continentale s'attache à étudier des solutions d'adaptation et d'atténuation afin d'éviter des situations dramatiques environnementales comme l'augmentation des sécheresses ou humaines comme l'émergence de zones de conflits et permettant de garantir la pérennité des stocks d'eau. L'ensemble de ces éléments montre toute l'importance de représenter les processus d'hydrologie continentale pour le suivi global de la dynamique climatique \citep{alkama2008,douville2016}.\\

 
La figure \ref{cycleeau} recense l'ensemble des processus qui agissent sur les particules d'eau et décrivent le cycle de l'eau. 

\begin{figure}[h!]
 \centerline{\includegraphics[scale=0.35]{cycleeau}}
 \caption{Représentation schématique du grand cycle de l'eau. Source : \url{www.eaufrance.fr}}
 \label{cycleeau}
\end{figure} 

Ainsi en isolant une particule d'eau océanique, il est possible de tracer les différents chemins qui s'offrent à elle. Sous l'effet de forçages atmosphériques, la particule d'eau s'évapore, elle s'élève par flottabilité et vient alimenter l'atmosphère en vapeur. Dans l'atmosphère, la particule est soumise en continu à des contraintes dynamiques et thermodynamiques qui peuvent la transporter sur des distances plus ou moins grandes vers des zones où les conditions sont favorables à une modification de ses propriétés et notamment à sa condensation: nous observons ce phénomène par la formation de nuages. Si ces nuages deviennent précipitants alors la goutte d'eau retourne vers la surface, qu'elle soit océanique ou continentale, où différents itinéraires s'offrent à elle. Dans le premier cas, la goutte d'eau rejoint son point de départ: l'océan et le cycle est fermé. Dans le second cas, la goutte d'eau peut ruisseler pour rejoindre une rivière puis un fleuve, être stockée dans un lac pour finalement revenir à son point de départ: l'océan. La goutte d'eau peut aussi s'évaporer directement après avoir atteint la surface et par conséquent revenir dans l'atmosphère. Enfin elle peut interagir avec la surface terrestre et le sol pour revenir dans l'atmosphère par évaporation ou évapotranspiration selon qu'elle a été interceptée par la végétation, captée par ses racines ou encore s'infiltrer dans le sol pour alimenter les aquifères. Ces différents itinéraires sont accessibles à toute goutte d'eau liquide. Dans le cas de précipitations solides, l'eau peut être stockée sous forme de neige ou de glace au niveau de la banquise ou des régions montagneuses. Tant qu'elle ne fond pas, cette eau reste stockée et ne participe pas aux écoulements de surface. \\

À chaque étape du cycle correspond une échelle de temps qui caractérise les processus physiques et les interactions des compartiments entre eux. Ces échelles ont une très grande importance dans la caractérisation des phénomènes physiques et dans la compréhension des rétroactions entre les compartiments et leur évolution à court et long terme. Ainsi les sécheresses, même saisonnières, peuvent conduire à des déficits en eaux souterraines et avoir un effet irréversible à l'échelle humaine du fait d'un "effet mémoire" des aquifères \citep{lam2011,cuthbert2019}. Ces effets sont plus ou moins locaux et il est primordial d'intégrer les facteurs anthropiques, de plus en plus présents, dans la représentation du cycle de l'eau \citep{abbott2019}. Ces facteurs anthropiques agissent de façon plurielle que ce soit au travers de pollutions, de l'utilisation de l'eau pour l'agriculture ("eaux vertes"), de la modification d'occupation des sols ou par la perturbation des équilibres bioécologiques et des écoulements en lien avec le changement climatique. \\

Les lacs représentent environ 20\% de tout le stock en eau douce de surface \citep{messager2016}. Inégalement répartis à la surface terrestre, on compte pas moins de 117 millions de lacs, dont la superficie dépasse 0.002 km$^{2}$, soit l'équivalent de 3.7\% des terres émergées \citep{verpoorter2014}. Les régions qui bénéficient d'une densité lacustre élevée se trouvent principalement dans les hautes latitudes de l'hémisphère nord comme la Scandinavie ou le nord canadien.\\
Malgré l'abondance de cette ressource dans des régions en première ligne face au changement climatique, les processus hydrologiques associés sont peu ou mal représentés dans les modèles hydrologiques et climatiques globaux. De plus, la pression environnementale croissante sur les lacs a déjà commencé à altérer cette ressource vitale. Des études récentes montrent les effets des altérations climatiques et anthropiques sur les systèmes lacustres et mettent en garde contre les conséquences irréversibles pouvant advenir \citep{woolway2020, jenny2020}. Ces impacts se présentent sous des formes diverses qui vont de la réduction de la couverture en glace \citep{sharma2019} à des modifications conséquentes des stocks \citep{wang2018} en passant par l'acidification des eaux \citep{phillips2015} ou l'augmentation des concentrations en micropolluants et microplastiques \citep{eerkes2015, schwarzenbach2006}, voir l'asséchement total menant à la disparition de l'écosystème lacustre.\\
Au-delà des impacts sur les propriétés physiques et chimiques associées aux lacs, c'est donc tout l'écosystème lacustre qui subit des modifications. La structure écologique, notamment les réseaux trophiques, et le fonctionnement de ces écosystèmes entretiennent un vaste tissu de services, appelés services écosystémiques, dont la société en retire un bénéfice socio-économique. Une structure non exhaustive est schématisée par la Figure \ref{fig:services_eco}. Les modifications locales des conditions sont ainsi sans frontières thématiques et altèrent l'ensemble de ces services.

\begin{figure}[h!]
 \centerline{\includegraphics[width=0.9\textwidth]{regulation}}
 \caption{Principaux services écosystémiques associés au lacs. Inspirée des travaux de \citet{schallenberg2013}.}
 \label{fig:services_eco}
\end{figure} 

La modélisation joue un rôle majeur dans la description des processus environnementaux puisqu'elle permet dans certains cas de pallier les limites de l'observation. La complexité des processus et la diversité des échelles spatiales et temporelles font que l'observation reste limitée voire impossible dans certaines circonstances. En complément de ces mesures, les modèles rendent compte d'une représentation de la réalité simplifiée sur des échelles spatio-temporelles plus étendues. \\
Depuis les travaux pionniers de \citet{manabe1969} à la fin des années 60 ou de \citet{deardorff1977}, les modèles de surface (LSM) proposent aujourd'hui une description plus réaliste de processus physiques hétérogènes et de leur complexité à l'interface sol-atmosphère \citep{levis2010}. Cependant ces modèles étaient limités car ils ne pouvaient pas initier les transferts latéraux de masse. C'est ainsi que les modèles de routage en rivière (RRM) ont vu le jour en permettant alors le transfert d'eau issu des modèles de surface à travers le réseau hydrographique et par la même occasion de fermer le cycle de l'eau. Les RRMs sont aujourd'hui indispensables à la simulation des débits mais aussi à la caractérisation de l'hydrologie locale, régionale et globale que ce soit dans le cadre de la prévision des crues et des sécheresses ou plus généralement dans le suivi de la ressource en eau \citep{ducharne2003, lucas2003, lam2011, zajac2017}.\\
Malgré un développement précoce du bilan d'énergie des lacs à l'échelle globale \citep{lemoigne2016,piccolroaz2020, woolway2017a}, les lacs ont mis plus de temps à être considérés comme une composante hydrologique essentielle dans les RRMs. Du fait de leur prédominance dans certains processus hydrologiques régionaux et grâce à la meilleure résolution des modèles de climat, les lacs représentent, depuis quelques années, un intérêt majeur dans les développements hydrologiques et climatiques régionaux et globaux \citep{bowling2010,cherkauer2010,burek2013}.\\
\clearpage

\subsubsection*{\underline{{\fontfamily{lmss}\selectfont Objectifs d'étude}}}

Au sein des développements des modèles de surface utilisés au Centre National de Recherche Météorologiques (CNRM) pour la prévision hydrologique et climatique, ma thèse vise à répondre à des objectifs explicites et d'actualités concernant l'hydrologie et, plus spécifiquement, l'intégration des lacs dans un système hydrographique global. Ce travail s'appuie sur les développements récents du modèle de routage CTRIP couplé à la plateforme de modélisation SURFEX \citep{decharme2019} en vue d'une paramétrisation, à l'échelle du globe, de la dynamique massique des lacs. Plus particulièrement, cette thèse s'attache à quantifier l'effet du bilan de masse des lacs sur la modélisation de l'hydrologie continentale et les enjeux inhérents. \\
Pour cela, la thèse aborde les objectifs de recherche suivants: \\

\begin{itemize}
\item \textbf{Développer} un modèle non calibré de bilan de masse MLake capable d'améliorer la simulation des débits de rivières par CTRIP à l'échelle globale;\\

\item \textbf{Proposer} un diagnostic sur les variations de niveau d'eau dans les lacs pour un suivi dans le passé, le présent et le futur;\\

\item \textbf{Améliorer} la caractérisation des zones à enjeux par le biais d'une représentation de la bathymétrie des lacs applicable à l'échelle globale.\\
\end{itemize}

\subsubsection*{\underline{{\fontfamily{lmss}\selectfont Plan du manuscrit}}}

\noindent Pour répondre à ces objectifs de recherche, le manuscrit est divisé en quatre chapitres. \\

Le premier chapitre pose le contexte de l'étude ainsi que son cadre théorique. Pour cela une présentation des bilans régissant le cycle de l'eau est proposée en guise d'introduction aux techniques à notre disposition pour l'observer et le modéliser. Par ailleurs, des considérations théoriques en limnologie et hydrologie lacustre sont proposées pour assurer une compréhension des enjeux du développement du bilan de masse des lacs. \\

Le chapitre 2 décrit, de façon succinte, les modèles utilisés dans cette thèse et notamment les processus physiques nécessaires à la production et au transfert de masse à travers les différents compartiments. Ce chapitre s'attache à décrire la plateforme de modélisation de la surface SURFEX, le modèle de surface ISBA, le modèle de routage CTRIP ainsi que le modèle résolvant le bilan d'énergie des lacs FLake. Enfin il présente le modèle de bilan de masse MLake, au coeur de cette thèse, ses hypothèses ainsi que les étapes de son développement. \\

Le chapitre 3 porte sur l'évaluation et la validation de MLake dans une configuration locale restreinte au bassin versant du Rhône. L'évaluation sur cette zone est motivée par la disponibilité de forçages haute résolution issus de la chaîne opérationelle SAFRAN-ISBA-MODCOU et d'un réseau de mesures conséquent. Elle consiste, premièrement, à analyser les performances du système CTRIP-MLake par rapport aux simulations de référence CTRIP sur cette zone. S'en suit une double validation effectuée d'abord en comparant les simulations et les observations de débits sur trois stations de jaugeage, puis en analysant la cohérence des variations de niveau du Léman. \\

Après cette évaluation locale, le chapitre 4 propose une évaluation et une validation régionale sur trois bassins versants présentant des conditions hydrométéorologiques variées. Après cette validation, une simulation à l'échelle globale a permis de confirmier l'intérêt de prendre en compte le bilan de masse des lacs dans les régions où leur densité est forte.\\

Les perspectives amenées par ce travail de thèse, et notamment le besoin de dissocier la dynamique de masse propre aux barrages des lacs naturels, sont abordées dans une conclusion globale. Plus généralement, l'importance de l'impact anthropique est indéniable et la modélisation hydrologique doit nécessairement prendre en compte ces effets. Enfin les avancées dans le domaine de l'hydrologie lacustre passent aussi par une approche géomorphologique à développer afin d'intégrer pleinement l'hypsométrie des lacs dans la résolution du bilan de masse.
