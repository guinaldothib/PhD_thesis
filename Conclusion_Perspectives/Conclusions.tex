\chapter*{{\fontfamily{lmss}\selectfont Conclusion et perspectives}}
\label{chap:conclu_persp}

\addstarredchapter{Conclusion générale et Perspectives}
\markboth{\uppercase{Conclusion générale et Perspectives}}{}

L'importance de l'eau sur Terre est indéniable tant pour l'expansion et le maintien de la vie que pour le développement du système socio-économique. Néanmoins, cette ressource est fragile et sa disponibilité est limitée mais surtout inégalement répartie à la surface du globe. Pour rappel, quatre milliards de personnes font face, au moins une fois par an, à des restrictions quantitatives d'eau. Ces disparités sont exacerbées par des altérations externes d'origine anthropique et aussi à cause du dérèglement irréfutable de la dynamique climatique. \\
Le mouvement et le renouvellement de l'eau sur la planète s'effectue sous la forme d'un cycle que tentent de représenter les scientifiques en caractérisant ses différentes composantes ainsi que leurs interdépendances. L'hydrologue, en s'intéressant à la partie continentale du cycle, possède des outils d'observation et de modélisation qui lui permettent d'accéder à une description efficace et réaliste des processus mis en jeu. Ainsi, la modélisation pallie les limites spatio-temporelles de l'observation en proposant une approche simplifiée des processus physiques. C'est dans ce contexte que sont nés les systèmes couplant les modèles de surface aux modèles de routage en rivière, et décrivant respectivement les transferts verticaux et transfert latéraux de flux. \\
Dans ce cycle de l'eau, les lacs constituent une ressource importante, environ 20\% de l'eau douce de surface et surtout très accessible. En dépit de leur abondance et de leur rôle sur les transferts d'eau à l'échelle régionale \citep{bowling2010}, ces réservoirs ne sont que très rarement pris en compte dans les modèles hydrologiques globaux \citep{downing2010}.\\

C'est face à ce constat que le cadre de cette thèse a été défini. En s'appuyant sur les développements effectués au Centre National de Recherches Météorologiques, cette thèse intègre une paramétrisation du bilan de masse dans le système hydrologique global ISBA-CTRIP. Plus particulièrement, les objectifs d'étude de la thèse sont de développer un modèle de bilan de masse MLake pour améliorer la simulation des débits de rivières, d'introduire un diagnostic pour le suivi des niveaux d'eau de lacs et enfin de caractériser les stocks d'eau lacustres par une meilleure représentation de la bathymétrie.\\

Dans un premier temps, le travail s'est appuyé sur une description succincte des outils existants pour observer et modéliser le cycle de l'eau. La mise en contexte a été suivie d'une brève introduction à la limnologie afin d'étayer et de justifier l'intérêt d'une représentation des systèmes lacustres en hydrologie. Enfin l'état des lieux des connaissances actuelles en modélisation des lacs a mis en évidence la prédominance des développements de paramétrisation du bilan d'énergie par rapport à ceux du bilan de masse.\\

Le chapitre \ref{chap:intro} a introduit le cadre théorique dans lequel s'insèrent les outils de modélisation utiles pour les diverses études menées au cours de la thèse présentées dans le chapitre \ref{chap:descriptions}. Ce dernier a mis l'accent sur la plateforme de modélisation de surface SURFEX \citep{masson2000} comportant notamment le modèle de surface ISBA \citep{noilhan1989} ainsi que sur le modèle de routage en rivière CTRIP auquel il est couplé \citep{decharme2019}. C'est à partir de ces outils qu'a été développé le modèle MLake résolvant le bilan de masse pour les lacs à 1/12°.\\
Ce modèle présente l'avantage de se baser sur une paramétrisation non calibrée et s'adaptant à la diversité de lacs présents à la surface du globe. Pour ce faire, le développement s'est appuyé sur l'utilisation conjointe de la carte d'occupation des sols ECOCLIMAP \citep{faroux2013} et de la base de données de profondeurs des lacs GLDB \citep{choulga2019}. L'intérêt d'utiliser ces bases, au-delà de leur disponibilité, est qu'elles sont cohérentes avec le modèle FLake \citep{mironov2008} résolvant le bilan d'énergie pour les lacs.\\
La première étape a consisté à agréger l'information globale de présence des lacs à la résolution kilométrique pour proposer une carte globale de lacs auxquels a été attribuée un identifiant unique et une profondeur moyenne. La deuxième étape a abouti à l'intégration de ces informations dans le réseau de rivière initial de CTRIP et à la correction d'éventuelles erreurs de routage grâce à l'application d'un masque de réseau. Une gestion du partage des flux d'eau issus des forçages est le résultat, dans un troisième temps, de la mise en place d'un masque de ruissellement. Ce masque, complémentaire à celui de réseau, assure la fermeture du bilan et un partage correct entre les flux s'écoulant en rivière et ceux stockés dans les lacs. Tout ce travail préliminaire a permis la construction d'un réseau réaliste et robuste pour la résolution des processus physiques. Ainsi, à chaque pas de temps du modèle MLake, le bilan de masse est effectué par la résolution d'une équation dont la variable pronostique est le stock d'eau dans le bassin.\\

L'évaluation de cette paramétrisation a d'abord été effectuée sur le bassin versant du Rhône. Ce choix a été motivé par la présence d'un réseau de mesures fiables et étendues mais aussi par la disponibilité de forçages à haute résolution issus de la chaîne hydrométérologique opérationnelle SAFRAN-ISBA-MODCOU \citep{lemoigne2020}. Le bassin du Rhône est aussi le seul bassin versant français possédant une assez grande variété de lacs inclus dans un contexte hydrologique complexe et diversifié. Le modèle CTRIP-MLake a fait l'objet d'une évaluation en mode offline, sur la période 1960-2016, avec une correction des forçages au niveau des lacs par une estimation de l'évaporation issue d'une simulation globale FLake. Cette évaluation locale a permis de confirmer l'intérêt d'ajouter la dynamique des lacs pour la représentation des débits de rivières locaux et régionaux. En se basant sur une équation empirique de déversoir rectangulaire à seuil épais, il est maintenant possible de corriger les débits simulés en aval du bassin et d'améliorer significativement les performances de CTRIP. Ces améliorations portent principalement sur un lissage des hydrographes en lien avec une réduction de la variabilité et de l'amplitude. Au-delà de ces améliorations, MLake apporte aussi un diagnostic correct sur les marnages du Léman, validé par une comparaison avec des mesures \textit{in situ}. Néanmoins, ces résultats sont à nuancer dû fait de la prédominance des facteurs anthropiques sur le bassin et notamment de la régulation des eaux du Rhône et de ces affluents principaux.\\

Une deuxième évaluation s'est intéressée, initialement, à trois bassins versants décrivant des conditions hydroclimatiques différentes: le bassin de l'Angara, le bassin de la Neva et enfin le bassin du Nil Blanc. Cette évaluation s'est effectuée en mode offline avec des ruissellements et drainages issus d'une simulation globale d'ISBA forcés par des réanalyses Earth2Observe, corrigés pour les lacs par les évaporations d'une simulation globale où ISBA a été remplacé par FLake. L'application de ce modèle sur les trois bassins versants confirme les résultats de l'évaluation locale sur les débits et les marnages. Seul le cas du Nil Blanc et donc du lac Victoria présente des résultats peu probants. Dans tous les cas, on observe une réduction de la variabilité et de l'amplitude des débits se traduisant par un lissage des hydrographes. Cela met en lumière la représentation correcte de l'effet tampon des lacs par l'écrêtement des débits de pointes et un soutien accru des étiages grâce à l'intégration du module MLake. En outre, le diagnostic sur les marnages est particulièrement performant pour les lacs avec un cycle saisonnier bien respecté. Seul un décalage temporel systématique apparaît sur les cycles du lac Baïkal et du lac Ladoga. L'origine de ce décalage est identifié comme probablement lié à la résolution du bilan d'énergie dans ISBA. Il semblerait que l'utilisation d'un unique bilan pour la canopée et la neige induit une sous-estimation des cumuls de neige et de la durée de l'enneigement. La réduction des caractéristiques isolante de la couche de neige sur les propriétés thermiques du sol entraine alors une fonte précoce et plus intense de la glace du sol associée à une sous-estimation des ruissellements dans les zones de densité forestière élevée \citep{napoly2020}.\\
En suivant cette évaluation, les résultats d'une simulation préliminaire ont confirmé la validité de MLake à l'échelle globale et plus particulièrement sur les zones de grande densité lacustre. Même s'il est préférable de rester prudent vu que tous les processus ne sont pas encore intégrés, cette simulation préliminaire est encourageante par les nettes améliorations qu'elle présente.\\
Le test de sensibilité à la largeur du seuil semble indiquer que la largeur de la rivière en aval du lac est un bon prédicteur de la largeur du seuil. Ces résultats restent quand même contraints par des éléments externes comme la morphologie du lac et notamment sa bathymétrie qui sont à ce jour peu documentées.\\
Des questions persistent néanmoins notamment sur la prescription d'une hypsométrie appliquée à l'échelle globale afin de mieux caractériser la dynamique des lacs et permettre un suivi plus précis des variations de stocks.\\


Les objectifs principaux énoncés au début de ce projet de recherche ont été remplis. Toutefois il reste des incertitudes à éclaircir et des biais à corriger. Les performances limitées du modèle sur certaines zones sont notamment liées aux limites qui émergent de certaines hypothèses de développement. Un des défauts de MLake est la redistribution spatiale des variables. À chaque pas de temps, les variables décrivant le stock et les hauteurs d'eau sont affectées à toutes les cellules de ce lac dans le masque de réseau. Au pas de temps journalier, cette hypothèse n'est pas tellement restrictive pour des lacs de petites dimensions. Par contre, la validité de l'hypothèse sur des lacs plus importants comme le lac Baïkal semble discutable. Dans le cadre du développement d'une morphologie plus précise, il convient de s'intéresser aussi à d'autres formes pour représenter la surface du lac. Une surface elliptique pourrait, par exemple, être testée et comparée à la version actuelle. Enfin la couverture en glace qui peut se former à la surface du lac est prise en compte dans le bilan d'énergie mais pas dans le bilan de masse. Cela introduit des biais sur les calculs de masses et de débits notamment pour les lacs de hautes latitudes recouverts de glace la majeure partie de l'année. En allant plus loin sur la représentation des débits de sortie, il est vrai que l'approche reste encore simple et mériterait un travail détaillé notamment pour la représentation de la géométrie de l'exutoire avec la prise en compte de la pente d'écoulement ou d'une largeur de seuil dynamique (par exemple sous la forme d'une fraction de la circonférence du lac).\\

Au-delà des hypothèses relatives à MLake, certains processus ne sont tout simplement pas encore pris en compte dans le modèle et participent à l'apparition d'incertitudes. Il est évident que le processus majeur à introduire dans CTRIP concerne l'anthropisation. Les activités anthropiques influencent la majorité des fleuves et lacs du monde et les futurs développements doivent se concentrer sur l'introduction des barrages-réservoirs et les prélèvements pour l'irrigation ou l'industrie. Si l'on regarde le Rhône, il est clair que l'influence de l'homme est un facteur majeur dans la régulation des débits et dans la modification des régimes hydrologiques. Au second plan, l'introduction des flux latéraux et verticaux sont indissociables du développement d'un modèle de lac. La contribution des échanges avec les aquifères dans les bassins endoréiques est indéniable mais reste peu documentée. Des incertitudes persistent sur ces échanges et la paramétrisation des flux aquifères-lacs assurerait une meilleure représentation du soutien des eaux des bassins endoréiques par des remontées du système hydrogéologique. Il en est de même pour les échanges latéraux entre le lac et les zones humides dont la saturation caractérise l'état d'humidité des sols.\\
À moyen terme le développement de MLake porte aussi sur un couplage direct avec la plateforme de modélisation SURFEX et plus particulièrement le modèle résolvant le bilan d'énergie pour les lacs, FLake.\\
Il est aussi important de travailler sur l'assimilation de donnée comme celle de la température de surface comme axe majeur pour corriger le bilan d'énergie du lac. Dans cette même optique, le couplage avec un modèle de Système Terre comme CNRM-ESM \citep{seferian2019} ou l'utilisation de CTRIP dans la représentation du cycle du carbone \citep{delire2020} nécessite la prise en compte des sédiments associés à la prédominance de processus hydrodynamiques comme les seiches.\\

Dans le futur, plusieurs pistes d'évolution sont alors à envisager afin d'obtenir un modèle de lac consistant et représentatif des processus clés pour le suivi hydrologique et climatique global. En plus des pistes de réflexions présentées précédemment, d'autres développements doivent être menés sur le modèle hydrologique de surface ISBA-CTRIP. La quantification des apports du nouveau schéma MEB pour la résolution d'un bilan multi-énergie sur les simulations hydrologiques est ainsi primordial. Il est envisageable que la représentation plus réaliste des processus de gel et dégel du sol corrige les décalages temporels qui apparaissent sur les simulations de débits dans les zones boréales. Une autre piste à exploiter est la mise en place d'une carte dynamique d'occupation des sols s'appuyant sur la distinction de zone saturée semi-permanente \citep{pekel2016} et une évolution des bases de données pour renforcer la précision des fractions de couverts. En parallèle du développement d'une hypsométrie pour les lacs, la prescription d'une profondeur dynamique pour FLake est envisageable pour corriger les termes d'évaporation dans le bilan d'énergie. Cette prescription s'appuiera sur le couplage entre MLake et SURFEX avec la prise en compte des rétroactions entre bilan d'énergie et de masse pour les lacs. Bien sûr, l'évolution temporelle de la profondeur moyenne sera associée à l'évolution d'une surface dynamique utile dans SURFEX pour anticiper l'emprise spatiale lacustre. Ce couplage est primordial pour l’utilisation du système dans un environnement de projections climatiques à l’échelle globale. \\ Certaines cartes comme le produit HydroLAKES \citep{messager2016} présentent aujourd'hui un intérêt particulier car elles fournissent des informations à haute résolution et adaptées à l'échelle globale. Dans une perspective plus lointaine d'une descente d'échelle, le modèle devra considérer des processus hydrodynamiques. Parmi ces processus, il est probable que le développement d'un schéma dynamique permette à CTRIP de proposer une modélisation de la propagation d'ondes dans le réseau hydrographique.\\

Dans tous les cas, ce travail de thèse pose les bases de la modélisation lacustre au sein des développements hydrologiques à Météo-France. Bien qu'il reste des pistes d'améliorations, MLake a été intégré dans le modèle global ISBA-CTRIP pour la simulation des débits des grands bassins fluviaux et propose maintenant un suivi des marnages pour les principaux lacs. En suivant les quelques pistes de réflexions proposées, il est tout à fait envisageable d'introduire cette paramétrisation en couplage avec un modèle de climat, comme CNRM-CM, pour les exercices de projections hydrologiques et climatiques à long terme.