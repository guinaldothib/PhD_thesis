\documentclass[a4paper,12pt,twoside]{StyleThese}

%!TEX root = Manuscrit.tex

\usepackage{amsmath,amssymb}
%\usepackage{forum,euler}
%\usepackage{palatino,eulervm}
\usepackage[tracking]{microtype}

\usepackage[sc,osf]{mathpazo}
\linespread{1.025}              
\usepackage[euler-digits,small]{eulervm}

%\usepackage{lmodern}
\usepackage[T1]{fontenc}
\usepackage[utf8]{inputenc}
\usepackage[square,sort,comma]{natbib} 
\bibliographystyle{abbrvnat}
\setcitestyle{authoryear,open={[},close={]}}
\usepackage[french]{babel}
\usepackage{cancel}
\usepackage{multirow}
\usepackage{supertabular}
\usepackage{algorithmic}
\usepackage{algorithm}
\usepackage{amsthm}
\usepackage{float}
\usepackage{tabularx}
\usepackage[justification=centering]{caption}
\usepackage{longtable}
\usepackage{titlesec}
\usepackage{setspace}
\usepackage{booktabs}
\usepackage{pdfpages}
\usepackage{titlesec}

\titlespacing\section{15pt}{12pt plus 4pt minus 2pt}{7pt plus 2pt minus 2pt}
\titlespacing\subsection{15pt}{12pt plus 4pt minus 2pt}{7pt plus 2pt minus 2pt}
\titlespacing\subsubsection{20pt}{12pt plus 4pt minus 2pt}{7pt plus 2pt minus 2pt}
\spacing{1.25}
%\usepackage[left=2cm,right=2cm,top=1.25cm,bottom=1.25cm,includefoot,includehead,headheight=13.6pt]{geometry}
%\usepackage{geometry}
\setlength{\parindent}{20pt}
\usepackage{geometry}
\geometry{hmargin=2.5cm,vmargin=2.5cm}
%\renewcommand{\baselinestretch}{1.05}

\usepackage{silence}

\WarningFilter{minitoc(hints)}{W0023}
\WarningFilter{minitoc(hints)}{W0024}
\WarningFilter{minitoc(hints)}{W0028}
\WarningFilter{minitoc(hints)}{W0030}

\usepackage{aecompl}
\usepackage{url}

% List of abbreviations
% Do not try putting acronyms in section titles, that would cause infinite loop of pdftex compilation
%\usepackage[printonlyused,withpage]{acronym}
\usepackage[withpage]{acronym}

% My pdf code

%\usepackage{ifpdf}
%%
%\ifpdf
%\usepackage[pdftex]{graphicx}
%\else
\usepackage{graphicx}
%\fi

\graphicspath{{images/}}

% Links in pdf
\usepackage{color}
\definecolor{linkcol}{rgb}{0,0,0}
\definecolor{citecol}{rgb}{0.5,0,0}

% Table of contents for each chapter

\usepackage[nottoc, notlof, notlot]{tocbibind}
\usepackage{minitoc}
\setcounter{minitocdepth}{2}
\mtcindent=15pt
\mtcselectlanguage{french}
\setcounter{mtc}{9}
% Use \minitoc where to put a table of contents

% definitions.
% -------------------

\setcounter{secnumdepth}{3}
\setcounter{tocdepth}{2}

% Some useful commands and shortcut for maths:  partial derivative and stuff

\newcommand{\pd}[2]{\frac{\partial #1}{\partial #2}}
\def\abs{\operatorname{abs}}
\def\argmax{\operatornamewithlimits{arg\,max}}
\def\argmin{\operatornamewithlimits{arg\,min}}
\def\diag{\operatorname{Diag}}
\newcommand{\eqRef}[1]{(\ref{#1})}

\usepackage{rotating}                    % Sideways of figures & tables
%\usepackage{bibunits}
%\usepackage[sectionbib]{chapterbib}          % Cross-reference package (Natural BiB)                 % Put References at the end of each chapter
                                         % Do not put 'sectionbib' option here.
                                         % Sectionbib option in 'natbib' will do.
\usepackage{fancyhdr}                    % Fancy Header and Footer

% \usepackage{txfonts}                     % Public Times New Roman text & math font

%%% Fancy Header %%%%%%%%%%%%%%%%%%%%%%%%%%%%%%%%%%%%%%%%%%%%%%%%%%%%%%%%%%%%%%%%%%
% Fancy Header Style Options

\pagestyle{fancy}                       % Sets fancy header and footer
\fancyfoot{}                            % Delete current footer settings

%\renewcommand{\chaptermark}[1]{         % Lower Case Chapter marker style
%  \markboth{\chaptername\ \thechapter.\ #1}}{}} %

%\renewcommand{\sectionmark}[1]{         % Lower case Section marker style
%  \markright{\thesection.\ #1}}         %

\fancyhead[LE,RO]{\bfseries\thepage}    % Page number (boldface) in left on even
% pages and right on odd pages
\fancyhead[RE]{\bfseries\nouppercase{\leftmark}}      % Chapter in the right on even pages
\fancyhead[LO]{\bfseries\nouppercase{\rightmark}}     % Section in the left on odd pages

\let\headruleORIG\headrule
\renewcommand{\headrule}{\color{black} \headruleORIG}
\renewcommand{\headrulewidth}{1.0pt}
\usepackage{colortbl}
\arrayrulecolor{black}

\fancypagestyle{plain}{
  \fancyhead{}
  \fancyfoot{}
  \renewcommand{\headrulewidth}{0pt}
}

%\usepackage{FrAlgorithm}
%\usepackage[noend]{FrAlgorithmic}
%\usepackage{scrextend}

%%% Clear Header %%%%%%%%%%%%%%%%%%%%%%%%%%%%%%%%%%%%%%%%%%%%%%%%%%%%%%%%%%%%%%%%%%
% Clear Header Style on the Last Empty Odd pages
\makeatletter

\def\cleardoublepage{\clearpage\if@twoside \ifodd\c@page\else%
  \hbox{}%
  \thispagestyle{empty}%              % Empty header styles
  \newpage%
  \if@twocolumn\hbox{}\newpage\fi\fi\fi}

\makeatother

%%%%%%%%%%%%%%%%%%%%%%%%%%%%%%%%%%%%%%%%%%%%%%%%%%%%%%%%%%%%%%%%%%%%%%%%%%%%%%%
% Prints your review date and 'Draft Version' (From Josullvn, CS, CMU)
\newcommand{\reviewtimetoday}[2]{\special{!userdict begin
    /bop-hook{gsave 20 710 translate 45 rotate 0.8 setgray
      /Times-Roman findfont 12 scalefont setfont 0 0   moveto (#1) show
      0 -12 moveto (#2) show grestore}def end}}
% You can turn on or off this option.
% \reviewtimetoday{\today}{Draft Version}
%%%%%%%%%%%%%%%%%%%%%%%%%%%%%%%%%%%%%%%%%%%%%%%%%%%%%%%%%%%%%%%%%%%%%%%%%%%%%%%

\newenvironment{maxime}[1]
{
\vspace*{0cm}
\hfill
\begin{minipage}{0.5\textwidth}%
%\rule[0.5ex]{\textwidth}{0.1mm}\\%
\hrulefill $\:$ {\bf #1}\\
%\vspace*{-0.25cm}
\it
}%
{%

\hrulefill
\vspace*{0.5cm}%
\end{minipage}
}

\let\minitocORIG\minitoc
\renewcommand{\minitoc}{\minitocORIG \vspace{1.5em}}

\usepackage{subfig}
\usepackage{multirow}
% \usepackage{slashbox}

\newenvironment{bulletList}%
{ \begin{list}%
	{$\bullet$}%
	{\setlength{\labelwidth}{25pt}%
	 \setlength{\leftmargin}{30pt}%
	 \setlength{\itemsep}{\parsep}}}%
{ \end{list} }


\newtheorem{definition}{Définition}
\renewcommand{\epsilon}{\varepsilon}

% centered page environment

\newenvironment{vcenterpage}
{\newpage\vspace*{\fill}\thispagestyle{empty}\renewcommand{\headrulewidth}{0pt}}
{\vspace*{\fill}}

% Hyperref code

\ifpdf
  \usepackage[pagebackref,hyperindex=true]{hyperref}
\else
  \usepackage[dvipdfm,pagebackref,hyperindex=true]{hyperref}
\fi

% nicer backref links
\renewcommand*{\backref}[1]{}
\renewcommand*{\backrefalt}[4]{%
\ifcase #1 %
(Non cité.)%
\or
(Cité en page~#2.)%
\else
(Cité en pages~#2.)%
\fi}
\renewcommand*{\backrefsep}{, }
\renewcommand*{\backreftwosep}{ et~}
\renewcommand*{\backreflastsep}{ et~}

% Change this to change the informations included in the pdf file

% See hyperref documentation for information on those parameters

\hypersetup
{
bookmarksopen=true,
pdftitle="Titre du manuscript",
pdfauthor="Votre nom", %auteur du document
pdfsubject="Description rapide du sujet du manuscrit", %sujet du document
%pdftoolbar=false, %barre d'outils non visible
pdfmenubar=true, %barre de menu visible
pdfhighlight=/O, %effet d'un clic sur un lien hypertexte
colorlinks=true, %couleurs sur les liens hypertextes
pdfpagemode=UseNone, %aucun mode de page
pdfpagelayout=SinglePage, %ouverture en simple page
pdffitwindow=true, %pages ouvertes entierement dans toute la fenetre
linkcolor=linkcol, %couleur des liens hypertextes internes
citecolor=citecol, %couleur des liens pour les citations
urlcolor=linkcol %couleur des liens pour les url
}

%epigraph style
%\usepackage{ebgaramond}
%\usepackage{epigraph}
%\setlength{\epigraphrule}{0pt}
%\setlength{\epigraphwidth}{0.6\textwidth}
% \renewcommand\textflush{flushepinormal}
% \renewenvironment{flushepinormal}{}{\vspace*{-\baselineskip}}

%\usepackage{pxfonts}
\usepackage{epigraph}
\renewcommand\textflush{flushright}

\usepackage{etoolbox}
\makeatletter
\newlength\epitextskip
\pretocmd{\@epitext}{\em}{}{}
\apptocmd{\@epitext}{\em}{}{}
\patchcmd{\epigraph}{\@epitext{#1}\\}{\@epitext{#1}\\[\epitextskip]}{}{}
\makeatother

\setlength\epigraphrule{0pt}
\setlength\epitextskip{2ex}
\setlength\epigraphwidth{.8\textwidth}

%Caption fontsize reduced
\captionsetup{font=small}
%footnote
%\makealetter
%\def\@makefnmark{\hbox{(\@textsuperscript{\normalfont\@thefnmark})}}
%\makeatother

%table en tableau
\addto\captionsfrench{\def\tablename{Tableau}}

%Footnote brackets
\renewcommand*{\thefootnote}{(\arabic{footnote})}

\begin{document}


%!TEX root = Manuscrit.tex

\vfill
\includepdf[scale=1.]{example.pdf}
\vfill

\cleardoublepage
\pagenumbering{roman}

\setcounter{page}{1}
\cleardoublepage
\chapter*{\fontfamily{lmss}\selectfont Remerciements}

\textit{Verba volant, scripta manent.}\\


\noindent Il est d’usage, en préambule d’un travail de thèse, de remercier ceux qui ont contribué à produire ce qui remplacera, pour certains, leur livre favori et, pour d’autres, servira d’excellent cale-meuble. Loin d’être un passage obligatoire, ce moment unique offre l’opportunité d’avoir une attention pour chacune des personnes ayant marqué ces trois années de travaux.\\

\noindent En premier lieu, je souhaiterais adresser un immense remerciement à mes directeurs de thèse. Tout d’abord aux deux officiels: Aaron A. Boone et Patrick Le Moigne, sans qui je n’aurais pas pris part à cette merveilleuse aventure. Merci pour votre excellence tant sur le plan scientifique qu'humain.
\noindent Ensuite j’aimerais remercier un directeur officieux: Simon Munier dont nos discussions sont un exemple parfait de la notion de relativité.\\

\noindent Merci à Christine Lac, sans jeu de mots, qui est une chercheuse talentueuse mais surtout une personne humaine et bienveillante.\\

\noindent Débuter un travail de thèse c’est aussi intégrer une équipe. D’abord MOSAYC, rapidement devenue SURFACE, j’aimerais remercier des personnes qui me sont chères: 
Delphine, dich dich à mon Google Python personnel. Adiou Marie, la serial potière du couloir. Le duo de l’extrême: Antoine et Stéphane. Adrien qui, même au GEMAP, a su me conseiller tout au long de la thèse, Sylvie pour son écoute, Diane pour sa bienveillance et sa capacité à faire la fête, Gaëtan et Malak les petits jeunes qui feront la fierté de l’équipe. Théo et Maxime les inoubliables stagiaires.\\
Une pensée aussi pour VILLE et VEGEO.

\noindent Dans la grande famille du CNRM, j’aimerais remercier tous les doctorants que j’ai pu rencontrer. D’abord les anciens, Hélène D., Carole P., Tiphaine S., Quentin R., Quentin F., César S., Léo D., Alexane L., Najda V., Yann C., Marine G., Thomas R. Puis les docteurs ou presques-docteurs Damien S., Zied.S., Clément S., Daniel. S., Martin C., Mayeul D.\\
\noindent Une attention particulière à Pierre-Antoine, mon petit karaté-geek. Mary pour l’aide, les conseils et les blagues sur les footballeuses.\\

\noindent Enfin les copains Météo qui m’ont accepté à leur table: Julie, Hélène, Marine, Alice, Lucas et Benoit. Je ne pouvais pas finir sans citer des copains en or: Thomas le plus rayonnant des Bretons, Olivier le plus Vendéen des punks, Hugo le plus chic des Basques, Marc le plus bel accent de France et Axel le plus beau mannequin maillot de bain moulant.\\

\noindent Le CNRM c’est aussi une équipe administrative dévouée: Ouria, Anita, Régine et ma chère Martine. Que la vie au CNRM est simple lorsqu’on est bien entouré! \\

\noindent J’aimerais ponctuer ce tour d’horizon du laboratoire en remerciant tout d’abord CTI pour leur aide et leur soutien. Vous êtes un pilier important, central et irremplaçable. \\
\noindent Je voudrais avoir une pensée particulière pour les copains/copines du vélo : Olivier T., Véronique M., Fred S., Jean-Marie D.\\
\noindent A la confluence de ces deux remerciements, je terminerais en remerciant celui qui est sûrement le  plus atypique et inoubliable des compagnons de route: Serge Blin.\\

\noindent Une réussite de thèse, ce sont aussi des personnes externes qui comptent beaucoup pour moi.\\
\noindent Elise, une rencontre par delà les étoiles. À nos bières du vendredi après-midi, à tes poules et à Locky.\\
\noindent Mes amis de longues dates. Arnaud présent depuis la 6ème, copilote de rodéo en tracteur tondeuse. David, présent depuis la 2nd, même si tu as eu un titre de docteur avant moi, tu restes avant tout un arracheur de dent.
Merci à tous les copains de JDA: Laurène, Cassandre, Axel, Marine, Paul.\\
Agathe, que le temps a passé depuis la maternelle. À nos soirées post-journée pédagogique, aux fourmis écrasées entre lames et lamelles, à la technique australienne. Je suis heureux que tu aies trouvé chaussure à ton pied. Thomas tu es un garçon exceptionnel, tu rayonnes et c'est toujours un plaisir que de passer du temps avec toi.\\

\noindent J’aimerais conclure en ayant une attention particulière pour ma famille. Papa, Maman, merci de votre patience, d’avoir contribué au développement de mon oreille interne, de m’avoir toujours soutenu dans mes projets (même lorsqu’il s’agissait de partir au Sud Soudan). Vincent,  je t’ai parfois fait porté le chapeau mais j’ai fait tout mon possible  pour te laisser de la place, que ce soit enfant lors de nos soirées pyjama improvisées ou plus tard lors des retours de fêtes à Juillan.\\
\noindent La famille s’agrandit aussi du côté du Sud-Est et cela grâce à Isa, Pascal, Axelle, Arthur et maintenant Elliot. Merci de m’avoir accepté. Je remercie aussi la Bonne Mère!\\

\noindent Je souhaite conclure avec la personne sans qui je ne serais jamais arrivé jusqu’ici. Pauline, merci pour le nombre incalculables d’heures passées à me soutenir, à me conseiller et surtout à m’écouter parler de mes lacs, de mes doutes comme de mes certitudes. Je mesure la chance que j’ai.

\cleardoublepage

\begin{vcenterpage}
\chapter*{{\fontfamily{lmss}\selectfont Résumé}}
\label{chap:résumé}
\addstarredchapter{Résumé}
\markboth{\uppercase{Résumé}}{}
\noindent L'hydrologie continentale s'intéresse à tous les aspects du cycle de l'eau des terres émergées et représente les flux de masse d'eau qui y sont échangés. Que ce soit dans le sous-sol ou dans le brassage continu des torrents, l'eau et les processus hydrologiques associés entretiennent un lien direct avec la dynamique atmosphérique et la variabilité climatique. De nombreuses publications révèlent l'importance des lacs dans ce cycle tout en pointant le manque de représentation dans les modèles de surface utilisés en applications climatiques.
\\
L'étude, à l'échelle globale, de ces processus hydrologiques s'appuie sur des techniques de modélisation qui, au CNRM, passent par un système couplé composé du modèle de surface ISBA et du modèle de routage en rivière CTRIP. Ces dernières années, les progrès réalisés sur les paramétrisations et la représentation des nouveaux processus dans ces modèles ont abouti à une nette amélioration des performances du système pour des applications hydrologiques couplées avec un modèle de climat.
\\
Dans la continuité de ces efforts, l'objectif principal de cette thèse est de développer une paramétrisation des lacs pour intégrer leur bilan de masse dans le modèle global CTRIP à 1/12°. Par ailleurs, le modèle développé, MLake, propose un diagnostic sur le marnage des lacs afin de permettre le suivi du niveau d'eau basé sur des mesures satellitaires.
\\
Bénéficiant d'un réseau de mesures dense et de forçages climatiques haute résolution, le bassin versant du Rhône a été choisi pour évaluer localement MLake sur la période 1960-2016. Les résultats sur trois stations de jaugeage montrent une nette progression des performances de CTRIP dans la simulation des débits du Rhône. Cela se caractérise par un effet tampon avec écrêtement des débits de crues et soutien à l'étiage. La confrontation du diagnostic sur les niveaux d'eau du Léman avec des mesures locales révèle une capacité du modèle à suivre les cycles annuels et interannuels du marnage.
\\
Une deuxième évaluation s'est ensuite portée à l'échelle globale pour confirmer le comportement du modèle dans des conditions hydroclimatiques contrastées. Cette évaluation confirme la capacité du modèle à simuler des débits réalistes mais révèle la perturbation importante du cycle hydrologique naturel par l'anthropisation. Enfin les résultats préliminaires d'une simulation globale démontrent l'intérêt d'utiliser MLake notamment par le fait que sur les 21\% des stations impactées, principalement dans les régions arctiques, 68\% d'entre elles voient un apport positif.
\\
Enfin, l'introduction d'une bathymétrie gaussienne en paramétrant l'hypsométrie des lacs, engage une réflexion sur les perspectives d'améliorations du modèle en vue d'un couplage avec un modèle de climat et de l'utilisation des données de la future mission spatiale SWOT.

\chapter*{{\fontfamily{lmss}\selectfont Abstract}}
\label{chap:abstract}
\addstarredchapter{Abstract}
\markboth{\uppercase{Abstract}}{}
\noindent The water cycle encompasses the main processes related to mass fluxes that influence the atmosphere and climate variability. More specifically, continental hydrology refers to the water transfer occurring at the land surface and sub-surface. Recent publications reveal the importance of lakes within the water cycle, but they also highlight the lack of representation of their dynamics in land surface models which are used for climate studies.
\\
Modelling is one of the main methods used for the representation of these processes at regional to global scales. The land surface model system used in this thesis is composed of the ISBA land surface model coupled to the river routing model TRIP that combines the CNRM’s latest developments for use in stand-alone hydrological applications or coupled to a climate model.
\\
This PhD is focused on the development and evaluation of lake mass-balance dynamics and water level diagnostics using a new non-calibrated model called MLake which has been incorporated into the 1/12° version of the CTRIP model. 
\\
Simulated river flows forced by high resolution hydrometeorological forcings are evaluated for the Rhone river basin against \textit{in situ} observations coming from three river gauges over the period 1960-2016. Results reveal the positive contribution of MLake in simulating Rhone discharge and in representing the lake buffer effects on peak discharge. Moreover, the evaluation of the simulated and observed water level variations show the ability of MLake to reproduce the natural seasonal and interannual cycles. 
\\
Based on the same framework, a final evaluation was conducted in order to assess the value of the non-calibrated MLake model for global hydrological applications. The results confirmed the capability of the model to simulate realistic river discharges worldwide. At 68\% of the 2000 river gauge stations impacted by the addition of the lake dynamics, which are mostly located within regions of high lake density, the new model resulted in improved simulated river discharge. The results also highlighted the strong effect of anthropization on the alterations of river dynamics, and the need for a global representation of human-impacted flows in the model. 
\\
This study has lead to several future perspectives, such as the incorporation of a parametrization of lake hypsometry for use at global scale under the hypothesis that lake bathymetry could be approximated by a Gaussian function. The implementation of such developments will improve the representation of vertical water dynamics and facilitate both the coupling of MLake within the CNRM earth system model framework and the future spatial mission SWOT for improved future global hydrological and water resource projections.

\end{vcenterpage}
%%
\cleardoublepage
%begin epigraphe
\vspace*{\fill} \epigraph{\itshape L’histoire d’un ruisseau, même de celui qui naît et se perd dans la mousse, est l’histoire de l’infini. Ces gouttelettes qui scintillent ont traversé le granit, le calcaire et l’argile ; elles ont été neige sur la froide montagne, molécule de vapeur dans la nuée, blanche écume sur la crête des flots ; le soleil, dans sa course journalière, les a fait resplendir des reflets les plus éclatants ; la pâle lumière de la lune les a vaguement irisées ; la foudre en a fait de l’hydrogène et de l’oxygène, puis d’un nouveau choc a fait ruisseler en eau des éléments primitifs.}{Élisée Reclus\\ (Histoire d'un ruisseau)}
\epigraph{\itshape Chaque homme n'est pas lui-même seulement. Il est aussi le point unique, particulier, toujours important, en lequel la vie de l'univers se condense d'une façon spéciale, qui ne se répète jamais.}{Herman Hesse \\ (Demian)}
\vfill\clearpage
%end epigraphe

\dominitoc
\tableofcontents

\adjustmtc
\listoffigures
\listoftables
\adjustmtc

\mainmatter

\chapter*{{\fontfamily{lmss}\selectfont Introduction générale}}
\label{chap:intro_géné}
 
\addstarredchapter{Introduction générale}
\markboth{\uppercase{Introduction générale}}{}

L' eau liquide est une ressource qui peut sembler abondante par son omniprésence à la surface de la Terre mais qui est à la fois rare et unique. Sa présence est aussi importante pour l'émergence et le maintien de la vie que pour les paysages qu'elle façonne, les territoires qu'elle délimite et les écosystèmes, peu connus mais riches, qu'elle héberge.\\
Pourtant, sa préservation et son accès sont de plus en plus mis en péril sous l'effet conjugué de la croissance démographique, du développement socio-économique et des dérèglements climatiques. Alors que l'accès à des conditions suffisantes d'alimentation en eau potable est un droit fondamental, le dernier rapport des Nations Unies sur la mise en valeur des ressources en eau fait état d'une situation alarmante. Au 21\ieme{} siècle, pas moins de deux milliards de personnes vivent dans des pays en situation de stress hydrique permanent et ce chiffre atteint quatre milliards si les pénuries saisonnières sont prises en compte \citep{WDDR2019}. Quasiment tous les pays sont touchés par des pénuries saisonnières même si les pays faisant face aux pénuries les plus élevées sont situés en Afrique du Nord et au Moyen/Proche-Orient (Figure \ref{stress_hydrique}).\\ 
Cette pression anthropique tend à croître puisque la consommation d'eau augmente de 1\% par an depuis 1980 et devrait continuer à augmenter jusqu'à 2050 pour être 20\% à 30\% plus élevée qu'aujourd'hui \citep{burek2016}.\\

\begin{figure}[h!]
\includegraphics[width=1.\textwidth]{stress_hydrique}
\caption{Carte des niveaux de stress hydrique pour l'année 2019. Source: \citet{WDDR2019}.}
\label{stress_hydrique}
\end{figure}


Lorsqu'on y regarde de plus près, seule une infime proportion, 2.5\%, de l'eau liquide est douce, dont les trois quarts sous forme de glace, ce qui rend cette ressource encore plus précieuse et limitée. Le tableau \ref{repartition_eau} représente la fraction des réserves totales des eaux douces et salées à l'échelle du globe répartie par réservoir et les temps de rétention associés. Ces données montrent la prépondérance de l'eau salée par rapport à l'eau douce mais aussi que les stocks peuvent être impactés de différentes manières par des perturbations externes, à cause de la variabilité de leurs temps de rétention. Ainsi les eaux souterraines, malgré un stock conséquent, sont, d'une part, difficilement accessibles et, d'autre part, possèdent des temps de rétention longs. Ces réserves d'eau sont particulièrement sensibles aux moindres perturbations telles que les abaissements de nappes qui peuvent avoir des conséquences quasi-irréversibles. D'un autre côté, les rivières ont des temps de rétention plus courts et donc un taux de renouvellement élevé. En contrepartie, les quantités stockées en surface sont moindres et la ressource est moins pérenne et plus sujette aux variations saisonnières. Dans un contexte où l'homme prélève un volume d'eau douce deux fois supérieur à toute la quantité qui ruisselle sur le globe il apparaît donc nécessaire de repenser notre mode de consommation. \\

\begin{table}[h!]
 \caption{Répartition globale des stocks d'eau dans les principaux compartiments en millions de $m^{3}$ et les temps de rétention associés.}
 \label{repartition_eau}
 \begin{tabularx}{\textwidth}{ccXXXX}
 \hline
 &\footnotesize{Nom}&\footnotesize{Volume (milliers de km$^3$)}& \footnotesize{Pourcentage des réserves totales d'eau}& \footnotesize{Pourcentage des réserves totales d'eau douce}& \footnotesize{Temps de rétention}\\
 \hline
 \multirow{3}{1.5cm}{Réserves salées}&Océans&1338000&96.6&-&3,100 ans\\
  &Aquifères&12570&0.91&-&300 ans\\
  &Lacs salés&85&0.006&-&10-1,000 ans\\
  \hline
 \multirow{8}{1.5cm}{Réserves eau douce} &Glaciers/Pergélisols&24364&\multirow{8}{1.5cm}{2.5}&69.5&16,000ans\\
  &Aquifères&10530&&30&300 ans\\
  &Lacs d'eau douce&91&&0.26&1-100ans\\
  &Humidité du sol&16.5&&0.05&280 jours\\
  &Atmosphère&12.9&&0.04&9 jours\\
  &Zones humides&11.5&&0.03&/\\
  &Rivières, fleuves&2.12&&0.006&12-20 jours\\ 
  &Biosphère&1.12&&0.003&/\\  
  \hline
 \end{tabularx}
\end{table}

La variabilité de la distribution spatio-temporelle de l'eau liquide est un moteur dans l'émergence d'enjeux à court et long terme. Les régions où la ressource se raréfie doivent faire face à des problèmes de pénuries, d'aridifications et de désertifications amenant les populations à se déplacer. À l'inverse, l'abondance provoque une augmentation de la récurrence d'évènements climatiques exceptionnels qui mettent en péril la sécurité de pays entiers. C'est par exemple le cas du Bangladesh où la montée des eaux océaniques accentue les conséquences liées aux inondations en période cyclonique.\\
Par ailleurs, l'accès à l'eau potable n'est pas seulement défini par des critères quantitatifs mais aussi qualitatifs. Ainsi, 65\% des eaux fluviales sont considérées comme menacées \citep{vorosmarty2010} et cette dégradation des écosystèmes aquatiques devient un enjeu majeur dans un contexte où nos sociétés ont besoin des services écosystémiques existant et où les conflits autour de la gestion de l'eau se multiplient. \\

Le changement climatique exacerbe ces difficultés et ces inégalités en provoquant des risques majeurs pour l'équilibre des écosystèmes globaux mais aussi pour nos sociétés. Alors qu'un milliard de personnes vit dans des zones inondables \citep{di2013}, le changement climatique contribue à une augmentation significative de l'occurence des crues à l'échelle du globe. Les conséquences de cette augmentation sont multiples et variées. Elles devraient aussi engendrer une augmentation de 580\% du nombre de personnes affectées par les crues \citep{alfieri2015} et altérer les régimes d'écoulements et les périodes d'enneigement à l'échelle continentale \citep{schneider2013, forzieri2014, ribes2019} et globale \citep{rodell2018}.\\


L'hydrologie, qui s'intéresse à tous les aspects du mouvement de l'eau sur Terre, ses conséquences sur l'environnement et nos sociétés, permet d'étudier ces phénomènes, à la confluence de domaines comme la glaciologie, la météorologie, la chimie ou encore la géographie. \'Etymologiquement, le mot "hydrologie" fait référence à l'étude de l'eau dans sa globalité et traduit plus précisément l'étude de son cycle. C'est pourquoi il incombe à l'hydrologue, au-delà de s'intéresser aux équations de bilan d'eau aux différentes échelles, de comprendre et d'illustrer ce système complexe, hétérogène et en constant renouvellement. \\
Les champs d'applications de l'hydrologie sont - donc - à la fois vastes et parfaitement définis par le cycle de l'eau. Celui-ci correspond au mouvement et renouvellement perpétuel de l'eau sur Terre que ce soit sous la forme de glace, de liquide ou de vapeur. Il décrit ainsi les connections entre les processus, plus ou moins distants, qui le composent (évaporation, infiltration, ruissellement). Il sert aussi de support à la représentation des principes physiques clés comme la conservation de la masse et illustre les changements globaux. \\
Les études dans le domaine de l'hydrologie continentale se portent plus particulièrement sur les échanges d'eau se produisant au niveau des terres émergées et répond aux enjeux liés à la ressource en eau et à la protection des biens et des personnes par la prévision du risque inondation ou de sécheresse. Sur des temps plus longs, l'hydrologie continentale s'attache à étudier des solutions d'adaptation et d'atténuation afin d'éviter des situations dramatiques environnementales comme l'augmentation des sécheresses ou humaines comme l'émergence de zones de conflits et permettant de garantir la pérennité des stocks d'eau. L'ensemble de ces éléments montre toute l'importance de représenter les processus d'hydrologie continentale pour le suivi global de la dynamique climatique \citep{alkama2008,douville2016}.\\

 
La figure \ref{cycleeau} recense l'ensemble des processus qui agissent sur les particules d'eau et décrivent le cycle de l'eau. 

\begin{figure}[h!]
 \centerline{\includegraphics[scale=0.35]{cycleeau}}
 \caption{Représentation schématique du grand cycle de l'eau. Source : \url{www.eaufrance.fr}}
 \label{cycleeau}
\end{figure} 

Ainsi en isolant une particule d'eau océanique, il est possible de tracer les différents chemins qui s'offrent à elle. Sous l'effet de forçages atmosphériques, la particule d'eau s'évapore, elle s'élève par flottabilité et vient alimenter l'atmosphère en vapeur. Dans l'atmosphère, la particule est soumise en continu à des contraintes dynamiques et thermodynamiques qui peuvent la transporter sur des distances plus ou moins grandes vers des zones où les conditions sont favorables à une modification de ses propriétés et notamment à sa condensation: nous observons ce phénomène par la formation de nuages. Si ces nuages deviennent précipitants alors la goutte d'eau retourne vers la surface, qu'elle soit océanique ou continentale, où différents itinéraires s'offrent à elle. Dans le premier cas, la goutte d'eau rejoint son point de départ: l'océan et le cycle est fermé. Dans le second cas, la goutte d'eau peut ruisseler pour rejoindre une rivière puis un fleuve, être stockée dans un lac pour finalement revenir à son point de départ: l'océan. La goutte d'eau peut aussi s'évaporer directement après avoir atteint la surface et par conséquent revenir dans l'atmosphère. Enfin elle peut interagir avec la surface terrestre et le sol pour revenir dans l'atmosphère par évaporation ou évapotranspiration selon qu'elle a été interceptée par la végétation, captée par ses racines ou encore s'infiltrer dans le sol pour alimenter les aquifères. Ces différents itinéraires sont accessibles à toute goutte d'eau liquide. Dans le cas de précipitations solides, l'eau peut être stockée sous forme de neige ou de glace au niveau de la banquise ou des régions montagneuses. Tant qu'elle ne fond pas, cette eau reste stockée et ne participe pas aux écoulements de surface. \\

À chaque étape du cycle correspond une échelle de temps qui caractérise les processus physiques et les interactions des compartiments entre eux. Ces échelles ont une très grande importance dans la caractérisation des phénomènes physiques et dans la compréhension des rétroactions entre les compartiments et leur évolution à court et long terme. Ainsi les sécheresses, même saisonnières, peuvent conduire à des déficits en eaux souterraines et avoir un effet irréversible à l'échelle humaine du fait d'un "effet mémoire" des aquifères \citep{lam2011,cuthbert2019}. Ces effets sont plus ou moins locaux et il est primordial d'intégrer les facteurs anthropiques, de plus en plus présents, dans la représentation du cycle de l'eau \citep{abbott2019}. Ces facteurs anthropiques agissent de façon plurielle que ce soit au travers de pollutions, de l'utilisation de l'eau pour l'agriculture ("eaux vertes"), de la modification d'occupation des sols ou par la perturbation des équilibres bioécologiques et des écoulements en lien avec le changement climatique. \\

Les lacs représentent environ 20\% de tout le stock en eau douce de surface \citep{messager2016}. Inégalement répartis à la surface terrestre, on compte pas moins de 117 millions de lacs, dont la superficie dépasse 0.002 km$^{2}$, soit l'équivalent de 3.7\% des terres émergées \citep{verpoorter2014}. Les régions qui bénéficient d'une densité lacustre élevée se trouvent principalement dans les hautes latitudes de l'hémisphère nord comme la Scandinavie ou le nord canadien.\\
Malgré l'abondance de cette ressource dans des régions en première ligne face au changement climatique, les processus hydrologiques associés sont peu ou mal représentés dans les modèles hydrologiques et climatiques globaux. De plus, la pression environnementale croissante sur les lacs a déjà commencé à altérer cette ressource vitale. Des études récentes montrent les effets des altérations climatiques et anthropiques sur les systèmes lacustres et mettent en garde contre les conséquences irréversibles pouvant advenir \citep{woolway2020, jenny2020}. Ces impacts se présentent sous des formes diverses qui vont de la réduction de la couverture en glace \citep{sharma2019} à des modifications conséquentes des stocks \citep{wang2018} en passant par l'acidification des eaux \citep{phillips2015} ou l'augmentation des concentrations en micropolluants et microplastiques \citep{eerkes2015, schwarzenbach2006}, voir l'asséchement total menant à la disparition de l'écosystème lacustre.\\
Au-delà des impacts sur les propriétés physiques et chimiques associées aux lacs, c'est donc tout l'écosystème lacustre qui subit des modifications. La structure écologique, notamment les réseaux trophiques, et le fonctionnement de ces écosystèmes entretiennent un vaste tissu de services, appelés services écosystémiques, dont la société en retire un bénéfice socio-économique. Une structure non exhaustive est schématisée par la Figure \ref{fig:services_eco}. Les modifications locales des conditions sont ainsi sans frontières thématiques et altèrent l'ensemble de ces services.

\begin{figure}[h!]
 \centerline{\includegraphics[width=0.9\textwidth]{regulation}}
 \caption{Principaux services écosystémiques associés au lacs. Inspirée des travaux de \citet{schallenberg2013}.}
 \label{fig:services_eco}
\end{figure} 

La modélisation joue un rôle majeur dans la description des processus environnementaux puisqu'elle permet dans certains cas de pallier les limites de l'observation. La complexité des processus et la diversité des échelles spatiales et temporelles font que l'observation reste limitée voire impossible dans certaines circonstances. En complément de ces mesures, les modèles rendent compte d'une représentation de la réalité simplifiée sur des échelles spatio-temporelles plus étendues. \\
Depuis les travaux pionniers de \citet{manabe1969} à la fin des années 60 ou de \citet{deardorff1977}, les modèles de surface (LSM) proposent aujourd'hui une description plus réaliste de processus physiques hétérogènes et de leur complexité à l'interface sol-atmosphère \citep{levis2010}. Cependant ces modèles étaient limités car ils ne pouvaient pas initier les transferts latéraux de masse. C'est ainsi que les modèles de routage en rivière (RRM) ont vu le jour en permettant alors le transfert d'eau issu des modèles de surface à travers le réseau hydrographique et par la même occasion de fermer le cycle de l'eau. Les RRMs sont aujourd'hui indispensables à la simulation des débits mais aussi à la caractérisation de l'hydrologie locale, régionale et globale que ce soit dans le cadre de la prévision des crues et des sécheresses ou plus généralement dans le suivi de la ressource en eau \citep{ducharne2003, lucas2003, lam2011, zajac2017}.\\
Malgré un développement précoce du bilan d'énergie des lacs à l'échelle globale \citep{lemoigne2016,piccolroaz2020, woolway2017a}, les lacs ont mis plus de temps à être considérés comme une composante hydrologique essentielle dans les RRMs. Du fait de leur prédominance dans certains processus hydrologiques régionaux et grâce à la meilleure résolution des modèles de climat, les lacs représentent, depuis quelques années, un intérêt majeur dans les développements hydrologiques et climatiques régionaux et globaux \citep{bowling2010,cherkauer2010,burek2013}.\\
\clearpage

\subsubsection*{\underline{{\fontfamily{lmss}\selectfont Objectifs d'étude}}}

Au sein des développements des modèles de surface utilisés au Centre National de Recherche Météorologiques (CNRM) pour la prévision hydrologique et climatique, ma thèse vise à répondre à des objectifs explicites et d'actualités concernant l'hydrologie et, plus spécifiquement, l'intégration des lacs dans un système hydrographique global. Ce travail s'appuie sur les développements récents du modèle de routage CTRIP couplé à la plateforme de modélisation SURFEX \citep{decharme2019} en vue d'une paramétrisation, à l'échelle du globe, de la dynamique massique des lacs. Plus particulièrement, cette thèse s'attache à quantifier l'effet du bilan de masse des lacs sur la modélisation de l'hydrologie continentale et les enjeux inhérents. \\
Pour cela, la thèse aborde les objectifs de recherche suivants: \\

\begin{itemize}
\item \textbf{Développer} un modèle non calibré de bilan de masse MLake capable d'améliorer la simulation des débits de rivières par CTRIP à l'échelle globale;\\

\item \textbf{Proposer} un diagnostic sur les variations de niveau d'eau dans les lacs pour un suivi dans le passé, le présent et le futur;\\

\item \textbf{Améliorer} la caractérisation des zones à enjeux par le biais d'une représentation de la bathymétrie des lacs applicable à l'échelle globale.\\
\end{itemize}

\subsubsection*{\underline{{\fontfamily{lmss}\selectfont Plan du manuscrit}}}

\noindent Pour répondre à ces objectifs de recherche, le manuscrit est divisé en quatre chapitres. \\

Le premier chapitre pose le contexte de l'étude ainsi que son cadre théorique. Pour cela une présentation des bilans régissant le cycle de l'eau est proposée en guise d'introduction aux techniques à notre disposition pour l'observer et le modéliser. Par ailleurs, des considérations théoriques en limnologie et hydrologie lacustre sont proposées pour assurer une compréhension des enjeux du développement du bilan de masse des lacs. \\

Le chapitre 2 décrit, de façon succinte, les modèles utilisés dans cette thèse et notamment les processus physiques nécessaires à la production et au transfert de masse à travers les différents compartiments. Ce chapitre s'attache à décrire la plateforme de modélisation de la surface SURFEX, le modèle de surface ISBA, le modèle de routage CTRIP ainsi que le modèle résolvant le bilan d'énergie des lacs FLake. Enfin il présente le modèle de bilan de masse MLake, au coeur de cette thèse, ses hypothèses ainsi que les étapes de son développement. \\

Le chapitre 3 porte sur l'évaluation et la validation de MLake dans une configuration locale restreinte au bassin versant du Rhône. L'évaluation sur cette zone est motivée par la disponibilité de forçages haute résolution issus de la chaîne opérationelle SAFRAN-ISBA-MODCOU et d'un réseau de mesures conséquent. Elle consiste, premièrement, à analyser les performances du système CTRIP-MLake par rapport aux simulations de référence CTRIP sur cette zone. S'en suit une double validation effectuée d'abord en comparant les simulations et les observations de débits sur trois stations de jaugeage, puis en analysant la cohérence des variations de niveau du Léman. \\

Après cette évaluation locale, le chapitre 4 propose une évaluation et une validation régionale sur trois bassins versants présentant des conditions hydrométéorologiques variées. Après cette validation, une simulation à l'échelle globale a permis de confirmier l'intérêt de prendre en compte le bilan de masse des lacs dans les régions où leur densité est forte.\\

Les perspectives amenées par ce travail de thèse, et notamment le besoin de dissocier la dynamique de masse propre aux barrages des lacs naturels, sont abordées dans une conclusion globale. Plus généralement, l'importance de l'impact anthropique est indéniable et la modélisation hydrologique doit nécessairement prendre en compte ces effets. Enfin les avancées dans le domaine de l'hydrologie lacustre passent aussi par une approche géomorphologique à développer afin d'intégrer pleinement l'hypsométrie des lacs dans la résolution du bilan de masse.

\cleardoublepage
%!TEX root = Manuscrit.tex
\chapter{{\fontfamily{lmss}\selectfont Une introduction à l'hydrologie continentale: Contexte de l'étude}}
\label{chap:intro}
\minitoc

L'intérêt de ce chapitre introductif est d'aborder les principes généraux qui régissent les processus de production et de transfert des masses d'eau au sein du système Terre et dont les lacs en sont une composante. Ainsi la description des différents bilans qui composent le cycle de l'eau et les moyens à notre disposition pour l'observer et le modéliser vont être abordés. La description exhaustive des composantes du cycle de l'eau pourrait à elle seule faire l'objet d'un travail de recherche et c'est pour cela que l'accent est mis sur la composante centrale de cette thèse: les lacs.\\
Un bref état des lieux sur les connaissances en limnologie physique et dynamique est proposé afin de comprendre les enjeux de la modélisation lacustre et les processus inhérents à ces plans d'eau tout en permettant de mieux circonscrire la problématique liée au bilan d'eau lacustre.\\
D'un point de vue thématique, ce chapitre sert aussi à détailler le contexte général et les objectifs à atteindre pour représenter la dynamique des lacs à l'échelle globale.

\section{{\fontfamily{lmss}\selectfont Les différents bilans appliqués à l'hydrologie continentale}}
\label{sec:intro_bilans}

Les processus physiques, quels qu'ils soient, sont dépendants d'une source d'énergie. Cela s'applique évidemment aux processus hydrométéorologiques dont la principale source d'énergie provient du rayonnement solaire. Au sommet de l'atmosphère, le rayonnement solaire, dont le spectre est représenté sur la figure \ref{ray_solaire}, a une valeur constante égale à 1 368 $W.m^{-2}$.

\begin{figure}[h!]
\centering
 \includegraphics[width=0.8\textwidth]{spectre-solaire}
 \caption{Distribution spectrale du rayonnement solaire au sommet de l'atmosphère et au niveau du sol. Source: \citet{malardel2005}.}
 \label{ray_solaire}
\end{figure} 

~\\
Le rayonnement solaire se répartit en tout point de notre planète suivant des caractéristiques spatiales (comme la longitude et la latitude) et temporelles (telles que l'heure ou le jour de l'année). Minimale aux pôles et maximale à l'équateur, la variation d'intensité du rayonnement engendre un bilan radiatif, en moyenne annuelle, déficitaire pour les régions polaires et excédentaire pour les régions équatoriales (Figure \ref{bilan_zonal}). 
Pourtant le système surface-atmosphère reste équilibré énergétiquement à l'échelle globale grâce à une compensation de ce déséquilibre radiatif par le biais des circulations atmosphériques et océaniques qui transportent l'énergie des régions tropicales aux régions polaires (Figure \ref{energie2}). En France, le Gulf Stream est une manifestation visible de ce type de mécanisme océanique global qui permet à la façade Atlantique de bénéficier d'une température océanique douce tout au long de l'année.\\

\begin{figure}[h!]
    \begin{minipage}[c]{.46\linewidth}
        \centering
        \includegraphics[scale=0.6]{bilan_zonal}
    \caption{Bilan radiatif zonal en moyenne annuelle et contribution du rayonnement solaire (courbe rouge), du rayonnement infra-rouge émis (courbe verte) et du rayonnement absorbé (courbe bleue). Source: \citet{malardel2005}.}  
     \label{bilan_zonal}     
    \end{minipage}
    \hfill%
    \begin{minipage}[c]{.46\linewidth}
        \centering
        \includegraphics[scale=0.6]{transport_energie}
        \caption{Représentation du transport d'énergie vers les pôles par les océans (surface verte) et l'atmosphère (surface bleue). Source: \citet{malardel2005}.}
        \label{energie2}
    \end{minipage}
\end{figure}

\noindent En pénétrant dans l'atmosphère, le rayonnement solaire est soumis à de nombreux mécanismes qui déterminent la quantité résiduelle d'énergie atteignant effectivement la surface de la Terre. Si l'on se soustrait aux hypothèses de l'optique et qu'on isole un rayon solaire, il est possible de décrire un ensemble de phénomènes qui vont agir sur ce rayon, comme la réflexion, la transmission ou l'absorption, représenté sur la figure \ref{radia_budget}. En moyenne climatologique, chaque processus interagit de façon plus ou moins importante avec le rayonnement. La part de réflexion sans changement de longueur d'onde est de 30\% tandis que l'absorption du rayonnement par l'air et les nuages compte pour 19\%. Finalement, seulement 51\% de l'énergie solaire initiale est effectivement absorbée par la surface terrestre.\\

\begin{figure}[h!]
 \centerline{\includegraphics[width=0.75\textwidth]{bilan_radiatif}}
 \caption{Bilan radiatif global au niveau de la surface terrestre et contribution des différents processus. Adapté de \citet{trenberth2009}.}
 \label{radia_budget}
\end{figure}

\noindent Grâce à cet apport d'énergie et à ses variations spatio-temporelles, deux grands cycles d'échanges se mettent en place au niveau de la surface et redistribuent les excédents entre les différents compartiments du système Terre:\\ 

\begin{itemize}
\item[$\bullet$] le cycle de l'énergie, décrit par le premier principe de la thermodynamique;
\item[$\bullet$] le cycle de l'eau, décrit par l'équation de continuité.
\end{itemize}
~\\
~\\
~\\
~\\
Ces cycles sont dépendants l'un de l'autre et interagissent en permanence; en témoigne l'augmentation de la température de surface recevant une certaine quantité d'énergie (un lac par exemple) qui se traduit par une augmentation des flux d'évaporation associés à une perte de masse.\\
Pour compléter les cycles globaux, il faut noter qu'il existe un troisième cycle, celui du carbone, primordial dans l'étude du système Terre et notamment dans un contexte de changement climatique d'origine anthropique, mais qui ne sera pas détaillé dans ce manuscrit.

\subsection{{\fontfamily{lmss}\selectfont Le bilan énergétique à la surface de la Terre}}
\label{sec:bilan_energie}

Comme une source d'énergie est indispenable à l'évolution des processus hydro-météorologiques, il est naturel de débuter la description des bilans de surface par celle du bilan énergétique. Le système considéré lors de l'étude du bilan énergétique de surface consiste en un couple défini par la surface et la couche limite atmosphérique situé au-dessus. La surface correspond à la partie, considérée immobile, où se développent les activités humaines et dont les caractéristiques propres conditionnent la limite basse de l'atmosphère. Lieu d'échanges énergétiques et d'humidité, elle est aussi la source de pollutions et de turbulence atmosphérique.\\
La couche limite atmosphérique, quant à elle, est la partie de l'atmosphère directement impactée par la surface, par frottements, et dont les temps caractéristiques d'évolutions sont courts. L'ordre de grandeur de son épaisseur est le kilomètre et elle est fortement impactée par le cycle diurne. La turbulence y est faible la nuit à cause de la stabilisation de l'atmosphère et plus forte la journée sous l'effet de la convection. Par ailleurs, c'est bien souvent à son sommet que l'on peut observer la formation de nuages. \\

Ce couple échange constamment de l'énergie sous forme de chaleur, du rayonnement et de la matière ce qui influence les caractéristiques propres à chacun et contraint leurs évolutions. Ainsi c'est la connaissance de ces échanges et notamment l'évolution des variables qui les caractérisent qui conditionne l'étude du cycle énergétique. Les variables énergétiques de surface ont des temps de réponse rapides et sont contraintes par les capacités thermiques des couches qu'elles représentent. La différence de température entre une rivière et sa berge lors d'une journée estivale est un exemple facilement observable des variations énergétiques de surface. L'apparition de gradients est conditionnée par des paramètres essentiels comme la profondeur de pénétration du rayonnement, la capacité d'absorption de la surface et l'albédo. Au niveau d'un sol nu la profondeur de pénétration est de l'ordre de quelques millimètres alors qu'elle peut atteindre plusieurs mètres pour les grands fleuves. Ces différences contribuent à la formation de zones plus ou moins propices aux échanges énergétiques (par exemple avec l'apparition de brises) mais contribuent aussi à l'hétérogénéité des réponses de ces surfaces face à des contraintes extérieures. Par conséquent, même si les échanges dépendent du type de surface et des variables thermodynamiques dans l'environnement, il est possible de dégager des grandes tendances de flux échangés intervenant à l'interface surface-atmosphère comme résumé par la figure \ref{radia_budget} en section précédente.
\\

\noindent Le bilan énergétique de surface consiste en un équilibre entre le bilan radiatif (apport ou perte d'énergie), le flux de conduction dans le milieu (par exemple le sol ou l'eau) et les flux convectifs liés à l'activité turbulente.\\

\noindent Le bilan d'énergie en surface s'écrit sous la forme: 

\begin{equation}
R_{n} = LE + H + G
\end{equation}
avec $R_{n}$, le flux radiatif net correspondant à la différence entre le flux radiatif reçu et le flux radiatif émis par la surface, $H$ le flux de chaleur sensible, $LE$ le flux de chaleur latente et $G$ le flux de chaleur échangé par conduction dans le milieu considéré. Tous ces flux s'expriment en W.m$^{-2}$. \\
\clearpage
\noindent Ces différents termes sont illustrés sur la figure \ref{termes_bilan}.\\
\begin{figure}[h!]
 \centerline{\includegraphics[width=0.85\textwidth]{termes_bilan}}
 \caption{Termes du bilan d'énergie de surface.}
 \label{termes_bilan}
\end{figure}

Le bilan radiatif net se décompose en deux types de rayonnements. D'un côté, le rayonnement solaire à courtes longueurs d'ondes et de l'autre le rayonnement infrarouge à ondes longues. Le rayonnement solaire issu de la photosphère est assimilé à celui d'un corps noir de température 6000 K dont le maximum d'intensité est émis dans le visible (0.48 $\mu.m$). La surface de la Terre ayant, quant à elle, une température moyenne aux alentours de 300 K, émet une intensité maximale dans l'infrarouge (10 $\mu. m$) \footnote{Ces longueurs d'ondes, de maximum d'émission $\lambda_{max}$, sont issues de la loi de déplacement de Wien selon $\lambda_{max} = \dfrac{a}{T}$. $a$ est une constante égale 2897 $\mu.m.K$ et T est la température du corps noir ($K$).}. \\
~\\

\noindent Le bilan radiatif, qui évolue selon le cycle diurne, équivaut donc à la somme algébrique des composantes ascendantes et descendantes des rayonnements solaire et infrarouge et s'écrit:

\begin{equation}
R_{n} = SW\downarrow + SW\uparrow + LW\uparrow + LW\downarrow
\end{equation}
avec $SW\downarrow$ le rayonnement solaire incident transmis jusqu'à la surface, $SW\uparrow$ le rayonnement solaire réfléchi par la surface, $LW\uparrow$ le rayonnement infrarouge émis par la surface, $LW\downarrow$ le rayonnement infrarouge reçu par la surface. Tous ces flux s'expriment en W.m$^{-2}$.
\\

\noindent Le bilan sur les courtes longueurs d'ondes dépend du rayonnement solaire incident et de l'albédo $\alpha$ de la surface (fraction du rayonnement réfléchi) tel que: \begin{equation}
SW_{total} = SW\downarrow + SW\uparrow = (1-\alpha) \: SW\downarrow
\end{equation}
\\
De la même façon, le bilan sur les longueurs d'ondes infrarouges dépend de la capacité d'un corps à émettre de l'énergie c'est-à-dire son émittance. Cette émittance dérive de la température d'un corps noir et est décrite par le loi de Stefan: 
\begin{equation}
M(T) = \sigma T^{4}
\end{equation}
avec $\sigma$ la constante de Stefan-Boltzmann qui vaut $5.67.10^{-8}$ W.m$^{-2}$.K$^{-4}$. \\

\noindent Le bilan sur le rayonnement infrarouge se divise aussi suivant une part d'infrarouge reçue $LW\downarrow$ et une part émise par la surface $LW\uparrow$ qui peut se résumer suivant:
\begin{equation}
LW_{total} = LW\uparrow + LW\downarrow = -\left(\epsilon \sigma T^{4} + (1-\epsilon)LW\downarrow \right) + LW\downarrow
\end{equation}
avec $\epsilon$ l'émissivité (-) et $T$ la température du corps (K).
\\

\noindent Le bilan convectif est représenté par les flux turbulents, en surface, de chaleur sensible $H$ et de chaleur latente $LE$. La chaleur sensible est définie comme la quantité d'énergie nécessaire pour augmenter la température d'un corps sans changement d'état (en W.m$^{-2}$). La chaleur latente est la quantité d'énergie nécessaire pour augmenter la température d'un corps lors d'un changement d'état (en W.m$^{-2}$).\\
Les flux turbulents en surface peuvent s'exprimer de la façon suivante:
\begin{align}
H  = \rho_{0}c_{p}\overline{w'\theta'}
\\
LE = \rho_{0}L_{v}\overline{w'q'}
\end{align}
avec $\rho_{0}$ la masse volumique de l'air (kg.m$^{-3}$), $c_{p}$ la capacité thermique à pression constante de l'air (J.kg$^{-1}$.K$^{-1}$), $L_{v}$ la chaleur latente de vaporisation de l'air (J.kg$^{-1}$), $\overline{w'\theta'}$ le flux cinématique vertical de chaleur (m.K.s$^{-1}$), $\overline{w'q'}$ le flux cinématique vertical d'humidité (m.s$^{-1}$).
\\

\noindent Enfin le flux de conduction représente les échanges de chaleur par conduction thermique entre la surface et le milieu considéré. Ce bilan s'écrit suivant la loi de Fourier qui détermine la quantité de chaleur transmise par agitation thermique et dépend des caractéristiques de conduction du milieu et donc de sa composition:
\begin{equation}
G = -\lambda \overset{\rightarrow}{grad}T 
\end{equation}
avec $\lambda$ la conductivité thermique (W.m$^{-1}$.K$^{-1}$), $T$ la température du milieu (K). \\

\noindent Tous ces termes (rayonnement et flux de chaleur) sont, par convention, positifs s'ils sont reçus par la surface et négatifs s'ils sont émis par la surface.\\ 
Le bilan énergétique à la surface s'équilibre donc suivant:
\begin{equation}
(1-\alpha) \: SW\downarrow  + \epsilon(LW\downarrow-\sigma T^{4}) = H + LE + G
\end{equation} 

\noindent Pendant la journée, le rayonnement solaire réchauffe la surface en lien avec une augmentation des flux infrarouges (la surface est assimilée à un corps noir). L'augmentation de la température de la surface engendre le déclenchement de phénomènes convectifs et un mélange par turbulence qui va réchauffer l'atmosphère. Pendant la nuit sur Terre, le rayonnement infrarouge, typiquement déficitaire, est compensé par un flux de chaleur positif provenant du sous-sol ou par un flux turbulent (si la différence de température entre la surface et la partie basse de l'atmosphère est positive). Ce système complexe est fondamental pour l'étude des phénomènes de basses couches, tel que le brouillard, et l'étude des surfaces, pour le suivi des sécheresses.

\subsection{{\fontfamily{lmss}\selectfont Le bilan hydrologique}}
\label{sec:bilan_hydrologique}

Comme nous l'avons évoqué dans l'introduction générale, le cycle de l'eau présente différents enjeux comme la gestion quantitative de l'eau en tant que ressource ou levier économique. Cette question est d'ores et déjà d'une importance capitale et tend à s'accentuer dans le contexte d'évolutions globales auquel devra faire face l'humanité dans les prochaines décennies. \\
Ainsi le bilan hydrologique permet de quantifier les rétroactions qui lient le climat à la variabilité spatiale et temporelle de la ressource en eau. Par exemple, la question des conséquences d'un déficit en pluie sur l'humidité des sols, les débits fluviaux et la recharge des eaux souterraines peut être abordée en étudiant ce bilan.\\
Les études les plus récentes révèlent qu'une personne sur cinq  dans le monde n'a pas accès à une quantité suffisante d'eau et seulement un tiers a accès à une eau de qualité acceptable \citep{who2010}. Ce stress est particulièrement inégal à travers le monde: certaines régions concentrent une grande quantité d'eau, par exemple la région des Grands Lacs Américains contient 20\% des réserves mondiales d'eau douce de surface \citep{messager2016}, tandis que d'autres souffrent d'un stress hydrique important comme par exemple les pays de la péninsule arabique. Ces inégalités tendent, par ailleurs, à s'accentuer par une rapide dégradation de la qualité et une réduction de la disponibilité en eau \citep{WDDR2019}. Au cours des 50 dernières années, la consommation d'eau a doublé et cette tendance s'accélère encore puisque l'utilisation d'eau augmente chaque année pour satisfaire les besoins accrus par la croissance démographique et le développement économique \citep{wada2013}. Associées à ces enjeux, émergent des problématiques sur l'approvisionnement en eau qui aggravent les pénuries saisonnières et mettent en exergue les modifications dues au changement climatique. Il est, par conséquent, primordial d'accroître la compréhension, l'observation et l'anticipation des processus du cycle de l'eau pour prévoir et anticiper les futures évolutions, pour limiter l'appauvrissement de la ressource et assurer la sécurité des personnes et des biens.
\\

\noindent Lorsque l'on parle de bilan hydrologique, il est question de tous les phénomènes induisant un mouvement et un renouvellement de l'eau sur Terre. Ce bilan concerne l'eau continentale, de surface et souterraine, l'eau des mers et des océans ainsi que l'eau atmosphérique et représente l'état de ce système sur une période donnée. \\
La répartition de l'eau sur Terre respecte une distribution variable suivant les différents compartiments que l'on tente d'observer et de modéliser. En moyenne, la quantité d'eau stockée est plus ou moins stable et les océans représentent le principal réservoir d'eau liquide avec 75\% des ressources mondiales sous forme d'eau salée, non directement potable.\\ Cependant, comme nous l'avons dit en introduction de ce manuscrit, chaque compartiment du cycle de l'eau possède des caractéristiques spatio-temporelles propres qui définissent les échanges entre les réservoirs. Une connaissance précise de la répartition entre ces différents compartiments et de leurs interactions assure une meilleure compréhension des enjeux liés à l'eau mais informe aussi sur la sensibilité de chaque compartiment face aux perturbations potentielles.\\

Le cycle de l'eau est complexe et la connaissance limitée de certains processus ralentit sa modélisation. En prenant en compte tous les processus du cycle de l'eau il est possible de le diviser en trois catégories: \\

\begin{itemize}
\item[$\bullet$] les précipitations;
\item[$\bullet$] les écoulements;
\item[$\bullet$] l'évaporation.
\end{itemize}
~\\
Ces trois classes quantifient de manière précise les différents échanges et processus qui décrivent le cycle de l'eau. \\
~\\
L'équation qui traduit les flux se base sur le principe de continuité. Elle caractérise le bilan de quantité d'eau entre l'entrée et la sortie de chaque système, pour une période donnée sous la forme: 

\begin{equation}
\frac{dS}{dt} = E - S 
\end{equation}
avec ${dS}$ la variation de stock pendant le temps considéré, $E$ et $S$ les quantités d'eau entrant et sortant de ce système. \\

\noindent Ce bilan d'eau est généralement exprimé en volume même si la hauteur d'eau, aussi appelée "lame d'eau" (définie comme le rapport du volume ruisselé sur la surface drainée), est privilégiée en hydrologie pour exprimer les quantités d'eau  en $mm$. Dans le cadre de cette thèse, les stocks du bilan d'eau sont exprimés sous forme de masse (kg).\\

Afin d'avoir une vision générale de ce cycle, il est nécessaire de préciser une notion importante en hydrologie: l'échelle. Quelle soit temporelle ou spatiale, l'échelle à laquelle on se place pour étudier le cycle de l'eau est importante car d'elle dépend la qualité des forçages et des données d'évaluation. De l'échelle dépend aussi la caractérisation de la variabilité spatiale des processus, principal enjeu de l'hydrologie et de la représentation de phénomènes tels que les crues \citep{bloschl1995,beven2001}. Précédemment l'échelle temporelle à été abordée afin de caractériser la sensibilité des compartiments hydrologiques à des évolutions de leurs propriétés. En hydrologie, deux échelles spatiales sont couramment utilisées avec pour chacune des conditions d'applications spécifiques à respecter. \\
Pour commencer, la première échelle correspond à la vision la plus complète du bilan d'eau c'est-à-dire l'échelle du globe. Cette échelle est caractérisée par l'équation du bilan hydrique: 
\begin{equation}
\frac{d\overline{S}}{dt} = \overline{P} - (\overline{R_{tot}} + \overline{ET}) = 0
\end{equation}
avec $\overline{S}$ le stock moyen annuel, $\overline{P}$ les précipitations moyennes annuelles, $\overline{R_{tot}}$ la valeur moyenne du ruissellement total annuel et $\overline{ET}$ l'évapotranspiration moyenne annuelle. Toutes les variables sont exprimées en mm.s$^{-1}$.
\\

Lorsqu'on applique l'équation de continuité au cycle de l'eau global, il en résulte, qu'indépendamment des processus que l'on étudie, le stock global S est conservé. Plus précisément, malgré les évolutions et les modifications qui interviennent aux échelles plus petites, la quantité d'eau, sous toutes ses formes, n'évolue pas et cela indépendamment du temps. C'est une condition importante dans le développement de modèles décrivant le cycle de l'eau car elle garantit la fermeture du bilan d'eau dans le système et assure une meilleure compréhension des flux de masse entre les compartiments.
\\

\noindent La deuxième échelle importante en hydrologie est celle bassin versant. Le bassin versant est une unité géographique hydrologiquement close sur laquelle se base l'étude du cycle hydrologique. Le bilan hydrologique du bassin versant est réalisé à un exutoire situé en aval (Figure \ref{bv}). Plus précisément, un bassin versant est la zone couvrant toute la surface topographique drainée par un cours d'eau et ses affluents à l'amont d'une section choisie. Par conséquent, une goutte d'eau tombant à l'extérieur de la ligne de partage des eaux (souvent une ligne topographique) ne peut pas rejoindre l'exutoire et ne contribuera donc pas au débit de ce bassin. \\

 \begin{figure}[h!]
 \centerline{\includegraphics[width=0.8\textwidth]{bassin_versant}}
 \caption{Représentation d'un bassin versant et de ses composantes principales.}
 \label{bv}
\end{figure}

Sur une année hydrologique $i$, l'équation de continuité s'applique à cette unité hydrologique sous la forme: 
\begin{equation}
\Delta S_{i} = P_{i} - (R_{i}+ET_{i})
\end{equation}
avec respectivement $S_{i}$, $P_{i}$, $R_{i}$ et $ET_{i}$ le stock annuel, le volume précipité, le volume ruisselé et le volume évapotranspiré au cours de l'année hydrologique $i$. Toutes les variables sont exprimées en mm.\\

Contrairement à l'année civile, une année hydrologique couvre le cycle annuel de l'eau à l'échelle du bassin versant: cette période est définie sur 12 mois à partir du mois de plus basses eaux. En France, l'année hydrologique débuté au 1$^{er}$ septembre et se termine le 31 août de l'année suivante.\\

Le bassin versant est caractérisé par des paramètres physiques et un comportement hydrologique qui vont déterminer sa réponse à des événements hydrométéorologiques. La surface drainée, le coefficient de ruissellement, la longueur hydraulique, la pente, la forme du bassin et son temps de concentration sont les caractéristiques essentielles qui permettent de décrire le bassin versant. \\

À l'opposé de l'échelle globale, ce bilan hydrique n'est pas constant et évolue suivant des facteurs climatiques, morphologiques ou géologiques en induisant une évolution à court, moyen ou long terme pour les variables étudiées. Dans ce cas, le stock associé évolue au détriment ou au profit des bassins versants voisins. À titre d'exemple, la variabilité climatique impacte fortement les bassins versants méditerranéens qui voient une intensification régionale des extrêmes de pluie \citep{tramblay2013,ribes2019} avec, paradoxalement, une diminution des cumuls et des durées de précipitations \citep{folton2019}. Malgré tout, les réponses hydrologiques à ces perturbations sont localement variables, notamment en matière de débit, et des zones climatologiquement proches peuvent engendrer des réactions hétérogènes. Ces perturbations dépendent fortement des caractéristiques intrinsèques des bassins versants et ne sont que peu liées aux changements de régimes de précipitations. Ces différences rendent les réponses inégales au sein d'une même zone géographique.

\section{{\fontfamily{lmss}\selectfont L'observation comme première approche}}
\label{sec:observations}

Depuis le début de l'humanité, l'observation est la pierre angulaire de la compréhension de l'environnement qui nous entoure. Galilée, Newton, Maxwell et tant d'autres ont vu émerger leurs découvertes de l'expérience. Aujourd'hui, l'observation est toujours au cœur de la science que ce soit en physique, en chimie, en sociologie ou bien en anthropologie. L'hydrologie ne fait pas exception et cette section balaie de façon non exhaustive les différents outils à notre disposition pour observer et étudier les processus hydrométéorologiques.\\ 
Que les observations soient adaptées à la représentation locale des processus ou bien porteuses d'une vision plus globale des mouvements d'eau, les observations sont nécessaires à la calibration et la validation d'études. Les développements techniques, par exemple dans le domaine spatial ou de la physique ondulatoire, ont permis une diversification des instruments qui, aujourd'hui, ne sont plus seulement des adaptations d'outils éprouvés dans d'autres domaines mais sont bien des moyens spécifiquement dédiés à l'observation des surfaces d'eau libre continentale.

\subsection{{\fontfamily{lmss}\selectfont Les mesures \textit{in situ}}}

Instruments historiques et essentiels à l'étude du cycle hydrologique, les stations \textit{in situ} ont permis un suivi de l'eau précis, à long terme et représentant spatialement les divers réservoirs hydrologiques. Même s'il reste des biais considérables suivant les techniques, aujourd'hui tous les compartiments du cycle de l'eau sont observés.\\

Dans le cadre du suivi des cours d'eau, l'observation quantitative et qualitative est possible soit par l'implantation de stations permanentes, soit par le biais de campagnes de mesures. Parmi les instruments permanents utilisés en hydrologie, les stations limnimétriques, composées d'une échelle limnimétrique, de capteurs et d'un enregistreur, mesurent en continu la hauteur de la surface libre d'un cours d'eau par rapport à sa hauteur initiale. L'échelle limnimétrique, dont l'altitude est rattachée au système universel de référence d'altitude (NGF IGN69) est fixée au bord du cours d'eau de sorte à mesurer le marnage (différence entre la cote à l'étiage et la cote des hautes eaux). La hauteur d'eau est lue grâce au limnigraphe, généralement via un radar ou un ultrason puis est transmise à un enregistreur qui l'archive en vue d'une reconstitution des chroniques de débits. Connaissant la hauteur d'eau, le débit est déduit d'une courbe de tarage spécifiquement construite pour la station. En complément, des campagnes de jaugeages sont effectuées régulièrement pour affiner la connaissance des couples débit/hauteur d'eau et de leurs évolutions. Ces campagnes ont pour but de déterminer les plages de variabilité des débits ainsi que les modifications morphologiques des cours d'eau par le biais de méthodes comme le jaugeage au moulinet ou l'ADCP (Acoustic Doopler Current Profiler). \\
\noindent Les réseaux d'observations sont entretenus par divers organismes qui traitent et mettent à disposition ces données. En France les hauteurs d'eau d'environ 3200 stations sont archivées, analysées et distribuées librement par la Banque Hydro (http://www.hydro.eaufrance.fr/). À l'échelle globale, les mesures de débits en rivières des principaux bassins mondiaux sont collectées et diffusées par le GRDC (Globale Runoff Data Center, figure \ref{grdc}). Ce centre de données international, sous l'autorité de l'Organisation Mondiale de la Météorologie (OMM), regroupe et met à disposition environ 4000 stations réparties sur 30 pays et plus de 9900 stations lorsque les archives de stations sont incluses, pour des chroniques pouvant remonter jusqu'à 200 ans. \\

\begin{figure}[h!]
 \centerline{\includegraphics[width=1.\textwidth]{grdc}}
 \caption{Carte des stations de mesures \textit{in situ} du réseau GRDC: Source \url{https://www.bafg.de/GRDC/}}
 \label{grdc}
\end{figure}

De nombreux instruments ont vu le jour pour capturer la diversité des variables et des paramètres qui caractérisent les lacs. Pour ce qui est des caractéristiques thermiques et optiques, l'observation profite d'avancées notables. Dans le cadre du suivi de la température des eaux, des appareils de mesures thermiques par profilage (profileur température-profondeur) ou par mouillage (thermomètres immergés) sont aujourd'hui disponibles. Les propriétés optiques reposent, quant à elles, sur des techniques de mesures de la transparence de l'eau comme par exemple la mesure par disque de Secchi: constitué d'un disque de 20 cm de diamètre divisé en quatre quadrants (2 peints en noir et 2 peints en blanc) et lesté sur une échelle graduée, le principe repose sur une relation entre la transparence de l'eau et la profondeur de disparition du disque. \\

Le suivi hydrodynamique à l'intérieur des bassins lacustres souffre d'un manque de mesures directes \textit{in situ} et se fait plutôt par le biais de variables diagnostiques comme la cote d'eau ou par l'utilisation d'objets dérivants à la surface. Certains lacs sont, aujourd'hui, instrumentés avec une association de stations limnimétriques à flotteurs et de courantomètres afin de caractériser les mouvements d'eau. Sur ce principe et pour combler ce manque, le projet lExplore lancé en 2019 par l'Ecole Polytechnique Fédérale de Lausanne a pour objectif de collecter un maximum d'informations en continu sur la physique, la chimie et la dynamique du Léman (https://wp.unil.ch/lexplore/).
Ces stations de mesures s'appuient sur des techniques éprouvées et fiables mais elles n'informent que sur une zone géographique relativement peu étendue et dépendante de la physionomie des environnements proches et des contraintes climatiques locales. \\

\noindent Ces disparités fortes entre les territoires donnent lieu à une hétérogénéité dans les mesures et limitent l'extrapolation aux zones non jaugées. Si l'on se place le long d'un cours d'eau, la couverture spatiale du réseau de mesures est bien inférieure à l'échelle temporelle de variations des processus tel que le débit. Un autre inconvénient se pose en période d'inondations où la structure dynamique des rivières rend particulièrement difficile la connaissance des débits par mesure des hauteurs d'eau du fait des variations haute fréquence, du transport solide mais aussi du débordement de la rivière qui rend caduque les courbes de tarage. De plus, ces dernières décennies les mesures \textit{in situ} en hydrologie ont souffert de nombreuses discontinuités dans leurs chroniques et pour beaucoup ne respectent plus les exigences de qualité et d'accessibilité en temps réel \citep{shiklomanov2002, hannah2011, van2016}. Tous ces inconvénients s'ajoutent à des limites d'ordre économique et humain comme les coûts de maintenance élevés, une stagnation du déploiement d'instruments dans des bassins non jaugés et l'avènement des techniques de télédétection entraînant ainsi une diminution conséquente du nombre de stations de mesures \textit{in situ} \citep{pavelsky2014}

\subsection{{\fontfamily{lmss}\selectfont La télédétection}}
Les besoins croissants de notre société pour une gestion quantitative de la ressource en eau se sont vite confrontés aux limites de l'observation \textit{in situ}. Le développement du suivi des surfaces continentales par satellites a offert une solution alternative face à la nécessité d'observer de façon homogène des systèmes isolés, peu accessibles et donc peu instrumentés.\\
Les données satellitaires sont, aujourd'hui, pleinement intégrées à la chaîne d'études scientifiques et possèdent un poids important dans la modélisation des processus de surface notamment par le développement des techniques d'assimilation de données. Plus particulièrement, les développements se sont axés sur deux champs complémentaires: celui de la mesure altimétrique et celui de la mesure optique. Cela a conduit à la première mission spatiale haute résolution dont l'un des objectifs principal est l'étude et le suivi de l'hydrologie continentale: la mission Surface Water and Ocean Topography (SWOT; \url{https://swot.jpl.nasa.gov/}).

\subsubsection{{\fontfamily{lmss}\selectfont La mesure de la hauteur des lacs}}
\citet{alsdorf2003} a démontré que les questions liées à la gestion quantitative de la ressource en eau dans un contexte de changement climatique et de croissance démographique ne pouvaient se reposer exclusivement sur un réseau composé de stations \textit{in situ}. Cette idée avait déjà germé à la fin des années 70 quand, profitant du succès des missions spatiales en océanographie comme GEOS-3 (1975) et SEASAT (1978), l'hydrologie a développé la première mission de mesure altimétrique nadir dédiée: GEOSAT (1985). Depuis cette époque et profitant de l'élan donné par les premières missions conjointes CNES/NASA pour l'océanographie comme TOPEX/POSEIDON en 1992 \citep{fu1994}, l'altimétrie satellite radar a apporté une amélioration notable dans le suivi des surfaces et des hauteurs d'eaux continentales et notamment pour le suivi des lacs \citep{calmant2008,abarca2012}\\

\noindent Le principe de l'altimétrie repose sur la mesure du temps de trajet aller-retour d'une onde réfléchie à la surface observée. Le temps de trajet de l'onde est ensuite converti pour retrouver la distance entre le satellite et la surface située à la verticale de celui-ci. Connaissant, de manière précise, l'altitude du satellite par rapport à son ellipsoïde de référence (erreur centimétrique sur la mesure de l'orbite), il est aisé d'extraire la cote d'eau de la surface observée (Figure \ref{altimetry}) à travers l'équation: 
\begin{equation}
H = Alt - R + Corr
\end{equation}
avec $H$ la hauteur mesurée de la surface réfléchissante, $Alt$ l'altitude du satellite par rapport à l'ellipsoïde, $R$ la différence d'altitude entre le satellite et la surface réfléchie appelée "range", $Corr$ les facteurs appliqués pour notamment compenser les effets de l'ionosphère sur le faisceau.\\ 

\begin{figure}[h!]
 \centering
 \includegraphics[width=0.60\textwidth]{nadir}
 \caption{Principe de l'altimétrie satellite nadir. Adapté de : \url{www.cnes.fr}}
  \label{altimetry}
\end{figure}

L'avènement de ce type d'observation a eu un impact majeur sur l'hydrologie continentale et la mise en place d'un système de surveillance des lacs \citep{birkett1995} et des rivières \citep{birkett1998,kouraev2004}. De plus, la réanalyse des différentes missions a démontré l'intérêt de reconstruire des chroniques pour la gestion des eaux continentales, technique qui donne des résultats très précis sur les hauteurs de surface libre des lacs \citep{berry2005}. \\
Cependant, l'altimétrie nadir se concentre sur les zones directement à la verticale du satellite limitant ainsi la couverture spatiale et la quantité de points couvrant effectivement un cours d'eau ou un lac. De plus, le choix de l'orbite résulte d'un compromis entre répétitivité temporelle et spatiale, cette durée variant entre 10 jours et 35 jours introduit des biais dans le suivi des surfaces.

\subsubsection{{\fontfamily{lmss}\selectfont La mesure de l'étendue des lacs}}

L'altimétrie joue un rôle essentiel pour le suivi des hauteurs d'eau, cependant pour avoir une vision complète sur la dynamique des eaux continentales et l'évolution de leur stock il est nécessaire de connaître leur étendue. Dans la constellation de satellites, des instruments optiques suivent et cartographient les étendues d'eau afin de compléter le spectre d'observations. À la différence de l'altimétrie nadir, les instruments optiques observent, de façon passive, une zone dans la direction perpendiculaire à l'azimut \footnote{L'azimut correspond à la dimension parallèle à l'avancée du satellite. La dimension perpendiculaire est appelée portée.} du satellite. Pour observer cette zone, les imageurs possèdent une visée oblique qui va définir la taille et la résolution de l'image. \\
Du fait de la  résolution limitée des images de ces instruments, de nouveaux instruments, utilisant des techniques comme le radar à synthèse d'ouverture (SAR pour Synthetic Aperture Radar en anglais) sont, aujourd'hui, embarqués à bord des satellites optiques (Figure \ref{sar}).\\

\begin{figure}[h!]
 \centerline{\includegraphics[scale=0.3]{sar}}
 \caption{Principe de la technique SAR. Source: \citet{calmant2008}.}
 \label{sar}
\end{figure}

\noindent Cette technique se base sur des principes d'optique ondulatoire et la modification de la phase d'une onde induite par la réflexion sur une surface. Dans cette configuration, une zone est observée par plusieurs faisceaux issus du même radar. La combinaison des informations provenant de ces multiples faisceaux donne une mesure de l'amplitude et de la phase de l'onde rétrodiffusée depuis le point. L'intersection de tous ces faisceaux réduit la surface observée et simule donc l'ouverture d'un instrument "synthétique" dont l'antenne est plus grande que la fenêtre d'émission initiale des satellites. \\

\noindent Ces missions spatiales ont abouti à une meilleure cartographie des surfaces en eau notamment en assurant une distinction entre lacs, rivières et plaines d'inondation. Cependant la distinction des différents types de zones humides reste notamment limitée par la résolution des instruments. Des avancées en imagerie optique ont comblé le fossé qui existait sur le suivi à long terme de ces surfaces et la distinction avec les surfaces à proximité. Ainsi, \citet{pekel2016} a traité des millions d'images Landsat à une résolution de 30m quantifiant ainsi les évolutions de surface en eaux sur les 30 dernières années. Ces cartes à très haute résolution constituent une avancée majeure car elles assurent un suivi global mais aussi régional des surfaces recouvertes de façon permanente ou semi-permanente en eau tout en appréhendant les causes de ces modifications (Figure \ref{baikal}).

\begin{figure}[h!]
 \centerline{\includegraphics[scale=0.30]{baikal}}
 \caption{Masque des eaux permanentes du bassin du lac Baïkal issu de l'analyse d'image Landsat; Source: JRC, \citet{pekel2016}.}
 \label{baikal}
\end{figure}


\subsection{{\fontfamily{lmss}\selectfont Vers des missions dédiées à l'étude du cycle de l'eau à l'échelle globale}}

Un pas a été franchi dans le suivi global des eaux continentales avec l'arrivée de missions spatiales gravimétriques comme GRACE (Gravity Recover And Climate Experiment) en 2002 et GRACE-FO (Follow-On) en 2018. Ces missions ont la particularité de ne pas effectuer de mesure directe vers la surface de la Terre mais d'utiliser les fluctuations d'un signal dans le domaine des micro-ondes entre deux satellites identiques gravitant à une distance de 220 km l'un de l'autre. Ces fluctuations sont attribuées à des  variations du champ gravitationnel terrestre qui, aux échelles de temps mensuelles et interannuelles, sont imputées à des anomalies du stock global d'eau liquide et à des redistributions dans les réservoirs de surface et souterrains \citep{tapley2004}. Cette mission spatiale est utilisée dans les étapes de validation des modèles hydrologiques \citep{niu2006} mais aussi dans la recherche des conséquences du changement climatique et de l'anthropisation sur les évolutions du stock en eau \citep{rodell2018}.\\

Les temps de revisite assez longs et la courte durée de vie des missions obligent à s'appuyer sur, au minimum, deux missions spatiales pour étudier précisément des étendues d'eau. Par conséquent, la détermination de chroniques de suivi de ces étendues passe obligatoirement par une phase de réanalyse et de mise en cohérence. Dans la suite du travail de \citet{alsdorf2003} il a été mentionné la nécessité d'une mission spatiale unique couvrant toutes les spécificités de l'hydrologie continentale. Ces caractéristiques doivent reposer sur une résolution spatio-temporelle fine: environ 100m pour une durée de revisite de quelques jours. Ces recommandations ont pour but de surveiller les variations de hauteurs des principaux cours d'eau et lacs sur des durées adaptées à l'étude de la dynamique des rivières et notamment des ondes de crues. Cette mission doit aussi être particulièrement adaptée à l'étude de bassins non (ou très peu) jaugés dans un cadre global et s'adapter aux contraintes temporelles des dynamiques de l'hydrologie continentale \citep{alsdorf2007}. De ces besoins est né le projet conjoint CNES/NASA Surface Water and Ocean Topography (SWOT) dont le lancement du satellite est prévu en 2021. Les instruments et les caractéristiques de cette mission sont regroupés dans le rapport de \citet{fu2012}.\\

\noindent La principale innovation de la mission SWOT concerne l'utilisation d'un interféromètre InSAR (Synthetic Aperture Radar Interferometer) en bande Ka (fréquence de 35.75 GHz) comme charge utile principale \citep[figure \ref{fig_swot},][]{biancamaria2016}. De part et d'autre du satellite, deux fauchées de 50km, séparées l'une de l'autre de 20km, produisent un interférogramme traité pour produire une image des hauteurs d'eau. Le principe de l'interférométrie se base sur une triangulation du signal. Les signaux rétrodiffusés sont captés par deux antennes et les hauteurs d'eau sont déduites de la différence de phase entre ces signaux.\\
\begin{figure}[h!]
    \centering
    \includegraphics[width=0.6\textwidth]{SWOT}     
    \caption{Illustration de la mission spatiale SWOT et de son instrument KaRIn. Source: NASA.}  
    \label{fig_swot}
\end{figure}

\noindent Cette technique assure une résolution pour les données brutes d'environ 6m dans la direction azimuth et dans un intervalle variant de 10 à 60 m pour la portée. La résolution horizontale attendue des produits issus de ces mesures (comme les hauteurs d'eau) est de l'ordre de 100 m.\\
Grâce à ces caractéristiques SWOT effectuera des mesures de très haute précision. Pour les hauteurs d'eau les précisions attendues sont centimétriques et pour les pentes la précision sera de l'ordre de 1.7 cm/km. De plus, l'objectif de SWOT est de fournir des informations nécessaires à la production de masques d'étendues d'eau avec un seuil de détection allant de 100 m pour la largeur de rivière et de 250 m x 250 m pour les lacs et plaines d'inondation. Le cycle temporel permet une couverture globale tous les 21 jours avec une cartographie de la majorité des étendues d'eau et rivières tous les 10 jours. Les applications de cette mission sont nombreuses, par exemple, la mesure des débits, associée à la mesure des variations de hauteur, est utile pour le suivi des variations de stocks d'environ deux tiers des lacs et réservoirs du monde. Cela permettra une étude globale du cycle de l'eau répondant aux contraintes de la modélisation \citep{biancamaria2016,cretaux2016}.\\

\noindent Des études ont déjà démontré l'intérêt des observations synthétiques SWOT pour le suivi des lacs \citep{lee2010, cretaux2016, gao2016}. SWOT est adapté à la compréhension du rôle des lacs dans le système hydrologique global et plus particulièrement dans la modélisation de leur dynamique. Le consensus sur la distribution globale des lacs n'est pas strict. Cela est notamment dû à la dépendance de l'estimation de leur distribution aux techniques d'observations et d'analyse. SWOT joue un rôle important dans la fourniture de données de distribution, mais assure aussi une information sur la dynamique spatio-temporelle de ces systèmes. Ainsi 65\% des variations du stock des eaux lacustres doivent être suivis par la mission SWOT \citep{biancamaria2009} avec une estimation précise de la densité de lacs dont l'extension spatiale est supérieure à 0.06 $km^{2}$ (équivalent au lac du Capitello en Corse). Ces données sont essentielles à une bonne compréhension du rôle des lacs dans le cycle hydrologique.


\section{{\fontfamily{lmss}\selectfont La modélisation du cycle de l'eau}}

La complexité des processus liés au cycle de l'eau et la diversité des échelles spatiales et temporelles associées rendent les études grandeur nature impossibles à mener. Afin de se soustraire à cette difficulté, il est nécessaire d'utiliser des modèles numériques qui schématisent les principes physiques étudiés. Ces modèles rendent compte d'une certaine réalité tout en s'appuyant sur des connaissances précises des processus et des observations. Ils sont adaptés aux conditions morphologiques et climatiques des zones étudiées mais aussi à l'échelle d'étude.  

\subsection{{\fontfamily{lmss}\selectfont Les modèles}}

Il est de plus en plus courant, même en dehors des sciences, d'entendre parler de "modèle". Quotidiennement lorsque l'on regarde les prévisions météorologiques elles sont, en partie, issues de modèles atmosphériques. Aussi les différents exercices de projections climatiques du Groupe d'experts Intergouvernemental sur l'\'Evolution du Climat (GIEC) ont mis en avant les "modèles de climat". Plus récemment encore, la crise sanitaire a mis en lumière les différents modèles épidémiologiques. \\
Quelle que soit la thématique étudiée, un modèle est défini par un ensemble de paramètres, de variables, d'équations et de conditions aux limites qui constituent la structure générale et l'état d'un système.\\

\noindent Dans une vision souvent trop manichéenne, l'observation et la modélisation sont opposées alors qu'elles sont justement complémentaires. L'observation est à l'origine du développement de nos connaissances actuelles, cependant ses limites apparaissent rapidement lorsque les systèmes étudiés sont aussi complexes que le cycle de l'eau. Pour appréhender cet environnement, l'hydrologue et plus généralement le scientifique s'est, donc, doté d'outils pour représenter les processus étudiés. \\

\noindent Les modèles hydrologiques sont nés de cet intérêt pour l'étude de tout ou une partie du cycle de l'eau. Il serait, bien sûr, naïf de croire que les enjeux de compréhension sont résolus simplement par le développement de modèle. En effet, les modèles reposent sur les connaissances à l'état de l'art des systèmes et proposent une vision simplifiée des phénomènes réels.\\
En ce sens, l'hydrologue modélisateur doit poser des hypothèses sur la représentation des systèmes et donner un cadre au modèle afin de garantir la description la plus correcte possible. Ce cadre repose donc sur un travail amont non négligeable qui est présenté dans la suite de la section. Il est, par conséquent, important de définir les processus que l'on veut représenter aux échelles spatio-temporelles de l'étude. Puis il faut fixer le cadre scientifique et notamment réfléchir à l'utilisation qui en sera faite pour déterminer quelle famille de modèle choisir.
\subsection{{\fontfamily{lmss}\selectfont Les composantes essentielles}}

La modélisation hydrologique est un sujet partagé par plusieurs disciplines scientifiques s'inscrivant dans un contexte d'amélioration de la connaissance des systèmes de surface. De plus, les observations sont entachées d'erreurs et, seules, ne sont pas suffisantes pour caractériser les processus physiques notamment pour le suivi, la prévision et la prévention. Le développement rigoureux d'un modèle hydrologique nécessite toutefois de connaître et de choisir judicieusement les composantes essentielles à la représentation réaliste de l'environnement d'étude. Celles-ci sont à la confluence de différents processus en entrées et sorties liés par les bilans de masse et d'énergie. De plus, elles participent à la production de flux entre l'amont et l'exutoire des bassins. Les composantes essentielles à la modélisation hydrologique grande échelle sont détaillées dans les sections suivantes.

\subsubsection{{\fontfamily{lmss}\selectfont Les précipitations}}

Les précipitations désignent l’ensemble des hydrométéores qui, après condensation, arrivent au sol. Qu’importe leur phase ou type (pluie, neige, grêle), les précipitations sont classées en deux catégories décrites par leur cumul (en mm) ou leur intensité (en mm.s$^{-1}$):\\
~\\
~\\
\begin{itemize}
			\item[$\bullet$] Les pluies convectives associées à l’élévation rapide d’une masse d’air chargée d’humidité et résultant d’une instabilité verticale de l’air;
			\item[$\bullet$] Les pluies stratiformes associées aux zones de basses pressions et résultant d’une condensation verticale lente et uniforme d’une masse d’air humide.
\end{itemize}

\noindent L’estimation du cumul et de l’intensité des précipitations est un enjeu majeur des études hydrologiques et de leurs applications \citep{winter1995}. En effet, les précipitations sont le forçage le plus important pour l’estimation du bilan en eau \citep{yilmaz2005,stisen2012}. Toutefois, la précision sur la mesure des précipitations dépend de nombreux facteurs comme le relief et l'évolution spatio-temporelle de la perturbation qui conditionnent la réaction du bassin versant et donc le type d'écoulement. L'estimation des cumuls de précipitations est la source majeure d'incertitudes en hydrologie \citep{fekete2004}. Cette estimation repose sur un réseau d'observations dense et éprouvé mais dont la variabilité spatiale est la plus complexe à appréhender. \\

Le réseau d'observations pour les précipitations se compose de trois principaux types d'appareils. \\

\begin{itemize}
\item[$\bullet$] le pluviomètre qui mesure le cumul d'eau tombée dans un intervalle de temps donné en utilisant un auget, un cylindre gradué ou un capteur optique; 
\item[$\bullet$] le pluviographe qui enregistre la hauteur instantanée d'eau;
\item[$\bullet$] le radar, instrument le plus récent, qui estime l'intensité de précipitations, par le biais de la mesure de réflectivité, sur de grandes superficies et une hauteur intégrée.
\end{itemize}
~\\
À ce jour, Météo-France opère un peu moins de 3000 stations pluviométriques (Figure \ref{pluvio}) complétées par 32 radars météorologiques (Figure \ref{systeme_mesures}) qui donnent une bonne couverture spatiale du territoire métropolitain. \\

\begin{figure}[h!]
    \begin{minipage}[c]{.46\textwidth}
        \centering
        \includegraphics[scale=0.5]{carte_reseau}
        \caption{Réseau d'observations au sol des précipitations opéré par Météo-France. \'Edition du 24/02/2020. Source: \url{http://pluiesextremes.meteo.fr/}}
        \label{pluvio}
     \end{minipage}
     \hfill%
    \begin{minipage}[c]{.46\textwidth}
     \includegraphics[scale=0.5]{reseau_aramis}
     \caption{Réseau Aramis des radars météorologiques opérés par Météo-France au 31/08/2019. Source: \url{http://meteofrance.fr}}  
     \label{systeme_mesures}   
    \end{minipage}
\end{figure}

\noindent La précision et la fiabilité de chaque instrument dépend des conditions météorologiques. Par exemple, les mesures par pluviomètre sont biaisées dans les zones exposées aux vents ou dans le cas de précipitations neigeuses. Dans tous les cas et malgré les progrès techniques des mesures, la couverture spatiale hétérogène d'un territoire reste le facteur limitant dans la mesure de précipitations avec l'apparition de zones blanches \citep{maddox2002} qui impactent la précision de la réponse modélisée et introduit des erreurs \citep{segond2007}. La qualité de l’estimation dépend donc de la disponibilité des données \citep{ly2013} ce qui peut amener des biais importants dans les modèles hydrologiques notamment dans le cas de modèles distribués, sensibles aux positions des stations de mesures \citep{bell2000, nicotina2008}. Des études ont démontré la grande variabilité des prévisions hydrologiques par rapport à la précision des mesures de précipitations \citep{kavetski2006}. Une estimation précise de la distribution spatiale et temporelle des chroniques de pluies est, par conséquent, essentielle dans ces modèles \citep{jatho2010,mercogliano2013}. 

\subsubsection{{\fontfamily{lmss}\selectfont L'évaporation}}

L'évaporation correspond au changement de phase d'une particule d'eau liquide sous forme gazeuse. En modélisation, ce processus est généralement couplé à la transpiration pour évaluer les flux de masses échangés entre la surface et l'atmosphère. Dans le cas particulier d'un changement de phase entre une particule de glace ou de neige sous forme gazeuse, on parle de sublimation.\\

L’évaporation est le terme qui assure le couplage entre le bilan d'énergie et le bilan de masse de surface. La part de rayonnement solaire absorbée par la surface contribue à son réchauffement et au changement d'état de son contenu en eau. Ce processus physique s'observe sur des surfaces humides (humidité dans le sol ou surface saturée) et dépend de l'humidité relative \footnote{Rapport entre la pression de vapeur saturante et la pression de vapeur dans l'air.} à pression et température constantes. \\
La transpiration correspond à la réponse de la végétation à ce même rayonnement solaire. Les végétaux captent de l'eau via les racines ou l'interceptent via les feuilles. Cette eau circule au sein de l'ensemble de la plante pour ensuite rejoindre les stomates, orifices qui régulent les échanges gazeux entre la plante et l'atmosphère, où elle s'évapore si elle n'est pas utilisée pour la photosynthèse.\\
L'évapotranspiration est un processus qui varie dans le temps (cycle diurne, saisonnier) et dans l'espace (latitude, longitude) suivant la quantité de rayonnement solaire reçu. Ainsi le volume d'eau évapotranspiré est plus important, à même latitude, en été qu'en hiver. \\

Comme pour l'évaporation, la sublimation est liée à un déséquilibre entre la pression de vapeur saturante et la pression de l'air à l'interface glace-atmosphère. Son impact est particulièrement important dans les régions arctiques ou montagneuses où elle retarde les ruissellements printaniers associés à la fonte nivale \citep{box2001,vionnet2014,stigter2018}. Dans les régions arctiques, la fraction des précipitations qui rejoint l'atmosphère par sublimation est comprise entre 10 à 50\% des précipitations avec des disparités expliquées par les approches utilisées, la localisation et la période d'observation \citep{pomeroy1999,groot2013}. Dans certaines régions, le taux de sublimation peut même atteindre 100\% \citep{liston2004}.\\

Pour ce qui est de l'estimation de l'évaporation, il existe aujourd'hui deux approches complémentaires, la première se basant sur des formules empiriques issues de l'expérience et une seconde profitant des développements en télédétection spatiale.\\
La méthode empirique d'estimation recommandée par la Food and Agriculture Organisation (FAO) \citep{allen1998} est celle de Penman-Monteith \citep{monteith1965}. Cette équation se base sur la caractérisation d'une surface de référence recouverte par une végétation de type gazon non soumis à un stress hydrique, de hauteur uniforme égale 0,12 m, d'albédo 0,23 et d'une résistance de surface de 70 $s.m^{-1}$. D'autres méthodes empiriques se basent sur l'estimation de l'évaporation en utilisant des instruments tels que le lysimètre ou le bac à évaporation. À cela s'ajoute l'existence d'un réseau mondial d'observations FLUXNET qui permet d'avoir accès à des données d'évaporation, issues de mesures des flux turbulents, pour plus de 500 sites \citep{baldocchi2001}. La sublimation est plus difficilement observable, des techniques similaires existent (suivi des flux turbulents ou gravimétriques) mais c'est bien souvent sur la modélisation que repose son estimation \citep{macdonald2010,groot2013}.\\
Les méthodes de télédétection se basent sur une combinaison de variables de surface tels que la température de surface, le Normalized Difference Vegetation Index (NDVI) ou encore l'humidité du sol pour estimer les flux turbulents de chaleur latente. Ainsi des produits d'évapotranspiration sont disponibles à l'échelle globale à partir de données optiques issues de missions spatiales comme MODIS \citep[Moderate-Resolution Imaging Spectroradiometer,][]{salomonson2002}.
\clearpage
\subsubsection{{\fontfamily{lmss}\selectfont Ruissellement/Infiltration}}
\label{sec:ruissellement}

Comme nous l'avons vu en introduction de ce manuscrit, plusieurs itinéraires s'offrent à une goutte d'eau de surface: elle peut s'écouler sous forme de ruissellement ou bien s'infiltrer et contribuer aux écoulements de sub-surface. Le ruissellement correspond à toute l'eau liquide s'écoulant à la surface du sol. L'infiltration, quant à elle, correspond à la part d'eau liquide qui s'infiltre dans le sol par différents processus. À l'interface entre sol et sous-sol se trouve une zone importante qui assure les transferts d'eau pour l'alimentation racinaire des végétaux.\\

Le fait qu'une goutte d'eau ruisselle ou s'infiltre dépend des propriétés hydriques du sol qui varient spatialement et temporellement. Ainsi le couvert végétal mais aussi le type de roches constituant le sous-sol vont agir sur la porosité du sol et sur la conductivité hydraulique. Par exemple, les sols bétonnés anthropisés sont plus favorables au ruissellement que les sols nus naturels ce qui explique l'accentuation des vitesses d'écoulement des eaux et donc l'intensité des crues dans les zones urbanisées \citep{nirupama2007,fox2012}. Le ruissellement n'est pas uniquement lié aux précipitations et dans des régions comme les zones arctiques, c'est la fonte nivo-glaciaire qui les alimente. Pour autant les processus de ruissellement et leurs caractéristiques restent similaires.\\
Les ruissellements de surface se produisent de deux façons: dans le cas où l'intensité de la pluie est supérieure à la capacité d'infiltration du sol on parle de ruissellement Hortonien et dans le cas où le sol est préalablement saturé, on parle de ruissellement de Dunne.\\
Le couple ruissellement/infiltration repose donc sur des contraintes physiques particulières qui sont dépendantes de facteurs internes et externes. 
Dans le cas où le sol est non saturé, l'eau peut s'infiltrer dans des pores et, sous l'effet de la gravité, percoler verticalement pour alimenter un réseau souterrain constitué de rivières et d'aquifères. Ces réservoirs souterrains représentent, par définition, l'ensemble des zones comprenant la partie saturée permanente, appelée nappe phréatique, ainsi que sa zone d'infiltration. Dans le cycle de l'eau, les rivières sont généralement connectées à un réseau de sub-surface alimenté par un aquifère dont les échanges dépendent du gradient de charge hydraulique entre la rivière et l'aquifère.

\subsubsection{{\fontfamily{lmss}\selectfont Les débits}}

Les composantes du cycle hydrologique présentées dans les paragraphes précédents participent à la production des masses d'eau qui transitent au niveau du sol et du sous-sol et assurent la continuité du cycle de l'eau. Les échanges entre ces réservoirs s'effectuent par le biais de transferts latéraux définis sous forme de débits. Ces flux correspondent, par définition, à la quantité d'eau qui traverse une section choisie sur une période fixée. Seuls les débits de surface participant aux échanges entre les rivières et les lacs seront abordés ici.\\

L’étude des écoulements d'eau à surface libre est la composante du bilan hydrologique directement accessible pour l'observation et l'exploitation et donc une des données les plus étudiées en hydrologie. Développée au 19\ieme{} siècle, cette branche de l’hydraulique étudie les écoulements dont l’interface entre l’eau et l’air est libre. Suite aux développements théoriques comme l’introduction de la formule de Manning-Strickler ou de techniques comme l’ADCP, la connaissance des débits de la plupart des rivières est aujourd'hui assez précise (en général moins de 5\% d'incertitude) et seulement limitée par les coûts économiques et humains des campagnes de mesures.\\

La connaissance du débit des rivières est essentielle pour la caractérisation du continuum écologique et du bon état écologique des masses d’eau associées. Les affluents transportent des éléments nutritifs et renouvellent les eaux nécessaires au développement de la vie dans les masses d’eau. Les émissaires, quant à eux, permettent un équilibre volumique en évacuant le trop-plein d’eau en période de hautes eaux. 

\subsection{{\fontfamily{lmss}\selectfont Les modèles hydrologiques}}
Le concept général de modélisation hydrologique consiste à déterminer numériquement l'impact d'une modification, par exemple une pluie, sur le système et ses processus, par exemple une crue. Ce concept est présenté sur la figure \ref{fonction-hydro}. \\


\noindent Pour cela, le modèle s'attache à produire l'hydrogramme en un point d'un bassin, pris typiquement à l'exutoire, en réponse à la pluie nette tombée sur la totalité du bassin. Plus particulièrement, les modèles hydrologiques déterminent, grâce à une fonction de production, la quantité d'eau qui participe aux écoulements. Cette fonction caractérise la fraction de pluie nette \footnote{Quantité de pluie qui ruisselle strictement à la surface du terrain en réponse à une averse.} qui s'écoule effectivement en un point donné. Ces modèles évaluent ensuite la répartition temporelle des écoulements à l'exutoire connaissant la fonction de transfert du bassin. Cette fonction détermine l'hydrogramme de crue à partir du hyétogramme produit grâce la fonction de production.

Cette discrétisation peut se retrouver dans la philosophie des modèles hydrologiques suivant qu'ils traitent de la fonction de production ou de la fonction de transfert. \\

\begin{figure}[h!]
 \centerline{\includegraphics[width=0.95\textwidth]{fonctions_hydrologiques}}
 \caption{Représentation schématique du principe de la modélisation hydrologique prenant en entrée un hyétogramme de pluie et produisant en sortie l'hydrogramme correspondant.}
  \label{fonction-hydro}
\end{figure}

\begin{figure}[h!]
 \centerline{\includegraphics[width=1.\textwidth]{modeles}}
 \caption{Classification des modèles hydrologiques suivant le type de processus et la dimension spatiale.}
  \label{modele}
\end{figure}
\clearpage

\noindent Dans un soucis de clarté, les différents modèles hydrologiques sont couramment classés en groupes basés sur des critères communs. Ces critères s'appuient sur une discrétisation qui peut être spatiale, temporelle ou par la méthode de résolution des processus. Dans la suite de cette section, nous nous focaliserons sur la discrétisation spatiale et par processus.
La figure \ref{modele} donne une proposition de classification pour les modèles hydrologiques.


\subsubsection{{\fontfamily{lmss}\selectfont Classification suivant la méthode de résolution}}

Une première classification des modèles hydrologiques consiste à les séparer suivant la méthode dont les processus hydrologiques sont définis.\\

Les modèles à base physique ou "mécanistes" obéissent à une structure contrainte par le principe de conservation, utilisent des lois empiriques (notamment concernant les frottements) et résolvent les équations de Saint-Venant (ces équations sont la forme des équations de Navier-Stokes intégrées selon la hauteur). Elles participent au transfert de masse vers l'aval du bassin. Dans le cas d'un écoulement unidirectionnel sans transport solide décrit par la figure \ref{troncon}, ces équations sont composées de l'équation de continuité: \\
\begin{equation}
\frac{\partial h}{\partial t} + \frac{\partial h \overline{u}}{\partial x}=0
\end{equation}

et de l'équation de conservation de la quantité de mouvement:
\begin{equation}
\rho \frac{\partial \overline{u}}{\partial t} + \rho\overline{u} \frac{\partial \overline{u}}{\partial x} = \rho g \sin\theta - \rho g \cos\theta\frac{\partial h}{\partial x} - \frac{\tau}{h}
\end{equation}
avec $\tau$ la contrainte de frottement, $\rho$ la masse volumique de l'eau, $\overline{u}$ la vitesse moyenne, $h$ la hauteur caractéristique, $\theta$ l'angle entre l'horizontale et un vecteur normal à la pente du lit de la rivière. \\

\begin{figure}[h!]
 \centerline{\includegraphics[width=0.75\textwidth]{troncon_riviere}}
 \caption{Schéma présentant les différents paramètres sur un tronçon de rivière en écoulement unidirectionnel non permanent.}
  \label{troncon}
\end{figure}

Les modèles mécanistes prennent en compte explicitement un ensemble de processus physiques, souvent les processus prédominants, et ont donc des domaines de validité très grands. De plus, les équations résolues permettent un éventail d'applications large et une compréhension précise des processus physiques car elles permettent de résoudre des problèmes uni- ou multi-dimensionnels. Ce type d'approche est utilisé dans de nombreux domaines comme la modélisation hydrodynamique pour l'étude de crues. Cependant, ces modèles sont contraints par la connaissance précise du bassin, les coûts de calcul conséquents et souvent limitant (dans le cas de résolution fine ou de zone d'étude grande) et reposent sur un réseau d'observations dense. \\

Les modèles empiriques, ou modèles pluie-débit, pour leur part s'appuient sur les observations pour reproduire une dynamique des variables de sortie (les débits) en fonction de variables d'entrée (les cumuls de précipitations). Ces modèles ne cherchent pas à identifier des mécanismes spécifiques ou à décrire des processus élémentaires. Ils s'appuient sur une analyse fréquentielle ajustée sur des observations pluie-débit pour reconstituer des séries chronologiques liant une intensité de pluie et à un débit (\textit{e.g.} lois de type Gumbel ou Generalized Extreme Value pour la construction de courbe Intensité-Durée-Fréquence). Une étape de calage est obligatoire afin de déterminer les fonctions et les paramètres qui s'ajustent au mieux aux observations. \\
Ce type de modèle s'adapte particulièrement bien à un réseau de mesures parcellaires mais constitué de longues séries temporelles. De plus, ces modèles peuvent s'appuyer sur des techniques modernes telle que l'intelligence artificielle afin d'affiner les relations statistiques et de gérer des quantités de données importantes. Par contre, il reste un inconvénient majeur à ces modèles: l'hypothèse principale considère que le schéma physique reste stable dans le temps ce qui implique que pour tout changement de physique il faut reprendre le modèle à la base et redéfinir ses caractéristiques. Ce type d'approche empêche, par ailleurs, toute interprétation physique des résultats. Un exemple de modèle est celui de "Fabret" décrit par l'équation:
\begin{equation}
Q(t + \Delta t) = \frac{a-1}{a}Q(t) + \frac{S.b(t)}{3,6 .a.\Delta t'}\prod(t)
\end{equation}
avec $a$ le coefficient de décrue, $b(t)$ un coefficient de production de pluie, $S$ la surface du bassin versant, $\Delta t'$ le délai de prise en compte de la pluie et $\prod(t)$ le cumul de pluie entre t et t+$\Delta$t. \\

Les modèles conceptuels peuvent être vus comme des modèles de complexité intermédiaire. En effet, la paramétrisation ne se base pas explicitement sur des lois physiques comme dans les modèles mécanistes, pour autant elle apporte une grande adaptabilité avec des coûts de calculs souvent raisonnables. L'idée est de représenter les processus par des approches simplifiées, des analogies ou des lois empiriques. Une analogie traditionnellement utilisée est l'approche 'réservoir' qui compartimente les différents processus comme pour le modèle LISFLOOD du CEPMMT \citep[Centre Européen pour les Prévisions Météorologiques à Moyen Terme,][]{burek2013}. Ces compartiments échangent des flux résultant d'une résolution du bilan de masse sur chaque réservoir de sorte que: 
\begin{equation}
\frac{dS}{dt} + \tau S = cste
\end{equation}
avec $S$ le stock et $\tau$ le temps caractéristique. \\

\noindent Le réservoir a une capacité de rétention proportionnelle au ruissellement reçu. La représentation des flux de masses au sein d'un bassin versant est ainsi vue comme un ensemble de réservoirs connectés dont les caractéristiques sont définies \textit{a priori}. Ces modèles sont adaptés à des applications diverses comme la prévision du risque inondation ou les études climatiques. L'inconvénient majeur porte sur l'obligation de caler certains paramètres ce qui peut rendre ces modèles complexes et difficilement exportables notamment dans le cas où une représentation détaillée de chaque réservoir est indispensable. En particulier, la nécessité de caler un grand nombre de paramètres peut conduire à des problèmes d'équifinalité liés au fait que plusieurs combinaisons de paramètres peuvent amener à des performances égales, les processus représentés par ces paramètres se compensant les uns avec les autres.

\subsubsection{Classification suivant l'échelle spatiale}

Une autre manière de définir des classes de modèles repose sur une discrétisation spatiale. En effet, l'hétérogénéité spatiale est un enjeu majeur dans le développement des modèles hydrologiques car elle conditionne leur performance et leur adaptabilité \citep{bloschl1995}. Les modèles peuvent être divisés en trois familles suivant l'unité élémentaire considérée. L'unité choisie traduit alors le niveau de détail attendu pour la représentation des processus par le modèle.\\

La première famille est celle des modèles globaux dont l'unité élémentaire est le bassin versant. Ces modèles ont des applications à l'échelle globale et prennent les bassins versants comme des entités uniques. Dans ce cas il n'y a aucune prise en compte des variabilités spatiales des paramètres. En outre, ces modèles sont très utiles pour la prévision des crues (notamment les modèles de type GR) car simples, robustes et bien adaptés à chaque bassin versant via le calage des paramètres.\\
Ici, ces modèles ne sont pas utilisés car le but est de caractériser des processus, à la fois pour mieux comprendre les mécanismes du cycle de l'eau et pour pouvoir anticiper les impacts des changements climatiques et anthropiques sur ces mécanismes.\\

Il existe également des modèles dit "distribués" ou "spatialisés" qui prennent en compte explicitement la variabilité spatiale des caractéristiques du bassin et la variabilité spatiale des forçages. Dans ce cas, un maillage régulier est utilisé dont l'unité élémentaire est une cellule de cette maille. C'est le maillage le mieux adapté pour expliciter la variabilité spatiale. Dans un cadre d'étude plus large, ce genre de modèle apporte des informations sur l'évolution future des systèmes, par exemple dans le contexte de changement climatique. En contrepartie, ce type d'approche demande de grande quantité de données à fournir afin de décrire, de façon assez détaillée, la zone étudiée et par conséquent de grandes ressources de calcul. Un autre inconvénient est le problème de surparamétrage qui se pose lors de la spatialisation. En effet, afin de caractériser chaque zone unitaire il faut un nombre équivalent d'observations indépendantes ce qui devient compliqué sur des territoires très hétérogènes.\\

La dernière famille de modèles hydrologiques est dite "semi-distribuée" et se situe dans un entre-deux. Les surfaces sont classées par types dont les comportements hydrologiques sont comparables pour chaque classe. L'unité élémentaire est le sous-bassin versant. Par conséquent, les bassins versants sont divisés en sous-bassins caractérisés par des processus et des paramètres similaires (\textit{e.g}. discrétisation par altitude ou par sous-bassins versants géologiques). Ce type d'approche présente aussi un bon compromis entre la prise en compte de la variabilité spatiale et le contexte opérationnel. Le modèle TOPMODEL \citep[TOPography based hydrological MODEL,][]{beven1979} est un exemple de modèle semi-distribué notamment utilisé en couplage avec un modèle de surface pour la prévision des crues rapides au CNRM. Ce modèle calcule des échanges d'eaux latéraux en prenant en compte seulement le ruissellement de Dunne suivant un indice topographique préalablement affecté à chaque sous-bassin. Ces indices se basent sur la capacité de rétention en eau du sol en un point par rapport à la pente et l'aire drainée sur ce même point. \\

De plus en plus, les modèles rassemblent les caractéristiques de plusieurs familles pour combiner plusieurs avantages. En effet, la structure des modèles font qu'ils négligent généralement la dynamique de la végétation et sont sensibles à la variabilité spatiale de processus comme la précipitation. Pour la prévision du risque de crues en temps réel, telle qu'effectuée au Service Central d'Hydrométéorologie et d'Appui à la Prévision des Inondations (SCHAPI), des modèles semi-empiriques semi-conceptuels, comme le modèle Génie Rural pour la Prévision \citep[GRP,][]{tangara2005,berthet2010}, sont utilisés. Le modèle, calé grâce à des chroniques de pluie et de débit observées, utilise la pluie nette (fraction des précipitations qui contribue totalement à l'écoulement) sur le bassin versant en entrée pour déterminer le débit à l'exutoire. Une partie conceptuelle, sous la forme d'un réservoir de production, assure la transformation de cette pluie en ruissellement transféré ensuite à un deuxième réservoir, le réservoir de transfert, qui calcule la propagation dans le réseau hydrologique.

\subsection{{\fontfamily{lmss}\selectfont Les modèles de routage en rivières}}

Les modèles de surface effectuent les bilans d'énergie et d'eau sur chaque maille en se focalisant sur les transferts verticaux entre l'atmosphère, la surface et le souterrain. Les modèles de routage représentent la fonction qui assure le transfert latéral de masse d'une maille amont à une maille aval du réseau, ces transferts comprennent le ruissellement jusqu'à la rivière puis la propagation dans le réseau. Les modèles de routage en rivières (RRM) sont des modèles hydrologiques qui s'attachent à convertir le ruissellement total \footnote{Volume d'eau disponible pour l'écoulement.} généré par la fonction de production en débit, dont l'écoulement gravitaire est imposé par la topographie. Ces modèles sont indispensables à la fermeture du cycle de l'eau du continuum continent-océan-atmosphère. Ces modèles sont aussi utiles pour comprendre des phénomènes tels que les apports d'eau douce en mer Méditerranée modifiant notamment la salinité et les circulations océaniques, facteurs importants des épisodes méditerranéens \citep{sauvage2018} ou les effets du Rhône sur la circulation du Léman \citep{halder2013}. \\ 

Les modèles de routage se répartissent en deux classes: les modèles hydrodynamiques et hydrologiques. Dans le cas d'un modèle hydrodynamique, le routage se base sur la résolution des équations de Saint-Venant (equation 1.17) auxquelles sont appliquées des hypothèses simplificatrices utiles à la réduction des coûts de calcul. Ces hypothèses portent sur la caractérisation de l'onde de propagation suivant un terme prédominant dans les équations de quantité de mouvement. \\

Lorsque l'amplitude et le temps de variation de la hauteur sont faibles, l'écoulement se rapproche alors d'un état de régime permanent; la vitesse d'écoulement de chaque section s'adapte quasi-instantanément à une modification de la profondeur de la rivière. Dans ce cas de figure, les termes de pression et d'inertie sont négligeables et on parle d'onde \textbf{cinématique}. Le système se réduit à un équilibre entre le terme de frottement et le terme de gravité associé à l'équation de continuité:
\begin{equation}
\frac{\partial h}{\partial t} + c(h) \frac{\partial h}{\partial x}=0
\end{equation}
avec $c$ la vitesse de propagation de l'onde sur la section.\\

Dans certains cas de figure, l'amplitude est trop importante pour négliger le gradient de pression; on parle alors d'onde \textbf{diffusive}. Les hypothèses d'ondes cinématique et diffusive sont facilement vérifiées dans un contexte de crue lente où la propagation de l'onde est peu impactée par les termes inertiels. À l'inverse, lorsque que la pente est faible, l'équilibre se fait entre le terme d'inertie et le gradient de pression: on parle d'onde \textbf{dynamique}.\\
Ces modèles hydrauliques sont couramment utilisés dans des études locales et régionales mais nécessitent des informations précises sur la topographie et les caractéristiques physiques des tronçons de rivière considérés.\\

Dans le cas d'un modèle de routage hydrologique l'approche privilégiée est conceptuelle. Les équations représentent la continuité massique ou volumique sur une section de rivière en estimant un volume stocké sur la base des débits entrants et sortants. Ce schéma est résolu de proche en proche et le diagnostic effectué à partir du volume calculé permet de déterminer les nouvelles conditions d'écoulement.\\

\noindent Le choix des équations conditionne le calcul des vitesses d'écoulement et permet de classer les différents modèles de routage de rivières. Dans le cas du modèle TRIP (Total Runoff Integrating Pathways), utilisé au CNRM, la vitesse d'écoulement est constante et uniforme ce qui rend les effets de la résolution, et indirectement d'une meilleure représentation de la topographie, minimes sur les débits simulés. Par ailleurs, les modèles historiques à réservoirs linéaires considéraient une vitesse constante dans le temps mais spatialement dépendante de paramètres tels que la topographie ou les caractéristiques physiques des tronçons de rivières. Avec une résolution grossière, ces modèles lissaient les débits par une limitation des effets de la topographie sur les vitesses notamment sur les événements de crue \citep{vorosmarty1989}. Dans les modèles actuels, le consensus se porte sur le choix d'une vitesse variable dans le temps et dans l'espace dépendante de l'équation de Manning: \begin{equation}
\overline{u} = K_{s}R_{h}^{\frac{2}{3}}i^{\frac{1}{2}}
\end{equation}
avec $\overline{u}$ la vitesse moyenne de l'écoulement dans la section (m.s$^{-1}$), $K_{s}$ le coefficient de rugosité de Strickler (m$^{\frac{1}{3}}$.s$^{-1}$), $i$ la pente hydraulique (m.m$^{-1}$) et $R_{h}$ le rayon hydraulique (m).\\

Par ailleurs, les schémas de routage se différencient aussi par leurs paramétrisations physiques. Les plus complets vont associer les aquifères, les plaines d'inondations, les lacs et les barrages \citep{hanasaki2006, lam2011,yamazaki2011, burek2013, decharme2019} tandis que les autres, souvent les modèles historiques, se limitent à la partie fluviale \citep{vorosmarty1989,coe1998}.\\

\noindent Nous verrons en détail dans le chapitre 2 la version de TRIP développée au CNRM et les hypothèses qui permettent une application à l'échelle globale couplée avec le schéma de surface ISBA du modèle de climat ARPEGE.

\section{{\fontfamily{lmss}\selectfont Introduction à la limnologie}}
\label{sec:limnologie}

Les lacs dont la superficie est supérieure à 0.002 km$^{2}$ (soit une maille de 140m x 140m) sont au nombre de 117 millions et représentent 3.7\% des terres émergées \citep{verpoorter2014}. Cependant, leur distribution géographique est inégale avec une densité lacustre particulièrement élevée dans l’hémisphère Nord et plus particulièrement dans les hautes latitudes comme en Scandinavie et au Canada \citep[Figure \ref{hydrolakes}]{downing2006}. \\

\begin{figure}[h!]
 \centering
 \includegraphics[width=0.85 \textwidth]{HydroLAKES}
 \caption{Carte mondiale représentant les lacs et réservoirs dont la superficie dépasse 10 ha et issue de la base Hydrolakes. Source: \citet{messager2016}.}
 \label{hydrolakes}
\end{figure}

Les lacs jouent un rôle triple à l’échelle de la planète. En modulant les amplitudes diurnes et saisonnières de la température de la couche limite de surface, ils sont des acteurs majeurs dans les échanges de flux énergétiques entre la surface et l'atmosphère  \citep{long2007}. Ils représentent aussi une source secondaire d’humidité et peuvent déclencher ou amplifier des conditions de précipitations \citep{miner1997}. Les lacs jouent aussi un rôle de zone tampon hydrologique entre l’amont et l’aval d'un cours d'eau \citep{spence2006}. Totalement intégrés dans le système hydrologique, ils interagissent avec tous les composants du cycle de l'eau \citep{muller2014}. Leur hydrologie dépend fortement des apports en eau par les affluents (\textit{e.g.} en période de crues ou de fonte nivale), de l’évaporation estivale et des conditions météorologiques \citep{marsh1996, winter2004}. Ces inter-dépendances particulièrement élevées amènent à des situations historiques de baisse du niveau d'eau \citep{gronewold2016, wurtsbaugh2017}. Les lacs sont, par ailleurs, des indicateurs de développement socio-économique en offrant un nombre important de services écosystémiques \citep{sarch2000,schindler2009}. Parmi ces services, les lacs sont une source d'eau douce non négligeable et de nourriture pérenne ainsi qu'un lieu attractif pour le tourisme \citep{sterner2020}. Ces services directement ou indirectement fournis par les zones lacustres sont des leviers économiques et de développement locaux et régionaux dont le bénéfice est estimé entre 169 et 403 USD par habitant et par an \citep{reynaud2017}. \\

Ces zones d'intérêt unique et leur sensibilité face à des modifications des conditions externes en font des indicateurs fiables des signatures du changement climatique ou de pollutions d'origine anthropique \citep{vincent2009,IPCC2013}. Ainsi, les lacs voient leurs paramètres physiques influencés par les modifications climatiques \citep{adrian2009}. Parmi les plus notables il est possible de noter  la hausse de la température de surface \citep{oreilly2015,woolway2017a}, la réduction de la durée de couverture en glace \citep{sharma2019}, la modification de leur dynamique donnant lieu à des changements de régime \citep{woolway2019} et le dérèglement de nombreux équilibres chimiques, biologiques et physiques \citep{yvon2012,kraemer2017}. La variation du stock en eau de ces dépressions est, entre autres, responsable de ces changements par une dynamique directement corrélée aux modifications des régimes pluviométriques et/ou à une pression croissante de l'anthropisation \citep{kolding2012,gownaris2018}.\\
Constituant une composante essentielle des recherches sur le cycle hydrologique, l’étude des lacs à l’échelle globale a un intérêt particulier dans les domaines de l’environnement, de l’agriculture, de la météorologie et de la gestion de la ressource \citep{schindler2009,seekell2014}. L’eau stockée tout au long du cycle hydrologique réagit au moindre changement d’équilibre, qu’il soit atmosphérique ou hydrogéologique \citep{dinka2014, bouchez2015, wang2018}. Cette eau peut aussi influencer de nombreux domaines comme l’écologie \citep{dudgeon2006} ou l’économie \citep{rast2000}. Par ailleurs, les petits lacs sont des proxys de changements rapides alors que les grands lacs sont révélateurs de changements de plus grande amplitude. \\
%\clearpage
Il est courant de séparer l'étude des propriétés physiques des lacs en deux branches qui, loin d'être équitablement développées, sont complémentaires. D'un côté se trouve l'étude des caractéristiques physiques, telles que la température, la densité ou encore les propriétés optiques et acoustiques, regroupée dans le domaine de la \textit{limnologie physique}. De l'autre se trouve l'étude des mouvements verticaux et horizontaux de ces plans d'eau regroupée dans le domaine de la \textit{limnologie dynamique}. Il est évident que ces deux branches ne résument pas à elles seules la complexité des processus lacustres qui font des lacs des zones à la fois uniques du point de vue de la biodiversité et des caractéristiques chimiques. Pour avoir une vision complète d'autres domaines, tout autant nécessaires à la représentation des processus intrinsèques aux lacs, devraient être étudiés, comme par exemple l'étude du cycle du carbone \citep{tranvik2009}.

\subsection{{\fontfamily{lmss}\selectfont Limnologie physique: conséquence du forçage thermique}}
\label{sec:limno_physique}

Le riche héritage scientifique de la limnologie physique provient de l'importance de la température dans les processus lacustres \citep{adrian2009}. Comme composants du système Terre, les lacs sont soumis au bilan d'énergie de surface et à des contraintes thermiques qui modifient leurs caractéristiques. En retour, les variations du bilan interne forcent les propriétés de l'environnement proche. Que ce soit à des échelles temporelles courtes ou longues, la présence de lacs en tant que machines thermiques ne peut pas être négligée \citep{balsamo2017}.
\subsubsection{{\fontfamily{lmss}\selectfont Le bilan énergétique des lacs}}

À l'intérieur du système Terre, les différents composants, soumis à des forçages extérieurs, possèdent leurs propres bilans thermiques. Les lacs ne font pas défaut à cette règle et leur bilan énergétique se compose des mêmes termes que le bilan global avec un équilibre entre le forçage radiatif, la convection et le stockage de chaleur comme illustré sur la figure \ref{ener_lac}. 

\begin{figure}[h!]
 \centering
 \includegraphics[width=0.55 \textwidth]{bilan_ener_lac}
 \caption{Bilan d'énergie à la surface du lac et quelques-uns des contributeurs à sa modification.}
 \label{ener_lac}
\end{figure}


L'apport énergétique du rayonnement solaire, qui dépend des paramètres atmosphériques, de la position géographique et varie dans le temps, suit les règles vues dans la Section \ref{sec:bilan_energie}. Le bilan radiatif moyen au niveau des lacs est positif avec des pertes par émission d'infrarouge compensées par une forte absorption du rayonnement solaire. \\

Ce bilan est tout de même inégal avec les lacs tropicaux (Lac Tanganyika, Lac Turkana) recevant en moyenne 292 W.m$^{-2}$ quand les lacs boréaux (Lac Onega, Lac Saimaa) n'en reçoivent en moyenne que 106 W.m$^{-2}$. Des disparités existent aussi dans la transparence des lacs et leur capacité à réfléchir le rayonnement solaire. Ainsi l'albédo d'un lac non gelé se situe en moyenne aux alentours de 0.06 induisant une faible réflexion du rayonnement incident à la surface et donc une pénétration plus importante du rayonnement dans le lac en comparaison d'un sol nu. Le coefficient d'absorption du lac est sélectif et dépend de la longueur d'onde mais aussi de la concentration en matières dissoutes et en suspension. Pour l'eau pure, ce coefficient se situe autour de 0.48 $\mu m$\footnote{Cette valeur correspond à une longueur d'onde dans le bleu}. Ce coefficient sert dans l'estimation de la capacité d'absorption décrite par la loi de Beer-Lambert:

\begin{equation}
\label{eq:beer_lambert}
I(z) = I_{0}e^{-\eta \: z}
\end{equation}
avec $I(z)$ l'intensité lumineuse à la profondeur $z$ , $I_{0}$ l'intensité lumineuse avant pénétration dans le milieu et $\eta$ le coefficient d'absorption du milieu ($m^{-1}$).
\noindent Ce coefficient d'absorption permet aussi de déterminer les zones d'intérêt particuliers comme la zone photique, siège des principaux processus biologiques.\\

La partie infrarouge du bilan radiatif modifie les propriétés thermiques du lac en tant que source d'énergie quand le lac se réchauffe et en tant que puits quand il se refroidit. En moyenne annuelle, ce bilan énergétique est déficitaire par rapport à l'atmosphère et les lacs émettent un rayonnement infrarouge thermique aux alentours de 10 $\mu m$ \citep{touchart2002}.\\
Concernant le bilan calorifique, les lacs échangent de la chaleur sensible par convection avec l'atmosphère. Bien que fortement dépendants du gradient thermique à l'interface lac/atmosphère, ces échanges peuvent résulter d'efforts mécaniques, tel que le vent, qui vont modifier les conditions de stabilité de la surface. Si les contraintes mécaniques persistent et qu'elles déplacent suffisamment de masses, des remontées d'eau froide au niveau des berges des lacs peuvent apparaître. Ces phénomènes, communément appelés "upwellings", participent aux mélanges des eaux, à leur oxygénation et à l'apport de nutriments.\\

Le dernier terme du bilan d'énergie correspond au bilan de chaleur latente qui se résume pour les lacs au terme puits d'évaporation imposé par les conditions thermodynamiques, mécaniques ou morphologiques. L'évaporation lacustre dépend, en premier lieu, des conditions de pression à sa surface suivant la loi de Dalton. Cette loi exprime la quantité évaporée comme proportionnelle à la différence entre pression de vapeur à la surface du plan d'eau et la pression partielle de vapeur dans l'atmosphère. Cela se résume en disant que l'évaporation potentielle augmente lorsque la pression atmosphérique diminue (typiquement pour les lacs d'altitude tels que les lacs du plateau Tibétain) ou que l'humidité relative de l'air situé au-dessus est faible (et directement lié à la température de l'air). Dans un second temps, les efforts mécaniques, de façon similaire au bilan de chaleur sensible, influencent la quantité d'eau évaporée. L'évaporation est proportionnelle, jusqu'à une certaine limite, à la vitesse du vent qui évacue les couches atmosphériques superficielles humides et les remplace par des couches où l'humidité relative est plus faible. Enfin l'évaporation dépend des caractéristiques physiques du lac et notamment de sa superficie. Ainsi plus un lac a une emprise spatiale importante plus l'air en mouvement au-dessus aura de chance d'atteindre une humidité relative maximale. Le taux d'évaporation lacustre est très variable suivant les lacs mais reste le terme dominant du bilan thermique des lacs tropicaux soumis aux alizés tel que le lac Tchad et pouvant perdre plus de 2 m d'eau par an par simple évaporation \citep{bouchez2015,pham2020}.

\subsubsection{{\fontfamily{lmss}\selectfont La température au sein des lacs}}
Le bilan énergétique modifie directement la température des lacs et, en premier lieu, la température de surface. La température d'un lac varie ainsi au cours de l'année suivant les conditions atmosphériques et le cycle saisonnier mais le lac subit aussi des modifications thermiques journalière liées au cycle diurne. Bien sûr ces variations ne sont pas comparables à celle d'un sol nu mais sont source de turbulence et assurent un bon fonctionnement écologique du plan d'eau \citep{bouffard2019}.\\
Les échanges énergétiques entre le lac et son environnement se traduisent directement par les évolutions internes de sa température. La répartition thermique horizontale et verticale conditionne de nombreux processus physiques et biologiques. Parmi ceux-ci la stratification joue un rôle majeur dans l'évaluation de l'état écologique d'un plan d'eau notamment en modifiant les échanges verticaux d'oxygène et donc les conditions propices à la production primaire \citep{elcci2008,piccolroaz2015}.\\
\clearpage
Par définition, la température et la pression influencent la masse volumique de l'eau douce. À une pression de surface constante de 1013,25 hPa, le profil de température suit une parabole dont le foyer se situe à 3.98 °C, appelée température de densité maximale (Figure \ref{water_density}).

\begin{figure}[h!]
 \centerline{\includegraphics[width=0.65 \textwidth]{water_density}}
 \caption{Évolution de la densité de l'eau douce en fonction de la température.}
 \label{water_density}
\end{figure}

\noindent Cela signifie qu'une particule d'eau subissant un refroidissement à pression constante voit sa densité diminuer et aura tendance à plonger au fond du lac si les conditions le permettent. Tant que la température de la particule reste au dessus de la limite de 3.98°C elle continue de plonger sous des couches d'eau plus chaudes et donc moins denses. Par contre, si la température continue de décroître le volume de la particule commence à se dilater et voit alors sa densité augmenter de nouveau. Étant donnée la forme de l'équation d'évolution de la densité de l'eau, un écart thermique constant se traduira par un écart de densité d'autant plus grand qu'on s'éloigne du foyer de la parabole. Cet effet est vrai pour les eaux de surface mais se complexifie en profondeur par l'action de la pression hydrostatique qui comprime les particules d'eau. En plongeant, la particule subit alors une variation adiabatique de sa température causée par un transfert du travail des forces de pression vers la particule sous forme de chaleur. Ces effets de pression biaisent les mesures de températures de profondeur. Pour se soustraire à ces effets il est donc préférable d'utiliser des paramètres non biaisés comme la température et la densité potentielle. Les effets thermiques ne sont pas l'objet de cette étude mais sont essentiels pour comprendre la façon dont sont construits les modèles thermiques de lac, le lecteur se tournera donc vers des articles spécialisés pour obtenir plus de détails.

\subsubsection{{\fontfamily{lmss}\selectfont Stratification}}

L'étude et la connaissance du profil de température vertical au sein des lacs a mis à jour un phénomène déjà connu en océanographie issu de la relation non-linéaire entre la température et la densité de l'eau: \textbf{la stratification verticale saisonnière}. Cette caractéristique distingue les lacs des cours d'eau qui l'alimentent et conditionne toute la chaîne trophique.
Dans un lac d'eau douce, la stratification apparaît lorsqu'est observé un étagement des masses d'eau suivant un profil de température qui est soit direct soit inversé (Figure \ref{stratification}). \\

\begin{figure}[h!]
 \centerline{\includegraphics[width=0.95 \textwidth]{stratification}}
 \caption{Exemple de cas de stratification pour un lac en zone tempérée. En été, l'étagement des eaux se fait en stratification directe. En hiver, la situation s'inverse et une stratification inversée se met en place.}
 \label{stratification}
\end{figure}

Dans le sens direct, la température diminue avec la profondeur amenant à un étagement des masses d'eau avec les eaux chaudes et moins denses en surface et les eaux froides et denses en profondeur. Pour les lacs assez profonds, la température diminue jusqu'à la valeur de densité maximale soit environ 4 °C. Dans le sens inverse, la température croît avec la profondeur. Cette situation se rencontre lorsque les eaux plus froides que la température maximale de densité se retrouvent au dessus de masses d'eau à la température de densité maximale.\\ 
Les états de stratification trouvent généralement leurs origines dans un forçage thermique comme la température de l'air ou dans un forçage mécanique comme le vent \citep{snortheim2017}. La stratification est d'autant plus forte que le forçage est intense et dure dans le temps, ainsi les longues périodes de gel rencontrées en Sibérie accentuent le gradient de température entre l'interface glace-eau liquide et le fond du lac. Parmi tous les états possibles, il existe aussi une situation non stratifiée appelée \textbf{homothermie} qui correspond à un profil de température homogène sur toute la profondeur du lac. Suivant la localisation et les forçages externes, les états de stratification peuvent se succéder ou alors un seul état peut dominer tout au long de l'année comme c'est le cas pour les lacs tropicaux.\\
En se basant sur cette micticité, trois grandes familles de lacs se distinguent:\\

\begin{itemize}
\item[$\bullet$] les lacs \textbf{méromictiques} dont les eaux se mélangent moins d'une fois par an;

\item[$\bullet$] les lacs \textbf{monomictiques} dont les eaux se mélangent une seule fois par an. On parle de lac monomictique chaud lorsque le mélange s'effectue à l'automne sans que la température de l'eau ne descende au dessous de 4 °C. À l'inverse, la température de l'eau d'un lac monomictique froid ne dépasse jamais 4 °C et le brassage des eaux se produit en été;

\item[$\bullet$] les lacs \textbf{polymictiques} dont les eaux se mélangent plus d'une fois par an. C'est notamment le cas des lacs \textbf{dimictiques} dont les eaux se mélangent deux fois par an comme présenté sur la figure \ref{stratification}. La polymicticité n'a pas de limite et certains lacs peuvent avoir des eaux qui se mélangent quotidiennement.
\end{itemize}
~\\
Bien sûr cette classification est générale et ne prend pas en compte les détails distinguant les lacs dont les eaux ne se mélangent que partiellement ou même de façon irrégulière.\\

\noindent Dans un cadre idéal, cet étagement saisonnier des masses d'eau serait associé à un profil de température décroissant de manière exponentielle. Or les profils observés présentent une partie concave au niveau de la couche de surface puis une inversion du profil dans une couche intermédiaire pour tendre vers un profil convexe exponentiel dans la couche la plus profonde. Cette "anomalie" a amené une distinction entre ces couches saisonnières. Se distingue dans un premier temps, la couche superficielle nommée \textit{épilimnion} qui est directement influencée par le cycle diurne ou le vent. C'est une couche soumise à la turbulence et couramment appelée couche de mélange pour traduire les processus convectifs qui s'y produisent. Dans un second temps, se trouve la \textit{thermocline} qui est définie soit comme le point d'inflexion du profil de température, soit comme la couche ayant un gradient de température de plus d'un degré par mètre. Quelle que soit sa définition, cette couche représente une "barrière thermique" entre la surface et le fond du lac qui inhibe tous les échanges gazeux ou nutritifs \citep{shimoda2011}. Enfin au-delà de cette couche intermédiaire existe \textit{l'hypolimnion} qui correspond aux eaux profondes dont la température est constante proche de 4 °C. Cette couche n'existe pas dans tous les lacs et dépend notamment de la profondeur du lac. Du fait de la présence de la thermocline au-dessus, cette couche est considérée comme relativement stable d'un point de vue mécanique et souvent isolée de l'influence de la surface \citep{wetzel1983}.\\
La stratification est notamment dépendante de la température moyenne de la colonne d'eau \citep{lewis1996,kraemer2015} et de la morphologie du lac \citep{macintyre2010,butcher2015} avec un impact fort sur les conditions de mélange.


\subsubsection{{\fontfamily{lmss}\selectfont Les lacs comme conditions à la limite des modèles atmosphériques}}

Ces conditions thermiques et énergétiques particulières font que les lacs ont une influence sur les échanges de flux d'énergie avec l'atmosphère par rapport aux autres surfaces continentales \citep{lemoigne2013,potes2017}. Les flux de chaleur latente, pouvant être importants, font des lacs des sources d’humidité qui fixent des conditions limites locales et régionales à l'atmosphère en modifiant les caractéristiques de la couche limite atmosphérique \citep{verburg2010,li2015,woolway2017b}. Plus globalement, les conditions atmosphériques modifient les caractéristiques de la surface des lacs et notamment celles d'albédo, de rugosité et de capacité thermique. La section précédente a montré que cette dernière est, elle-même, dépendante du vent et des conditions de stratification et de mélange \citep{bouffard2019}. Ces modifications entrainent aussi une modification du bilan d'énergie de surface et des flux d'énergie avec l'atmosphère \citep{dutra2010}. La prise en compte du bilan d'énergie associé aux lacs en tant que source d'humidité statique corrige donc les termes de flux de chaleur. Cette correction est due à une augmentation de l'évaporation potentielle. Par conséquent, la présence de lacs peut avoir une incidence sur le climat régional tel qu'en Afrique de l’Est ou en Scandinavie où ils modifient les régimes de pluies que ce soit en intensité mais aussi en couverture spatiale \citep{samuelsson2010,thiery2015}.\\

\begin{figure}[h!]
 \centerline{\includegraphics[width=0.9\textwidth]{lake_breeze}}
 \caption{(a) Occurence des sommets protubérants diurnes et nocturnes détectés par satellite au dessus de la zone des Grands Lacs Africains. (b) Nombre de sommets protubérants nocturnes détectés par satellite sur la période 2005-2013. Source: \citet{thiery2017}.}
  \label{lake-breeze}
\end{figure}
\clearpage

Les lacs jouent un rôle à l'échelle locale dans le développement et l'intensification de cellules convectives comme au niveau du Lac Majeur \citep{pujol2011}, du Lac Victoria \citep{thiery2016} ou du Lac Malawi \citep{koseki2019}. Ces signatures influencent notamment le régime de précipitations journalières. Au niveau du lac Victoria, le cycle diurne de précipitations convectives au dessus du lac est modifié par rapport aux surfaces environnantes avec une activité convective maximale pendant la nuit \citep{thiery2017}. Ce phénomène est lié au processus de brise qui apparaît à la frontière entre le lac et la rive et qui provoque une divergence diurne du flux atmosphérique et une convergence nocturne. Associées à l'humidification continue de l'atmosphère par le plan d'eau, il est possible de détecter des anomalies positives de précipitations sur le lac au cours de la nuit \citep[figure \ref{lake-breeze},][]{thiery2016, koseki2019}.

À l'échelle régionale, les lacs influencent les conditions atmosphériques en réponse à des conditions synoptiques particulières. Les effets de lacs sur les précipitations sont particulièrement documentés du fait de leurs fréquences et de leurs intensités sur le pourtour des Grand Lacs Américains \citep{niziol1995}. Dans le cas de conditions synoptiques où des masses d'air froides polaires sont advectées sur les Grand Lacs Américains, la présence de cette source humidité enrichit rapidement les masses d'air en vapeur d'eau et engendrent des cumuls de neige sur les côtes sud-est opposées au flux principal. Ces cumuls peuvent atteindre plusieurs dizaines de centimètres en quelques heures (Figure \ref{lake-effect-2}).
~\\

\begin{figure}[h!]
    \begin{minipage}[c]{.45\linewidth}
        \centering
        \includegraphics[scale=0.3]{lake_effect}       
    \end{minipage}
    \hfill%
    \begin{minipage}[c]{.45\linewidth}
        \centering
        \includegraphics[scale=0.05]{lake_effect_2}
    \end{minipage}
  \caption{Processus de formation des phénomènes d'enneigement extrême par effet lac et composition colorée du phénomène aux États-Unis. Source photo et schéma adapté de: \url{https://www.weather.gov}}
   \label{lake-effect-2}
\end{figure}

L'intégration des lacs dans les modèles de prévisions du temps améliore la prévision de précipitations à l'échelle régionale. Ainsi, la présence de lacs provoquerait une augmentation des cumuls convectifs de l'ordre de 20\% et des anomalies négatives jusqu'à 70\% en juin en Finlande \citep{samuelsson2010}. Ces effets ont des conséquences majeures dans les applications météorologiques et climatiques puisque ces améliorations ont abouti à une réduction des biais systématiques des modèles climatiques \citep{lemoigne2016}. \\
Dans la même optique, \citet{balsamo2012} confirme une réduction des erreurs de prévisions météorologiques sur les températures de l'air au printemps et en été, dans ces mêmes régions, par la résolution du bilan d'énergie lacustre dans le couplage surface-atmosphère.

\subsubsection{{\fontfamily{lmss}\selectfont L'importance des lacs en climat}}

Le bilan énergétique des lacs est aussi important pour les études climatiques. Les lacs ont une influence sur les climats locaux et régionaux aux échelles saisonnières, interannuelles et séculaires \citep{bonan1995,samuelsson2010, bowling2010, martynov2012,lemoigne2016}. Les lacs font varier la dynamique des stocks d'énergie de surface mais conditionnent aussi les échanges entre les différents acteurs de ce bilan d'énergie. Ainsi \citet{dutra2010} démontre l'influence de l'ajout des lacs dans le partage des flux d'énergie de surface à l'échelle globale. L'ajout d'une paramétrisation du bilan d'énergie propre aux lacs améliore les estimations de stockage énergétique et influe sur les estimations d'évapotranspiration. Plusieurs études ont démontré qu'à l'échelle régionale, la température de l'air à 2 m (T2m) est modifiée par la présence de lacs et cela proportionnellement à leur densité \citep{samuelsson2010,lemoigne2016}. Dans ses travaux, \citet{krinner2003} a montré que l'influence climatique des lacs était particulièrement visible en été dans les régions boréales lorsque ceux-ci sont libre de glace. L'inertie thermique intrinsèque aux lacs provoque un décalage ainsi qu'une atténuation des réponses de températures de surface par rapport aux échelles temporelles classiques de variations des températures atmosphériques. En d'autres termes, des anomalies négatives de T2m au cours du  printemps/été et positives au cours de l'automne/hiver apparaissent. Loin d'être négligeables, ces anomalies de température sont, en moyenne, de l'ordre du degré et des anomalies dépassant 1.5 °C ont été mesurées autour des lacs Ladoga et Onega (Russie) \citep[Figure \ref{samuelsson},][]{samuelsson2010}. \\
\clearpage
\begin{figure}[h!]
 \centerline{\includegraphics[scale=0.35]{samuelsson}}
 \caption{Différence de température de l'air à 2m entre une simulation avec et sans lac (en °C). Source: \citet{samuelsson2010}.}
 \label{samuelsson}
\end{figure}

Les lacs font partie de boucles de rétroactions climatiques les rendant fortement dépendants des forçages atmosphériques \citep{de2006,lei2014,mao2018}. Une hausse généralisée de la température moyenne de l’air provoque une augmentation de l’évapotranspiration potentielle mais entraine aussi une accélération de la fonte glaciaire ainsi qu'une modification du régime de ruissellement. Cela est encore plus visible dans les régions boréales où le réseau lacustre est particulièrement dense et est dépendant du comportement hydrologique régional. L’évolution des conditions atmosphériques conduit à une modification de ce réseau par l’apparition ou la disparition de lacs et donc la modification des conditions atmosphériques et hydrologiques \citep{chen2013}. \\

\subsection{{\fontfamily{lmss}\selectfont Limnologie dynamique: les lacs en mouvement perpétuel}}
\label{sec:limno_dyn}

Contrairement à certaines idées reçues les eaux d'un lac ne stagnent pas. Les variations de son volume sont caractérisées par une vaste gamme de fréquences allant de variations basses fréquences, comme la baisse progressive des niveaux du lac Tchad, vers des fréquences plus élevées associées à des modifications soudaines de l'apport en eau ou du déversement lié à l'activité anthropique. Plusieurs processus engendrent ces mouvements et assurent une redistribution verticale et horizontale des eaux. Les mouvements d'eau des lacs suivent les lois de l'hydrodynamique et leurs dimensions relativement faibles contribuent à l'apparition de mouvements spécifiques. Dans tous les cas, le volume d'eau contenu dans les lacs est fortement dépendant des contraintes amont du bassin versant qui domine son alimentation à travers le bilan hydrologique.

\subsubsection{{\fontfamily{lmss}\selectfont Foyer de mouvements horizontaux}}

De façon similaire aux océans, les lacs, et notamment les plus grands, sont soumis à des mouvements horizontaux tels que les vagues et les courants. Les vagues sont des ondes progressives présentent à la surface du lac dépendant des conditions de vitesse et de direction du vent. De nombreuses études \citep{mccombs2014,ji2017,grieco2019} ont permis de mieux comprendre ces phénomènes et leurs conséquences parfois historiques comme le naufrage du SS Edmund Fitzgerald en 1975 sur le lac Supérieur attribué à des vagues d'environ 7.5 m de haut \citep{hultquist2006}. Les courants lacustres sont la traduction des efforts mécaniques qui agissent au sein du lac. Contrairement aux courants océaniques, les courants lacustres ont la particularité de se produire dans des bassins de dimensions relativement petites \citep{beletsky1999,laval2003,amadori2018}. Ils ont un aspect primordial en limnologie car ils informent sur la redistribution de matières en suspension comme les sédiments ou encore sur les producteurs primaires comme le phytoplancton. Une estimation précise de ces courants assure aussi une meilleure anticipation des évolutions de températures et les transferts de polluants au sein du bassin lacustre \citep{baracchini2020}.

\subsubsection{{\fontfamily{lmss}\selectfont Mouvements verticaux}}

La convection joue un rôle capital dans la modification de la structure verticale du lac. Qu'elle soit forcée, notamment sous l'effet du vent lors d'épisode d'upwelling, ou liée à des courants de densité, lors des phases de déstratification, la convection permet un brassage des eaux nécessaire à l'équilibre écologique des plans d'eau \citep{bouffard2019}. Elle assure une oxygénation des eaux profondes tout en garantissant un apport de nutriments nécessaires au cycle trophique \citep{schladow2002,pernica2017}.\\
Cependant ce sont les mouvements verticaux liés au marnage qui constituent le sujet de cette thèse. Le marnage est l'oscillation saisonnière du niveau de surface d'un plan d'eau entre ses hautes et ses basses eaux. Ces variations dépendent de facteurs climatiques, morphologiques et mécaniques qui modifient la réponse en surface. 
L'amplitude de variation provient de trois processus principaux: deux processus ondulatoires et un processus hydrologique.\\
Aux échelles de temps courtes, le niveau d'eau est perturbé par des phénomènes oscillatoires tels que les seiches ou les ondes de crue. Même si les seiches ne sont pas prises en compte dans cette étude, il est intéressant de les définir. Contrairement aux marées qui proviennent d'un interaction gravitationnelle, les seiches caractérisent l'oscillation de la surface du lac ayant pour origine l'entrée en résonance d'ondes stationnaires au sein du bassin \citep{rueda2002}. Elles sont l'expression des conditions aux limites induites par la morphologie du lac sur les ondes qui le parcourent. Dans ce cas, les seiches sont dites "externes" et proviennent soit d'une différence brusque de pression atmosphérique entre deux rives soit d'une accumulation de masse sur une rive par le vent. Un autre type de seiche dites "internes" correspond au phénomène similaire appliqué à l'oscillation d'une surface interne au lac comme la thermocline.\\

Dans le cadre de cette étude, plusieurs échelles de temps de travail sont utilisées mais restent toutes supérieures a l'échelle journalière. Dans ce contexte les variations provoquées par des ondes de crues ou des seiches sont négligeables. Ce qui est important ici, ce sont les évolutions de niveau associées à des variations de volume basées sur une modification des conditions hydrologiques locales et régionales sur des échelles temporelles saisonnières ou intersaisonnières. Les marnages, à ces échelles de temps, sont donc liés aux processus de transfert de masse intrinsèques aux lacs et aujourd'hui modifiés de façon notable par la présence de l'Homme. 

\subsection{{\fontfamily{lmss}\selectfont Hydrologie lacustre}}

Les évolutions du niveau des lacs et plus globalement la tendance de ces variations, à différentes échelles de temps, sont intimement liées au contexte global et régional qui contraint la réponse hydrologique. Ces questions restent primordiales dans la gestion de la ressource en eau et du développement des sociétés. Il est donc essentiel d'évaluer le stock disponible afin de garantir une consommation d'eau sobre qui ne dépasse pas la quantité qui ruisselle des surfaces vers les océans \citep{oki2006}. Ces estimations doivent passer par des analyses pertinentes des variations à plusieurs échelles spatiales afin d’appréhender les évolutions à long terme.\\

Le bilan hydrologique appliqué aux lacs est nécessaire pour caractériser l'état de la ressource et ses interactions avec les autres composantes à plusieurs échelles temporelles. Les échelles longues informent sur la pérennité du stock en eau face aux risques d'assèchement, les échelles saisonnières déterminent les régimes hydrologiques naturels dont le marnage est la conséquence observable, enfin les courtes échelles temporelles sont révélatrices de variations hautes fréquences en lien notamment avec les prélèvements anthropiques pour l'industrie ou l'agriculture. Dans tous les cas de figures, le lac interagit avec le bassin versant qui l'alimente. Cette intégration dans un continnum est nécessaire pour éviter une vision manichéenne qui caractérise le couple rivière-lac comme discontinu. \\
Dans de nombreuses régions, les réseaux de rivières et de lacs sont connectés \citep{kratz2000} et participent à la construction du réseau hydrologique régional. Les rivières ne sont pas uniquement de longs réseaux continus mais plutôt une succession de branches connectées par des lacs plus ou moins petits. La densité de ces lacs, leur taille, leur position et l'aire drainée influencent ce transfert de masse. Malgré cette importance, le prise en compte de cette continuité est souvent ignorée dans les études \citep{jones2010}. Pourtant la prévision des variations du bilan d'eau est importante dans les régions où la modification de la distribution spatiale et temporelle d'eau douce coïncide avec une croissance démographique rapide et une évolution climatique \citep{schewe2014}. \\

De la même façon que pour le bassin versant, un bilan hydrologique pour le lac est défini par une équation de bilan de masse: \\

\begin{equation}
\frac{dV}{dt} = (P-E) + \frac{Q_{in}-Q_{out}}{A}
\end{equation}
~\\
Au-delà de la simplicité relative de l'équation de bilan, c'est la considération multi-factorielle qui rend l'analyse complexe. Chaque composant du bilan dépend de facteurs externes propres qui font émerger des questions récurrentes sur l'influence notamment du bassin de drainage, de la prédominance d'un des processus sur les autres mais aussi de la saisonnalité. Ainsi connaître l'hydrologie des lacs c'est pouvoir caractériser les dépendances qui existent avec des facteurs externes au bassin pour une préservation quantitative et qualitative de leurs eaux. Les lacs sont un des maillons du continuum hydrologique dont chaque composant influence les autres à des échelles annuelles ou pluriannuelles. À titre d'exemple, la réduction des stocks glaciaires sur le plateau Tibétain impacte radicalement les réserves d'eau douce et ajoute une tension supplémentaire à ces zones essentielles pour l'approvisionnement en eau d'une grande partie de la population mondiale \footnote{Le plateau Tibétain est couramment désigné comme étant le "château d'eau de l'Asie".}. Le bilan hydrologique influence aussi le bilan énergétique notamment en été où les eaux s'écoulant du lac alimentent le bassin aval en eau épilimnique chaude. Cet effet tend à persister à l'automne et s'inverse au printemps avec une alimentation en eaux froides. \\

De l'analyse de ce bilan, il est possible de classer les lacs suivant le type de connexion avec le réseau hydrographique: leur rhéisme. D'un côté se trouvent les lacs exoréiques, connectés au bassin aval par un écoulement à l'exutoire; le bilan hydrologique est généralement bénéficiaire. De l'autre côté, les lacs endoréiques sont caractérisés par l'absence d'exutoire et un bilan hydrologique généralement dominé par la corrélation atmosphérique du couple précipitations-évaporation. \\
Les interconnexions sont accentuées dans les bassins endoréiques, comme dans le cas du lac Tchad ou la mer d'Aral où des facteurs naturels ou anthropiques altèrent l'équilibre hydrologique. L'anthropisation joue un rôle non négligeable dans la modification des conditions hydrologiques de ces bassins. En conjonction avec des modifications des régimes atmosphériques, la baisse du niveau d'eau, d'origine anthropique, a provoqué une quasi-disparition de ces lacs avec des conséquences sanitaires et écologiques dramatiques \citep{philip2007,gao2011}.\\

Un exemple pédagogique de l'effet conjoint de l'anthropisation et de la variabilité climatique est le cas de la mer d'Aral (Figure \ref{aral}).
Située dans une zone aride d'Asie à la frontière entre le Kazakhstan et l'Ouzbékistan, la mer d'Aral est soumise à des conditions météorologiques où la perte par évaporation est quasiment 10 fois plus importante que l'apport par précipitations. Les niveaux d'eau de la mer étaient, à l'origine, quasi-équilibrés par un apport complémentaire en eau assuré par deux fleuves. La construction de nombreux barrages et le développement de l'irrigation ont provoqué la diminution du débit entrant dans le lac de 16.7 km$^{3}$.an$^{-1}$ dans les années 1970 à 4.2 km$^{3}$.an$^{-1}$ à la fin des années 1980. En conséquence, la mer d'Aral a vu son niveau d'eau baisser d'environ 70 cm.an$^{-1}$ amenant la surface de la mer de 67 000 km$^{2}$ à 16 000 km$^{2}$ et le volume de 1 083 km$^{3}$ à 100 km$^3$ \citep{cretaux2005}.\\

\begin{figure}[h!]
 \centerline{\includegraphics[scale=0.5]{aral}}
 \caption{Suivi des berges de la mer d'Aral par imagerie satellite sur la période 2000-2008. Source: NASA Earth Observatory.}
 \label{aral}
\end{figure}
\clearpage
\noindent Les lacs, par leur place spéciale dans le cycle hydrologique, offrent une source d’informations importante sur l’effet du changement climatique et les conséquences à différentes échelles \citep{williamson2009}. Comme nous venons de le voir, une des conséquences principales, si ce n'est une des plus visibles, est l’inexorable diminution du niveau des lacs et de leurs stockes \citep{tao2015,wurtsbaugh2017,wang2018storage,busker2019}. Cette raréfaction de la ressource amène à des tensions régionales voire une impossibilité pour les populations à satisfaire leurs besoins vitaux. Ces tensions ne sont pas forcément l'apanage de pays manquant d'infrastructures ou de moyens et touchent aussi des régions considérées comme hydrologiquement développées sinon conscientes des enjeux\footnote{Par exemple, les conflits en Californie entre les villes de Los Angeles et Owens ont pour origine un approvisionnement en eau jugé inégal.}. \\
\noindent Les lacs sont aussi des acteurs importants dans la régulation du cycle du carbone en agissant comme de véritables puits de dioxyde de carbone et capables de stocker des taux de carbone organique supérieurs aux océans \citep{cole2007}. \\

Cependant la caractéristique qui nous intéresse dans le cadre de cette étude correspond plus particulièrement à la capacité des lacs à amortir les débits régionaux. Cette effet est couramment désigné sous le terme d'\textbf{effet tampon}. L'effet tampon est caractérisé par le temps moyen de rétention, paramètre défini comme le temps nécessaire à une goutte d'eau qui entre dans un système pour en sortir soit par l'exutoire soit par évaporation. Ce temps moyen est de l'ordre de 5 ans pour un lac dont la superficie est d'au moins 10 hectares \citep{messager2016}. Ce paramètre est dépendant de paramètres locaux comme les caractéristiques physiques du lac, le climat ou les régimes hydrologiques locaux. \citet{bowling2010} a montré que 80\% de l'eau issue de la fonte nivale en Arctique est stockée dans des lacs atténuant ainsi le pic de débit printanier. \\
La représentation des échanges de masses d'eau dans un modèle assure une meilleure compréhension des phénomènes liant les différents compartiments (notamment les interactions entre les villes, les surfaces végétales, les aquifères et les surfaces aquatiques). L’étude du régime hydrologique des lacs permet donc d'obtenir une vision complète du cycle de l’eau pour la compréhension de phénomènes hydrologiques ou l’étude d’impact des barrages sur le corridor écologique d’un cours d’eau. S'intéresser aux variations spatio-temporelles intrinsèques aux lacs est aussi nécessaire d'un point de vue opérationnel en tant qu'indicateur de sécheresses que dans une optique de prévention du risque inondations \citep{oki2006}. Ainsi \citet{zajac2017} propose une étude globale portant à la fois sur les lacs et les réservoirs avec l'intégration complète de ces retenues dans le réseau de rivières. Cette étude souligne l'atténuation et le décalage temporel du transfert d'eau au sein d'un bassin. \\

Cependant, la modélisation hydrologique des lacs par la quantification des flux entrants et sortants se heurte à des contraintes de précision et de mesures des paramètres hydrologiques et météorologiques. Certaines régions comme l'Afrique de l'Est sont fortement dépendantes des conditions d'évaporation et l'intégration des lacs peut modifier grandement l'amplitude des débits simulés \citep{zajac2017}. 
Malgré une connaissance qui évolue rapidement \citep{gibson2006,swenson2009,gronewold2016}, les variations spatiales et temporelles liées au stockage d’eau des lacs ne font pas, à ce jour, l’objet d’études complètes assurant un suivi global \citep{alsdorf2003}. D’une part cette variabilité est principalement due à un réseau de mesures \textit{in situ} éparse et hétérogène \citep{alsdorf2007} mais aussi à des sources d'incertitudes pesant sur les variables d'entrée comme les cumuls de précipitations \citep{fekete2004}. D’autre part, le développement de modèles de surface continentale ou de rivière a vu le jour à la fin des années 1980 sans considération de l’activité des lacs et des zones humides dans les calculs de vitesses d’écoulements \citep{downing2010}. À ce jour, un nombre croissant d’études s’intéresse à la quantification des lacs et de leurs propriétés \citep{doll2003,downing2006,mcdonald2012,verpoorter2014}. Cependant, ces études ne représentent qu’une vision à un instant donné et ne considèrent pas les phénomènes amenant des évolutions à court terme (\textit{e.g.} apparition et disparition de lacs thermokarstiques issus de la fonte du permafrost). En outre, des études ont montré l'impact des retenues sur les débits de surface en se focalisant principalement sur l'impact des réservoirs et des rivières anthropisées sans intégration globale d'une dynamique lacustre dans un réseau de rivières \citep{haddeland2006,hanasaki2006,doll2009,zhou2016}.

\section{{\fontfamily{lmss}\selectfont \'Etat de l'art de la modélisation des lacs}}

C'est à la fin des années 90 que le besoin d’une meilleure description du cycle de l’eau à l’échelle globale émergea \citep{alsdorf2003}. Motivée par une stagnation voire une diminution du nombre de stations de jaugeage, cette évolution fait écho à l’intérêt croissant pour les phénomènes climatiques et leurs manifestations locales et régionales. Des modèles hydrologiques représentant différents processus comme les régimes de crues \citep[LISFLOOD,][]{de2000} ou la relation pluie-débit \citep{perrin2003} ont ainsi vu le jour. \\
Des études ont démontré le rôle central des lacs dans des disciplines scientifiques aussi diverses que l’écologie, la biochimie ou la météorologie (section \ref{sec:limnologie}). Malgré cela les groupes de modélisation se sont longtemps attachés à développer des outils se focalisant sur les ruissellements de surface, les échanges souterrains et le routage et négligeant les paramétrisations des transferts latéraux \citep{davison2016}. Ainsi, la dynamique des lacs est un des processus majeurs encore sous-représenté dans les développements hydrologiques globaux  \citep{gronewold2020}. \\
Après avoir été considérés d’intérêt mineur dans les études de processus à échelle globale et ignorés dans les modéles \citep{downing2010}, les lacs font, aujourd’hui, l’objet d’un intérêt particulier comme composants du cycle hydrologique et climatique à toutes les échelles spatiales. Les modèles lacustres sont utilisés pour une meilleure compréhension des bilans d’eau et d'énergie et pour affiner la connaissance des rétroactions climatiques, notamment dans les régions arctiques ou en Afrique de l'Est, régions où la densité lacustre est particulièrement élevée. 

\subsection{{\fontfamily{lmss}\selectfont Des modèles thermiques de lacs...}}

La section \ref{sec:limno_physique} a montré que la modification des contraintes de surface dues à la présence de lacs ne peut être négligée dans les régions où leur densité est importante car ces plans d'eau influencent les conditions climatiques locales. Au regard de l’impact qu’ont les lacs sur de nombreuses régions du monde, il est nécessaire d’inclure ces éléments dans le couplage avec des modèles atmosphériques et climatiques en vue d’affiner les prévisions hydrologiques et climatiques.\\

Les disparités qui apparaissent dans les flux de chaleur entre un sol nu et un lac ainsi que le besoin d'un meilleure représentation de l'interface surface-atmosphère ont poussé la communauté scientifique à d'abord s'intéresser à la thermodynamique des lacs et ses interactions avec l'atmosphère \citep{hostetler1993,goyette2000}. Sur cette base, de nombreux modèles thermodynamiques ont été développés dont le tableau \ref{tab_thermo} donne un aperçu. Le principe de ces modèles consiste en la résolution d'équations thermodynamiques en couplage avec l'atmosphère pour estimer les échanges de chaleur et de moment mais aussi pour déterminer l'évolution des caractéristiques intrinsèques aux lacs. Parmi cette diversité de modèles deux familles se dégagent suivant que la représentation des échanges est uni- ou tri-dimensionnelle. 

\begin{table}[h!]
 \caption{Principaux modèles résolvant le bilan d'énergie pour les lacs}
 \label{tab_thermo}
 \begin{tabularx}{\textwidth}{XcXX}
 \hline
 Nom & Référence & Paramétrisation &  Dimension\\
 \hline
  Hostetler&\citet{hostetler1993}&semi-empirique 1D&Multi-couches\\
  MyLake&\citet{saloranta2007}&semi-empirique 1D&Multi-couches\\
  CLM4-LISSS&\citet{subin2012}&semi-empirique 1D&Multi-couches\\
  ALBM&\citet{tan2015}&semi-empirique 1D&Multi-couches\\
  LAKEoneD&\citet{joehnk2001}&k-$\epsilon$ turbulence 1D&Multi-couches\\
  Simstrat&\citet{goudsmit2002}&k-$\epsilon$ turbulence 1D&Multi-couches\\
  Simstrat&\citet{goudsmit2002}&k-$\epsilon$ turbulence 1D&Multi-couches\\
  LAKE&\citet{stepanenko2016}&k-$\epsilon$ turbulence 1D&Multi-couches\\
  FLake& \citet{mironov2008} &bulk 1D&Double-couches\\
  GLM&\citet{hipsey2019}&bulk 1D&Multi-couches\\
  POM adaptation&\citet{song2004}&3D&Multi-couches\\ 
  \hline
 \end{tabularx}
\end{table}

\noindent Le niveau de détail d’un modèle n’est pas nécessairement lié à la qualité de la modélisation. Les modèles très détaillés apportent plus d’incertitudes du fait de l’ajout de paramètres qui nécessitent aussi une plus grande connaissance de la zone étudiée et l'ajout de nouvelles variables qu'il faut pouvoir initialiser. En outre, l’échelle d’application est un point essentiel du développement. Elle conditionne l’utilisation du modèle à l’échelle d'étude et contraint le niveau de détail attendu pour les processus sur la base du nombre d'observations nécessaires à l'évaluation. Dans cette optique, \citet{swayne2003} a montré que l’utilisation d’un modèle pouvait être justifiée par les dimensions des lacs modélisés sans pour autant être une condition nécessaire. Ainsi plus un lac est grand, plus un modèle détaillé sera performant. Cette règle explique notamment l’utilisation de modèles d’océan tri-dimensionnels comme les modèles Nucleus for European Modelling of the Ocean (NEMO) ou Princeton Ocean Model (POM) pour les Grands Lacs d’Amérique ou le Lac Baïkal. La modélisation de ces lacs nécessite un maillage spatial à résolution fine afin de rendre compte du caractère non-uniforme des températures de surface \citep{leon2007}. De plus, les dimensions de ces lacs engendrent un hydrodynamisme plus proche des conditions d'une mer intérieure ou d'une partie d'océan. Cependant le couplage des modules de lacs avec des modèles atmosphériques ajoute une complexité non négligeable et requiert une capacité de calcul qui est parfois coûteuse pour des études à grande échelle. \\
Par conséquent, les modèles 1D ont été adoptés dans le cadre d’études climatiques en se basant uniquement sur la description verticale du profil de température des lacs. Parmi ces modèles 1D, les plus notables sont ceux qui appartiennent à la famille des modèles types Hostetler \citep{hostetler1993} comme CLM4-LISS \citep{subin2012}, utilisés pour représenter les lacs dans une approche climatique globale ou régionale \citep{martynov2012}, ou ceux qui appartiennent aux modèles de type k-$\epsilon$. Ces derniers basent la paramétrisation de la turbulence sur les équations uni-dimensionnelles d'énergie cinétique turbulente \citep{stepanenko2013}. 
Nombre de ces modèles sont aujourd'hui intégrés à des modèles numériques de surface couplés à des modèles d’atmosphère et de climat \citep{mackay2009, thiery2016, salgado2010, lemoigne2016}.\\

Le modèle utilisé pour la modélisation des échanges de flux de chaleur avec l’atmosphère par le Centre National de Recherches Météorologiques (CNRM), FLake (Mironov, 2008), fait partie des modèles "bulk" uni-dimensionnels. Nous verrons ses spécifités dans le chapitre suivant.

\subsection{{\fontfamily{lmss}\selectfont ... vers des modèles hydrologiques}}

Les modèles thermiques sont essentiels à une meilleure connaissance des échanges énergétiques en couplage avec les modèles atmosphériques et climatiques mais ne représentent pas la dynamique du bilan de masse d’eau. Les épaisseurs d'eau de lacs sont pour ainsi dire statiques et la capacité d'évaporation des lacs devient alors quasi-infinie. La représentation des composantes interagissant avec le lac conditionne la capacité d’un modèle à résoudre le bilan hydrologique. D’un autre point de vue, la difficulté qui réside dans la modélisation des lacs est la description de l’interaction entre le lac et les autres composantes du bilan hydrologique qui ne sont pas toutes directement mesurables (\textit{e.g.}: les échanges d’eau avec les aquifères). \\
Depuis quelques années, les lacs sont considérés dans les modèles principalement sous l'impulsion d'une meilleure résolution des modèles de climat (General Circulation Model, GCM). Les lacs qui ne représentaient qu'une fraction négligeable d'une cellule de maille se retrouvent aujourd'hui à couvrir une voire plusieurs cellules, obligeant la communauté scientifique à considérer leurs processus. Des modèles ont été développés afin de suivre l’évolution de systèmes lacustres individuels ou régionaux et le tableau \ref{tab_masse} récapitule les différents modèles hydrologiques. \\

\begin{table}[h!]
\centering
 \caption{Principaux modèles résolvant un bilan de masse pour les lacs}
 \label{tab_masse}
 \begin{tabularx}{\textwidth}{cc}
 \hline
 Model name & Reference \\
 \hline
  VIC&\citet{cherkauer2003}\\
  WaterGap&\citet{hunger2008} \\
  Jena Adaptable Modelling System&\citet{krause2010}\\
  HYPER&\citet{lindstrom2010}\\
  LISFLOOD&\citet{burek2013}\\
  CLM 4.5&\citet{thiery2017b}\\
  General Lake Model&\citet{hipsey2019}\\
  Community Water Model&\citet{burek2019}\\
  \hline
 \end{tabularx}
\end{table}

Parmi ces modèles, seuls deux modèles globaux, à ma connaissance, intègrent un modèle de bilan de masse des lacs couplé à un réseau de rivières et à un modèle atmosphérique: Variable Infiltration Capacity (VIC) et LISFLOOD. \\

Le modèle VIC, développé par \citet{liang1996}, est d'un intérêt particulier par son application couplée à un modèle atmosphérique. VIC est un modèle hydrologique semi-distribué qui a fait l’objet de validations à l’échelle régionale \citep[avec une résolution de 1/8°:][]{maurer2002} et à l’échelle globale \citep[avec une résolution de 2°:][]{nijssen2001}.  Dans cette approche, chaque maille est divisée sous forme de tuiles (sol nu, végétation, rivières, lacs) ayant une paramétrisation propre. À cela s'ajoute un modèle de routage, présenté en figure \ref{vic_lake}, introduit pour transférer les volumes d’eau modélisés de la surface vers les rivières. Des estimations de débits en sortie de chaque cellule du modèle sont ensuite générées par la résolution des équations de Saint-Venant. À l'origine décrit par \citet{cherkauer2003}, le modèle de lac a été plus largement présenté et validé sur la région arctique dans la version de \citet{bowling2010}. La figure \ref{vic_lake} schématise le fonctionnement du modèle de lac dans VIC.\\

\begin{figure}[h!]
 \centerline{\includegraphics[scale=0.7]{VIC_lake}}
 \caption{Représentation schématique du modèle de lac intégré à VIC. Source: \citet{cherkauer2003}.}
 \label{vic_lake}
\end{figure}
\clearpage
Le modèle résout, à la fois, les équations de bilan énergétique et hydrologique sur chaque cellule du modèle afin de déterminer un profil vertical saisonnier de température. Au sein d'une tuile de type "lac", tous les ruissellements et drainages s'écoulent directement dans celle-ci en modifiant en retour son stock. Lorsque tous les termes entrants et sortants sont connus, une équation de déversoir basée sur une hypothèse de seuil épais rectangulaire calcule le débit en sortie du lac. 
\begin{equation}
Q= c_{d}b\sqrt{g} [\frac{2}{3} (z-z_{min})^\frac{3}{2}]
\end{equation}
avec $Q$ le débit de déversement (m$^{3}$.s$^{-1}$), $c_{d}$ un coefficient de débit rendant compte des phénomènes turbulents aux abords du seuil, $b$ la largeur de l'écoulement au niveau du seuil (m), $g$ la constante de gravité (m.s$^{-2}$), $z$ la cote de surface libre mesurée (m), $z_{min}$ la cote de surface libre minimale du lac (m).\\

Dans cette version de VIC, le modèle de lac est enrichi par un algorithme décrivant explicitement les échanges de sub-surface entre la zone lacustre et la zone humide adjacente. Les échanges sont liés à la différence de charge entre ces deux zones. Dans le cas où la charge en eau est plus importante dans la zone humide, l'écoulement s'effectue vers le lac et inversement lorsque la charge est plus importante dans le lac. Enfin le module a été amélioré pour rendre compte des cycles de gel et de dégel par une meilleure représentation de l'évaporation et de l'albédo de surface. Cette amélioration passe par la définition de l'hypsométrie de lac.
\begin{equation}
A(h) = A_{min} \sqrt{\frac{h}{h_{min}}}
\end{equation}
avec $h$ la profondeur totale du lac, $A(h)$ l'aire du lac à la cote $h$, $h_{min}$ la profondeur du lac par rapport au seuil et $A_{min}$ l'aire du lac au niveau du seuil. Toutes les profondeurs sont en $m$ et les aires en $m^{2}$.\\

\citet{bowling2010} a démontré l’efficacité du modèle à l'échelle globale dans l’analyse de la variation du stockage d'eau dans les régions arctiques par sa capacité à expliquer ces variations lors des périodes de dégel. L'ajout des lacs atténue aussi les débits des rivières obtenus dans les simulations régionales. \\

Le second modèle prenant explicitement en compte la dynamique des lacs est LISFLOOD \citep{burek2013}. Ce modèle hydrologique semi-distribué simule les processus de transfert d'eau à l'échelle de grands bassins versants \citep{de2000}. Dans sa version simplifiée, le modèle explicite les processus souterrains, le routage en rivière, les lacs et l'anthropisation par le biais des barrages-réservoirs. Le schéma de routage utilise une approche d'onde cinématique en réponse à un ruissellement de surface et de sub-surface généré par le modèle de surface H-TESSEL du CEPMMT \citep[figure \ref{lisflood},][]{balsamo2009}. 

\begin{figure}[h!]
 \centerline{\includegraphics[scale=0.3]{lisflood}}
 \caption{Schéma des processus modélisés dans le modèle LISFLOOD. Source: \citet{burek2013}.}
 \label{lisflood}
\end{figure}


La physique du modèle, similaire à celle du modèle VIC, résout une équation de bilan de masse sur un point du réseau de routage. Le débit à l'éxutoire est modélisé par modification de l'équation de seuil de Poleni avec un déversoir rectangulaire: 
\begin{equation}
Q = \mu L\sqrt{2g}.H^{\frac{3}{2}}
\end{equation}
avec $Q$ le débit à l'exutoire (m$^{3}$.s$^{-1}$), $g$ l'accélération gravitationnelle à la surface de la Terre (m.s$^{-2}$), $\mu$ le coefficient de débit dépendant de la géométrie du déversoir (généralement entre 0.5 et 0.8), $L$ la largeur du déversoir et $H$ la cote d'eau au dessus du déversoir (m).\\

Dans ce modèle, une courbe hypsométrique est prescrite \textit{a priori} et considère que le volume évolue linéairement par rapport à la cote d'eau.
Ce modèle a notamment été utilisé afin d'estimer l'impact de la présence des lacs et réservoirs pour l'amélioration de la simulation des débits dans le système de prévention des crues globale GloFas \citep{alfieri2013}.
\clearpage

\section{{\fontfamily{lmss}\selectfont Conclusion}}

Le cycle de l'eau est une représentation du mouvement et du renouvellement perpétuel de l'eau dans le système global. Que les processus étudiés se concentrent sur les quantités d'eau produites, stockées ou transférées il est aujourd'hui démontré que ce cycle évolue dans le temps et l'espace sous l'effet de contraintes directes ou indirectes. \\
L'hydrologie s'applique à étudier l'hydrosphère, ses liens avec les compartiments du système Terre comme l'atmosphère mais aussi les rétroactions qui existent avec la biosphère dont l'homme fait partie.
Du point de vue d'une unité hydrologiquement close telle que le bassin versant, le bilan d'eau, qui décrit les échanges de ce cycle, est caractérisé par ses composantes essentielles. Ces composantes sont influencées par des paramètres physiques et physiographiques responsables d'un partitionnement, en moyenne stable, de l'eau dans les différents réservoirs. \\
Afin d'accroître la connaissance de ces processus et de les étudier il est primordial d'utiliser des méthodes adaptées à l'échelle spatiale et temporelle souhaitée. Dans le cadre d'études locales, l'hydrologie s'appuie sur des techniques d'observation éprouvées dont la fiabilité et la précision ne sont plus à démontrer. Malgré cela, ces techniques ne donnent que des informations parcellaires sur des aspects locaux. Ainsi le développement de l'altimétrie et de l'imagerie satellitaire à donné à l'hydrologie les moyens nécessaires pour un suivi et une gestion de la ressource en eau à l'échelle globale. S'appuyant d'abord sur les missions spatiales dédiées à l'océanographie, l'hydrologie spatiale a su se faire une place jusqu'à aboutir à une mission spatiale présentant un volet spécifique pour l'hydrologie continentale: la mission spatiale Surface Water and Ocean Topography.\\
Ces observations serviront notamment de données de calibration et de validation des modèles de surface et des modèles de routage associés qui contribueront à une meilleure connaissance des rétroactions avec le climat, au suivi saisonnier des sécheresses et à la prévention du risque inondation en temps réel. Ces modèles représentent les différents processus de la manière la plus détaillée tout en s'adaptant aux considérations spatiales voulues. Il apparaît aujourd'hui que ces modèles sont nécessaires à la connaissance des conséquences du changement climatique sur cette ressource.\\

Les lacs font partie intégrante du cycle global de l'eau et en sont mêmes les composants principaux dans les régions boréales. Ces étendues modifient les propriétés de la couche limite atmosphérique en contribuant à la modification des bilans d'énergie et d'eau. À l'échelle régionale, leur capacité thermique spécifique provoque des anomalies de températures. Les lacs sont, à l'échelle de temps de l'humanité, des sources d'humidité quasi-infinie induisant une augmentation de l'évaporation potentielle pouvant provoquer une modification locale des régimes de pluies convectives. Enfin, complètement intégrés dans le réseau hydrographique global les lacs sont des zones tampons atténuant la propagation d'ondes de crues voire se comportant comme des réservoirs collectant l'eau d'un bassin. Les lacs sont des sentinelles des évolutions climatiques mais aussi de l'anthropisation; la mer d'Aral en est un exemple notable. Même s'ils ne représentent qu'une faible part de l'eau douce globale, ces réservoirs sont directement accessibles et sont vulnérables face aux altérations et aux pollutions continentales. 
La compréhension de la dynamique et des échanges avec les compartiments hydrologiques est, par conséquent, nécessaire pour comprendre et anticiper les évolutions futures des stocks et de leur transfert aval.\\

\noindent Maintenant que le cadre théorique a été mis en place, il convient d'apporter un regard sur les techniques et méthodes représentant les processus de surface et développées au CNRM. À travers le chapitre suivant c'est donc une description des outils mis à notre disposition ainsi que des outils développés dans cette thèse qui sont détaillés.


\cleardoublepage
\chapter{{\fontfamily{lmss}\selectfont Description des données et modèles à notre disposition}}
\label{chap:descriptions}
\minitoc

\noindent Après avoir mis en place le contexte théorique encadrant les processus hydrologiques et ceux spécifiques aux lacs, ce chapitre aborde les techniques de modélisation développées au CNRM.\\
La modélisation des surfaces repose sur une plateforme externalisée, SURFEX, qui, associée à la carte d'occupation des sols ECOCLIMAP-II, fournit les conditions basses des bilans d'énergie et d'eau pour les modèles atmosphériques. En hydrologie, ce sont plus particulièrement le modèle de surface ISBA et le modèle de routage en rivière CTRIP qui nous intéressent. ISBA calcule le ruissellement et le drainage à l'interface sol-végétation-atmosphère tandis que CTRIP transfère ces volumes d'eau horizontalement sous forme de débits.\\
Au sein de cette plateforme, seul le bilan d'énergie associé aux lacs est représenté par le biais du modèle FLake. C'est dans cette optique que la fin du chapitre s'arrêtera plus spécifiquement sur le modèle de bilan de masse des lacs développé dans cette thèse: MLake.

\section{{\fontfamily{lmss}\selectfont Les bases de données}}
\subsection{{\fontfamily{lmss}\selectfont ECOCLIMAP}}
\label{sec:ECOCLIMAP}

Avant de vouloir modéliser la surface et les processus associés, il faut pouvoir distinguer les différents couverts, les classer et déterminer les propriétés intrinsèques qui les caractérisent. Dans ce contexte, il est important de s'appuyer sur les outils à notre disposition pour gèrer la répartition et l'hétérogénéité de la surface. À l'échelle globale, cette information provient de l'analyse de facteurs climatiques et d'observations, agrégés sous format numérique et indiquant la proportion de chaque couvert contenue au sein d'une maille à la résolution fixée. Au CNRM, la discrétisation du sol en différents couverts se base sur des données d'occupation des sols issues d'ECOCLIMAP-II \citep[figure \ref{ecoclimap},][]{faroux2013}.\\

\begin{figure}[h!]
\centering
  \includegraphics[scale=0.1]{ECO25}
  \caption{Carte d'occupation des sols issue d'ECOCLIMAP-II.}
  \label{ecoclimap}
\end{figure}

\noindent ECOCLIMAP est une base de données globale d'occupation des sols et de paramètres de surface à la résolution kilométrique issue de la mutualisation d'une carte d'occupation des sols et d'informations satellitaires. Cette base donne la répartition et la fraction des surfaces naturelles, urbanisées et marines, leurs variabilités spatio-temporelles ainsi que les paramètres physiques associés\footnote{On compte parmi ces paramètres l'albédo, l'indice de surface foliaire ou la longueur de rugosité.} à la résolution utilisée par le modèle. La version ECOCLIMAP-II a été utilisée dans le cadre de ces travaux. En plus de proposer la fraction couverte par chaque type de surface, ECOCLIMAP-II classe les surfaces continentales naturelles suivant 19 sous-classes, dont les types fonctionnels de végétation, regroupés dans le tableau \ref{eco_patch}, offrent une discrétisation plus précise du couvert et garantissent une meilleure quantification des évolutions propres à chaque type. À ces 19 couverts végétaux s'ajoutent trois couverts pour les mers, les lacs et les rivières.\\
ECOCLIMAP-II est, donc, un outil dynamique rendant compte du type de surface et de sa couverture spatiale utile à la modélisation météorologique et climatique.

~\\

%ECOCLIMAP-II Patches
\begin{table}[h!]
 \centering
 \caption{Présentation des 19 types de végétation d'ECOCLIMAP-II.}
 \label{eco_patch}
 \begin{tabularx}{0.75\textwidth}{Xc}
 \hline
 \hline
  1 & Sol nu\\
  2 & Roche nue\\
  3 & Neige et glace permanente\\
  4 & Feuillu tempéré à feuilles caduques\\
  5 & Conifère boréal persistant\\
  6 & Feuillu tropical persistant\\
  7 & Culture de type C3\\
  8 & Culture de type C4\\
  9 & Culture irriguée\\
  10 & Prairie tempérée\\
  11 & Prairie tropicale\\
  12 & Tourbières, parcs irrigués\\
  13 & Feuillu tropical à feuilles caduques\\
  14 & Feuillu tempéré à feuilles persistantes\\
  15 & Conifère tempéré persistant\\
  16 & Feuillu boréal à feuilles caduques \\
  17 & Conifère boréal à épines caduques\\
  18 & Prairie boréale\\
  19 & Buissons, arbustes \\
\hline
\hline
 \end{tabularx}
\end{table}

\clearpage

\subsection{{\fontfamily{lmss}\selectfont Global Lake DataBase}}

La couverture spatiale des lacs fournie par ECOCLIMAP-II indique la position des plans d'eau sur Terre et permet de connaître leur extension spatiale (Figure \ref{eco_lake}). 

\begin{figure}[h!]
 \includegraphics[width=1.\textwidth]{eco_lake}
 \caption{Carte des fractions de lacs dans ECOCLIMAP-II. Chaque pixel bleu indique la présence d'un pixel identifié comme lac.}
 \label{eco_lake}
\end{figure}

Pour caractériser la morphologie de chaque lac, il est nécessaire d'en connaître la profondeur moyenne, variable essentielle à la compréhension des processus lacustres \citep{hakanson2005}. Contrairement aux données de surface, dont la connaissance est facilitée par le développement des mesures satellitaires, la mesure de la profondeur moyenne des lacs à l'échelle globale est rendue difficile par les coûts tant humain que financier, limitant alors le développement de bases de données cohérentes. 
Parmi les quelques bases de données existantes \citep{lehner2004,verpoorter2014,messager2016} la base de données GLDB \citep{kourzeneva2012} a été spécialement développée pour le besoin de la prévision numérique du temps. Son principal avantage est d'être en cohérence avec ECOCLIMAP-II pour prescrire la profondeur moyenne de près de 15 000 lacs et une bathymétrie précise de 36 autres lacs \citep{toptunova2019,choulga2019}. Pour les lacs n'étant pas référencés avec une profondeur moyenne précise dans la base GLDB, celle-ci prend une valeur par défaut égale à 10 m.\\

\noindent La version de GLDB utilisée dans cette étude présente les avantages suivants:\\

\begin{itemize}
\item[$\bullet$] l'ajout de valeur par défaut pour les réservoirs et lacs n'ayant pas de données;
\item[$\bullet$] l'intégration de bathymétries détaillées pour la majorité des lacs finlandais;
\item[$\bullet$] la correction des profondeurs moyennes pour la zone boréale en s'appuyant sur des cartes géologiques ainsi que sur une méthode analytique basée sur l'étude de la surface et du type de climat \citep{choulga2014};
\item[$\bullet$] la distinction entre les lacs d'eau douce et les lacs salés.
\end{itemize}
~\\
Comme les données globales de volumes sont encore plus rares et souvent issues d'extrapolations statistiques ou de calcul indirect, l'initialisation des stocks d'eau dans les lacs se basera sur l'information couplée entre ECOCLIMAP-II et GLDB. 

\section{{\fontfamily{lmss}\selectfont SURFEX}}
\label{sec:SURFEX}
La modélisation des échanges à l'interface surface-atmosphère présente un intérêt majeur pour une meilleure compréhension des couplages entre atmosphère, surface et sous-sol, pour l'apport d'informations nécessaires à la prévision des phénomènes extrêmes ou pour une meilleure représentation des conditions à la limite turbulentes et radiatives en surface.\\
La représentation détaillée des surfaces est nécessaire pour répondre aux besoins de la météorologie opérationnelle et de la prise en compte de l'hydrologie dans les études climatiques. Contrairement à certaines paramétrisations physiques du modèle Meso-NH \citep{lac2018}, la modélisation des surfaces a été externalisée pour donner le jour à la plateforme SURFEX \citep[Surface Externalisée]{masson2013}. Utilisée en couplage avec un modèle d'atmosphère (comme AROME ou ARPEGE), de climat \citet[CNRM-CM]{voldoire2019}) ou en mode "off-line", c'est-à-dire sans rétroaction de la surface sur l'atmosphère, cette plateforme concrétise les efforts de mutualisation afin de garantir l'utilisation des modèles de surface dans de nombreux domaines  tels que la prévention du risque avalanche ou la modélisation des flux énergétiques en ville \citep{vionnet2012,schoetter2017,lemoigne2020}. \\
Cette plateforme simule les flux d'énergie, de masse et de quantité de mouvement à l'interface surface-atmosphère en résolvant les bilans d'eau et d'énergie utiles notamment à la simulation des évolutions du stock en eau de surface et de sub-surface. SURFEX, contraint par des forçages atmosphériques (température, humidité, vent, pression, rayonnement solaire et infrarouge, pluie et neige), simule l'évolution des variables de surface (comme la température de surface) et du sol pour les surfaces continentales et résout les bilans d'eau et d'énergie. Tout cela participe à la fermeture des bilans pour le continuum surface-atmosphère-océan dans le cas d'un couplage avec un modèle atmosphérique et hydrologique. \\
\clearpage
\noindent La représentation des surfaces dans SURFEX adopte une approche par tuiles (Figure \ref{surfex}), qui rend compte, d'une part, de l'hétérogénéité des surfaces à l'intérieur des mailles d'étude, et d'autre part, de leur variabilité spatio-temporelle.

~\\

\begin{itemize}
\medbreak
\item[$\bullet$] une tuile décrivant les \textbf{surfaces urbanisées} modélisées par le modèle Town Energy Balance \citep[TEB,][]{masson2000,lemonsu2004}. TEB se base sur une approche de rue en forme de canyon où des bilans distincts sont calculés pour chaque composant de ce système\footnote{Cette approche prend comme composant, un toit, une rue et deux murs placés face à face.}. Ce modèle a de nombreuses applications dont l'étude de l'ilôt de chaleur urbain et son interaction avec le climat \citep{daniel2019}; 
\bigbreak
\item[$\bullet$] une tuile pour les \textbf{mers et océans}. Plusieurs approches avec des degrés de complexité variés existent. Pour des temps d'expériences courts, une approche simple prescrit la température de surface (SST: Sea Surface Temperature) puis calcule la longueur de rugosité par la formule de Charnock afin d'estimer les flux de surface. Pour des temps d'expériences plus longs et afin de prendre en compte le cycle diurne de température, un modèle de couche de mélange unidimensionnel simule l'évolution de la SST, des courants, de la salinité et du transport turbulent vertical \citep{lebeaupin2009};
\bigbreak
\item[$\bullet$] une tuile pour les \textbf{surfaces continentales naturelles} avec comme modèle utilisé Interaction Sol Biosphère Atmosphère \citep[ISBA,][]{noilhan1989}. Ce modèle intervient directement dans cette thèse et la suite de ce chapitre détaillera de façon plus approfondie sa physique;
\bigbreak
\item[$\bullet$] une tuile \textbf{lac} associée au modèle thermique unidimensionnel FLake \citep{mironov2008}. Ce modèle a été développé pour les besoins de la prévision numérique du temps, des études climatiques et pour le traitement des lacs dans les modèles d'environnement. Le modèle se base sur une approche d'auto-similarité du profil vertical de température pour déterminer la structure thermique interne au lac et les conditions de mélange à différentes profondeurs pour des pas de temps allant de quelques jours à plusieurs années. Comme pour ISBA, ce modèle est utilisé dans cette thèse et sera détaillé plus loin dans ce chapitre. \\
\end{itemize}

\begin{figure}[h!]
  \centering
  \includegraphics[width=0.65\textwidth]{surfex}
  \caption{Représentation de l'approche par tuile dans SURFEX et le couplage avec un modèle d'atmosphère. Source: \url{https://www.umr-cnrm.fr/surfex}}
  \label{surfex}
\end{figure}

\clearpage

Une des applications de SURFEX en hydrologie consiste à étudier le cycle de l'eau de l'échelle du bassin versant à l'échelle globale. Lors du couplage hydrologique, SURFEX génère le ruissellement et le drainage qui alimentent ensuite un modèle de routage en rivière, pour simuler les débits, ou un modèle hydrogéologique, pour simuler, en complément des débits, les hauteurs de nappes. Au sein du CNRM, SURFEX est couplé à plusieurs types de modèles hydrologiques. À l'échelle du bassin versant, SURFEX peut être couplé avec TOPMODEL afin de modéliser les crues rapides notamment sur le pourtour méditerranéen \citep{vincendon2010}. Aux échelles régionales, il est utilisé dans la chaîne de modélisation hydrométéorologique SAFRAN-ISBA-MODCOU que nous aurons l'occasion de détailler plus tard. Enfin, SURFEX est couplé au modèle hydrologique global CTRIP \citep{decharme2007}. \\

\noindent Dans la suite du chapitre, une attention particulière va être portée sur les modèles qui ont été plus spécifiquement utilisés dans le cadre de cette thèse:

\begin{itemize}
\medbreak
\item[$\bullet$] le modèle ISBA dans sa version historique force-restore et la version plus récente multi-couches diffusive;
\medbreak
\item[$\bullet$] le modèle CTRIP qui est le modèle global de routage de l'eau en rivière;
\medbreak
\item[$\bullet$] le modèle FLake pour la représentation du bilan d'énergie des lacs;
\medbreak
\item[$\bullet$] le modèle MLake, développé pendant cette thèse, qui rend compte de la dynamique massique des lacs à l'échelle globale et des interactions avec le réseau de rivières.
\medbreak
\end{itemize}
\section{{\fontfamily{lmss}\selectfont Le modèle de surface ISBA}}
\label{sec:isba}

La suite de ce chapitre s'attache à la description des deux composantes essentielles à la modélisation hydrologique développées et utilisées au CNRM: le modèle couplé ISBA-CTRIP \citep{decharme2007}. Ce qui suit dans cette section détaille les caractéristiques d'abord en matière d'estimation des flux liés à la résolution du bilan d'énergie et d'eau de surface par ISBA.\\

Le modèle ISBA \citep{noilhan1989} se base sur un schéma de transfert sol-végétation-atmosphère qui simule les échanges d'eau et d'énergie entre les composantes. L'avantage de ce modèle est de considérer les paramètres essentiels à la connaissance de l'état physique de la surface tels que le contenu en eau, sa phase, les échanges d'énergie dans le sol, la quantité d'eau interceptée par la canopée, l'évapotranspiration, le drainage et le ruissellement. Ce modèle est, aujourd'hui, couplé, par le biais de SURFEX, aux modèles atmosphériques et climatiques utilisés à Météo-France \citep{voldoire2019}. Ce modèle a été complété depuis par des développements pour une meilleure prise en compte de la neige, des processus liés à la photosynthèse ou encore du partitionnement radiatif entre la forêt et le sol sous-jacent. Les différents schémas introduits dans ISBA sont détaillés dans le tableau \ref{schema_isba}.
Dans le cadre des travaux de cette thèse, ISBA a servi à déterminer le ruissellement et le drainage générés par les surfaces hors zones lacustres en réponse aux forçages atmosphériques. \\
\noindent Dans la suite du chapitre, le schéma ISBA 3-couches puis le schéma multi-couches diffusif seront détaillés après une brève description de la version historique "force-restore" 2 couches.

%Historique ISBA
\begin{table}[h!]
 \caption{Présentation de l'évolution des options de physique dans ISBA:  ${\ast}$ représente des améliorations supplémentaires aux processus physiques.}
 \label{schema_isba}
 \begin{tabularx}{\textwidth}{XXX}
 \hline
 Processus & Nom du schéma & Référence\\
 \hline
 \multirow{3}{4cm}{Sol} & Force-Restore 2 couches & \citet{noilhan1989}\\
 & Force-Restore 3 couches$^{\ast}$ & \citet{boone1999}\\
 & Diffusion sur 5-couches & \citet{boone2000}\\
 & Diffusion sur N-couches$^{\ast}$ & \citet{decharme2011}\\
 \hline
 \multirow{3}{4cm}{Neige}& couche unique & \citet{douville1995}\\
 & ES 3 couches &  \citet{boone2001}\\
 & ES 12 couches$^{\ast}$ & \citet{decharme2016}\\
 \hline
 \multirow{2}{4cm}{Hydrologie} & Ruissellement par saturation & \citet{habets1999}\\
 & Ruissellement de Dunne influencé par la topographie &\citet{decharme2007}\\
 \hline
 \multirow{2}{4cm}{Couvert forestier}& bulk modèle Multi-Energy-Balance (MEB) & \citet{boone2017}\\
 & bulk MEB avec litière & \citet{napoly2017}\\
 \hline
 \multirow{3}{4cm}{Cycle du carbone et photosynthèse} & ISBA-Ag-s avec LAI prescrit & \citet{calvet1998}\\
 & ISBA-Ag-s avec LAI dynamique & \citet{calvet2001}\\
 & ISBA-Ag-s CC & \citet{gibelin2008}\\
 \hline
 \end{tabularx}
\end{table}

\subsection{{\fontfamily{lmss}\selectfont Version historique: ISBA force-restore}}
\label{subsec:ISBA-FR}

Dans sa version historique, le modèle ISBA décrivait l'évolution du bilan d'énergie et du bilan de masse de la surface par une approche "force-restore" \citep[ou "forçage-relaxation"]{deardorff1977} sur un sol à deux couches où l'évaporation et le drainage étaient explicitement résolus \citep{mahfouf1996}. Cette approche prenait en compte huit variables pronostiques: $T_{s}$ la température de surface, $T_{p}$ la température profonde, $W_{r}$ le réservoir d'interception, $\omega_{s}$ le contenu en eau de surface, $\omega_{p}$ le contenu en eau profond, $W_{n}$ le contenu en eau de la neige, $\rho_{n}$ la densité de la neige et $\alpha_{n}$ l'albédo de la neige.\\

Ce modèle est bien adapté pour les intégrations numériques à court terme, comme pour les prévisions météorologiques à courte et moyenne échéance. \noindent
Cependant cette approche est limitée dans sa description des processus physiques plus complexes comme les mécanismes de transferts diffusifs dans le sol. Le fait de développer le modèle à 3 couches ISBA-3L a permis la prise en compte explicite d'une couche de sol supplémentaire pour séparer la couche racinaire (impactant potentiellement le bilan en eau par absorption) et la couche sous-racinaire \citep{boone1999} tout en assurant l'évolution temporelle du contenu en eau.\\

Par la suite \citet{boone2000} a développé une version plus complète d'ISBA basée sur un schéma de sol multi-couches résolvant explicitement les lois de Darcy et de Fourier pour les transferts diffusifs dans le sol appelés ISBA-DF (ISBA- explicit vertical Diffusion model).
Aujourd'hui ISBA-DF considère 12 couches de sols ainsi qu'une meilleure représentation de la zone racinaire \citep{decharme2011}.\\
\noindent Ces évolutions ont été concomitantes à l'amélioration des processus sous-mailles portant sur la prise en compte des hétérogénéités spatiales et temporelles (précipitations, topographie ou encore type de végétation). La paramétrisation des ruissellements par \citet{decharme2006} et la modélisation du ruissellement de surface sous-maille par \citet{habets1999} constituent des améliorations significatives du modèle.\\

\noindent Dans la version historique du modèle couplé ISBA-CTRIP, la représentation des sols et les ruissellement associés provenaient de la version "force-restore" d'ISBA. Ce modèle a ensuite laissé la place au modèle ISBA-DF qui est utilisé dans la suite de la thèse en mode forcé\footnote{Cela signifie que les forçages atmosphériques sont prescrits sans tenir compte des rétroactions de la surface sur l'atmosphère.}. De cette façon, les incertitudes liées aux processus simulés par un modèle atmosphérique et non nécessaires ici sont filtrées.

\subsubsection*{{\fontfamily{lmss}\selectfont Fraction de couvert}}
Le modèle ne discrétise pas seulement le sol en trois couches mais prend aussi en compte trois réservoirs distincts: le réservoir de végétation, le réservoir de sol et le réservoir de neige. \\
Ces trois composants permettent de considérer des processus contribuant à une meilleure représentation du cycle de l'eau et de l'énergie. La paramétrisation de la surface couverte par chacun des réservoirs est faite sous-maille suivant le schéma \ref{fraction_sol}.

\begin{figure}[h!]
\centering
\includegraphics[width=1.0\textwidth]{fraction_sol}
\caption{Représentation sous-maille de la surface dans ISBA suivant la canopée, le sol et le manteau neigeux.}
\label{fraction_sol}
\end{figure}

\noindent Ainsi il est possible de retrouver:\\

\begin{itemize}
\item[$\bullet$] $veg$ la fraction de sol recouverte par la canopée;
\item[$\bullet$] $p_{sn}$ la fraction totale de surface couverte de neige composée des fractions de sol $p_{sn,g}$ et de végétation $p_{sn,v}$ recouvertes de neige.
\end{itemize}
\clearpage
\noindent La fraction totale de surface couverte de neige se distingue par deux composantes définies pour le schéma mono-couche tel que:

\begin{align}\label{eq:frac_snow}
p_{sn,v} = \frac{h_{s}}{h_{s}+\omega_{sv}z_{0}}\\
p_{sn,g} = \frac{W_{s}}{W_{s}+W_{crn}}
\end{align}
avec $h_{s}$ l'épaisseur totale de neige (m). $z_{0}$ la longueur de rugosité (m). $\omega_{sv}$ est un paramètre empirique fixé à 2 \citep{decharme2019}. $W_{s}$ l'équivalent en eau de la neige (kg.m$^{-2}$). $W_{crn}$ l'équivalent critique en eau de la neige égal par définition à 10 kg.m$^{-2}$.\\

\noindent La fraction totale de neige est enfin calculée par:
\begin{equation}
p_{sn} = (1-veg)p_{sn,g}+veg \: p_{sn,v}
\end{equation}
avec $veg$ la fraction de végétation sur la maille. Cette fraction de végétation varie selon le type de sol (\textit{e.g.} 0.0 pour un sol nu et 0.95 pour une prairie) et de façon exponentielle suivant l'indice de surface foliaire, LAI, issu d'ECOCLIMAP\footnote{Le $LAI$, Leaf Area Index, est une grandeur qui informe sur la densité de végétation sur la surface du sol.}.\\

\subsection{{\fontfamily{lmss}\selectfont Le modèle ISBA-3L}}
\label{subsec:ISBA-3L}

La version d'ISBA avec une discréatisation du sol en 3 couches a été initialement développée afin de distinguer les flux d'eau dans la zone influencée par les processus racinaires et la couche sous-racinaire. Le principe général du modèle est dicté par les principes de conservation de l'énergie et de la masse. 

\subsubsection{{\fontfamily{lmss}\selectfont Température du sol et bilan d'énergie}}
\label{subsubsec:energie}
La température de la couche superficielle du sol $T_s$ (assimilée à une couche d'épaisseur 1 cm) assure la représentation du bilan d'énergie de surface dans ISBA-3L.\\

\noindent Le cycle diurne de la température dépend d'une part du flux de chaleur vertical dans le sol G et d'autre part de la température moyenne du sol profond $T_{2}$ (Figure \ref{boone}) sur une durée temporelle $\tau$ fixée à une journée (en $s$) suivant la formulation de \citet{bhumralkar1975} et \citet{blackadar1976}:

\begin{equation}
\begin{cases}
\label{eq_t_surf_3L}
\dfrac{\partial T_{s}}{\partial t} = C_{T}G - \dfrac{2\pi}{\tau}(T_{s}-T_{2})\\

\\

\dfrac{\partial T_{2}}{\partial t} =\dfrac{1}{\tau}(T_{s}-T_{2})
\end{cases}
\end{equation}
avec $C_{T}$ la capacité calorifique du sol (J.kg$^{-1}$.K$^{-1}$).\\

\begin{figure}[h!]
  \includegraphics[width=1.\textwidth]{boone.png}
  \caption{Discrétisation du sol dans les différentes versions du modèle ISBA, d'après \citet{boone2000}}
  \label{boone}
\end{figure}

\noindent La capacité calorifique du sol est dépendante notamment de la discrétisation du sol entre les différents couverts selon la figure \ref{fraction_sol}:

\begin{equation}
C_{T} = \dfrac{1}{\left[\dfrac{(1-veg)(1-p_{sn,g})}{C_{g}}+\dfrac{veg(1-p_{sn,v})}{C_{v}}+\dfrac{p_{sn}}{C_{s}}\right]}
\end{equation}
où $veg$ est la fraction de végétation dans un pixel ISBA prescrit par ECOCLIMAP-II ou issue d'observations. $C_{g}$, $C_{v}$ et $C_{s}$ sont les capacités calorifiques respectivement du sol, de la canopée et de la neige (J.kg$^{-1}$.K$^{-1}$). \\

\noindent Considérant le volume de sol comme infiniment petit, il est possible de négliger les variations temporelles d'énergie dans le sol et de réduire l'équation de conservation de l'énergie pour donner une estimation de l'évolution du flux de chaleur vertical dans le sol:

\begin{equation}
\label{eq_fluxsol_3L}
G = R_{n} - H - LE
\end{equation}
avec $R_{n}$ le rayonnement net (W.m$^{-2}$), H le flux de chaleur sensible (W.m$^{-2}$) et LE le flux de chaleur latente (W.m$^{-2}$)\\

\noindent Conformément à ce que nous avons vu à la section \ref{sec:bilan_energie}, le flux de chaleur sensible s'écrit comme : 

\begin{equation}
\label{eq_sensheat}
H = \rho_{a}c_{p}C_{H}V_{a}(T_{s}-T_{a})
\end{equation}
où $\rho_{a}$ est la masse volumique de l'air (kg.m$^{-3}$), $c_{p}$ la capacité calorifique à pression constante de l'air (J.kg$^{-1}$.K$^{-1}$), $V_{a}$ la vitesse du vent (m.s$^{-1}$) et $C_{H}$ le coefficient d'échange dépendant des conditions de stabilité de l'air et de la rugosité de surface.\\

\noindent En général, les températures de l'Eq.\ref{eq_sensheat} sont exprimées en température potentielle, mais pour simplifier, nous les avons approchées dans ce manuscrit en utilisant la température réelle de l'air.
\\

\noindent Le flux de chaleur latente $LE$, quant à lui, est la somme de l'évaporation d'eau liquide pour un sol nu $E_{g}$ (kg.m$^{-2}$.s$^{-1}$), de l'évapotranspiration de la végétation $E_{v}$ (kg.m$^{-2}$.s$^{-1}$) et de la sublimation de la neige $E_{s}$ (kg.m$^{-2}$.s$^{-1}$) selon l'équation:

\begin{equation}
\label{eq_latentheat}
LE = L_{v}(E_{g}+E_{v})+L_{s}E_{s}
\end{equation}
avec $L_{v}$ et $L_{s}$ la capacité calorifique respectivement de vaporisation et de sublimation (J.kg$^{-1}$). \\

\noindent L'évaporation pour le sol nu est donnée par:
\begin{equation}
\label{eq_evap_3L}
E_{g} = (1-veg)\rho_{a}C_{H}V_{a}[h_{u}q_{sat}(T_{s})-q_{a}]
\end{equation}

\noindent et l'évapotranspiration au niveau de la végétation par:

\begin{equation}
\label{eq_etr_3L}
E_{v} = E_{c} + E_{tr}=veg\rho_{a}C_{H}V_{a}h_{v}[q_{sat}(T_{s})-q_{a}]
\end{equation}
avec $veg$ la fraction de sol couverte par la végétation, $q_{sat}(T_{s})$ l'humidité spécifique saturante à la surface (kg.kg$^{-1}$), $q_{a}$ l'humidité spécifique de l'air (kg.kg$^{-1}$), $h_{u}$ l'humidité relative à la surface et $h_{v}$ le coefficient adimensionnel de Halstead. \\

\noindent Ce dernier coefficient assure une distinction entre l'évaporation directe de la végétation $E_{c}$ et la transpiration des feuilles $E_{tr}$ selon l'équation définie dans \citet{noilhan1989}:

\begin{equation}
h_{v} = (1- \delta)\frac{R_{a}}{R_{a}+R_{s}}+\delta
\end{equation}
où $R_{a}$ est la résistance aérodynamique (s.m$^{-1}$), $R_{s}$ est la résistance stomatique (s.m$^{-1}$) et $\delta$ la fraction de feuillage interceptant l'eau. \\


\subsubsection{{\fontfamily{lmss}\selectfont Bilan en eau}}

Le bilan en eau de surface est principalement influencé par la quantité de précipitations reçue au niveau du sol. Cependant toutes les précipitations ne rejoignent pas directement le sol et une distinction doit être faite entre la part interceptée par le canopée, la part stockée au niveau de la neige et enfin la part précipitante directement au sol. Le bilan en eau global d'ISBA pour la surface est définie tel que:

\begin{equation} \label{bilan_eau_3L}
\frac{dW}{dt} = \frac{dW_{g}}{dt} + \frac{dW_{n}}{dt} + \frac{dW_{r}}{dt}
\end{equation}
avec $W_{g}$ le stock en eau du sol (kg.m$^{-2}$), $W_{n}$ le stock en eau dans le réservoir de neige (kg.m$^{-2}$) et $W_{r}$ le stock en eau de la canopée (kg.m$^{-2}$).\\

\noindent Ce schéma reprend l'approche réservoir de \citet{deardorff1978} pour représenter l'évolution temporelle des masses d'eau stockées. Pour le réservoir de végétation l'équation de masse s'écrit:

\begin{equation}
\label{eq_waterfx_veg}
\frac{dW_{r}}{dt} = (1-p_{sn,v})vegP_{r}-(E_{c}+d_{r})
\end{equation}
$p_{sn,v}$ correspond à la fraction de la végétation recouverte de neige. $veg\,P_{r}$ est la fraction de précipitation interceptée par la canopée (kg.m$^{-2}$.s$^{-1}$). $E_{c}$ est l'estimation de l'évaporation directe de l'eau interceptée par la végétation. $d_{r}$ est la composante du stock d'interception qui contribue au ruissellement de surface lorsque ce réservoir est saturé.\\

\noindent Une paramétrisation de la contribution du réservoir de la canopée au ruissellement a été proposée par \citet{mahfouf1995} à l'échelle globale selon une fonction exponentielle:

\begin{equation}
d_{r}=P_{r}e^{\dfrac{\mu(W_{r}-W_{r,max})}{P_{r}\Delta t}}
\end{equation}
$\mu$ représente la fraction de la maille effectivement mouillée, elle est fixée à 0.1. $W_{r,max}$ est la capacité maximale du réservoir de la canopée proportionnelle à la densité de feuillage.\\

\noindent L'évolution temporelle de la masse d'eau stockée par le réservoir de sol est définie par:

\begin{equation}
\label{eq_waterfx_sol}
\frac{dW_{g}}{dt} = I_{r} - E_{g} - E_{tr}  - D 
\end{equation}
avec $I_{r}$ l'infiltration réelle (kg.m$^{-2}$.s$^{-1}$). $E_{g}$ l'évaporation du sol nu définie par l'Eq. \ref{eq_evap_3L} (kg.m$^{-2}$.s$^{-1}$). $D$ représente le puits de masse par drainage (kg.m$^{-2}$.s$^{-1}$).\\

\noindent Enfin l'évolution temporelle dans le réservoir de neige s'écrit:
\begin{equation}
\label{eq_waterfx_sn}
 \frac{dW_{n}}{dt}  = P_{n} + p_{sn,g}[ P_{r}(1-veg) + d_{r} ] - E_{s} - S_{m}
\end{equation}
avec $P_{n}$ les précipitations neigeuses (kg.m$^{-2}$.s$^{-1}$). $E_{s}$ l'évaporation du manteau neigeux (kg.m$^{-2}$.s$^{-1}$). $S_{m}$ la masse de neige fondue quittant le réservoir (kg.m$^{-2}$.s$^{-1}$).\\

\noindent Ce bilan global est complété par des bilans hydrologiques spécifiques décrivant les flux de masse au sein des différentes couches du sol. Ces bilans sont définis sur les trois couches hydrologiques prescrites selon la figure \ref{isba_3L}. Ces trois couches correspondent pour la profondeur comprise entre 0 et $d_1$ à la couche superficielle, pour la profondeur comprise entre 0 et $d_2$ à la couche racinaire (cela implique que la couche superficielle est incluse dans la couche de profondeur $d_2$) et pour la profondeur entre $d_2$ et $d_3$ à la couche sous-racinaire. \\

\noindent Chacune des trois couches possède une équation de bilan en eau distincte liée aux autres couches par le biais de transferts verticaux de contenu en eau. Ainsi la teneur en eau de la couche superficielle $\omega_1$ contribue à la teneur en eau de la couche racinaire $\omega_2$ suivant une approche force-restore \citep{deardorff1978}.\\ 
\noindent La couche sous-racinaire a été introduite par \citet{boone1999} pour différencier la couche racinaire de la profondeur totale du sol et pour prendre en compte un contenu en eau distinct et non influencé par la végétation, $\omega_3$, de cette zone. Toutes les valeurs de teneurs en eau sont bornées par une valeur minimale $\omega_{min}$ qui évite que le sol ne s'assèche totalement (en effet même dans un sol très sec une fine pellicule d'eau reste liée aux grains par adsorption) et une valeur maximale $\omega_{sat}$, définie pour caractériser la porosité du sol.\\

\begin{figure}[h!]
\centering
\includegraphics[width=0.7\textwidth]{ISBA_3L}
\caption{Représentation du bilan d'eau comme modélisé par ISBA-3L.}
\label{isba_3L}
\end{figure}
~\\

\noindent Les équations qui régissent l'évolution de la teneur en eau de ces différentes couches sont de la forme:
\begin{equation}
 \begin{cases}
 \dfrac{\partial \omega_{1}}{\partial t} =& \dfrac{C_{1}}{\rho_{\omega}d_{1}}(I_{r}-E_{g}-Q_{fz_{1}})-D_{1} \qquad 
 \hskip3.15cm
 \left(\omega_{min} <\omega_{1} <  \omega_{sat}\right)
 \\
 \dfrac{\partial \omega_{2}}{\partial t} =& \dfrac{1}{\rho_{\omega}d_{2}}(I_{r}-E_{g}-E_{tr}-Q_{fz_{1}}-Q_{fz_{2}})-K_{2}-D_{2} \qquad 
 \left(\omega_{min} <\omega_{2} <  \omega_{sat}\right)
 \\
 \dfrac{\partial \omega_{3}}{\partial t} =& \dfrac{d_{2}}{d_3-d_2}(K_{2}+D_{2})-K_{3} \qquad 
 \hskip3.85cm
 \left(\omega_{min} <\omega_{3} <  \omega_{sat}\right)
 \end{cases}
\end{equation}
 %
avec $C_{1}$ le coefficient de relaxation du sol contrôlant les échanges d'humidité entre la surface et l'atmosphère, $\rho_{\omega}$ la masse volumique de l'eau (kg.m$^{-3}$). $D_{1}$ (s$^{-1}$) et $D_{2}$ (s$^{-1}$) les termes de diffusion de l'humidité. $Q_{fz_{1}}$ et $Q_{fz_{2}}$ représentent les flux respectivement de surface et sub-surface lors du gel/dégel du sol (kg.m$^{-2}$). $K_{2}$ (s$^{-1}$) et $K_{3}$ (s$^{-1}$) sont les termes de drainage gravitationnels. $I_{r}$ correspond à l'infiltration réelle (kg.m$^{-2}$).

\subsubsection{{\fontfamily{lmss}\selectfont Ruissellement et infiltration}}

Les deux mécanismes participant à la production de ruissellement, présentés dans la section \ref{sec:ruissellement}, sont représentés dans ISBA par des paramétrisations sous-maille.
Le ruissellement de Dunne, généré par une saturation du réservoir de sol, est paramétré suivant deux approches. La première, introduite par \citet{habets1999}, s'appuie sur le schéma proposé dans le modèle hydrologique Variable Infiltration Capacity \citet[VIC,][]{dumenil1992}. Le ruissellement se base sur une discrétisation de chaque cellule en réservoirs élémentaires auxquels sont affectés des capacités d'infiltration propres $I_{pr}$. Lorsque la capacité des réservoirs non-saturés est connue, alors tous les réservoirs ayant une capacité inférieure à $I_{pr}$ contribuent à l'estimation de la fraction de cellule saturée $A$.
La deuxième méthode se base sur l'approche TOPMODEL développée par \citet{habets2001} et étendue par \citet{decharme2006}. Cette méthode détermine la fraction de maille saturée $f_{sat}$ afin d'estimer la fraction de précipitations atteignant directement le sol. 
Le ruissellement résultant s'exprime selon les équations:
%
\begin{equation}\label{eq_runoff}
\begin{cases} 
Q_{D}^{VIC}&= \int_{I_{pr}}^{I_{pr}+P_{r}} A(I)dI
\\
Q_{D}^{TOP}&= f_{sat}(1-veg)P_{r}
\end{cases}
\end{equation}
%
avec $P_{r}$ le cumul de précipitations (kg.m$^{-2}$.s$^{-1}$) et $f_{sat}$ la fraction de cellule saturée. \\

\noindent Le ruissellement de Horton, atteint lorsque l'intensité des précipitations dépasse le taux d'infiltration du sol, a été paramétré par \citet{decharme2006} selon:
\begin{equation}
Q_{H} = (1-\delta_{f})max(0,S_{m}+(1-veg)P - I_{no\_gel})+\delta_{f}max(0,S_{m}+(1-veg)P-I_{gel})
\end{equation}
avec $\delta_{f}$ la fraction de sol gelé, $I_{no\_gel}$ le taux d'infiltration d'un sol non gelé et  $I_{gel}$ le taux d'infiltration d'un sol gelé. \\

\noindent Le ruissellement simulé par ISBA se résume donc à:
\begin{equation}
Q_{ISBA}=Q_{D}+(1-f_{sat})Q_{H}
\end{equation}

\noindent En ce qui concerne l'infiltration réelle $I_{r}$, elle est définie comme la différence du potentiel maximal d'infiltration du sol $I_{p}$ et la part du stock ne s'infiltrant pas mais participant au ruissellement $Q_{ISBA}$ tel que:
\begin{equation} \label{eq_infiltration}
I_{r}=I_{p}-Q_{ISBA}
\end{equation}

L'infiltration potentielle maximale est définit par:

\begin{equation}
I_{p} = \left[(1-veg)P_r + d_r\right]\left(1-p_n\right)+ p_n \,S_m
\end{equation}

soit pour l'infiltration réelle:
\begin{equation}
I_{r}={\rm min}\bigg\lbrace
I_{p}, \,
\left[(1-veg)P_r + d_r
\right]\left(1-p_n\right)
+ p_n \,S_m
-Q_{ISBA}
\bigg\rbrace
\end{equation}

\subsubsection{{\fontfamily{lmss}\selectfont Diffusion}}
%
La diffusion verticale de l'humidité dans le sol est un processus clé dans les échanges entre les différentes couches du sol et caractérise la capacité d'un liquide à monter ou descendre dans un milieu poreux. Dans le schéma d'ISBA-3L deux types de diffusion sont décrites suivant le milieu dans lequel elle apparaît.\\ 
La diffusion entre la couche superficielle $d_{1}$ et la couche racinaire $d_{2}$ est proportionnelle à la teneur en eau de la première couche et le contenu en eau superficielle à l'équilibre entre les forces de gravité et de capillarité noté $\omega_{w_{eq}}$ suivant l'équation:
\begin{equation}
D_{1} = \frac{C_{2}}{\tau}(\omega_{1}-\omega_{w_{eq}})
\end{equation}
où $C_{2}$ est le coefficient de relaxation qui détermine la vitesse de rétablissement de l'équilibre hydrique entre les deux couches et $\tau$ la constante de relaxation égale à un jour (en s).

La diffusion entre la couche racinaire et sous-racinaire est directement calculée suivant la différence de teneur en eau:

\begin{equation}
D_{2} = \frac{C_{4}}{\tau}(\omega_{2}-\omega_{w_{3}})
\end{equation}
avec $C_{4}$ le coefficient de relaxation qui détermine la vitesse de rétablissement de l'équilibre hydrique entre les deux couches.

\subsubsection{{\fontfamily{lmss}\selectfont Drainage}}
%
De la même façon le drainage gravitationnel est séparé en deux termes. Le premier définit le drainage gravitationnel entre la couche racinaire et la couche sous-racinaire $K_{2}$ et le deuxième caractérise le drainage gravitationnel entre la couche sous-racinaire et le sous-sol. Ces termes apparaissent lorsque la teneur en eau est supérieure à la capacité totale du sol $\omega_{fc}$. Les deux termes de drainage sont calculés suivant les équations introduites par \citet{mahfouf1996} pour $K_{2}$ et \citet{boone1999} pour $K_{3}$:

\begin{align}
K_{2} =& \frac{d_{3}}{d_{2}}\frac{C_{3}}{\tau}max(0,\omega_{2}-\omega_{fc})
\\
K_{3} =& \frac{d_{3}}{(d_{3}-d_{2})}\frac{C_{3}}{\tau}max(0,\omega_{3}-\omega_{fc})
\end{align} 
%
Depuis cette version initiale, \citet{habets1999} a introduit un drainage résiduel qui correspond à la valeur minimale assurant un soutien aux étiages en périodes sèches.\\

\noindent Il est important de noter que tous les coefficients forçage-relaxation ainsi que les paramètres hydrologiques associés à chaque couche sont dépendants de l'humidité du sol ainsi que des propriétés de texture du sol développées par \citet{noilhan1995} elles-mêmes provenant des paramètres de \citet{clapp1978}. Elles ont été réécrites par \citet{decharme2006} afin de tenir compte du profil exponentiel de conductivité hydraulique à saturation tel que:

\begin{equation}
k_{sat}(z) = k_{sat,c} e^{-f(z-d_{c})}
\end{equation}
où $k_{sat,c}$ est la valeur minimale de la conductivité hydraulique à saturation (m.s$^{-1}$), $f$ est le coefficient de décroissance (m$^{-1}$) et $d_{c}$ la profondeur minimale de la zone saturée (m).\\

Cette dernière équation amène à écrire la conductivité hydraulique du sol suivant le potentiel hydrique de la couche $i$, $w_{i}$, selon:

\begin{equation}
k(w_{i},z) = k_{sat(z)} \left(\frac{w_{i}}{w_{sat}}\right)^{2b+3}
\end{equation}
avec $w_{sat}$ l'humidité du sol à saturation et $b$ la pente de la courbe de rétention hydrique.

\subsection{{\fontfamily{lmss}\selectfont Le modèle ISBA-DF}}
\label{sec:ISBA-DF}


\noindent La version diffusive d'ISBA reprend la structure d'ISBA-3L pour ce qui est de la structuration du sol mais divise celui-ci en N couches (le nombre de couches par défaut est de 12) dont la profondeur totale atteint 12 m \citep{decharme2013}. En outre, la surface est une couche explicite qui n'est plus contenue dans la deuxième couche comme cela est le cas dans la méthode "force-restore".\\
%
L'épaisseur de chaque couche est prescrite afin de réduire au maximum les artefacts introduits par la résolution numérique en différences finies. Ces profondeurs sont prescrites en mètres: 0.01 ,0.04, 0.2, 0.4, 0.6, 0.8, 1, 1.5, 2, 3, 5, 8 et 12, mais peuvent être modifiées par l'utilisateur pour les adapter à des cas spécifiques.\\

\noindent Du point de vue des bilans d'énergie et d'eau, la méthode de résolution s'apparente à la méthode "force-restore" d'ISBA-3L à laquelle s'ajoute une résolution explicite des équations verticales de Fourier et de Darcy assurant une meilleure représentation de l'hétérogénéité verticale des propriétés thermiques et hydrauliques du sol \citep{decharme2016}. 
Le drainage est similaire sur le plan conceptuel mais a une formule différente alors que les autres schémas, tels que le ruissellement de surface ou la prise en compte du manteau neigeux, restent identiques. \\

\noindent Concernant le schéma de neige, seule l'expression donnant la fraction recouvrant le sol nu diffère légèrement de l'Eq. \ref{eq:frac_snow} avec:

\begin{equation}
p_{sn,g} = min(1, \frac{D_{n}}{D_{ng}})
\end{equation}
avec $D_{n}$ la profondeur totale de sol (similaire à $h_{s}$ mais la notation est modifiée afin de bien distinguer les deux schémas, ici il faut comprendre que la profondeur totale est équivalente à la somme des épaisseurs de chaque couche) et $D_{ng}$ une profondeur limite pour la neige dans le sol fixée à 0.01 m. \\

\subsubsection{{\fontfamily{lmss}\selectfont Température du sol et bilan d'énergie}}

\noindent Le flux de chaleur dans le sol est déterminé par l'équation unidimensionnelle de transfert thermique de Fourier:
\begin{equation}
\label{eq:heat_trans}
c_{g}\frac{\partial T_{g} (z)}{\partial t} = \frac{\partial }{\partial z}\left[\lambda_{g}(z)\frac{\partial T_{g} (z)}{\partial z}\right] + \frac{L_{f}Q_{fz}(z)}{\Delta z}
\end{equation}
avec $c_{g}$ la capacité calorifique du sol (J.m$^{-3}$.K$^{-1}$), $T_{g}$ la température du sol (K), $\lambda_{g}$ la conductivité thermique du sol (W.m$^{-1}$.K$^{-1}$), $L_{f}$  la chaleur latente de fusion (3.337.10$^{5}$ J.kg$^{-1}$), $Q_{fz}$ le flux de gel/dégel pour le contenu en eau du sol dans chaque couche (kg.m$^{-2}$.s$^{-1}$).\\

\noindent La spécificité du schéma de la version récente d'ISBA-DF est la séparation du bilan d'énergie en deux schémas indépendants: l'un associé au manteau neigeux et l'autre au continuum sol-végétation-plaines d'inondations \citep{decharme2019}. Les équations décrivant l'évolution de la température de surface $T_{g1}$ et de la couche $i$, $T_{i}$ sont:

\begin{align} \label{eq:heat_trans_discret}
\frac{\partial T_{g1}}{\partial t}=& C_{T}[G - \frac{\overline{\lambda}}{\Delta\widetilde{z}}(T_{s}-T_{2})] \\
\frac{\partial T_{i}}{\partial t}=& \frac{1}{c_{gi}}\frac{1}{\Delta z_{i}}[ \frac{\overline{\lambda_{i-1}}}{\Delta\widetilde{z}_{i-1}}(T_{i-1}-T_{i})-\frac{\overline{\lambda_{i}}}{\Delta\widetilde{z_{i}}}(T_{i}-T_{i+1})] \qquad \forall i =2,N
\end{align}
%
avec $\Delta z_{i}$ l'épaisseur de la couche i (m), $\Delta\widetilde{z_{i}}$ l'épaisseur entre les centres de deux couches successives (m), $C_{T}$ la capacité calorifique de la surface (K.m$^{-2}$.J$^{-1}$), $c_{gi}$ est la capacité calorifique totale du sol pour la couche i (J.m$^{-3}$.K$^{-1}$) et $\overline{\lambda_{i}}$ représente la conductivité thermique moyenne à l'interface entre deux couches successives (W.m$^{-1}$.K$^{-1}$).\\

L'épaisseur entre les centres de chaque couche $\Delta\widetilde{z_{i}}$ est calculée en prenant la moyenne des épaisseurs de deux couches successives tandis que la conductivité thermique moyenne de ces mêmes couches correspond à une moyenne harmonique de chaque couche pondérée par l'épaisseur:

\begin{equation}
\overline{\lambda_{i}} = \frac{(\Delta z_{i+1}\lambda_{i+1}+\Delta z_{i}\lambda z_{i})}{\Delta z_{i+1}+\Delta z_{i}}
\end{equation}

\noindent La capacité calorifique totale du sol $c_{gi}$, dépendante de la porosité du sol, de la teneur en eau du sol et de sa conductivité, est la somme des capacités calorifiques de l'eau et de l'air comme proposé par \citet{peters1998}. Afin de résoudre ces équations, un schéma numérique implicite "backward" basé sur la méthode des différences finies a été introduit dans le modèle ISBA-DF.

\subsubsection{\fontfamily{lmss}\selectfont Le bilan en eau}
Le bilan en eau dans le modèle ISBA-DF utilise une version des équations de Richards qui permet de décrire les flux massiques dans le sol suivant la loi de Darcy. Les évolutions massiques sont alors exprimées sous forme de bilan volumique et les gradients hydrauliques sous forme de charge en eau. Le principal avantage d'utiliser cette forme d'équations est la possibilité d'applications à tous types de sols qu'ils soient homogènes ou hétérogènes, saturés ou non-saturés.\\

\noindent Le bilan en eau général d'ISBA-DF reprend l'aspect général de l'Eq. \eqref{bilan_eau_3L}. Seul le terme de ruissellement $S_{m}$ issu du manteau neigeux est corrigé afin de prendre en compte des processus supplémentaires comme la percolation de l'eau de fonte à travers le manteau neigeux, le gel de l'eau de fonte et la modification des contenus en eau pour les différentes couches.\\

\noindent En considérant le même nombre N de couches que précédemment établi, la combinaison de l'équation de continuité et de la loi de Darcy amène à l'équation de Richards où les flux en eau $F$ sont paramétrés suivant le jeu d'équations suivant pour le flux de masse et de vapeur:

\begin{align}
\frac{\partial \omega_{1}}{\partial t} =& \frac{1}{\Delta z_{1}}[-\overline{k_{1}}(\frac{\Psi_{1}-\Psi_{2}}{\Delta \widetilde{z_{1}}}+1)- \overline{\nu_{1}} (\frac{\Psi_{1}-\Psi_{2}}{\Delta \widetilde{z_{1}}})+\frac{Q_{src}-Q_{fz}-Q_{sb}}{\rho_{\omega}}]
\\
\frac{\partial \omega_{i}}{\partial t} =& \frac{1}{\Delta z_{i}} [F_{i-1}-F_{i}+\frac{Q_{src_{i}}-Q_{fz_{i}}-Q_{sb_{i}}}{\rho_{\omega}}]
\end{align}
%
avec le flux à l'interface entre deux couches calculé par:
\begin{equation}
F_{i}=\overline{k_{i}}\left(\frac{\Psi_{i}-\Psi_{i+1}}{\Delta \widetilde{z_{i}}}+1\right) + \overline{\nu_{i}}\left(\frac{\Psi_{i}-\Psi{i+1}}{\Delta \widetilde{z_{i}}}\right)
\end{equation}
où $\overline{k_{1}}$ est la conductivité hydraulique moyenne de la couche superficielle (m.s$^{-1}$), $\overline{k_{i}}$ est la moyenne des conductivités hydrauliques entre les centres de deux couches successives (m.s$^{-1}$) soit :
%
\begin{equation}
\overline{k_{i}}= \sqrt{ k_{i}\left(\Psi_{i})  \, k_{i+1}(\Psi_{i+1}\right)}
\end{equation}
%
$\overline{\nu_{1}}$ est la conductivité isotherme de vapeur de la couche superficielle (m.s$^{-1}$), $\overline{\nu_{i}}$ est la moyenne des conductivités isothermes de vapeur entre les centres de deux couches successives calculées selon l'approche de \citet{braud1993} (m.s$^{-1}$), $\Psi_{i}$ le potentiel hydrique de la couche $i$ (m), $Q_{src}$ l'ensemble des termes sources/puits du sol telles que l'évapotranspiration et l'infiltration (kg.m$^{-2}$.s$^{-1}$) et $Q_{sb}$ les ruissellements de sub-surface simulés par l'approche TOPMODEL par rapport à la topographie sous-maille (kg.m$^{-2}$.s$^{-1}$).\\

\noindent Cette dernière équation est simplifiée pour la couche $N$ la plus profonde du sol où les gradients de potentiel hydrique sont négligeables pour devenir:
%
\begin{equation}
F_{N} = k_{N}
\end{equation}
%
avec $k_{N}$ le conductivité hydraulique de la couche $N$.\\

\noindent Cette hypothèse assure une condition à la limite basse pour le drainage même si la description des eaux souterraines et donc la présence d'aquifères nécessite une paramétrisation différente. Pour plus de détails sur la paramétrisation des aquifères, le lecteur se tournera vers la thèse de \citet{vergnes2012these}. Le lecteur se référera à \citet{boone2000} pour la méthode de résolution.\\
Dans cette approche, le sol est discrétisé en N couches dont l'épaisseur de la couche $j$ est $\Delta z_{j}$ et l'épaisseur entre le milieu de deux couches successives est $\Delta \widetilde{z_{j}}$. 
Les différentes discrétisations sont illustrées par la figure \ref{boone} de la section \ref{subsec:ISBA-3L}.\\

\subsubsection{\fontfamily{lmss}\selectfont Les processus de gel/dégel du sol}

Afin de compléter la représentation du sol, un processus de gel du sol a été mis au point par \citet{decharme2016} dans le but d'améliorer la dynamique des régions sujettes à ce phénomène. \\
La tendance du sol à geler est résolue explicitement sur chaque couche de sol en prenant en compte les effets de sublimation de la glace et le rayonnement de la végétation \citep{boone2000}. Ce changement de phase est simulé à chaque pas de temps si la disponibilité en énergie et en eau est suffisante. L'ajout de ces processus se traduit par des termes supplémentaires de changement de phase dans les bilans d'énergie et d'eau pour la couche superficielle et la couche racinaire. En terme de contenu en eau, l'évolution temporelle est définie par les équations:

 \begin{align}
 \dfrac{\partial \omega_{1,f}}{\partial t} =& \dfrac{1}{d_{1} \rho_{\omega}}\left(F_{1,\omega} - E_{g,f}\right) \qquad 
 \hskip.36in
 \left(0 \leq\omega_{1,f} \leq  \omega_{sat} - \omega_{min}\right)
 \\
 \dfrac{\partial \omega_{2,f}}{\partial t} =& \dfrac{1}{(d_{2}-d_{1})\rho\omega} F_{2,\omega} \qquad 
 \hskip.46in
 \left(0 \leq \omega_{2,f} \leq  \omega_{sat} - \omega_{min}\right)
 \end{align}
avec $\omega_{1,f}$ et $\omega_{2,f}$ respectivement le contenu en eau volumique équivalent pour la glace dans la couche superficielle et la couche racinaire ($m^{3}.m^{-3}$). $F_{1,\omega}$ et $F_{2,\omega}$ ($kg.m^{-2}.s^{-1}$) respectivement le terme de changement de phase dans la couche superficielle et dans la couche racinaire. $E_{g,f}$ ($kg.m^{-2}.s^{-1}$) le flux de sublimation à la surface en présence de glace.\\

Les flux de masse dus à la formation de glace ou à la fonte sont exprimés suivant les équations:

\begin{align}
 F_{1,\omega} =& (1-p_{n,g}) (F_{1,m}-F_{1,f})
 \\
 F_{2,\omega} =& (1-p_{n,g}) (F_{2,m}-F_{2,f}) 
\end{align}
avec l'indice numérique représentant la couche (1 pour la couche superficielle et 2 la couche racinaire). L'indice alphabétique informe sur le gel ($f$) ou la fonte ($m$).\\

  De plus, le contenu en eau maximal susceptible de geler est déterminé par l'approche énergétique de Gibbs en fonction de la température \citep{fuchs1978}. Ce paramètre introduit une limite pour la formation de la glace dans le sol. Cette approche permet aussi de déterminer la température maximale de changement d'état de l'eau \citep{decharme2016}.\\
  
Ce phénomène de gel-dégel est important en hydrologie car il influence la capacité de rétention en eau du sol. En effet, l'eau gelée est prise en compte dans la couche de sol ce qui laisse moins d'espace à l'eau liquide tout en modifiant les flux de masse par capillarité ou drainage.\\

Enfin, la végétation possède un effet de rayonnement non négligeable qui est pris en compte dans les équations de production de glace ou sa fonte par le biais du coefficient $K_{I,s}$ \citep{boone2000}:

\begin{equation}
K_{I,s} = \left(1-\dfrac{veg}{K_{I,2}}\right) \left(1-\dfrac{LAI}{K_{I,3}}\right)
\end{equation} 
avec $K_{I,2}$ et $K_{I,3}$ des coefficients sans dimension pour la végétation déterminés par \citet{giard2000}.\\
\noindent Cette équation assure un ralentissement du processus de gel lorsque la fraction $veg$ augmente et donc qu'une partie de l'énergie radiative est utilisée pour réchauffer ou refroidir la végétation présente sur la maille.
\subsection{\fontfamily{lmss}\selectfont Le traitement spécifique de la neige dans ISBA}

La neige modifie de façon significative le bilan d'énergie de surface, notamment les flux de conduction du sol, par une réduction de la longueur de rugosité et l'augmentation de l'albédo de surface. Ces composantes sont donc indispensables pour une modélisation réaliste des interactions neige-atmosphère et des processus hydrologiques. \\
Les lacs se situent principalement dans des zones soumises aux précipitations neigeuses et couvertes de sols pouvant geler. Cela s'observe de manière générale et plus spécifiquement pour les lacs étudiés dans le cadre de cette thèse (voir notamment au chapitre \ref{chap:etude-globale}). Même si les processus neigeux ne figurent pas au cœur du travail présenté, il est nécessaire d'en tracer les grandes lignes. \\
Plusieurs schémas de neige sont proposés dans ISBA:

\begin{itemize}
\item[$\bullet$] un schéma mono-couche;
\medbreak
\item[$\bullet$] un schéma multi-couches ISBA-Explicit Snow \citep[ES,][]{boone2001};
\medbreak
\item[$\bullet$] un schéma complexe pour le suivi des propriétés du manteaux neigeux et le risque d'avalanches CROCUS \citep{vionnet2012}.
\end{itemize}
\medbreak
La mise en place d'un schéma explicite de neige assure la résolution des gradients thermiques et de densité présents dans le manteau neigeux. Le schéma permet aussi la distinction entre le bilan d'énergie du sol et du manteau neigeux ainsi que l'estimation des échanges thermiques conductifs entre le sol et la neige et la simulation de l'évolution des réservoirs d'eau.

\subsubsection{\fontfamily{lmss}\selectfont Le schéma mono-couche}

Le schéma mono-couche possède trois variables pronostiques: l'équivalent en eau $W_{n}$ (Snow Water Equivalent, SWE), l'albédo $\alpha_{n}$ et la densité moyenne de la neige $\rho_{n}$.
La première est régie par l'Eq. \eqref{eq_waterfx_sn} et estime les évolutions du manteau neigeux  suivant les précipitations neigeuses et la sublimation de la neige. La densité de neige décrit l'état du manteau neigeux et varie, à un taux fixe \citep{verseghy1991} jusqu'à une valeur seuil de 300 kg.m$^{-3}$. Enfin l'albédo de la neige est une variable permettant de corriger l'estimation d'albédo moyenne $\alpha_{t}$ sur la cellule en prenant en compte les phénomènes de fonte et de vieillissement de la neige tel que:

\begin{equation}
\alpha_{t} = (1-\rho_{sn}\alpha + p_{sn}\alpha_{n})
\end{equation}
avec $\alpha$ l'albédo du couvert de sol sur la maille.\\

\noindent Concernant le bilan d'énergie, la forme du bilan pour le manteau neigeux peut se résumer sous la forme:

\begin{equation}
c_{n} \frac{dT_{n}}{dt} = R_{n,n} - H_{n} - LE_{S,n} - G_{n} + F_{n}
\end{equation}

\noindent L'apport du schéma ES par \citet{boone2001} est de mieux détailler la représentation des processus au sein du manteau neigeux et notamment les échanges d'eau entre les différentes couches de neige. Le modèle ISBA-ES a été développé dans une optique de couplage avec les modèles atmosphériques et les modèles hydrologiques distribués. Ce schéma divise le manteau neigeux en trois couches utilisant quatre variables pronostiques pour décrire l'état du manteau neigeux à chaque pas de temps: l'équivalent en eau du manteau neigeux, la chaleur stockée par la neige $H_{s}$, l'épaisseur de la couche $D$ et l'albédo $\alpha_{n}$.\\
Ainsi le réservoir de neige est assimilé à un réservoir qui vide une partie de son contenu dans la couche du dessous si sa capacité maximale est atteinte et où l'eau qui a percolé peut à nouveau geler. Le bilan simplifié s'écrit donc:

\begin{equation}
 \frac{dW_{n}}{dt}  = P_{n} - E_{s} - T_{r}
\end{equation}
avec $T_{r}$ la composante prenant en compte les phénomènes de ruissellement mais aussi de percolation et de gel/dégel à travers le manteau.\\

L'ajout de ces derniers processus implique un possible retard entre la fonte de surface\footnote{La fonte de surface débute lorsque la température de la neige, $T_{n}$ est supérieure à 273.15 K.} et le ruissellement qui s'écoule hors du manteau neigeux avec un effet sur l'hydrologie des bassins versants soumis à des régimes de type montagneux ou arctique.

Pour des détails plus précis, le lecteur se tournera vers les études de \citet{boone2001, vionnet2012, decharme2016}


\section{{\fontfamily{lmss}\selectfont Le modèle de routage en rivière: CTRIP}}
\label{sec:CTRIP}

Les modèles de surface tels qu'ISBA assurent la production des flux de surface et de sub-surface mais n'ont pas vocation à transformer ces quantités d'eau en débits transférés dans les bassins versants considérés. Ce transfert est simulé par le biais de modèles dédiés appelés modèles de routage en rivière.

\subsection{\fontfamily{lmss}\selectfont Version native: le modèle TRIP}

Plusieurs schémas de routage ont été développés dans les années 90 \citep{vorosmarty1989, coe1998, hagemann1997, fekete2001}. Le modèle initial utilisé au CNRM était celui de \citet{oki1998} appelé Total Runoff Integrating Pathways (TRIP). Ce schéma utilise une vitesse constante et uniforme de 0.5 m.s$^{-1}$ pour transférer les masses d'eau sur des grands bassins fluviaux à la résolution de 1° x 1° \footnote{Cette résolution correspond à une distance de 110 km en longitude et 110 km en latitude au niveau de l'équateur.}. De plus, chaque cellule contient un unique tronçon de rivière. Pour déterminer l'évolution temporelle des masses d'eau, TRIP résout une équation de bilan de masse avec comme variable pronostique la stock d'eau contenu dans un tronçon de rivière sur la maille. Ainsi sur chaque cellule du bassin, le débit entrant est issu des cellules amonts puis ajouté au ruissellement et drainage de la cellule étudiée pour déterminer le débit sortant à transférer à la cellule aval. L'ordre des cellules est spécifié par un réseau de routage intrinsèque à TRIP sous l'hypothèse que chaque tronçon de rivière peut recevoir une certaine quantité d'eau de plusieurs affluents mais ne peut transférer son stock qu'à travers un unique exutoire. À l'échelle de travail, il n'est pas possible de représenter chaque tronçon de rivière de manière correcte et réaliste ce qui oblige à regrouper plusieurs tronçons de rivière sous la forme d'un tronçon rectangulaire équivalent.

\subsection{\fontfamily{lmss}\selectfont La version CNRM: CTRIP}
\subsubsection{\fontfamily{lmss}\selectfont Principe général}


\noindent Dans le cadre d'applications spécifiques au CNRM, TRIP a subit de nombreuses améliorations qui ont abouti au développement d'une version de TRIP propre au CNRM, appelée CTRIP et présentée sur la figure \ref{decharme2019}.\\

\begin{figure}[h!]
\centering
  \includegraphics[scale=0.35]{Decharme2019.png}
  \caption{Schéma des différents processus du modèle couplé ISBA-TRIP.}
  \label{decharme2019}
\end{figure} 


CTRIP est un modèle de routage en rivière modulaire qui donne la possibilité de simuler le réservoir d'eau souterrain \citep{decharme2010} ainsi que les interactions entre la rivière et les plaines d'inondations \citep{decharme2008} ou encore la dynamique bidimensionnelle des aquifères \citep{vergnes2012}. Ainsi le réservoir de surface a été complété par un réservoir d'eaux souterraines afin de retarder la contribution du drainage dans la production de débit \citep{decharme2010}. CTRIP donne aussi la possibilité de considérer les écoulements paramétrés par une une vitesse variable dérivée d'une équation de Manning à la place de la vitesse constante initialement utilisée dans TRIP. Enfin le modèle a été couplé à ISBA par \citet{decharme2006}. Le couplage assure l'alimentation du système de routage en ruissellement de surface et en drainage par ISBA en réponse aux forçages atmosphériques afin de simuler le débit des grands bassins fluviaux et la gestion de leurs eaux à l'échelle globale. En plus des transferts latéraux, il considère les échanges verticaux et permet notamment de simuler les remontées capillaires du réservoir souterrain de CTRIP vers la zone racinaire d'ISBA ainsi que l'évaporation sur les zones inondées. Ce dernier point est essentiel puisque c'est sur ce principe que se base la prise en compte de l'évaporation sur les lacs. Ce couplage est aussi une pierre angulaire dans la caractérisation du cycle hydrologique régional \citep{decharme2008}, global \citep{alkama2010, decharme2012}, est intégré aux modèles de climat \citep{voldoire2019}, de modélisation du Système Terre \citep{seferian2019} et permet une meilleure modélisation des flux du cycle du carbone de surface \citep{delire2020}. CTRIP représente aujourd'hui le routage en rivières en global à la résolution 1/12° comme représenté sur la figure \ref{12d_globe}.\\

\begin{figure}[h!]
 \centering
 \includegraphics[width=1.\textwidth]{12d_dense}
 \caption{Réseau de rivière CTRIP au 1/12° à l'échelle globale.}
 \label{12d_globe}
\end{figure}

~\\

Dans la version la plus complète, le modèle CTRIP comprend trois réservoirs: un réservoir de surface $S$ pour la stock en eau des rivières, un réservoir $F$ pour le stock contenu dans les plaines d'inondations et un réservoir $G$ pour les eaux souterraines. L'évolution temporelle de ces stocks est régie par un système d'équations basé sur une approche d'onde cinématique:\\

\begin{eqnarray}
\dfrac{\partial S}{\partial t} &=&  Q_{in}^{S} - Q_{out}^{S} - Q_{RF} \\ 
\dfrac{\partial F}{\partial t} &=& Q_{RF}+ W_{F} \\ 
\dfrac{\partial G}{\partial t} &=& Q_{RG}
\end{eqnarray}

\noindent $Q_{in}^{S}$ (kg.s$^{-1}$) représente la somme des contributions des débits amont $Q_{in, TRIP}^{S}$ et du ruissellement sur la cellule $ Q_{in, R}^{S}$ tel que $Q_{in}^{S} = Q_{in, R}^{S}+ Q_{in, TRIP}^{S}$. $Q_{RG}$ correspond aux flux échangés avec le réservoir profond comptés positivement lorsqu'ils sont reçus et négativement lorsqu'ils sont perdus (kg.s$^{-1}$). $Q_{RF}$ représente les échanges de masse entre la rivière et les plaines d'inondations adjacente comptés positivement lorsqu'ils sont reçus et négativement lorsqu'ils sont perdus (kg.s$^{-1}$). \\
\clearpage
\noindent Le bilan sur les plaines d'inondations fait apparaître un terme de flux de masse $W_{F}$ qui rend compte du couplage entre CTRIP et ISBA et de la résolution d'un bilan de masse spécifique mise en place lorsque la hauteur d'eau dans la rivière dépasse une hauteur critique de débordement tel que:

\begin{equation}
W_{F} = P_{F} - I_{F} - E_{F}
\end{equation}
avec $P_{F}$ la fraction des précipitations captée par la plaine d'inondations (kg.s$^{-1}$), $I_{F}$ la part d'infiltration (kg.s$^{-1}$) et $E_{F}$ la fraction du volume contenu dans les plaines s'évaporant (kg.s$^{-1}$).\\

\noindent Ce schéma d'inondation ne sera pas utilisé dans l'évaluation du modèle de lacs développé dans cette thèse et tous les détails du modèle sont précisés dans \citet{decharme2010}.

\subsubsection{\fontfamily{lmss}\selectfont Paramétrisation d'un écoulement à vitesse variable}
\label{subsec:vitesse_var}

Pour le réservoir de surface, la masse quittant le tronçon de rivière est déterminée proportionnellement à la masse $S$ présente dans le tronçon et des caractéristiques morphologiques de la rivière par l'équation:

\begin{equation}
Q_{out}^{s}=\frac{\nu_{s}}{L_{riv}}S
\end{equation}
avec $\nu_{s}$ la vitesse d'écoulement (m.s$^{-1}$) et $L_{riv}$ (m) la longueur du tronçon de rivière pondérée par un coefficient de méandrement de 1.4 \citep{oki1998}.\\

\noindent Les caractéristiques d'un tronçon de rivière CTRIP sont représentées par la figure \ref{troncon_riv}. \\

\begin{figure}[h!]
\centering
\includegraphics[width=0.6\textwidth]{troncon_riv_new}
\caption{Représentation d'un tronçon de rivière dans CTRIP et ses caractéristiques associées.}
\label{troncon_riv}
\end{figure}

\noindent La vitesse d'écoulement n'est plus constante dans la version actuelle du modèle car peu réaliste pour décrire les débits des grands bassins fluviaux à l'échelle globale. Une vitesse variable, pour les flux dans les rivières et les plaines d'inondations, à été introduite par \citet{decharme2010} dans CTRIP afin d'accroître le réalisme en liant la vitesse d'écoulement aux caractéristiques du tronçon de rivière selon une résolution de type Manning \citep{arora1999} pour une section rectangulaire:

\begin{equation}
\nu_{s} =  \dfrac{s_{riv}^{\frac{1}{2}}}{n_{riv}}\left(\frac{W_{riv}.h_{s}}{W_{riv}+2h_{s}}\right)^{\frac{2}{3}}
\end{equation}
$s_{riv}$ la pente du lit de la rivière (m.m$^{-1}$), $n_{riv}$ le coefficient de rugosité de Manning (s.m$^{\frac{-1}{3}}$), $W_{riv}$ la largeur de la rivière (m) et $h_{s}$ la cote de surface de la rivière (m). 
\noindent Le rapport $\frac{W_{riv}.h_{s}}{W_{riv}+2h_{s}}$ est aussi appelé rayon hydraulique.\\

\noindent Cette dernière hauteur est proportionnelle à la masse contenue dans le tronçon de rivière et inversement proportionnelle au produit de la longueur par la largeur du tronçon:

\begin{equation}
h_{s} = \frac{S}{\rho_{\omega}L_{riv}W_{riv}}
\end{equation}
ave $\rho_{\omega}$ la masse volumique de l'eau (kg.m$^{-3}$).
\subsubsection{\fontfamily{lmss}\selectfont La dynamique des eaux souterraines}

Même s'il ne constitue pas le cœur de la présente thèse, le modèle d'aquifère a été utilisé dans certains cas d'étude et mérite donc d'être succinctement détaillé.
Le schéma résout une équation diffusive bidimensionnelle des écoulements souterrains avec pour variable pronostique la charge piézométrique $h_{\omega}$. Afin de prendre en compte la rotondité de la Terre dans les équations, celles-ci sont résolues en coordonnées sphériques selon:

\begin{eqnarray}
\omega_{eff} \dfrac{\partial h_{\omega}}{\partial t} &=& \dfrac{1}{r^{2} cos(\phi)}\left[\dfrac{\partial}{\partial \theta}\left(\frac{T_{\theta}}{cos(\phi)}\dfrac{\partial h_{\omega}}{\partial \theta}\right)+\dfrac{\partial}{\partial \phi}\left(T_{\phi}cos(\phi)\dfrac{\partial h_{\omega}}{\partial \phi}\right)\right] \\
& & + \dfrac{1}{\rho_{omega}}(Q_{sb}+Q_{sg}+Q_{ice}+Q_{RG}) \nonumber
\end{eqnarray}
où $\omega_{eff}$ est la porosité effective du sol (m$^{3}$.m$^{-3}$), $\theta$ la longitude, $\phi$ la latitude, $r$ le rayon de la Terre (m), $T_{\theta}$ et $T_{\phi}$ représentent la transmissivité respectivement selon l'axe des longitudes et l'axe des latitudes (m.s$^{-1}$), $Q_{sb}$ est le drainage provenant du couplage avec le modèle ISBA (kg.m$^{-2}$.s$^{-1}$), $Q_{sg}$ correspond aux flux de masse sol-aquifères résolus suivant l'équation de Darcy (kg.m$^{-2}$.s$^{-1}$), $Q_{ice}$ correspond au ruissellement provenant du manteau neigeux (kg.m$^{-2}$.s$^{-1}$) et $Q_{RG}$ est le flux de masse entre la rivière et l'aquifère (kg.m$^{-2}$.s$^{-1}$). \\

\noindent Les flux de masse transférés par capillarité à travers le sol $Q_{sg}$ ne sont pas uniformes sur les mailles CTRIP car ils ne sont effectifs que dans des plaines alluviales ou des zones planes. Ainsi certaines informations, comme la variabilité spatiale de la topographie sous-maille, doivent être partagées entre ISBA et CTRIP pour pouvoir considérer ces remontées d'eau par capillarité. L'équation régissant ces flux de masses a été développé par \citet{vergnes2014} selon:

\begin{equation}
Q_{sg} = f_{wtd}\rho_{\omega}k_{N}\left(\frac{\Psi_{N}-\Psi_{sat}}{z_{N}-z_{wtd}}+1\right)+(1-f_{wtd})\rho_{\omega}k_{N}
\end{equation}
$f_{wtd}$ est la fraction de maille où sont effectifs ces échanges. $k_{N}$, $z_{N}$ et $\Psi_{N}$ sont respectivement la conductivité hydraulique (m.s$^{-1}$), la profondeur (m) et le potentiel hydrique (m) de la dernière couche de sol $N$ dans le modèle ISBA (zone racinaire). $z_{wtd}$ est la profondeur de l'aquifère (m). $\Psi_{sat}$ est le potentiel hydrique du sol à saturation (m).\\
 
\noindent Les échanges entre la rivière et l'aquifère sont paramétrés suivant le rapport entre la charge hydraulique de la rivière et celle de l'aquifère pour garantir que le flux soit dirigé vers l'aquifère lorsque la charge piézométrique est inférieure à l'altitude du lit de la rivière et inversement lorsqu'elle est supérieure. La profondeur réaliste des aquifères est garantie par l'ajout d'une condition limite basse où la profondeur maximale est fixée à 1000 m.\\
La résolution de ce système d'équations se fait par un schéma numérique implicite en différences finies qui provient du modèle hydrogéologique MODCOU \citet{ledoux1989} au pas de temps journalier.

\subsection{\fontfamily{lmss}\selectfont CTRIP 12D}
\label{subsec:CTRIP_12D}

\noindent La version la plus récente de CTRIP a vu des améliorations significatives lors des dernières années.\\
En premier lieu, la résolution du modèle a été affinée pour passer d'une résolution de 0.5° x 0.5° \footnote{équivalent à environ 40 km pour la France métropolitaine.} à la résolution 1/12° \footnote{équivalent à environ 6-8 km en direction azimutale pour la France métropolitaine.}. Cette résolution assure une meilleure représentation de la structure des rivières avec une meilleure prise en compte des méandres des grands fleuves mais aussi une discrétisation plus performante entre les différents cours d'eau d'un même réseau (Figure \ref{ctrip_france}).

\begin{figure}[!h]
     \subfloat[CTRIP à 0.5°\label{ctrip_fr_2d}]{%
       \includegraphics[width=0.45\textwidth]{CTRIP_2D.png}
     }
     \hfill
     \subfloat[CTRIP à 1/12°\label{ctrip_fr_12d}]{%
       \includegraphics[width=0.45\textwidth]{CTRIP_12D.png}
     }
     \hfill
     \caption{Représentation des rivières à différentes résolutions.}
     \label{ctrip_france}
\end{figure}

\subsection{\fontfamily{lmss}\selectfont Les caractéristiques nécessaires au fonctionnement de CTRIP}
\label{sec:ctrip_caracteristique}
L'ensemble des flux de masses vus ci-dessus sont calculés en fonction de paramètres et de caractéristiques intrinsèques à chaque tronçon de rivière eux-mêmes faisant partie d'une entité plus grande: le bassin versant. Ainsi la vitesse d'écoulement dépend de la localisation du segment dans le réseau mais aussi de la géomorphologie de la zone concernée. De plus, la qualité du routage dépend de paramètres essentiels à une reconstitution correcte du chevelu hydrologique et des apports des affluents. Par conséquent, il est primordial de définir les caractéristiques qui seront prescrites au modèle pour router les flux de masse. Les données initiales pour la détermination des caractéristiques sont différentes suivant la résolution du modèle et seule les données pour le réseau au 1/12° sont présentées ici. Les paramètres du modèle sont issus d'observations, de jeux de données haute résolution lorsque c'est possible (\textit{e.g.} la longueur des tronçons de rivière) ou de relations empiriques (\textit{e.g.} le coefficient de rugosité).

\subsubsection{\fontfamily{lmss}\selectfont La topographie}

Chaque cellule du réseau contient un unique tronçon de rivière. L'écoulement, et plus largement la structure du réseau au sein de ces tronçons, est contraint par la topographie. Cette topographie provient du modèle numérique de terrain (MNT) MERIT-DEM \citep[accessible à: \url{http://hydro.iis.u-tokyo.ac.jp/~yamadai/MERIT_DEM/ ,}][]{yamazaki2017} qui informe sur l'altitude de la surface à une résolution de 3 arcsec (soit 1/1200°)\footnote{équivalent à 90 m au niveau de l'équateur.}. Le réseau de drainage est, lui aussi, construit sur la base d'un MNT, cette fois-ci en utilisant MERIT-HYDRO \citep[accessible à: \url{http://hydro.iis.u-tokyo.ac.jp/~yamadai/MERIT_Hydro/}][]{yamazaki2019} à la même résolution de 90m. Ce réseau à haute résolution est ensuite projeté au 1/12° par un algorithme de Dominant River Tracing \citep{wu2011, wu2012}.\\

La méthode de projection consiste en un algorithme automatique d'extraction et de remontée en échelle d'un réseau de rivières basé sur des informations hydrographiques à haute résolution. Pour ce faire, la méthode se base sur les paramètres de direction d'écoulement, d'accumulation de flux et du réseau à haute résolution pour déterminer des motifs récurrents à plus haute résolution. L'avantage de ce type d'approche est de conserver l'écoulement général dans les bassins fluviaux en déterminant un trajet entre les cellules sources et les exutoires à haute résolution et en donnant la priorité au plus grands cours d'eau. Pour cela, un identifiant est attribué à chaque rivière et conservé tout au long du processus afin de garder la cohérence et l'identification des bassins entre les deux résolutions (1/12° et 1/1200°).

\subsubsection{\fontfamily{lmss}\selectfont Le séquençage}

Le séquençage des tronçons de rivière est une étape primordiale qui attribue à chaque nœud du réseau un numéro de séquence $SN$ qui définit sa localisation dans le chevelu. La résolution numérique de CTRIP itère sur la base de ce numéro afin d'assurer la résolution du transfert de l'eau depuis les cellules amont vers les cellules aval.\\
Dans cette méthode, les cellules les plus en amont se voient attribuer la valeur minimale de 1 qui est ensuite incrémentée pour chaque nouveau nœud aval rencontré. Dans le cas spécifique où une cellule reçoit un flux de plusieurs pixels amont, le numéro de séquence est attribué selon la règle:

\begin{equation}\label{eq:SN}
SN_{aval} = max(SN_{i,amont}) + 1
\end{equation}
avec $SN_{i,amont}$ le numéro de séquence de la rivière dont l'identifiant est $i$.

\subsubsection{\fontfamily{lmss}\selectfont Paramètres morphologiques de la rivière}
\label{subsec:param_riv}

Différentes caractéristiques morphologiques sont nécessaires au routage (voir figure \ref{troncon_riv} en section \ref{subsec:vitesse_var}). Ces paramètres conditionnent le transfert horizontal mais aussi la structure de chaque tronçon de rivière et la cohérence générale des bassins. \\

\noindent Précédemment, la longueur de rivière $L_{riv}$ (m) a été introduite. Cette longueur correspond à la distance parcourue par le tronçon de rivière au sein de la cellule, calculée à partir du réseau haute résolution MERIT-HYDRO. Au 1/12°, les longueurs sont considérées comme suffisamment réalistes pour se soustraire au coefficient de méandrement. Elles sont par ailleurs bornées entre 1000 m et 20000 m.\\

\noindent De façon similaire la pente de la rivière $s_{riv}$ ($m$) est calculée à partir de la différence d'altitude provenant du MERIT-DEM entre l'amont et l'aval de chaque tronçon de rivière. Afin de garantir un écoulement dans tous les tronçons, une valeur minimale est prescrite égale à 10$^{-4}$ (m). La formule pour la pente se résume à:

\begin{equation}
s = max\left(\frac{z_{amont} - z_{aval}}{L_{riv}},10^{-4}\right)
\end{equation}
~\\

Un paramètre géomorphologique essentiel pour ce travail de thèse est la largeur de la rivière. Celle-ci est calculée suivant une formule empirique validée sur la France par \citet{vergnes2014} et étendue à l'échelle globale \citep{decharme2019}. La formulation se base sur la relation entre le débit moyen annuel de la rivière $Q_{mean}$ ($m^{3}.s^{-1}$) sur la période 1958-2010 et la largeur de la rivière $W_{riv}$ ($m$) selon:

\begin{equation}
W_{riv} = max(W_{min}, \alpha Q_{mean}^{\beta})
\end{equation}
avec $\alpha$ et $\beta$ des coefficients empiriques égaux respectivement à 5.41 et 0.59. Une valeur minimale $W_{min}$ est prescrite égale à 30 m.\\

Enfin la prescription de la profondeur de la rivière $h_{riv}$ ($m$) dans chaque cellule est donnée par la formule empirique introduite par \citet{vergnes2014}:

\begin{equation}
h_{riv} = 1.4 W_{riv}^{0.28}
\end{equation}

Sur la France métropolitaine, cette relation a été affiné et validé par \citep{vergnes2018} suivant:

\begin{equation}
h_{riv} = 0.14 W_{riv}^{0.53}
\end{equation}

\subsubsection{\fontfamily{lmss}\selectfont Le coefficient de Manning}
Essentiel au calcul de la vitesse d'écoulement, le coefficient de Manning de rivière noté $n_{riv}$, caractérise la résistance, aussi appelée rugosité, du lit de la rivière sur l'écoulement. À l'échelle globale, l'estimation précise d'un coefficient de Manning est limitée par la connaissance d'une part du type de lit de rivière et d'autre part de la végétation qui recouvre ou non les berges.
Dans CTRIP, ce paramètre est calculé en deux étapes. Tout d'abord, un paramètre empirique $\alpha_{r}$ est introduit pour rendre compte de la variation linéaire du coefficient de Manning à travers le réseau hydrographique. Cela permet d'assurer l'attribution de valeur élevée de coefficient de Manning pour les torrents en tête de bassin et des faibles valeurs pour les larges embouchures de fleuves tout en rendant compte d'un certain réalisme. Ce paramètre dépend du numéro de séquence tel que:

\begin{equation}
\alpha_{r} = \left(\frac{SN_{max}-SN}{SN_{max}-SN_{min}}\right)
\end{equation}
avec $SN_{max}$ et $SN_{min}$, respectivement, le numéro de séquence maximal et minimal du bassin dans le réseau CTRIP.\\

\noindent Connaissant ce paramètre, le coefficient de Manning est déterminé comme la moyenne géométrique entre les coefficients des plaines d'inondations $n_{fld}$ et une valeur standard de 0.035 (m$^{-1/3}$.s) \citep{lucas2003, yamazaki2011}:

\begin{equation}
n_{riv} = 0.035^{1.0-\alpha_{r}}.n_{fld}^{\alpha_{r}}
\end{equation}
où $n_{fld}$ est définit comme la moyenne arithmétique des coefficients de chaque couvert présent sur la cellule pondéré par leur fraction (m$^{-1/3}$.s).

\section{{\fontfamily{lmss}\selectfont Le modèle de bilan d'énergie: FLake}}
\label{sec:FLake}
Dans la résolution du bilan de masse, il est nécessaire d'estimer les flux de masse échangés avec l'atmosphère par le biais des flux de chaleur turbulents. Le modèle FLake est déjà intégré à la plateforme SURFEX pour résoudre le bilan d'énergie des lacs à l'échelle globale \citep{salgado2010}.\\
FLake (Freshwater Lake model) est un modèle unidimensionnel simulant l'évolution du profil vertical de température au sein des lacs et résolvant le bilan d'énergie au sein des différentes couches qui les structurent, pour satisfaire les besoins de la prévision numérique du temps \citep{mironov2008, mironov2010}. Son utilisation en opérationnel a permis de réduire certains biais, notamment ceux sur les températures de l'air \citep{balsamo2012}. FLake est, aussi, intégré au modèle climatique CNRM-CM de Météo-France \citep{voldoire2019} et a été utilisé dans le cadre d'études climatiques afin d'améliorer la connaissance sur les interactions entre les lacs et le climat \citep{samuelsson2010, lemoigne2013,lemoigne2016}.\\

Aujourd'hui utilisé dans de nombreux services de prévisions météorologiques comme le DWD (Allemagne) ou le CEPMMT, FLake a l'avantage d'être relativement peu coûteux en ressources de calcul et cela sans le besoin d'effectuer de réglage ses paramètres. De plus, le modèle se base sur des paramètres externes issus d'observations le rendant ainsi théoriquement applicable à n'importe quelle situation.\\
\clearpage
La structure verticale des lacs modélisée par FLake est composée de deux couches comme illustré sur la figure \ref{flake}. La première représente la couche superficielle de mélange, aussi appelée épilimnion, directement influencée par les échanges de surface. La deuxième couche correspond à la thermocline s'étendant jusqu'au fond du lac. Cette division limite en partie l'utilisation de FLake à des lacs assez peu profonds \footnote{La profondeur maximale prescrite dans FLake est de l'ordre de 50-60 m.} \citep[]{lemoigne2016} même si les résultats sur des lacs profonds comme les Grands Lacs Africains montrent des performances acceptables \citep{thiery2015}.\\

\begin{figure}[h!]
  \centering
  \includegraphics[width=1.0\textwidth]{flake2}
  \caption{Présentation du profil de température et les variables pronostiques d'un lac de profondeur $h_{b}$ selon le modèle FLake. Adapté de \citet{mironov2008}.}
  \label{flake}
\end{figure}

\noindent Pour la résolution du bilan d'énergie, FLake considère cinq variables pronostiques: $h_{ML}$ la profondeur de la couche de mélange déterminée en prenant en compte les conditions de mélange décrits par les efforts mécaniques et convectifs (m). $T_{b}$, $T_{s}$ et $T_{MW}$ respectivement la température du fond du lac, la température de surface et la température moyenne de la colonne d'eau (K). $C_{T}$ un coefficient, appelée facteur de forme, décrivant le profil de température au sein de la thermocline (\ref{flake}). En plus de ces variables pronostiques, le modèle possède deux paramètres prescrits par défaut ou ajustables selon des observations: la profondeur du lac (m) et le coefficient d'extinction (m$^{-1}$). \\

\noindent Dans FLake, la profondeur du lac est supposée constante et les simulations ne considèrent pas de modifications du niveau d'eau. Le coefficient d'extinction, relié à la transparence du lac est généralement issu d'observations (mesuré par un disque de Secchi) et informe sur la capacité du rayonnement solaire à pénétrer dans le lac (voir Eq. \ref{eq:beer_lambert}).\\
En plus de résoudre le bilan d'énergie dans le lac, FLake modélise en option la structure thermique de la couverture en glace et neigeuse au dessus du lac et les échanges thermiques avec une couche de sédiments au fond du lac. Cette dernière option n'a pas été activée dans le cadre de cette thèse.

\subsection{{\fontfamily{lmss}\selectfont \'Evolution du profil de température dans la colonne d'eau}}
Dans le cas où la surface du lac est libre de glace, le température de la couche superficielle de mélange $T_{ML}$ est uniforme, égale à la température de surface $T_{s}$.\\
La température de la thermocline, dépendante de la profondeur, est paramétrée sur la base du concept d'auto-similarité proposé par \citet{kitaigorodskii1970}. Ce principe assure une conservation du profil de température et de ses caractéristiques sur toute l'épaisseur de la couche. \\
En d'autres termes, cela signifie qu'à une profondeur $z$ fixe au sein de la thermocline, la température est seulement dépendante de la fonction $\phi_{T}$. Cette fonction, aussi appelée fonction de forme, prescrit l'allure générale que suit le profil de température dans la thermocline. 
À une profondeur $z$ donnée et au pas de temps $t$, la température à l'intérieur du lac est définie par la fonction de forme $\phi_{T}$ définie comme:

\begin{equation}
\phi_{T}(\zeta) = \frac{T_{s}(t) - T(z,t)}{T_{s}(t)-T_{b}(t)}
\end{equation}
Les conditions aux limites auxquelles cette fonction doit satisfaire sont $\phi_{T}(0)=0$ et $\phi_{T}(1)=1$. \\

\noindent La fonction de forme dépend de la profondeur adimensionnelle $\zeta$ définie par:

\begin{equation}
\zeta = \frac{z-h_{ML}(t)}{h_{b}-h_{ML}(t)}
\end{equation}

\noindent qui doit elle-même satisfaire à des conditions aux limites pour la fonction de forme induisant que cette profondeur soit bornée par $\zeta(h_{ML})=0$ et $\zeta(h_{b})=1$. 
\clearpage

\noindent Pour calculer cette fonction, une approximation empirique se basant sur un polynôme du quatrième degré à été intégré à FLake et lie la fonction de forme au profil de température dans la thermocline selon: \\

\begin{equation}
\phi_{T} (\zeta)=\left(\frac{40}{3}C_{T}-\frac{20}{3}\right) \zeta +(18-30C_{T})\zeta^{2}+(20C_{T}-12)\zeta^{3}+\left(\frac{5}{3}-\frac{10}{3}C_{T}\right)\zeta^{4} 
\end{equation}

\noindent La double paramétrisation de l'évolution de la température qui résulte de cette approche se résume alors à:

\begin{align}\label{temp_flake}
T(z,t)=
\begin{cases}
 T_{s}(t) & \text{si 0 $\leq z$ $\leq h_{ML}$}\\
 T_{s}(t)-(T_{s}(t)-T_{b}(t))\Phi_{T}(\zeta )& \text{si $h_{ML} \leq$ z $\leq h_{b}$}
\end{cases}
\end{align}

\noindent L'utilisation de la fonction de forme $\Phi_{T}(\zeta )$ introduit une dépendance des équations de température à sa forme empirique approchée. Pour éviter ces approximations, la fonction de forme est remplacée dans les équations par une variable pronostique, le coefficient de forme $C_{T}$. En considérant l'Eq. \eqref{temp_flake} il est possible de lier les quatre variables pronostiques selon:

\begin{equation}
\overline{T}(z,t) = T_{s} - C_{T}(1-\frac{h_{ML}}{h_{b}})(T_{s}-T_{b})
\end{equation} 

\noindent Cette dernière équation permet de se soustraire à cette dépendance en introduisant la forme explicite du facteur de forme $C_{T}$ dans les équations de bilan d'énergie:

\begin{equation}
C_{T} = \int_{0}^{1} \Phi_{T}(\zeta) d\zeta
\end{equation}

\noindent L'évolution de ce coefficient est calculée suivant l'équation:

\begin{equation}
\frac{dC_{T}}{dt}= \text{sign}(\frac{dh}{dt})\frac{C_{max}-C_{min}}{t_{rc}}
\end{equation}
où $t_{rc}$ représente un coefficient de relaxation proportionnel au carré de l'épaisseur de la thermocline (s). sign est la fonction signe. $C_{max}$, $C_{min}$ sont respectivement la valeur maximale et minimale du coefficient de forme fixé à $0.8$ et $0.5$ et qui représentent les états particuliers où l'épaisseur de la couche de mélange augmente ($\dfrac{dh}{dt}>0$) ou se réduit ($\dfrac{dh}{dt}<0$).\\

\subsection{{\fontfamily{lmss}\selectfont Prise en compte des couches de neige et de glace}}

La représentation des couches de neige et glace repose sur un modèle thermodynamique à deux couches. De la même façon que le profil de température dans la thermocline, le modèle est basé sur une représentation paramétrique de la température suivant le principe d'auto-similarité.\\

\noindent L'évolution de la température dans la couche de neige et celle dans la couche de glace sont représentées par l'équation:

\begin{align}\label{temp_neige_lake}
T(z,t)=
\begin{cases}
 T_{f} - (T_{f} - T_{I}(t))\Phi_{I}(\zeta_{I}) & \text{si $-H_{I}(t) \leq z$ $\leq 0$}\\
 T_{I}(t)-(T_{I}(t)-T_{S}(t))\Phi_{S}(\zeta_{S})& \text{si $-[H_{I}+H_{S}](t) \leq$ z $\leq -H_{I}(t)$}
\end{cases}
\end{align}
où $T_{f}$ est la température de solidification de l'eau (K), $T_{I}$ est la température à l'interface neige-glace (K), $T_{S}$ est la température à l'interface neige-atmosphère (K), $H_{I}$ est l'épaisseur de la couche de glace (m) et $H_{S}$ est l'épaisseur de la couche de neige (m).\\

\noindent Le cumul de neige est une variable temporelle donnée par les forçages atmosphériques ou les observations. L'évolution associée de la couche de neige est calculée, sans fonte, selon:

\begin{equation}
\dfrac{d\rho_{S}H_{s}}{dt} = \left(\dfrac{dM_{S}}{dt}\right)_{a}
\end{equation}
avec $\rho_{S}$ la masse volumique de la neige (kg.m$^{-3}$), $M_{S}$ la masse de neige par unité de surface (kg).\\

L'épaississement de la couche de glace est initiée lorsque la température de la glace est inférieure à la température de solidification. Cette augmentation est corrélée à un dégagement de chaleur à la limite basse de la couche.\\
Les processus de dégel et de fonte sont aussi représentés dans le modèle et sont associés à des échanges de chaleurs contrôlés par l'équation de transfert de chaleur intégrée sur les couches considérées. Pour plus de détails, le lecteur se réfèrera au rapport technique de \citet{mironov2008}.\\

Comme présentée sur la figure \ref{flake} au début de cette section, la présence de glace et de neige modifie le profil de température de la colonne d'eau qui se trouve au-dessous. Cependant les processus thermodynamiques associés sont variés et complexes. C'est pourquoi, dans FLake, le profil de température dans la colonne d'eau reste inchangé lors des périodes de gel. Dans ce cas, la température à l'interface eau-glace est fixée à la température de solidification $T_{f}$.\\
Dans le cas où la température du fond $T_{b}$ est inférieure à la température de densité maximale, la couche de mélange et le facteur de forme restent inchangés, soit $\dfrac{dh}{dt}=0$ et $\dfrac{dC_{T}}{dt}=0$.\\
Si toute la couche est mélangée au moment du gel ($T_{S}=T_{b}=T_{MW}$) alors la couche de mélange est nulle ($h=0$) et le facteur de forme prend sa valeur minimale ($C_{t}=0.5$).

\subsection{{\fontfamily{lmss}\selectfont Température de peau}}

Plus récemment, \citet{lemoigne2016} a introduit un calcul de la température de peau dans le modèle FLake. Cette paramétrisation assure une meilleure estimation de la température de surface représentative du bilan d'énergie spécifique à l'interface lac-atmosphère. Cette température $\overline{T}_{0}$ est calculée sur une couche de surface constante, fixée à 1mm, telle que:

\begin{equation}
\overline{T}_{0} = \overline{T}_{-h} + \dfrac{h}{\lambda_{\omega}}(L^{*}+S^{*}-QH-QE)-\dfrac{1-\alpha_{\omega}}{h \: \lambda_{\omega}}I_{0}(1-e^{-kh})
\end{equation}
avec $I_{0}$ le forçage radiatif à la surface du lac (K.m.s$^{-1}$), $\overline{T}_{-h}$ la température de surface sans effet de peau (K), $\lambda_{\omega}$ la conductivité thermique de l'eau (W.m$^{-1}$.K$^{-1}$), $\alpha_{\omega}$ l'albédo, $L^{*}$ le rayonnement infrarouge net (W.m$^{-2}$), $S^{*}$ le rayonnement solaire net (W.m$^{-2}$). $QH$ et $QE$ respectivement les flux de chaleur latente et sensible (W.m$^{-2}$).\\

\section{{\fontfamily{lmss}\selectfont Le modèle de bilan de masse: MLake}}
\label{sec:MLake}

Avant de détailler le processus de bilan de masse des lacs et les équations mises en jeu, il est important de noter que le réseau CTRIP ne possède, initialement, aucune information sur les lacs et leurs connectivités avec les rivières. Au sein du maillage au 1/12°, toutes les cellules sont occupées par un unique tronçon de rivière décrit par des caractéristiques et une dynamique propre comme détaillées dans la section \ref{sec:CTRIP}. \\
\textbf{Cette thèse vise à développer un modèle de bilan de masse des lacs, MLake, pour ensuite l'intégrer, à l'échelle globale, dans le réseau de rivière CTRIP}. L'introduction des lacs dans CTRIP doit, par conséquent, satisfaire aux contraintes du réseau de routage: l'échelle de travail et le degré de complexité. L'échelle globale régit le cadre d'étude et nous oblige à développer un modèle à la fois simple, c'est-à-dire avec l'introduction d'un nombre de paramètres limités, sans pour autant sacrifier le réalisme physique et la structure du réseau hydrographique. \\
Le principe général du développement de MLake s'organise d'abord autour de la création d'une carte de lac globale identifiant les lacs uniques dans ECOCLIMAP-II (section \ref{sec:masque_lac}). À l'aide de cette carte, il est possible d'intégrer les lacs, par le biais d'un masque spécifique, dans le réseau spatialisé à 1/12° (\ref{sec:intégration}) et de corriger l'ordre des séquences dans le chevelu (section \ref{sec:chevelu}). La création d'un second masque pour la gestion des forçages est aussi utile pour s'assurer d'un partage correct du ruissellement et du drainage entre les rivières et les lacs (section \ref{sec:part_forcage}). Toute ces étapes sont primordiales dans la mise en place du schéma numérique de résolution du bilan de masse des lacs (section \ref{sec:routines}).

\subsection{{\fontfamily{lmss}\selectfont Création d'un masque de lac global}}
\label{sec:masque_lac}
D'un point de vue technique, le couvert de lac dans ECOCLIMAP-II informe de façon binaire sur la présence ou non d'un lac dans chaque cellule du maillage au 1/120°. Cette donnée est couplée à la base de données GLDB pour prescrire une profondeur moyenne sur ces mêmes cellules. Pour autant, ces données n'indiquent pas si deux cellules voisines identifiées comme lacs font partie d'un même lac. Autrement dit, l'association d'ECOCLIMAP-II avec GLDB ne permet pas d'extraire un masque de lacs nécessaire pour déterminer l'emprise en surface des lacs dans le maillage global. \\
Pour y remédier, la première étape a pour but d'agréger l'information sur la présence d'un lac au 1/120° dans ECOCLIMAP-II afin de construire une carte globale, ECOCLIMAP-agrégée, prescrivant les caractéristiques morphologiques, principalement la profondeur moyenne et l'aire de surface, à chaque lac identifié par un unique numéro et nécessaires à la résolution du bilan de masse et le suivi de la dynamique lacustre.\\
La méthode retenue consiste en une agrégation récursive des cellules identifiées comme lacs dans ECOCLIMAP-II sur la base d'une égalité entre les profondeurs moyennes prescrites dans GLDB. La figure \ref{recursion} présente une schématisation de la méthode employée.\\

\noindent L'algorithme parcourt chaque cellule au 1/120° et s'informe dans un premier temps sur la présence d'un lac dans celle-ci. L'algorithme parcourt la grille en débutant à la cellule \textbf{A1}. Dans le cas où la cellule contient un lac, la récursion est initiée sur une branche issue de cette cellule (soit \textbf{B1} soit \textbf{A2}). Chaque cellule voisine est donc interrogée sur la présence d'une cellule lac et, si oui, sur l'égalité des profondeurs moyennes de ces cellules. Tant que l'algorithme récupère une réponse positive, il parcourt la branche (ici la branche issue de \textbf{B1}). Si la condition d'arrêt est atteinte (cas où toutes les cellules voisines renvoient une réponse négative, ici \textbf{D2} et \textbf{C3}) alors l'algorithme remonte l'arborescence jusqu'à la dernière branche non exploitée (ici \textbf{B3}) et cela jusqu'à remonter vers le point de départ (\textbf{A1}) et exploiter une autre branche non exploitée (\textbf{A2}). Si toute l'arborescence est interrogée et renvoie une réponse négative (branche \textbf{B1} et \textbf{A2}), l'algorithme cherche la première cellule non exploitée dans la maille (\textbf{E1}). Chaque branche contenant un lac est identifiée par un unique numéro qui assure une distinction entre les lacs. Pour chacun des identifiants, la profondeur moyenne et l'aire de surface sont attribuées. La profondeur moyenne correspond à la profondeur utilisée pour vérifier la condition d'égalité, tandis que l'aire de surface correspond à la somme des aires de chaque cellule, au 1/120°, identifiée comme appartenant au lac.\\

\begin{figure}[h!]
\centering
\includegraphics[width=1.\textwidth]{recursion}
\caption{Représentation de l'algorithme d'agrégation et présentation du parcours récursif à partir de la carte d'occupation des sols ECOCLIMAP-II et de la base de données de profondeur moyenne GLDB.}
\label{recursion}
\end{figure}
\clearpage

Cette méthode est particulièrement efficace dans le cadre de grands lacs bien identifiables dans leur environnement régional comme les Grands Lacs Américains et Africains. Dans ce cas, même s'il y a des fausses détections ou des non détections dans la carte de départ, ces erreurs restent relativement faibles par rapport à la superficie totale du lac. Elle révèle par contre ses limites sur la distinction de petits lacs dans les régions de grande densité lacustre. Les corrections faites dans la base GLDB sur les profondeurs moyennes par \citet{choulga2014} se basent sur l'étude de couches géologiques afin de distinguer des zones lacustres spécifiques auxquelles s'applique une unique valeur. En agrégeant les lacs de ces zones, il est donc possible d'identifier, comme unique, plusieurs lacs, en limite de résolution, initialement distincts. Pour ces petits lacs, l'erreur relative est donc plus importante. Cette surestimation locale peut générer des anomalies hydrologiques pour un travail à fine échelle, mais ces anomalies sont généralement filtrées lors du passage à une résolution plus faible telle qu'à notre échelle de travail (1/12°).\\
Dans le cas spécifique des lacs de petites dimensions qui ne recouvrent pas complètement une cellule de 1/120° (e.g. les lacs thermo-karstiques), ceci ne sont pas considérés dans ce travail, ce qui nécessitera des développements spécifiques à moyen terme.


\subsection{{\fontfamily{lmss}\selectfont Intégration des lacs dans la réseau 12D}}
\label{sec:intégration}

Sur la base de cette carte globale, il est, dès lors, possible d'intégrer les lacs dans le réseau à 1/12°. Le choix des hypothèses à considérer est important pour assurer une distinction cohérente entre les cellules de lac et celles de rivière. En effet, chaque composante possède une hydrologie différente et il semble difficile de justifier, dans certains cas, la prédominance d'un lac par rapport à une rivière sur les flux dans le bassin. En d'autres termes, il est important que l'intégration se justifie à notre résolution par des considérations hydrologiques.\\

Dans CTRIP, chaque cellule représente un seul bief de rivière avec toutes les caractéristiques associées. La logique voudrait que chaque cellule identifiée comme une rivière dans le réseau de rivière CTRIP mais étant un lac dans ECOCLIMAP-agrégée soit remplacée par les caractéristiques de lacs. La résolution complète du bilan de masse est paramétrée sous-maille, c'est-à-dire que les processus physiques de transfert sont résolus à l'échelle du modèle. Au 1/12°, les rivières ne sont divisées que selon leur longueur et les largeurs de ces rivières ne peuvent déborder sur plusieurs cellules. \\
Pourtant dans le cas des lacs, il est possible de caractériser leurs emprises en surface de plusieurs manières. Dans la majorité des cas, les lacs, avec des dimensions spatiales faibles, peuvent être représentés par une seule cellule, facilement intégrable au réseau. Les problèmes d'intégration apparaissent dans le cas où un lac s'étend sur plusieurs cellules. Dans ce cas, il est impossible de dissocier les variables du lac car elles représentent le lac en une unique entité. Il convient de réunir plusieurs cellules pour décrire un seul processus. L'intégration d'un bilan de masse des lacs se retrouve donc contraint par la recherche d'un compromis entre la réalité du processus physique et la complexité de l'information spatiale. \\

\begin{figure}[h!]
  \centering
  \includegraphics[width=0.75\textwidth]{huziu}
  \caption{Représentation du continuum rivière-lac dans le modèle de climat régional canadien. Source: \citet{huziy2017}.}
  \label{huziu}
\end{figure}

Pour répondre à ce problème, \citet{huziy2017} propose une intégration distincte des lacs en séparant les lacs de petites tailles des grands lacs. Pour ce faire, ils développent deux classes de lacs: les lacs locaux et globaux. Un lac est considéré comme local lorsqu'il couvre moins de 60\% d'une cellule et est considéré comme global lorsqu'il recouvre au minimum deux cellules (même partiellement) à l'échelle du modèle. Les auteurs appliquent, pour chaque classe, une dynamique lacustre différente. Les lacs locaux sont considérés comme des extensions d'un tronçon de rivière qui contribuent à son débit aval sans être alimentés par le tronçon de rivière amont. La possibilité est laissée dans le cas de lacs globaux de les intégrer totalement dans le réseau de rivières qu'ils coupent en une partie amont et une partie aval (Figure \ref{huziu}). \\

Ce type de distinction n'est pas satisfaisant dans notre cadre d'étude car il ne prend pas en compte certains cas particuliers comme celui des lacs endoréiques et limite le rôle de la majorité des lacs à servir de tampon hydrologique. \\
\noindent La méthode appliquée, dans notre cas, se base sur la création d'un \textbf{masque de réseau} (Figure \ref{masque_reseau}) prenant en compte explicitement toutes les cellules d'un même lac depuis la carte ECOCLIMAP-agrégée (Figure \ref{masque_reseau}.a) pour les localiser correctement sur les tronçons de rivières. L'exemple d'illustration proposé ici porte sur le cas du lac du Bourget en Savoie (France).\\

\begin{figure}[!h]
\centering
       \includegraphics[width=1\textwidth]{masque_reseau}
     \caption{Schéma descriptif des étapes de la construction du masque de réseau sur le bassin du lac du Bourget à 1/12°. (a) Carte ECOCLIMAP-agrégée pour le lac du Bourget. (b) Identification de la plus grande rivière (tronçon jaune) s'écoulant à travers le lac du Bourget dans le MERIT-HYDRO. (b) Identification de la rivière correspondante à 1/12° (tronçon rose). (d) Création du masque de réseau associé au lac du Bourget à 1/12° suivant le tronçon de rivière rose.}
     \label{masque_reseau}
\end{figure}

\noindent La création du réseau de rivière CTRIP à 1/12° par l'algorithme DRT est susceptible de modifier localement la structure en privilégiant les plus grands cours d'eau. Il peut donc arriver que, sur la carte spatialisée, certains tronçons de rivières soient légèrement décalés pour assurer la cohérence du réseau. Ces adaptations entraînent la possibilité pour le lac, même s'il est correctement localisé à 1/120°, de se retrouver positionné sur un tronçon de rivière incorrect à 1/12°. Pour être sûr de la localisation, le réseau de rivière haute résolution MERIT-HYDRO (à 1/1200°) est utilisé afin d'identifier le numéro de rivière s'écoulant à travers le lac en question. La description de la méthode s'appuie sur la Figure \ref{masque_reseau} décrivant l'intégration du lac du Bourget dans le réseau CTRIP à 1/12°.\\
La création du masque a été développée en remontant de façon récursive la plus grande rivière dans le MERIT-HYDRO, en considérant l'accumulation de flux \footnote{L'accumulation de flux est un nombre, affecté à chaque cellule, qui correspond à la somme de toutes les cellules amont contribuant au flux de la cellule.}, traversant le lac au 1/120° (le tronçon de rivière en jaune dans la Figure \ref{masque_reseau}.b). Cette méthode lie chaque identifiant de lac à un identifiant de rivière à haute résolution et garantit que le continuum rivière-lac dans le bassin soit respecté. En effet, lors de la construction du réseau à 1/12°, chaque tronçon de rivière conserve le numéro d'identification attribué à 1/1200° (Figure \ref{masque_reseau}.c) et il en est de même pour le lac. Dans le cas où il est nécessaire de déplacer un tronçon de rivière, le lac le sera aussi.\\ 

\noindent Une question reste cependant en suspend concernant la prévalence d'une cellule lac par rapport à une cellule rivière. Dans certains cas, la fraction réelle de lac au sein d'une cellule peut s'avérer faible et donc le remplacement d'un tronçon de rivière par ce lac peut induire une modification trop importante de l'hydrologie. Il n'est d'ailleurs pas exclu que de fausses détections apparaissent dans ECOCLIMAP-II sur des rivières larges, comme l'Amazone, ou dans des régions avec une grande densité de lac. La méthode la plus adaptée semble être d'imposer un seuil sur la taille des lacs à partir duquel il est plus cohérent d'inclure un lac plutôt qu'une rivière dans le réseau. Comme ce type d'information est peu voire non existante à l'échelle globale, le choix arbitraire de ne considérer que les cellules dont le recouvrement par les lacs était d'au moins 50\% de l'aire d'une cellule CTRIP au 1/12° a donc été fait. \\
Ce choix permet notamment de filtrer les lacs trop petits qui, étant donnée leur réponse relativement rapide à l'échelle de notre étude, ne perturbent que peu la dynamique hydrologique. \\

Une autre question émerge ici quant à la cohérence hydrologique d'un tel masque avec des conséquences sur le comportement hydrologique du bassin. Dans l'exemple du lac du Bourget à 1/12°, le masque de réseau associé recouvre partiellement deux cellules. Une première cellule correspondant à la partie Nord du lac et une deuxième à la partie Sud (Figure \ref{masque_reseau}.d). Le lac du Bourget fait partie du bassin versant s'écoulant dans le Rhône par le biais du canal de Savières c'est-à-dire par sa partie Sud à 1/12°. La partie Nord, quant à elle, fait partie du bassin de la rivière Chéran qui se jette dans le Rhône plus en amont. À notre résolution, ces deux cellules ne contribuent donc pas aux mêmes écoulements avec la partie Nord et la partie Sud qui transfèrent de la masse vers deux bassins différents. Il y a ici l'apparition d'une incohérence induite par le choix d'échelle et les conflits entre des informations sous-maille. Pour pallier à ces problèmes, qui peuvent arriver sur d'autre lacs, le masque est lui-même corrigé par le biais d'un algorithme récursif remontant le tronçon de rivière identifié précédemment à 1/12° (tronçon rose sur la figure \ref{masque_reseau}.d). Depuis l'exutoire, l'algorithme identifie les cellules du masque de lac qui sont effectivement traversé par le tronçon de rivière. Le résultat pour le lac du Bourget est présenté sur la figure \ref{correction_reseau}. Le fait d'avoir un tel masque de réseau permet de dissocier ces deux cellules et de ne prendre en compte dans le bilan de masse que la partie Sud, cellule placée effectivement dans le bassin versant du lac. Ainsi même si une fraction du lac se trouve sur la partie Nord, le bilan de masse sur cette cellule restera celui d'une rivière.

\begin{figure}[h!]
\centering
       \includegraphics[width=0.65\textwidth]{correction_reseau}
       \caption{Masque de réseau pour le lac du Bourget et réseau CTRIP à 1/12° avant (a) et après correction (b) des incohérences hydrologiques.}
     \label{correction_reseau}
\end{figure}

~\\
~\\
~\\

\subsection{{\fontfamily{lmss}\selectfont Correction du chevelu hydrologique }}
\label{sec:chevelu}

Le travail d'intégration proposé dans cette thèse se base aussi sur la représentation du routage intrinsèque de CTRIP et notamment le travail sous forme de "nœud" de rivière. Comme illustré sur la figure \ref{algo_trip}, le réseau de rivières, en plus d'une représentation spatiale, est modélisé sous la forme d'une arborescence\footnote{\textit{tree representation}, en anglais.}. Ainsi tous les centres des cellules du réseau CTRIP sont associés à un nœud dans le réseau où est effective la résolution du bilan de masse. Ces nœuds sont ensuite reliés entre eux par des tronçons de rivière sous la condition qu'un nœud peut recevoir de la masse de plusieurs affluents mais ne peut la transférer qu'à un unique nœud aval. Dans la logique de CTRIP, chaque nœud tient compte des informations de la totalité de la cellule et notamment des données géomorphologiques et topographiques. Cette représentation en cascade assure un transfert amont-aval basé sur le numéro de séquence attribué à chaque nœud et présenté à la section \ref{sec:ctrip_caracteristique}. \\

\begin{figure}[h!]
\centering
       \includegraphics[width=1.\textwidth]{trip_network}
       \caption{Représentation du concept de réseau de routage tel qu'utilisé dans CTRIP au 1/12°. a) Réseau CTRIP sans lacs, b) Réseau CTRIP avec lacs.}
     \label{algo_trip}
\end{figure}

Dans le cas du modèle CTRIP sans MLake, le séquençage résout le bilan de masse  de l'amont du réseau ("headwater cell") vers l'exutoire ("outlet"). Le numéro de séquence minimal est attribué aux cellules les plus en amont puis est incrémenté à chaque pixel aval. Une incrémentation spécifique est définie au niveau d'une confluence suivant l'Eq. \ref{eq:SN}. Chaque nœud possède donc les paramètres de la rivière correspondant à sa localisation dans le chevelu comme définie dans la section \ref{sec:ctrip_caracteristique}.

\noindent Un schéma d'attribution du numéro de séquence similaire a été utilisé dans le cas des lacs. La spécificité repose sur l'attribution d'un unique nœud lac à la totalité des cellules recouvertes par un lac dans le masque de réseau.
Prenons l'exemple du chevelu représenté sur la figure \ref{algo_trip} et contenant un lac. Dans ce cas, toutes les cellules qui ont été précédemment identifiées comme des tronçons de rivières (ici \textbf{B-IV} et \textbf{C-V}) sont remplacées par un nœud unique regroupant la somme des cellules du lac sur le masque de réseau. L'introduction des lacs dans le réseau ne supprime pas les paramètres de rivières existants sur ces cellules et ne font que rajouter en surcouche les caractéristiques associées aux lacs. Cela évite la reconstruction du réseau dans le cas où MLake serait désactivé. \\

\subsection{{\fontfamily{lmss}\selectfont Gestion du partage des forçages }}
\label{sec:part_forcage}

Les problèmes d'intégration n'apparaissent pas seulement sur le comportement hydrologique dans le bassin mais aussi au niveau des processus comme l'interception des précipitations, le ruissellement ou les échanges de sub-surface qui sont spécifiques à chacune des cellules recouvertes. À l'inverse des caractéristiques morphologiques du lac, ces processus ne peuvent pas être spatialisés sur la totalité du lac.\\

L'approche sous forme de dualité rivière/lac amène des interrogations sur l'attribution des variables hydrologiques et le calcul des différentes composantes du bilan. Même si le lac est représenté par un seul nœud, il reste que son emprise spatiale modifie le partage des flux de masse au sein des cellules où il est présent. Pour détailler ce point, revenons au cas du lac du Bourget décrit sur la figure \ref{bourget}. 

\begin{figure}[h!]
\centering
  \includegraphics[width=0.65\textwidth]{masque_ruiss}  
  \caption{Représentation du masque réseau (a) et du masque de ruissellement (b) pour le lac du Bourget à 1/12° dans le réseau CTRIP.}
  \label{bourget}
\end{figure} 

\noindent Comme vu précédemment, à 1/12°, le lac du Bourget recouvre partiellement deux cellules mais seule la partie Sud, localisée dans le bassin versant, est considérée pour la résolution du bilan de masse.\\
Pourtant le lac récupère du ruissellement et intercepte des précipitations sur les deux cellules. Le bilan de masse est donc modifié par les contributions des deux cellules. Dans ce cas de figure, des problèmes de conservation peuvent apparaître et le masque de réseau n'est pas suffisant pour quantifier correctement l'interaction du lac avec de composantes du bilan comme les précipitations ou le ruissellement. Dans les faits, le lac intercepte une fraction des précipitations sur chaque cellule même si le volume résultant du bilan de masse sur ce lac participe seulement au débit du canal de Savières. \\

Il a donc semblé judicieux de créer un deuxième masque de lacs, le \textbf{masque de ruissellement} (Figure \ref{bourget}.b), spécifiquement pour le calcul des différentes variables hydrologiques prises en compte dans le bilan. Ce masque est complémentaire du masque de réseau et garantit la cohérence de l'hydrologie locale. \\

Ce masque est créé sur la base des informations à 1/120° fournies par ECOCLIMAP-agrégée. Pour chaque identifiant de lac, le masque de ruissellement est construit par interpolation de la donnée à 1/120° sur la grille 1/12°. Toutes les cellules à 1/12° contenant au moins une cellule de lac dans ECOCLIMAP-II (1/120°) sont alors considérées dans le réseau de ruissellement comme un lac, sans considération de bassin versant. Ce masque sert notamment pour le calcul de la masse d'eau interceptée et stockée par le lac en fonction de la fraction de cellule recouverte. Il doit, par contre, conserver la cohérence du réseau et le fait que certaines cellules rivière soient remplacées par des lacs, ce qui impose que la fraction de lac sur les cellules du \textbf{masque de réseau} soit ramenée à 1.\\

\subsection{{\fontfamily{lmss}\selectfont Processus physiques}}
\label{sec:routines}

Les deux étapes précédentes ont permis la construction du réseau de routage et l'initialisation des champs physiographiques et morphologiques des différentes entités. À partir de ces données, tous les éléments sont réunis pour effectuer la résolution numérique. La gestion du transfert d'eau dans le réseau est assurée par une équation du bilan de masse appliquée à l'ensemble de la surface du lac, supposée homogène. Le modèle CTRIP-MLake résout alors un bilan distinct pour les rivières et les lacs. Dans un souci de simplicité et afin d'accroître la flexibilité du modèle face à la diversité des dynamiques lacustres, le bilan de masse se base sur la résolution d'une équation dont la variable pronostique est le stock $V_{lake}$ (Figure \ref{masslake}):

\begin{equation}
\frac{dV_{lake}}{dt} = P_{ol}-E_{ol} +  R_{S}+ D +  Q_{in} - Q_{out} - Q_{gw} - Q_{p}
\end{equation}
$P_{ol}$ correspond aux précipitations directement interceptées par le lac (kg.m$^{-2}$.s$^{-1}$). $E_{ol}$ est l'évaporation estimée directement au-dessus du lac (kg.m$^{-2}$.s$^{-1}$). $Q_{gw}$ représente les flux de masse échangés avec les aquifères qui ne sont pas pris en compte dans cette thèse (kg.m$^{-2}$.s$^{-1}$). $Q_{p}$ représente les flux de masse issus de prélèvements anthropiques qui ne sont pas pris en compte dans cette thèse (kg.m$^{-2}$.s$^{-1}$).
Le ruissellement de surface $R_{S}$ (kg.m$^{-2}$.s$^{-1}$) et le drainage $D$ (kg.m$^{-2}$.s$^{-1}$) représentent la contribution des berges des lacs aux flux entrants.\\

\begin{figure}[h!]
  \includegraphics[width=1.\textwidth]{lac}
  \caption{Schéma des processus impliqués dans le bilan de masse d'un lac.}
  \label{masslake}
\end{figure}

\noindent L'initialisation du stock est effectuée par le biais des informations fournies par ECOCLIMAP-agrégée et GLDB. Le premier masque donne l'aire de surface pour chaque lac $A_{ECO}$ ($m^{2}$) et le deuxième informe sur la profondeur moyenne du lac $z_{moy,GLDB}$ ($m$). Le stock initial est ainsi déduit de la relation:
\begin{equation}
V_{lake,0} = A_{ECO}.z_{moy,GLDB}
\end{equation}

\noindent En ce qui concerne l'évaluation de la partie atmosphérique du bilan de masse( définit par le couple {précipitations, évaporation}), celui-ci est déterminé en calculant les précipitations interceptées sur le masque de ruissellement du lac $P_{ol}$ auquel est retranché l'estimation d'évaporation provenant d'une simulation FLake $E_{ol}$. \'A ce stade, il n'y a pas de rétroaction de la dynamique du lac sur l'évaporation, rétroaction qui sera introduite à terme par le couplage entre CTRIP-MLake et SURFEX. 

\noindent Comme vu dans la partie \ref{sec:part_forcage}, le ruissellement et le drainage sont calculés pour chaque lac en cohérence avec le masque de ruissellement tel que:

\begin{align}\label{mlake_rd}
\begin{cases}
 R_{S} = max(0, \sum_{p} r_{S}(p))\\
 D = max(0, \sum_{p} d_{S}(p))
\end{cases}
\end{align}
$r_{S}$ et $d_{S}$ sont respectivement le ruissellement de surface et le drainage sur chaque cellule du masque (kg.m$^{-2}$.s$^{-1}$). $p$ correspond aux cellules de lacs sur le masque de ruissellement. \\

\noindent De façon similaire la contribution des affluents aux flux entrants $Q_{in}$ est calculée en prenant la somme des débits provenant des cellules rivières amont sur le masque de réseau, tel que: 
\clearpage
\begin{align}\label{mlake_qin}
 Q_{in} = \sum^{l}_{k} q_{in}(k)
\end{align}
avec $q_{in}$ (kg.m$^{-2}$.s$^{-1}$) la contribution de l'affluent $k$ et $l$ le nombre d'affluents s'écoulant dans le masque de réseau du lac. \\

\noindent Le calcul des flux entrants détermine un état intermédiaire du lac défini par un volume $V_{lake}^{*}$ (kg) à l'instant $t$ tel que:

\begin{equation}
V^{*}_{lake}(t) = V_{lake}(t-\Delta t) + (P_{ol}(t) - E_{ol}(t) +  R_{S}(t)+ D(t) +  Q_{in}(t))\Delta t
\end{equation}
où $V(t-\Delta t)$ est le volume du lac au pas de temps précèdent.\\

\noindent En supposant que l'aire du lac reste constante quelle que soit la hauteur du lac \footnote{les lacs sont représentés sous forme prismatique.}, il est possible de définir une hauteur intermédiaire $h_{lake}^{*}$ (m) définie comme:

\begin{equation}
h_{lake}^{*}(t) = \frac {V_{lake}^{*}(t)}{A_{ECO}}
\end{equation}

\noindent Le débit s'écoulant hors du lac $Q_{out}$ est calculé sur la base d'une analogie avec un déversement de seuil qui lie la charge en eau au dessus d'un seuil au débit s'écoulant à travers ce seuil. Cette analogie est satisfaisante pour représenter la dynamique d'écoulement de l'eau s'écoulant principalement au-dessus de la contre-pente du lac. À 1/12°, les exutoires de lacs sont suffisamment étroits pour être considérés comme droit, de plus les effets de frottements sont négligeables et les lignes de courant rectilignes.\\

\begin{figure}[h!]
     \centering
     \subfloat[Cas d'un lac sans déversement\label{sub_qout1}]{%
       \includegraphics[width=0.49\textwidth]{q_ovfl_0}
     }
     \hfill
     \subfloat[Cas d'un lac avec déversement\label{subquout2}]{%
       \includegraphics[width=0.49\textwidth]{q_ovfl}
     }
     \hfill
     \caption{Schéma de déversement d'un lac. Les figures du haut représentent une vue de face. Les figures du bas représentent une vue transversale.}
     \label{qovfl}
\end{figure}

\noindent La charge en eau au-dessus du seuil est représentée par la hauteur relative, fruit de la comparaison entre $h_{lake}^{*}$ et la hauteur du seuil rectangulaire $h_{weir}$ (Figure \ref{qovfl}):

\begin{align}\label{q_out}
Q_{out}=
\begin{cases}
 0 & \text{si $h_{lake}^{*}<h_{weir}$ }\\
 C_{d} \rho_{\omega} \sqrt{2g}W_{weir}(h_{lake}^{*}-h_{weir})^{\frac{3}{2}}& \text{si $h_{lake}^{*}>h_{weir}$}
\end{cases}
\end{align}
$C_{d}$ est le coefficient de traînée adimensionnel associée au seuil égal à $0.485$ \citep{lencastre1963}, $W_{weir}$ est la largeur du seuil égale à la largeur de la rivière dans le réseau CTRIP au niveau de l'exutoire du lac (m). \\

Comme il n'existe aucune information globale sur la hauteur de la contre-pente ou le rapport entre profondeur du bassin lacustre et niveau d'eau, il a été décidé d'initialiser la hauteur du seuil au niveau initial du lac $z_{moy, GLDB}$. Le débit à l'exutoire devient donc seulement dépendant de la charge en eau au-dessus du seuil qui est atteinte après une phase de spin-up aboutissant à un régime d'équilibre pour le niveau du lac. Dans ce cas, la seule limitation est que le diagnostic sur le niveau du lac est construit en hauteur relative par rapport au niveau du lac et limite la comparaison directe avec des mesures absolues (comme les données d'altitude). \\

\noindent Le volume final du lac pour le pas de temps se résume donc à:

\begin{equation}
V_{lake}(t)=V^{*}_{lake}(t) - Q_{out}(t)\Delta t
\end{equation}

\subsubsection{{\fontfamily{lmss}\selectfont Organisation générale et paramètres introduits dans CTRIP}}
\begin{figure}
  \includegraphics[width=1.\textwidth]{code_overview.pdf}
  \caption{Organisation générale de la routine associée au module MLake dans la structure de CTRIP en mode offline.}
  \label{codeoverview}
\end{figure}
\noindent L'organisation générale du code de MLake est représentée sur la figure \ref{codeoverview}. L'introduction de MLake dans le réseau de routage ajoute sept paramètres et une variable dans le modèle sans engendrer de complexité supplémentaire. En plus des variables et paramètres prescrits, le modèle produit trois variables diagnostiques utiles à sa validation et plus généralement à l'étude de la dynamique lacustre: $lake\_in$ la somme des débits entrants dans le lac ($kg.m^{2}.s^{-1}$), $lake\_out$ le débit produit par déversement à l'exutoire du lac ($kg.m^{2}.s^{-1}$) et $lake\_h$ la variation relative de hauteurs du lac ($m$).\\
Le pas de temps de calcul choisis dans le cadre de ces travaux est un pas journalier.
\clearpage

\section{{\fontfamily{lmss}\selectfont Conclusion}}

La modélisation hydrologique à l'échelle régionale et globale développée au CNRM repose sur l'utilisation couplée du modèle ISBA, résolvant les bilans d'énergie et d'eau pour l'estimation de la production de ruissellement et de drainage, et du modèle de routage en rivière CTRIP, assurant le transfert d'eau de l'amont à l'aval des bassins versants. Ces modèles assurent la fermeture du bilan hydrologique global et sont aujourd'hui intégrés au modèle de Météo-France CNRM-CM utilisé pour étudier le climat et son évolution. Pour avoir une vision complète du cycle de l'eau et corriger les flux simulés des grands bassins fluviaux, il est nécessaire de prendre en compte la dynamique des masses d'eau lacustres en plus de leur bilan énergétique permis par l'introduction du modèle FLake dans la plateforme de modélisation de surface SURFEX. \\

À ce jour, CTRIP possède une résolution globale de 1/12° où chaque maille représente un unique tronçon de rivière dont la localisation provient de l'utilisation d'un modèle numérique de terrain à haute résolution. Cependant cette représentation occulte la dynamique hydrologique des régions où la densité de lac est importante. Le but de cette thèse est donc de corriger le réseau de rivière global en introduisant une paramétrisation des échanges en eau dans le continuum rivière-lac. Cela est permis par le couplage du modèle de bilan de masse de lac MLake à CTRIP. \\
Dans un premier temps, le réseau a été corrigé par l'introduction d'un masque de lac agrégé à l'échelle globale comptant pour tous les lacs issus de la carte ECOCLIMAP-II à 1 km de résolution. Cette carte rend compte des disparités entre les régions sur la disponibilité de la ressource en eau. En se basant sur cette carte, la paramétrisation introduite dans CTRIP assure une représentation plus réaliste des variations de masse et permet la prescription des conditions de stocks d'eau à l'échelle globale. À terme, le couplage de CTRIP-MLake avec le modèle de climat assurera le suivi de la ressource en eau et de ses variations régionales pour la prévision hydrologique globale. Les lacs étant des sentinelles du changement climatique, la prise en compte de leurs niveaux d'eau en tant que variables climatiques essentielles aide à la détection des zones de stress hydrique notamment en réponse à des modifications environnementales induites par le changement climatique. Cette modélisation couplée est justifiée par la dépendance de chaque variable hydrologique. Par exemple, une modification de la température de la colonne d'eau influence les taux d'évaporation potentielle qui agissent en retour sur les niveaux d'eau et donc l'emprise de surface. Ces modifications restent cependant relatives et spatialement inégales. Entre 1984 et 2015, 90 000 km$^{2}$ de surface en eau permanente ont disparu à travers le monde quand 180 000 km$^{2}$ sont apparus dans d'autres régions du monde \citep{pekel2016}. \\

\noindent Les deux chapitres qui suivent ont pour but d'évaluer et de valider cette nouvelle paramétrisation notamment par rapport aux débits produits et aux variations de hauteurs de lacs mais aussi en analysant la sensibilité du modèle à la largeur de l'exutoire des lacs. Ces évaluations se feront tout d'abord à l'échelle du bassin du Rhône grâce à l'utilisation de la chaîne SAFRAN-ISBA-MODCOU. Puis une évaluation à l'échelle globale sera détaillée sur trois bassins versants identifiés par des caractéristiques hydrologiques et climatiques différentes sur la base de forçages atmosphériques utilisés dans les études climatiques globales.

\cleardoublepage
\chapter{{\fontfamily{lmss}\selectfont \'Evaluation et validation locale: Le bassin versant du Rhône}}
\label{chap:etude-locale}
\minitoc

Les étapes d'évaluation et de validation sont incontournables dans le développement d'un modèle afin de quantifier les modifications induites par son intégration et de vérifier son bon fonctionnement. Pour cela, il convient de comparer les données simulées avant et après introduction de la nouvelle physique pour ensuite les évaluer sur la base d'observations locales. Les observations, et notamment les forçages atmosphériques, sont souvent de meilleure qualité que des données globales et permettent donc une comparaison précise. De plus, restreindre l'étude à une zone réduite engendre des coûts de calculs moindres et facilite la multiplication des configurations à tester. \\
La zone d'étude choisie dans cette thèse est le bassin versant du Rhône. Au vu de l'importance hydrologique du Léman dans le bassin versant et de l'existence de jeux de données climatiques et hydrologiques vastes et précis, le bassin rhodanien est une zone propice à l'évaluation et la validation du modèle.\\
~\\

\noindent Quatre stations de jaugeage sur le linéaire du Rhône ont été sélectionnées pour caractériser l'effet des lacs sur chaque région hydrographique du bassin\footnote{Partie d'un bassin hydrographique désignée comme le premier ordre de découpage hydrographique français. Il existe en France 24 régions hydrographiques dont quatre sur le bassin versant du Rhône.}. Profitant des forçages issus de la chaîne hydrométéorologique opérationelle SAFRAN-ISBA-MODCOU, il est possible de directement forcer CTRIP pour vérifier la cohérence des débits simulés avant et après l'introduction des lacs dans le modèle tout en réduisant sa sensibilité à la variabilité spatiale des forçages. En plus d'évaluer l'intérêt de MLake pour la simulation hydrologique, la validation se focalise aussi sur les performances de MLake à modéliser les débits du Rhône et les variations de niveau du Léman par rapport à des observations.

\section{{\fontfamily{lmss}\selectfont Le bassin versant du Rhône}}
\label{sec:bv-rhone}

\subsection*{Morphologie du bassin}

Long de 812 km, le Rhône assure un lien privilégié entre les glaciers alpins et la mer Méditerranée. Divisé entre la Suisse et la France, le Rhône draine une surface de 98 000 km$^{2}$ dont plus de 90 \% se trouve sur le territoire français (Figure \ref{bv_rhone_dem}). Cette frontière géographique est aussi une frontière morphologique qui divise le Rhône entre sa partie amont et sa partie aval. La partie amont correspond à la zone située entre la source du Rhône au glacier de Furka et l'exutoire du Léman à Genève. La partie aval, quant à elle, débute au sortir de Genève pour s'écouler et rejoindre la mer Méditerranée par son delta. Malgré cette distinction, le Rhône reste un fleuve alpin majoritairement influencé par les massifs montagneux qui l'alimentent puisque 50 \% de son bassin se situe au-dessus de 500 m (asl) et 15 \% au-dessus de 1500 m (asl). 

\begin{figure}[h!]
\centering
\includegraphics[scale=0.65]{bv_rhone_dem}
\caption{Bassin versant du Rhône et sa topographie.}
\label{bv_rhone_dem}
\end{figure}

\subsection*{Climat}
D'un point de vue climatique, le bassin versant du Rhône est particulièrement intéressant  car il peut être divisé en quatre zones climatiques dont les caractéristiques sont regroupées dans le tableau \ref{tab_rhone}\footnote{Pour plus d'informations sur la classification climatologique de Köppen-Geiger, le lecteur se tournera vers \citep{beck2017}}. La tête du bassin est soumise à un climat tempéré humide avec des cumuls de précipitations assez importants, principalement sous forme de neige. L'est et le nord du bassin sont influencés par un climat continental avec des hivers froids. Enfin la partie sud est influencée par le climat méditerranéen caractérisé par des étés secs et chauds, des cumuls annuels de précipitations faibles au contraire des intensités de précipitations saisonnières qui peuvent atteindre des valeurs extrêmes (plus de 300 mm en 12h).

\begin{table}[h!]
 \caption{Principales caractéristiques du bassin versant du Rhône avant son delta.}
 \label{tab_rhone}
 \begin{tabularx}{\textwidth}{cXXXX}
 \hline
 & Rhône alpestre & Haut Rhône français & Rhône moyen & Rhône inférieur\\
 \hline
  Altitude moyenne ($m$)&1655&&699&\\
  Surface drainée ($km^{2}$)&8000&12300&46150&29150\\
  Cumul annuel de précipitations ($mm$)&1000&900&890&695\\
  Débit moyen annuel ($m^{3}.s^{-1}$)&335&600&1400&1700\\ 
  Classification Köppen-Geiger &ET/Cfb&Cfb&Cfa&Csa\\
  \hline
 \end{tabularx}
\end{table}

\subsection*{Hydrologie}
\label{sec:hydrologie}

Avec un module aux alentours de 1700 $m^{3}.s^{-1}$, le Rhône est le plus puissant des fleuves français. La variété des climats sur son bassin lui confère aussi un régime hydrologique complexe caractérisé par un déphasage entre la tête et l'aval du bassin (Figure \ref{debit_r2d2}). La contribution de ses principaux affluents (Ain, Saône, Isère, Durance) est conséquente et compte pour 55\% du débit du Rhône à l'exutoire \citep{edl2019}. De plus, ces apports se classent selon trois composantes hydrologiques: pluviale, nivale et glaciaire. Cette triple alimentation assure un débit saisonnier constant mais présente une variabilité spatiale forte. \\

\begin{figure}[h!]
\centering
\includegraphics[width=1.\textwidth]{debit_stations_r2d2}
\caption{Chroniques de débits observés et cycle annuel issues de la Banque Hydro pour des stations situées dans les unités hydrographiques du Rhône. A) Porte du Scex (Suisse), B) Pougny, C) Mâcon, D) Valence, E) Beaucaire.}
\label{debit_r2d2}
\end{figure}

\noindent Suivant les régimes hydrologiques qui les caractérisent, le bassin versant se découpe en quatre sous-bassins hydrologiques:

\begin{itemize}
\item[$\bullet$] Sur la partie en amont du Léman, le Rhône possède un régime nivo-glaciaire dominé par les apports glaciaires. Ce régime s'identifie par des périodes de basses eaux hivernales (novembre-avril) et des périodes de hautes eaux au printemps issues de la fonte nivale;\\

\item[$\bullet$] Jusqu'à sa confluence avec la Saône, le régime reste nivo-glaciaire car influencé par de grands affluents (l'Arve, l'Ain, le Fier et le Guiers) dont l'alimentation saisonnière est régulière. Contrairement à la partie amont, les crues sur cette partie sont atténuées par la présence du Léman qui joue son rôle de tampon hydrologique;\\

\item[$\bullet$] Le régime de la partie médiane entre la Saône et l'Eyrieux est pluvial, notamment par l'apport des eaux de la Saône, soumis à un climat océanique. On observe, ici, une inversion des régimes de débits se traduisant par une période de hautes eaux en hiver et une période de basses eaux en été;\\

\item[$\bullet$] Enfin la partie aval, influencée par le climat méditerranéen, concourt à modifier totalement le régime hydrologique du Rhône. Sur cette zone, les cours d'eau souffrent de sévères étiages en été et participent à la propagation de crues rapides lors des épisodes méditerranéens automnaux.\\
\end{itemize}


En lien avec ces régimes hydrologiques et sur la base de leurs caractéristiques spatiales et temporelles, quatre grands types de crues se produisent sur le linéaire du Rhône :

\begin{itemize}
\item[$\bullet$] les \textbf{crues océaniques} se déclenchent suite à des cumuls de précipitations importants pendant les mois d'hiver sur la partie du bassin soumise à l'influence océanique venant de l'ouest et sont propagées par la Saône. Ces crues, de type fluvial, se caractérisent par des temps de concentration lents et une période de crue longue;\\

\item[$\bullet$] les \textbf{crues méditerranéennes} sont la conséquence directe des fortes intensités de précipitations tombant à l'automne sur le pourtour méditerranéen. Ces crues exceptionnelles se distinguent par des temps de concentration très courts et des pics de crues très importants issus de la contribution majeure des ruissellements torrentiels. Elles sont souvent associées à des crues dites "éclair" qui durent généralement quelques heures mais présentent des temps de réponse et des marnages importants;\\

\item[$\bullet$] Lorsque les épisodes méditerranéens ont des extensions spatiales importantes les crues associées n'ont plus ce caractère localisé, on parle alors de \textbf{crues méditerranéennes extensives}. Dans cette configuration, la crue est accentuée par la contribution des affluents et affecte la totalité du bassin rhodanien. C'est ce type de crue qui a provoqué la crue historique du Rhône de décembre 2003;\\

\item[$\bullet$] Lorsque les conditions météorologiques affectent la totalité du bassin alors il est possible de parler de \textbf{crues généralisées}. Ce type de crue est généralement provoqué par une combinaison des composantes océanique et méditerranéenne extensive.
\end{itemize} 
\clearpage
\subsection*{Intérêt économique}
Le corridor Rhône-Méditerranée a été, de tout temps, un axe économique majeur. Même s'il ne représente que 10 \% de la superficie du territoire il constitue un tissu économique pour le quart de la population française et plus du tiers des industries du bassin \citep{edl2019}. L'apport économique est aussi lié à l'attrait touristique de son bassin composé d'une importante diversité de territoire permettant notamment le développement d'activités nautiques. La gestion des eaux du Rhône a été cédée en 1933 et pour 90 ans à la Compagnie Nationale du Rhône (CNR)\footnote{L'échéance étant pour 2023, un projet de prolongation est en cours et devrait donner lieu, courant du printemps 2021, à un avenant ainsi qu'à un décret en conseil d'État pour une durée de concession allongée de 18 ans.}.\\

Les eaux du Rhône sont utilisées pour de nombreux usages. Entre autres, le potentiel énergétique du Rhône est important grâce à un débit conséquent. L'augmentation exponentielle du nombre d'ouvrages hydroélectriques, au milieu du 20\ieme{} siècle, donne aujourd'hui aux eaux du Rhône une importance capitale dans la production électrique française et suisse. Pour la partie suisse du Rhône, la capacité cumulée des réservoirs est de 1.2 km$^{3}$ pour une production de 1.5 milliard de kWh.an$^{-1}$ \citep{olivier2009}. Sur la partie française, le Rhône compte 20 centrales hydroélectriques gérées par la CNR. Pour satisfaire à la production électrique, un dédoublement du Rhône sur environ 180 km assure une alimentation constante en eau des stations hydroélectriques. En moyenne annuelle, ces centrales produisent 16400 GWh d'électricité, soit 93 \% de la production hydroélectrique française \citep{rhone2008}. \\
À cela s'ajoute l'utilisation des eaux du Rhône pour le refroidissement des centrales thermiques et nucléaires. Au total, les eaux du Rhône contribuent à 20 \% de la production électrique française. \\

Un autre vecteur économique important sur le Rhône concerne le transport, notamment sur son axe principal Rhône-Saône. Celui-ci est supporté par une forte anthropisation du linéaire fluvial et la construction de 14 écluses grand gabarit. Même s'il reste loin des grands fleuves comme le Saint-Laurent ou le Mississipi, le transport fluvial rhodanien s'élève à 22 millions de tonnes par an de marchandises. Ce transport s'axe principalement sur des fluxs de minéraux bruts (42 \%), de produits agricoles (13\%) et pétroliers (10 \%) \citep{rhone2008}. \\

Les eaux du Rhône sont, évidemment, d'une importance majeure en tant que ressource pour l'approvisionnement en eau potable et pour l'agriculture. Le Rhône est utilisé pour irriguer environ 108 000 hectares de terres agricoles sur une superficie totale de 190 000 hectares de surface irriguée \citep{rhone2008}.\\
Concernant l'eau potable, environ 0.2 km$^{3}$ d'eau est prélevé dans la nappe alluviale pour approvisionner 3 millions d'habitants \citep{olivier2009}.

\section{{\fontfamily{lmss}\selectfont Les lacs du bassin versant du Rhône}}

\subsection{{\fontfamily{lmss}\selectfont Caractéristiques générales}}
La France n'est pas un grand pays lacustre comparé à la Finlande ou au Canada mais elle compte un nombre important d'étangs, de lacs et de zones humides. Plus spécifiquement, 8.5\% du bassin du Rhône (France et Suisse incluses) est recouvert de lacs pour la majeure partie contenue dans cinq grands lacs \citep{olivier2009}: le Léman,  le lac du Bourget, le Lac d'Annecy, le lac de Serre-Ponçon et le lac de Sainte-Croix (Figure \ref{reseau_hydro_rhone}). Par ailleurs le lac du Bourget avec ses 3.6 km$^{3}$ est le plus grand lac naturel de France. Le tableau \ref{tab:lacs_rhone} présente les caractéristiques de ces cinq lacs.\\

\begin{figure}[h!]
     \subfloat[Principaux affluents du Rhône\label{rhone_riv}]{%
       \includegraphics[width=0.45\textwidth]{BV_rhone_rivieres}
     }
     \hfill
     \subfloat[Principaux lacs \label{rhone_lac}]{%
       \includegraphics[width=0.45\textwidth]{BV_rhone_lacs}
     }
     \hfill
     \caption{Représentation du réseau hydrographique du Rhône à 90m de résolution issue de MERIT-HYDRO.}
     \label{reseau_hydro_rhone}
\end{figure}

\begin{table}[h!]
 \caption{Caractéristiques des principaux lacs du bassin versant du Rhône.}
 \label{tab:lacs_rhone}
 \begin{tabularx}{\textwidth}{cXXXX}
 \hline
 Nom& Profondeur moyenne (m) & Profondeur maximale (m) & Superficie (km$^{2}$)& Volume (km$^{3}$)\\
 \hline
  Léman&154&310&580&89\\
  Lac du Bourget &85&145&44.5&3.6\\
  Lac d'Annecy &41&82&27.8&1.1\\
  Lac de Serre-Ponçon&72&90&28&1.3\\ 
  Lac de Sainte-Croix &30&93&22&0.76\\
  \hline
 \end{tabularx}
\end{table}

~\\
~\\
~\\
\subsection{{\fontfamily{lmss}\selectfont Le Léman et son rôle central}}
\label{sec:leman}

\'Etant données ses dimensions et sa localisation, le Léman joue un rôle central dans l'hydrologie du Rhône, il est donc important de s'intéresser brièvement à ses caractéristiques.\\

Le Léman a acquis ses lettres de noblesse en limnologie dès le 19\ieme{} siècle grâce au fondateur de cette science: le suisse François-Alphonse Forel. D'origine glaciaire et formé par l'effondrement d'une moraine, le Léman, avec ses 89 km$^{3}$, est le plus grand bassin d'eau douce d'Europe Occidentale \citep{cipel2019}. Sa morphologie est dominée par la présence des Alpes sur sa rive Sud-Est et du Jura sur sa rive Nord-Ouest donnant ainsi au lac une forme allongée dans une orientation Est-Ouest. Le Léman est constitué de deux grands bassins: à l'Est, le Grand Lac d'une superficie de 499 km$^{2}$ et de profondeur maximale 310 m, à l'Ouest, le Petit Lac d'une superficie de 81 km$^{2}$ et de profondeur maximale 76 m (Tableau \ref{tab_leman}).\\ 

\begin{table}[h!]
 \caption{Principales caractéristiques du Léman. Adapté de \citet{cipel2019}.}
 \label{tab_leman}
 \begin{tabularx}{\textwidth}{cXXX}
 \hline
 & Léman & Grand Lac & Petit Lac \\
 \hline
  Altitude moyenne (m)& 372.05 &&\\
  Surface libre (km$^{2}$&580.1&498.90&81.2\\
  Profondeur moyenne (m)&152.7&172&41\\
  Profondeur maximale (m)&309.7&309.7&76\\ 
  Volume (km$^{3}$) &89&86&3\\
  Temps de séjour théorique &\multicolumn{3}{c}{11 ans}\\
  \hline
 \end{tabularx}
\end{table}

Cette physionomie joue sur l'évolution des profils de température, avec une tendance du Petit Lac à réagir plus rapidement aux forçages extérieurs. D'un point de vue dynamique, chaque bassin engendre des courants généraux au sein du lac qui conduisent à la formation de gyres lacustres spécifiques pendant une grande partie de l'année \citep{lethi2012}. \\
Tout cela contribue à un répartition particulière des caractéristiques physiques du lac avec des eaux, en moyenne, plus chaudes au niveau du Petit Lac tandis que le Grand Lac présente des amplitudes saisonnières plus marquées.\\
Au-delà des légères disparités entre les bassins, le cycle annuel du profil vertical de température au sein du lac suit la même dynamique. Ainsi au printemps et en automne une stratification des eaux se met en place avant que l'hiver ne force le mélange des eaux pour tendre vers un profil homogène. Cette situation contribue au classement des eaux du Léman en bon état écologique \citep[Figure \ref{soulignacfig},][]{soulignac2019}. \\

\begin{figure}[h!]
\centering
\includegraphics[width=0.75\textwidth]{soulignac2019}
\caption{Distribution spatiale des états écologiques du Léman exprimée en terme de probabilité d'occurrence sur la base de 1000 échantillons prélevés en 2010. \textit{Chla}, \textit{NH4}, \textit{NO3}, \textit{TP} représente respectivement les concentrations en chlorophylle-a, ammonium, nitrate et phosphore total. \textit{SDD} est la profondeur du disque de Secchi. Source: \citet{soulignac2019}.}
\label{soulignacfig}
\end{figure}

Sur le plan hydrologique, le Léman est alimenté en eau douce par six affluents dont le principal est le Rhône avec ses 201 m$^{3}$.s$^{-1}$ (\url{https://www.hydrodaten.admin.ch/fr/2009.html}) (Figure \ref{bv-leman-cipel}). Du fait de l'assèchement continental des masses d'air océaniques venant de l'Ouest et du renforcement des pluies par effet orographique, la pluviométrie sur le bassin du Léman se répartit de façon croissante suivant un axe Ouest-Est. L'influence du climat montagnard assure un régime pluvio-nival avec un maximum entre le mois de mars et d'août et des basses eaux atteintes en hiver.\\

\begin{figure}[h!]
\includegraphics[width=1.\textwidth]{bv_leman}
\caption{Bassin versant du Léman et du Rhône aval jusqu'à la frontière franco-suisse. Les croix rouges localisent les stations de mesures d'où sont issues les observations de niveau d'eau fournies par Damien Bouffard (EAWAG/EPFL). Adapté de \citet{soulignac2019}.}
\label{bv-leman-cipel}
\end{figure}

\noindent Le Léman est anthropisé depuis le 19\ieme{} siècle, époque à laquelle la ville de Genève a construit un barrage pour alimenter les usines de la ville. La forte variation des niveaux du lac a contraint les cantons de Vaud et du Valais à co-signer, en 1884, un accord de gestion des eaux du Léman afin de garantir des variations acceptables. C'est sur cette base que s'est appuyée la construction du barrage poids de Seujet en 1995. Ce barrage de 73 m de long pour 1.5 m de haut fût construit avec comme double objectif de réguler les niveaux du Léman (par la même occasion le débit en sortie) et de produire de l'électricité pour la ville de Genève. Ainsi la convention signée oblige la ville de Genève a maintenir les niveaux du lac entre 372.3 m (asl\footnote{above sea level}) et 371.5 m (asl).\\
Cette régulation marque ainsi le pas entre le régime glaciaire du Rhône amont et le régime fluvial français, même si ce dernier reste dans un régime pluvio-fluvial grâce à l'apport des eaux de l'Arve \citep{ruiz2015}.

\section{{\fontfamily{lmss}\selectfont La chaîne SAFRAN-ISBA-MODCOU}}
SAFRAN-ISBA-MODCOU \citep{habets2008,lemoigne2020} désigne la chaîne hydrométérologique résultante de la collaboration entre le CNRM et Mines ParisTech \citep{etchevers2000}. Par la suite, d'autres partenaires dont le CETP \footnote{aujourd'hui LATMOS} et le Cemagref \footnote{aujourd'hui INRAE} ont été intégré au projet. Cette chaîne est composée de trois sous-systèmes: le Système d'Analyse Fournissant des Renseignements Atmosphériques à la Neige \citep[SAFRAN,][]{durand1993} fournissant les forçages atmosphériques, le modèle de surface ISBA résolvant les bilans d'eau et d'énergie et le modèle hydrogéologique MODCOU \citep{ledoux1989} qui simule les débits de rivières et la hauteur d'eau des aquifères. Ces données sont disponibles sur une grille régulière de 8 km projetée en Lambert II sur la France métropolitaine (Figure\ref{fig_sim}). \\

~\\
\noindent Ce système, à base physique, a été initialement validé sur des grands bassins hydrographiques comme le bassin Adour-Garonne \citep{voirin2003} ou le Rhône \citep{etchevers2001} avant d'être étendu à la France entière \citep{habets2008, quintana2008}. Il est utilisé à la fois en recherche et, depuis 2003, en opérationnel au sein de services comme la Direction de la Climatologie et des Services Climatiques (DCSC) de Météo-France. En parallèle, les travaux se sont portés sur la mise en place d'une base de données spatialisées des paramètres hydrométéorologiques depuis 1958 nécessaire à la prévision du risque inondation, la gestion de la ressource ou encore les effets du changement climatique appliqués en hydrologie \citep{soubeyroux2008, bonnet2017, dayon2018}.\\

\begin{figure}[h!]
\centering
\includegraphics[width=0.85\textwidth]{sim}
\caption{Représentation de la chaîne hydrométéorologique Safran-Isba-Modcou. Source: \citet{soubeyroux2008}.}
\label{fig_sim}
\end{figure}
~\\

Dans cette thèse, les ruissellements de surface et le drainage issus de SIM sur le bassin du Rhône ont été utilisés pour forcer CTRIP sur la période 1958-2016. Comme la grille régulière à 8 km de SIM est différente de la grille en longitude/latitude de CTRIP à 1/12°, les forçages atmosphériques ont été préalablement interpolés sur la grille CTRIP.\\
\clearpage
\noindent La suite du paragraphe décrit de façon succinte les deux sous-systèmes SAFRAN et MODCOU. Pour plus de détails le lecteur pourra se tourner vers les nombreuses publications disponibles.

\subsection*{Le système de réanalyse SAFRAN}

Développé au Centre d'\'Etudes de le Neige (CEN) pour la prévision du risque d'avalanches, SAFRAN fournit une analyse de huit variables atmosphériques ensuite utilisées par ISBA en tant que forçages atmosphériques. Ces huit variables sont: les précipitations liquides et solides, la température à 2 m, la vitesse du vent à 10 m, l'humidité spécifique à 2 m, la nébulosité, le rayonnement solaire et le rayonnement infrarouge (Figure \ref{safran}).\\
Le système repose sur le découpage du territoire en zones climatiques irrégulières pour lesquelles les variables atmosphériques sont considérées homogènes et seulement influencées par la topographie. Le zonage utilisé par Météo-France définit 615 zones qui ne dépassent pas 1000 km$^{2}$ de superficie. SAFRAN utilise alors un processus itératif de comparaison entre variables observées et analysées qui sont ensuite interpolées au pas de temps horaire suivant une méthode d'interpolation optimale. Cette interpolation se fait sur une grille régulière horizontale de 8 km couvrant la France métropolitaine ainsi que certaines zones extérieures faisant partie des bassins hydrographiques amont (\textit{e.g.} la Suisse pour le bassin du Rhône). Cette méthode d'optimisation s'appuie sur des observations ainsi que des ébauches issues du modèle atmosphérique globale ARPEGE pour produire des analyses sur les 9892 cellules de la grille.\\

\noindent En sortie, SAFRAN produit une analyse horaire des variables atmosphériques pour la période 1958-2016. D'abord validé sur le bassin du Rhône \citep{etchevers2001}, il a été étendu à l'ensemble du territoire métropolitain par \citet[][]{lemoigne2002}.

\begin{figure}[h!]
\includegraphics[scale=1]{safran}
\caption{Moyenne annuelle sur la période 1958-2018 de: a) température à 2m, b) humidité spécifique à 2m, c) vitesse du vent à 10m, d) précipitation totale annuelle, e) rayonnement solaire direct, f) rayonnement solaire diffus. Source: \citet{lemoigne2020}.}
\label{safran}
\end{figure}
\clearpage

\subsection*{Le modèle hydrogéologique MODCOU}

Le principe du modèle MODCOU repose sur la résolution d'une équation de diffusion pour le calcul des variations de niveaux des aquifères en réponse au ruissellement et au drainage issus d'ISBA. Connaissant ces niveaux, le modèle résout les équations de transfert à l'interface aquifère-rivière pour en déduire le stock de surface. Le transfert est ensuite assuré au sein de chaque bassin versant par un schéma numérique basé sur l'analyse des zones isochrones simulant les débits.\\
Le code initial de MODCOU a été employé pour développer la plateforme de modélisation EauDyssée utilisée pour simuler des bassins versants de tailles variés \citep{saleh2011}. Aujourd'hui ces modèles sont intégrés dans une plateforme de modélisation hydrogéologique Aqui-FR \citep{vergnes2020}.\\
Dans cette thèse, CTRIP s'occupe du routage en réponse aux forçages de SAFRAN-ISBA et le modèle MODCOU n'est donc pas utilisé.
\section{{\fontfamily{lmss}\selectfont Les configurations utilisées pour FLake et CTRIP}}
\label{sec:config_rhone}

Les flux de masse au niveau du lac Léman sont déduits d'une estimation du bilan entre l'évaporation et les précipitations. Les estimations d'évaporation tri-horaires ont été calculées par le biais de FLake avec la configuration proposée par \citet{lemoigne2016}. Parmi les paramètres importants, la profondeur est fixée à sa valeur maximale 60m et le coefficient d'extinction à 0.5 m$^{-1}$.\\
Les forçages générés par SAFRAN donnent donc une estimation des cumuls de précipitations solides/liquides au pas de temps horaire. Pour le lac, un calcul préliminaire simule un correction des forçages en prenant la différence entre la précipitation et l'évaporation au-dessus du lac. Cela permet de déduire la masse d'eau qui contribue directement au bilan sur le masque de ruissellement du lac (voir section \ref{sec:MLake}).\\

\begin{figure}[!h]
     \subfloat[Largeur des rivières]{%
       \includegraphics[width=0.5\textwidth]{river_width_rhone}
     }
     \hfill
     \subfloat[Numéro de séquence]{%
       \includegraphics[width=0.5\textwidth]{sequence_number_rhone}
     }
     \hfill
     \caption{Représentation de (a) la largeur des rivières et (b) du numéro de séquence sur le bassin du Rhône à 1/12° dans CTRIP avant l'introduction des lacs.}
     \label{param_ctrip}
\end{figure}

Concernant les paramètres de CTRIP, la configuration utilisée sur le bassin du Rhône prend seulement en compte le schéma d'aquifère qui est essentiel pour simuler correctement les débits dans la partie karstique du Rhône. La topographie et la largeur des rivières sont représentées sur la figure \ref{param_ctrip}. CTRIP est forcé par les sorties de SAFRAN-ISBA sur la période 1958-2016 et interpolées sur la grille CTRIP à 1/12°. Ces forçages sont ceux corrigés au niveau du Léman pour prendre en compte l'évaporation du lac sur cette même période.
~\\
~\\
~\\

\section{{\fontfamily{lmss}\selectfont Les données de validation }}
\label{sec:observations_rhone}
Dans le processus de développement d'un modèle il est nécessaire de vérifier que les processus physiques sont correctement représentés et il convient ensuite d'évaluer la qualité du modèle. Tout cela s'organise dans une étape de validation qui consiste en une comparaison des résultats de simulation avec des observations. Pour être significative, l'étape de validation doit se baser sur un grand nombre de données, ce qui est souvent le facteur limitant dans les études à grande échelle.\\
La validation de MLake sur le bassin versant du Rhône correspond finalement à une double validation. Tout d'abord, elle s'attache à déterminer les performances de CTRIP-MLake à simuler les débits du fleuve par rapport à une simulation de référence de CTRIP puis à les confronter à des observations. Ensuite comme MLake introduit une nouvelle variable diagnostique sur les variations de hauteur de lac, il est important de justifier cette introduction dans CTRIP.\\
Cette validation s'appuie sur des jeux de données présentés ici pour le bassin versant du Rhône.

\subsection*{{\fontfamily{lmss}\selectfont Débits}}

Le bassin du Rhône compte un nombre important de stations de jaugeage issues de la Banque Hydro et du GRDC. Il est évident que plus le nombre d'observations ayant passé le contrôle de qualité est important, meilleure est la qualité de l'analyse. Pour autant, le choix des stations doit respecter certaines règles. Ainsi les séries temporelles des stations doivent contenir au minimum, trois ans de données continues sur une période minimale totale de 10 ans. Dans le cas où deux stations se trouvent sur la même maille CTRIP, la station choisie est celle qui possède l'aire de drainage la plus grande. De plus, l'objet de l'étude porte sur l'influence des lacs dans un modèle de routage et de l'apport de leur dynamique sur les débits du Rhône. Pour cela, quatre sites d'études ont été choisis: Porte du Scex, Pougny, Valence et Beaucaire. Ce choix n'est pas arbitraire et repose sur des conditions strictes.\\

\noindent Sur la totalité des stations du bassin, un filtrage sur la disponibilité des données et la localisation a été effectué. Ainsi, seules les stations présentes sur le Rhône avec des données continues sur notre période d'étude (1958-2016) ont été sélectionnées. Parmi toutes les stations possibles, une seule station a été choisie par unité hydrographique dont une station de contrôle et trois stations d'évaluation (Figure \ref{fig_5stat}). \\

\begin{figure}[h!]
\centering
\includegraphics[width=0.8\textwidth]{BV_rhone_5stat}
\caption{Localisation des stations de jaugeage utilisées pour la validation des débits du modèle CTRIP-MLake.}
\label{fig_5stat}
\end{figure}

~\\

Le choix s'est donc porté sur:\\

\begin{itemize}
\item \textbf{Une station de contrôle amont}: la station de Porte du Scex qui se situe en amont du Léman. Cette station assure une cohérence dans les simulations et notamment contrôle que l'introduction de MLake ne modifie pas la stabilité du modèle dans la résolution de la dynamique en amont des lacs. Le cumul de précipitations moyen annuel au niveau de la station, issu des réanalyses SAFRAN, est de 1363 mm;\\

\item \textbf{Sur le Rhône amont français}: la station de Pougny qui est la station française directement en aval du Léman. Elle est d'une importance primordiale pour juger de l'influence directe du Léman sur les débits. Le cumul de précipitations moyen annuel au niveau de la station, issu des réanalyses SAFRAN, est de 1079 mm;\\

\item \textbf{Après la confluence de la Saône et de l'Isère}: la station de Valence. Comme les principaux affluents représentent plus de la moitié du débit (Figure \ref{sec:hydrologie}), cette station permet de quantifier l'importance de la dynamique du Rhône par rapport à celle des affluents avals. Le cumul de précipitations moyen annuel au niveau de la station, issu des réanalyses SAFRAN, est de 892 mm; \\

\item \textbf{À l'exutoire du bassin}: la station de Beaucaire. Située juste avant le début du Delta du Rhône, la station informe d'une part sur les apports d'eau issus de la totalité du bassin mais aussi sur l'influence de l'hydrologie locale. La propagation des crues est, dans cette unité hydrographique, très rapide du fait du climat méditerranéen et la station de Beaucaire nous permettra de valider le modèle dans un contexte où les débits sont très variables. Le cumul de précipitations moyen annuel au niveau de la station, issu des réanalyses SAFRAN, est de 655 mm.
\end{itemize}

\subsection*{{\fontfamily{lmss}\selectfont Hauteurs de lac}}

Pour les hauteurs de lac, la validation s'est concentrée sur le Léman qui influence directement l'alimentation du Rhône (section \ref{sec:leman}). Même si elles sont d'une importance majeure, il existe peu de données disponibles sur les hauteurs du Léman. Par ailleurs, que les données soient issues de stations limnimétriques ou d'observations satellitaires, les observations de cote d'eau ne sont pas directement accessibles.\\
Les données concernant le Léman ont été gracieusement fournies par Damien Bouffard (Eawag/EPFL) par le biais de l'Office Fédéral de l'Environnement Suisse. Ces données sont issues de trois stations de mesures placées sur les sites présentés en figure \ref{bv-leman-cipel} et fournissant des données continues entre 1974 et 2013. Deux stations sont situées sur la partie Grand Lac du Léman et une station est placée proche de l'exutoire. Ainsi, n'ayant aucune information sur la qualité des observations, les données des trois stations ont été moyennées pour calculer une hauteur moyenne du lac à comparer à la variable diagnostique issue de CTRIP-MLake.

\section{{\fontfamily{lmss}\selectfont Intégration des lacs sur le bassin versant du Rhône}}
Pour les quatre stations de référence, trois configurations de CTRIP-MLake ont été testées. Ces configurations, détaillées dans le tableau \ref{ctrip_config}, servent à évaluer le modèle CTRIP-MLake et à réaliser une analyse de sensibilité du modèle à la largeur du seuil du lac, seul paramètre ajustable. L'analyse statistique se base sur les scores présentés dans l'Annexe \ref{chap:critere-evaluation}.

{\renewcommand{\arraystretch}{1.1}
\begin{table}[h!]
 \caption{Configuration des différentes simulations effectuées sur le bassin versant du Rhône.}
 \label{ctrip_config}
 \begin{tabularx}{\textwidth}{p{3.5cm}p{3.5cm}p{7cm}}
 \hline
 Configuration&Forçages&Détails\\
 \hline
  $ctrip\_nolake$&SAFRAN-ISBA& \footnotesize{Simulation référence d'ISBA-CTRIP sans MLake}\\
  $ctrip\_mlake\_w1$&SAFRAN-ISBA&\footnotesize{Simulation CTRIP-MLake initialisée avec une largeur de seuil $weir\_w$ égale à la largeur de rivière aval}\\
  $ctrip\_mlake\_w0.5$&SAFRAN-ISBA&\footnotesize{Simulation CTRIP-MLake initialisée avec une largeur de seuil $weir\_w$ divisée par un facteur 2}\\
  $ctrip\_mlake\_w5$&SAFRAN-ISBA&\footnotesize{Simulation CTRIP-MLake initialisée avec une largeur de seuil $weir\_w$ multipliée par un facteur 5}\\
  \hline
 \end{tabularx}
\end{table}}


\subsection{{\fontfamily{lmss}\selectfont Apport des lacs sur les simulations de CTRIP}}
\label{subsec:apport}
Avant toute chose, il est important de vérifier que MLake a un effet significatif sur les simulations de CTRIP. Pour cela, il convient d'évaluer l'effet de MLake sur les performances de CTRIP en comparant les simulations à des sorties de référence CTRIP sans MLake. Cette partie se focalise donc principalement sur une analyse qualitative du processus de bilan d'eau lacustre et sur la sensibilité du modèle à la largeur du seuil du lac.\\

\noindent Le modèle a besoin d'une période de mise à l'équilibre (appelée spin-up) car la hauteur du seuil des lacs est initialisée \textit{a priori} et ne correspond pas à un état équilibré. Après plusieurs simulations, la durée de spin-up pour le Rhône est estimée à deux ans. Les résultats sur ces deux premières années de simulation ne sont donc pas pris en compte et la période d'évaluation sur le Rhône débute en 1960 et se termine en 2016.\\

\noindent La variabilité du Rhône est relativement élevée ce qui complique l'analyse qualitative des hydrographes. Dans un souci de lisibilité, la figure \ref{q_sensi_rhone} montre les résultats relatifs aux quatre stations de mesures choisies sur la période réduite 2000-2003. Néanmoins, l'analyse statistique et les scores portent bien sur la période d'étude complète 1960-2016.\\

\begin{figure}[h!]
\includegraphics[width=1.\textwidth]{subplot_q_sensi_rhone}
\caption{Hydrogramme du Rhône simulé par CTRIP-MLake pour les quatre stations de mesures sur la période 2000-2003. A) Porte du Scex, B) Pougny, C) Valence, D) Beaucaire.}
\label{q_sensi_rhone}
\end{figure}
\clearpage
Il n'apparaît aucune modification des débit amont au niveau de la station de contrôle. Cela confirme que MLake n'introduit pas de modification des débits en amont des lacs. \\
Par contre, sur l'ensemble des stations d'évaluation, la prise en compte des lacs provoque une contraction des débits du Rhône autour du module moyen annuel avec une réduction des pics de crues et une augmentation des débits d'étiage. Ainsi, la dynamique introduite par les lacs dans les simulations a tendance à lisser l'hydrogramme et cet effet est d'autant plus important que la station est proche du lac. Par exemple, le pic de débit simulé pour la crue extensive de l'automne 2002 est en nette diminution. L'introduction des lacs dans le modèle réduit de 17\% le pic de crue simulé du 26 novembre 2002 à la station de Beaucaire. Cette diminution amène le débit simulé de 12444 m$^{3}$.s$^{-1}$ dans la simulation de référence à une moyenne de 10362 m$^{3}$.s$^{-1}$ pour les simulations CTRIP-MLake. Sachant que le débit maximal observé ce jour là est de 10200 m$^{3}$.s$^{-1}$, la modélisation des lacs permet une meilleure représentation des crues spatialement étendues. Il est évident que des sources d'erreurs non prises en compte limitent l'interprétation de ces résultats mais ceux-ci sont encourageants.\\

Les effets de lacs se retrouvent aussi sur la partie amont du bassin avec un léger décalage temporel de l'hydrogramme. Cet effet est moins visible en aval du fait de la contribution importante des affluents du Rhône.\\ 
La réduction moyenne de variabilité temporelle pour les trois stations est de 17.8 \% même si  cette réduction est relativement plus forte pour les stations à l'aval (Valence, Beaucaire). D'un autre côté, le module annuel est significativement modifié pour la station de Pougny, avec une réduction de 30\%. Les autres stations voient leurs débits avals rester stable et les légers écarts sont introduits par l'utilisation de FLake dans CTRIP-MLake qui modifie les forçages (Tableau \ref{hydrology_metrics_rhone}).\\

%performance metrics for the river discharge for the Rhone river basin
\begin{table}[h!]
\footnotesize
	\caption{Comparaison des scores entre les débits observés et simulés au niveau des quatre stations de mesures}
	\label{hydrology_metrics_rhone}
	\centering
	\begin{tabularx}{\textwidth}{p{1.3cm}p{2.7cm}p{0.9cm}p{0.9cm}p{0.9cm}p{0.7cm}p{0.8cm}p{0.8cm}p{0.8cm}p{0.7cm}p{0.7cm}}
               \hline
		Station& Configuration & NSE &NSE log&KGE &NIC & $\overline{Q}$ \tiny{(m$^3$.s$^{-1}$)} &  $\sigma$ \tiny{(m$^3$.s$^{-1}$)} & RMSD \tiny{(m$^3$.s$^{-1}$)} &$\overline{Q_{s}}/\overline{Q_{o}}$ &$\sigma_{s}/\sigma_{o}$\\
               \hline
                \multirow{4}{4cm}{Porte du \\Scex}&\footnotesize{$ctrip\_nolake$}&-4.34&-3.9&0.31&-&231&288&229&1.26&2.91\\
		&\footnotesize{$ctrip\_mlake\_w1$}&-4.31&-3.74&-0.31&-&231&288&228&1.26&2.91\\
		&\footnotesize{$ctrip\_mlake\_w0.5$}&-4.28&-3.73&-0.30&-&231&287&227&1.26&2.90\\
		&\footnotesize{$ctrip\_mlake\_w5$}&-4.32&-3.74&-0.31&-&231&288&228&1.26&2.91\\
		&\footnotesize{$observations$}&-&-&-&-&182&99&-&-&-\\
               \hline
                \multirow{4}{4cm}{Pougny}&\footnotesize{$ctrip\_nolake$}&-1.58&-1.61&-0.07&-&336&317&245&1&2.10\\
		&\footnotesize{$ctrip\_mlake\_w1$}&-0.74&-0.60&0.33&0.32&340&269&201&1.01&1.80\\
		&\footnotesize{$ctrip\_mlake\_w0.5$}&-0.52&-0.39&0.41&0.39&340&251&188&1.01&1.64\\
		&\footnotesize{$ctrip\_mlake\_w5$}&-1.13&-1.12&0.17&0.17&340&298&223&1.01&1.95\\
		&\footnotesize{$observations$}&-&-&-&-&336&153&-&-&-\\
		\hline
		\multirow{4}{4cm}{Valence}&\footnotesize{$ctrip\_nolake$}&0.50&0.35&0.74&-&1568&1074&552&1.12&1.37\\
		&\footnotesize{$ctrip\_mlake\_w1$}&0.59&0.63&0.85&0.18&1589&831&502&1.14&1.05\\
		&\footnotesize{$ctrip\_mlake\_w0.5$}&0.62&0.64&0.83&0.24&1589&802&483&1.14&1.03\\
		&\footnotesize{$ctrip\_mlake\_w5$}&0.55&0.61&0.86&0.10&1589&875&525&1.14&1.12\\
		&\footnotesize{$observations$}&-&-&-&-&1400&782&-&-&-\\
		\hline
                \multirow{4}{4cm}{Beaucaire}&\footnotesize{$ctrip\_nolake$}&0.54&0.39&0.76&-&1925&1347&668&1.13&1.36\\
		&\footnotesize{$ctrip\_mlake\_w1$}&0.64&0.65&0.83&0.22&1948&1057&591&1.15&1.07\\
		&\footnotesize{$ctrip\_mlake\_w0.5$}&0.67&0.66&0.82&0.28&1948&1026&568&1.15&1.04\\
		&\footnotesize{$ctrip\_mlake\_w5$}&0.61&0.64&0.84&0.15&1948&1098&617&1.15&1.11\\
		&\footnotesize{$observations$}&-&-&-&-&1698&989&-&-&-\\
		\hline
	\end{tabularx}
\end{table}

La sensibilité à la largeur du seuil est nette et constante sur les trois stations d'évaluation. Le fait de réduire la largeur du seuil réduit l'amplitude des débits et augmente l'effet de lissage de l'hydrogramme. Ainsi, en utilisant le schéma CTRIP-MLake, les pics de crues sont plus faibles et les étiages plus importants. Cet effet est proportionnel à la largeur du seuil. Plus la largeur est grande, plus les débits simulés se rapprochent de la simulation de référence. Parmi les trois configurations, c'est pour la configuration $ctrip\_mlake\_w05$ que la variabilité est la plus faible avec une réduction moyenne du débit de 24\% par rapport aux simulations de référence. À l'inverse, l'augmentation de la largeur du seuil provoque un transfert d'eau plus rapide vers l'aval se traduisant par une variabilité moins forte du débit. Dans cette configuration la dynamique du Rhône se rapproche des simulations de référence et la variabilité moyenne, calculée à partir de l'écart-type, diminue de 10\%. \\

\noindent Ces conclusions concordent avec nos attentes concernant les effets de déversoirs. Un seuil plus large a tendance à renforcer la dynamique des débits en favorisant les temps de réponse courts. Cela se traduit par une baisse plus rapide des débits en période de basses eaux et une augmentation aussi plus importante en période de crue. À l'inverse, la diminution de la largeur du seuil allonge les temps de réponse du lac face aux forçages et engendre donc une variation plus lente des niveaux d'eau. L'alternance entre les périodes de basses et hautes eaux diminue et les amplitudes qui en résultent sont réduites.\\

\noindent La comparaison de CTRIP-MLake aux simulations de référence informe sur les apports de la nouvelle physique sur les simulations de débits. Cependant elle n'indique pas le réalisme du modèle et il donc est nécessaire de confronter ces résultats à des observations.


\subsection{{\fontfamily{lmss}\selectfont Validation du modèle CTRIP-MLake}}

L'objectif de cette deuxième étape d'évaluation est de valider les résultats des simulations du modèle CTRIP-MLake par rapport aux observations sur les stations choisies. Les simulations et leurs configurations restent similaires à celles utilisées dans la section précédente \ref{subsec:apport}.\\
Les résultats généraux présentés dans la suite s'appuient sur les figures \ref{q_rhone_obs}, \ref{q_rhone_moving}, \ref{seasonal_q_rhone} et le tableau \ref{hydrology_metrics_rhone}. De plus, l'analyse statistique se base sur les scores présentés dans l'Annexe \ref{chap:critere-evaluation}.

\begin{figure}[h!]
\includegraphics[width=1.\textwidth]{subplot_q_flake_rhone.png}
\caption{Hydrogrammes du Rhône simulés par CTRIP-MLake et observés pour les quatre stations de mesures sur la période 1960-2016. A) Porte du Scex, B) Pougny, C) Valence, D) Beaucaire.}
\label{q_rhone_obs}
\end{figure}

\begin{figure}[h!]
\includegraphics[width=1.\textwidth]{q_rhone_moving}
\caption{Hydrogrammes du Rhône simulés par CTRIP-MLake et observés pour les quatre stations de mesures sur la période 1960-2016 en moyenne glissante sur 30 jours. A) Porte du Scex, B) Pougny, C) Valence, D) Beaucaire.}
\label{q_rhone_moving}
\end{figure}

\begin{figure}[h!]
\includegraphics[width=1.\textwidth]{seasonal_q_flake_rhone.png}
\caption{Cycles saisonniers des débits du Rhône simulés par CTRIP-MLake et observés pour les quatre stations de mesures sur la période 1960-2016. A) Porte du Scex, B) Pougny, C) Valence, D) Beaucaire.}
\label{seasonal_q_rhone}
\end{figure}
\clearpage
\subsubsection*{{\fontfamily{lmss}\selectfont Débits}}
Dans sa partie amont, le Rhône présente un régime nivo-glaciaire unimodal dont le pic de débit se produit au début de l'été. Cette saisonnalité est bien représentée par CTRIP-MLake même si une surestimation des débits de hautes eaux et une sous-estimation des basses eaux persistent dans toutes les configurations. Malgré une réduction de la variabilité introduite par MLake, des biais persistent dans la simulation des extrêmes de débit notamment au niveau de la station de Pougny. \\

\noindent Ce manque de représentativité altère les performances du modèle et se traduit par des critères Nash-Sutcliffe Efficiency (NSE) négatifs pour toutes les configurations ($\overline{NSE}$ = -0.80). Il en est de même pour le logarithmique du NSE: $\overline{NSE}_{log}$ = -0.70. Ces deux scores traduisent la difficulté de CTRIP-MLake à reproduire la dynamique observée et notamment le NSE$_{log}$ informe sur la faible performance du modèle à simuler les étiages. Comme ces scores donnent un poids assez fort à la corrélation, il est intéressant de les recouper avec le critère Kling-Gupta Efficiency (KGE), un score hydrologique moins sensible à la corrélation (voir dans l'annexe \ref{chap:critere-evaluation}). Ce score est positif sur toutes les stations ($\overline{KGE}=0.30$) ce qui indique bien que la principale cause amenant à de faibles scores provient d'une faible corrélation entre les hydrogrammes simulés et observés. Le rapport des écarts-types confirme une trop grande variabilité des débits simulés, écart que l'on retrouve sur le cycle saisonnier (Figure \ref{seasonal_q_rhone}). En tout état de cause, même si les scores restent faibles, le Normalized Information Contribution (NIC), avec une valeur moyenne de 29\%, indique une contribution positive de MLake dans la simulation des débits du Rhône par CTRIP en sortie du Léman.\\

Dans sa partie aval, le régime du Rhône est pluvial bimodal avec une période de hautes eaux au printemps et une période d'étiages pendant l'été. Les stations choisies sur cette zone sont représentatives de la dynamique totale du bassin versant. Comme sur la partie amont, la saisonnalité est bien respectée même si la tendance à une surestimation des hautes eaux et une sous-estimation des étiages dans les simulations se retrouvent aussi sur cette partie du Rhône. Excepté le pic de débit printanier qui est mal représenté, la dynamique du débit entre fin juin et mars reste particulièrement bien simulée par le modèle. \\
Les résultats des stations de Valence et de Beaucaire sont similaires avec une variabilité du débit réduite de 22\% pour Valence et de 21\% pour Beaucaire. Ce résultat signifie que les lacs influencent les débits du bassin hydrographique jusqu'à l'aval et que leurs signaux dans la variation des débits ne sont pas dilués par la contribution des affluents. Les scores hydrologiques sont, par ailleurs, nettement améliorés avec des hausses moyennes de 0.09 pour le NSE à Valence et de 0.10 pour le NSE de Beaucaire (Tableau \ref{hydrology_metrics_rhone}). \\
Les résultats sur la partie aval confirment l'impact positif de l'introduction de MLake pour la représentation des pics de crues. En effet, l'ajout des lacs permet de retrouver des pics de crues plus proches des observations. De plus, il y a une amélioration dans la représentation des étiages avec des NSE$_{log}$ qui augmentent en moyenne de 0.28 à Valence et 0.25 à Beaucaire (Tableau \ref{hydrology_metrics_rhone}). La hausse est moins importante pour le critère KGE, par contre les valeurs atteintes dénotent une excellente représentation des débits sur ces stations ($\overline{KGE}_{valence}$ = 0.85, $\overline{KGE}_{beaucaire}$ = 0.83). Enfin l'amélioration qu'introduit MLake sur la simulation des débits est confirmée par le NIC qui se trouve en moyenne être de 17\% à Valence et de 22\% à Beaucaire.\\

\noindent Le critère NSE s'améliore sur le linéaire du fleuve avec les meilleurs scores pour les stations proches de l'exutoire. On remarque ainsi que pour la configuration $ctrip\_mlake\_w1$ le NSE passe de -0.74 à Pougny à 0.59 à Valence et enfin 0.64 à Beaucaire. Ces scores indiquent que le débit à l'exutoire d'un bassin versant représente mieux la dynamique totale du bassin mais aussi est moins influencé par les variations hautes fréquences apparaissant notamment à l'exutoire du Léman.\\
Ces résultats mettent en avant l'impact du modèle de lac sur le lissage des hydrogrammes et qui conduit à des améliorations significatives de l'amplitude et de la temporalité des simulations de débits sur le bassin versant. Ce phénomène est particulièrement visible sur l'alternance de crues entre 2002 et 2003 et les périodes d'étiages estivaux. \\

L'évaluation du modèle est finalement conclue par un test sur la sensibilité des débits simulés par le modèle à la largeur du seuil à l'exutoire du lac. Ce paramètre est est important dans le modèle et son évaluation en étant l'unique paramètre ajustable à ce stade. Ce test est réalisé suivant une approche 'one at a time', c'est-à-dire en prenant successivement des valeurs différentes. Ces facteurs multiplicatifs sont 0.5, 1 et 5. La valeur unitaire correspond à la valeur par défaut de la largeur de la rivière en aval du lac dans le réseau CTRIP.\\
Le test, dont la figure \ref{taylor_q_rhone} illustre les résultats, montre l'amélioration générale des performances dans les trois configurations par rapport à la simulation de référence CTRIP sans les lacs. La configuration $ctrip\_mlake\_w05$ se dégage des deux autres sans pour autant représenter une nette amélioration. Le diagramme montre aussi que l'élargissement de la largeur du seuil dégrade légèrement les scores tout en rapprochant les simulations de la configuration de référence sans lac.\\
Dans l'ensemble, la largeur du seuil semble être un paramètre robuste et la valeur initiale prescrite dans CTRIP peut donc être conservée sur le bassin versant du Rhône.

\begin{figure}[h!]
\centering
\includegraphics[width=0.95\textwidth]{taylor_q_rhone}
\caption{Diagramme de Taylor représentant les performances des différentes configurations à simuler les débits pour les quatre stations de mesures sur la période 1960-2016. A) Porte du Scex, B) Pougny, C) Valence, D) Beaucaire.}
\label{taylor_q_rhone}
\end{figure}
\clearpage
\subsubsection*{{\fontfamily{lmss}\selectfont Hauteurs}}

La variation de la cote d'eau est une variable diagnostique de MLake. Celle-ci est calculée sur la base du stock en eau à la fin du pas de temps et reflète les fluctuations saisonnières du volume d'eau du lac. Les résultats sont présentés sur la figure \ref{subplot_h_rhone}, figure \ref{season_h_rhone} ainsi que dans le tableau \ref{tab:level_metrics_rhone}.\\

\begin{figure}[h!]
\includegraphics[width=1.\textwidth]{subplot_h_rhone}
\caption{Séries temporelles des variations de niveau des eaux du Léman simulées par CTRIP-MLake et observées en trois stations sur la période 1974-2013.}
\label{subplot_h_rhone}
\end{figure}


%performance metrics for the lake levels
\begin{table}[h!]
	\caption{Scores détaillant les performances des différentes configurations pour la simulation des niveaux du Léman.}
	\label{tab:level_metrics_rhone}
	\centering
	\begin{tabularx}{\textwidth}{p{2cm}p{3cm}p{1cm}p{2cm}p{2cm}p{2cm}}
               \hline
		Lac & Configuration&$r$&$h_{max}$/$h_{min}$&$\sigma_{s}$ (m) &RMSD (m) \\
               \hline
                \multirow{3}{4cm}{Léman}&\footnotesize{$ctrip\_mlake\_w1$}&0.30&2.02/-0.84&0.55 (2.39&0.53\\
		&\footnotesize{$ctrip\_mlake\_w0.5$}&0.37&2.80/-1.22&0.81 &0.76\\
		&\footnotesize{$ctrip\_mlake\_w5$}&0.21&0.94/-0.3&0.20 &0.27\\
		&\footnotesize{$observations$}&0.23&0.53/-0.69&-&-\\
               \hline
	\end{tabularx}
\end{table} 

\noindent Dans l'ensemble, malgré une bonne représentation des variations de niveaux, les pics de débits et les étiages sont atteints prématurément par rapport aux observations et avec des amplitudes trop importantes. Alors que les amplitudes maximales observées restent autour de 0.5 m, cette surestimation systématique peut atteindre jusqu'à 2.8 m pour la configuration $ctrip\_mlake\_w05$. Seule la configuration $ctrip\_mlake\_w5$ donne des séries temporelles comparables aux séries observées.\\
Pour ce qui est des scores généraux, la corrélation est plutôt médiocre ($\bar{r}$ = 0.29) traduisant une faiblesse du modèle à représenter les variations de cote du Léman. Cela se traduit aussi par des écarts-types significatifs et un coefficient de variation moyen de 2.3 ($\overline{\sigma}_{s}$ = 0.52). Ces écarts sont expliqués en partie par la surestimation des hautes eaux dont le ratio moyen est de 2.6. Ces surestimations sont particulièrement fortes pour les configurations $ctrip\_mlake\_w05$ et $ctrip\_mlake\_w1$.\\

Contrairement aux résultats sur les débits et malgré une bonne corrélation relativement aux autres configurations (r = 0.37), $ctrip\_mlake\_w05$ présente de moins bons résultats. Cela s'explique par un temps de réponse plus long aux évolutions du forçage, les hauteurs d'eau qui en résultent sont logiquement plus élevées notamment en période de crue (les volumes additionnels étant stockés plus longtemps) et représentent une dynamique plus lente du lac. À l'inverse, la configuration $ctrip\_mlake\_w5$ présente les erreurs les plus faibles (RMSD = 0.27 m) ce qui se traduit par un cycle saisonnier plus réaliste (Figure \ref{season_h_rhone}).

\begin{figure}[h!]
\centering
\includegraphics[width=0.8\textwidth]{season_h_rhone}
\caption{Cycles saisonniers des variations de niveau des eaux du Léman simulés par CTRIP-MLake et observés en trois stations sur la période 1974-2013.}
\label{season_h_rhone}
\end{figure}

L'effet sous-jacent à cette sensibilité est que la dynamique du marnage est inversement proportionnelle à la largeur du seuil. Cela est physiquement correct puisque dans le cas d'une largeur de seuil plus grande, le temps de réponse aux forçages est atténué et le transfert d'eau vers l'aval plus long. De plus, l'analyse du cycle saisonnier indique que la configuration $ctrip\_mlake\_w5$ est celle qui respecte le mieux le cycle annuel des niveaux du Léman.


\subsection{{\fontfamily{lmss}\selectfont Discussions}}

\noindent Dans l'ensemble, les résultats indiquent un effet positif de l'ajout des lacs dans CTRIP sur le bassin versant du Rhône. Cet effet est particulièrement significatif pour le soutien des étiages dans la partie méditerranéenne, secteur à enjeux concernant la ressource en eau. L'apport des lacs sur le bassin versant du Rhône a déjà été mis en avant par \citet{zajac2017} et notre étude confirme l'intérêt de prendre en compte le bilan de masse des lacs sur cette zone. \\
Le test de sensibilité a montré que les débits et hauteurs simulés semblent relativement sensibles à la largeur du seuil de déversement. Toutefois seules trois configurations ont été testées et pour confirmer ces résultats, le test doit être étoffé pour prendre un intervalle de valeurs plus important. Une autre méthode pour affiner les résultats serait d'avoir accès à une estimation de la largeur de l'exutoire ou à une méthode de calcul indirect. Sur la base du travail de \citet{vergnes2012}, il serait ainsi opportun de tester la méthode de calcul des largeurs de rivière directement sur les débits en sortie de lac. Ce type d'information n'est à ce jour pas assez développé à l'échelle globale et souvent difficile à mesurer voire à estimer et varie notamment avec le niveau du lac. \\

\noindent Des biais importants perdurent dans les résultats notamment sur les cycles saisonniers entre observations et simulations. Quelles que soient les configurations choisies, les étiages restent sous-estimés et les pics de crues trop importants. \citet{decharme2010} a montré que ces défauts étaient intrinsèques au modèle CTRIP et ont été partiellement résolus par l'introduction d'un schéma d'aquifères \citep{vergnes2012}. Pourtant ces biais systématiques sont visibles dès l'exutoire du Léman et associés à des marnages beaucoup trop importants. Une des origines de ces différences provient de la présence du barrage de Seujet régulant l'exutoire du Léman. L'impact du barrage sur le cycle hydrologique du lac est clair avec un seuillage systématique des niveaux du lac lors des périodes de hautes eaux. Cela engendre une quasi-absence de pics de débits au printemps au profit d'une décroissance lente de l'hydrogramme. Lors des étiages, la présence des barrages-réservoirs assure des étiages moins importants et une disponibilité en eau accrue. Dans notre module, le choix a été fait de ne considérer que le bilan de masse naturel pour les lacs. Plus globalement, la pression anthropique sur le bassin versant du Rhône est très forte et limite donc la modélisation d'un comportement naturel du système hydrographique. La plupart des lacs sont contrôlés par des barrages et la Durance elle-même est dévié de façon significative. Dans sa version actuelle, MLake ne prend en compte ni les règles de gestion de barrage ni la pression anthropique présente sur ce matin. Comme nous avons vu dans la section \ref{sec:leman}, le marnage d'un réservoir est contrôlé dans un intervalle de niveau qui ne traduit pas la dynamique naturelle de celui-ci mais plutôt un besoin industriel ou social. Dans le cas du Léman, il est même soumis à un accord. Pour avoir une vision complète de la dynamique sur le bassin versant du Rhône et réduire les biais, il est par conséquent nécessaire de prendre en compte ces processus.
\clearpage

\section{{\fontfamily{lmss}\selectfont Conclusion}}

Dans son étude sur la France, \citet{vergnes2012} a mis en évidence l'apport posifif du schéma d'aquifère sur les débits d'étiage. Cependant l'ajout de ce module dans le modèle CTRIP ne corrige pas totalement les surestimations systématiques des simulations de débits.\\ 
L'ajout d'un module résolvant le bilan d'eau des lacs dans CTRIP engendre, sur le bassin versant du Rhône, une réduction des biais sur les débits et une meilleure représentation des étiages notamment sur la partie aval du bassin. Les résultats sur les pics de débits sont moins visibles même si une atténuation des débits de pointe est observée. Le fait de prendre en compte une dynamique lacustre, plus lente comparée à un tronçon de rivière, retarde légèrement les débits de pointe mais surtout atténue sensiblement leurs amplitudes. Cela s'explique par la capacité de rétention des lacs, moins impactés par les modifications atmosphériques à court terme, à assurer un soutien de l'étiage lors des périodes sèches. Cela est bien visible sur le cycle saisonnier des débits du Rhône avec un cycle globalement respecté sur chaque unité hydrographique. \\

Cette étude locale a montré l'intérêt de considérer les lacs dans l'hydrologie locale et régionale et cela même dans un bassin où l'importance des lacs est relativement modeste.
En se basant sur une approche de bilan de masse où les débits à l'exutoire sont représentés par une équation de déversoir rectangulaire à seuil épais, il est possible de réduire les biais sur les simulations de débits ainsi que de modéliser une dynamique des niveaux d'eau réaliste. Les résultats sont en nette amélioration par rapport à la version de CTRIP initiale. Ainsi la contribution moyenne de MLake aux performances sur les débits est de 23\% avec un critère KGE croissant sur le linéaire du Rhône ($\overline{KGE}$ = 0.66). L'apport principal de MLake est de proposer un diagnostic des variations de niveaux de lac avec des résultats probants pour le Léman dans la configuration $ctrip\_mlake\_w5$ (RMSD= 0.27 m et CV = 1.05).\\

Malgré tout des biais persistent notamment sur les débits de crues et marquent les limites du modèle. Le modèle développé ne prend pas en compte l'anthropisation qui modifie profondément la dynamique naturelle du lac. Ces biais relativement importants sur la partie amont ont, cependant, tendance à être gommés dans la partie aval du bassin.
Cette méthode montre des résultats satisfaisants qui confirment l'intérêt de son application à l'échelle globale. Pour cela il est nécessaire de trouver des sites d'études instrumentés. Au vu des résultats du test de sensibilité, l'étude globale doit porter sur un intervalle plus important de variation du facteur multiplicatif à appliquer à la largeur du seuil afin de caractériser plus précisément la réactivité du modèle à ce paramètre. Tout cela justifie l'implémentation à l'échelle globale de MLake pour vérifier sa cohérence et son applicabilité sur des bassins contrastés.

\cleardoublepage
\chapter{{\fontfamily{lmss}\selectfont \'Evaluation et validation globale}}
\label{chap:etude-globale}
\minitoc

\noindent Dans le chapitre précèdent, le modèle de bilan de masse MLake a été validé à 1/12° sur le bassin du Rhône. Cette évaluation locale nécessaire ne justifie pas, pour autant, la possibilité d'utiliser ce modèle pour des études globales. La confirmation de l'adéquation du modèle à l'échelle globale est menée dans un cadre plus large au sein de ce chapitre.\\
La version de CTRIP à 1/12° est en cours de validation à l'échelle globale et ne peut pas être utilisée en tant que telle pour évaluer et valider MLake à cette échelle. C'est pour cela que ce chapitre se base notamment sur le travail de \citet{decharme2019} pour valider d'un point de vue local le comportement de CTRIP-MLake en global à 1/12°.\\
L'évaluation du modèle est effectuée en mode off-line avec les mêmes forçages que ceux utilisés dans \citet{decharme2019} et interpolés à l'échelle de travail. Au niveau des modèles, le schéma ISBA est utilisé dans sa version diffusive alors que le modèle CTRIP ne tient pas compte des processus d'aquifères ni des plaines d'inondations. Ce choix est porté par le souci d'évaluer le modèle MLake indépendamment des autres processus. Pour autant la dernière partie de ce chapitre présentera les résultats préliminaires du modèle complet à l'échelle globale. 

\section{{\fontfamily{lmss}\selectfont Les sites d'études}}
\label{sec:bv-globe}

Le choix des sites d'études n'est aucunement arbitraire et est motivé par des contrastes climatiques et hydrologiques. En effet, il est essentiel de vérifier que le non-étalonnage du modèle ne provoque pas de sensibilité accrue aux conditions locales. Trois sites d'études présentant des enjeux intéressants ont donc été choisis: le bassin de l'Angara, du Nil Blanc et enfin de la Neva.

\subsection{{\fontfamily{lmss}\selectfont Bassin versant du lac Baïkal}}
\label{sec:baikal}
\subsubsection*{{\fontfamily{lmss}\selectfont Morphologie du bassin}}

Provenant du turc \textit{Bay Köl} ('lac sacré'), le lac Baïkal est remarquable par ses spécificités. Conséquence d'une subsidence et de la formation d'une zone de rift, le lac Baïkal occupe un fossé d'effondrement et bat les records du lac le plus profond (1640 m) et le plus vieux (environ 25 millions d'années) du monde. À cela s'ajoute que le lac Baïkal, avec un volume de 23 600 km$^{3}$ pourrait contenir l'équivalent des eaux des Grands Lacs Américains, soit 20\% des réserves d'eau douce lacustre \citep{brunello2003, messager2016}.\\
Pour ce qui est de ses caractéristiques, le lac Baïkal s'étire sur près de 650 km dans une orientation Nord-Est/Sud-Ouest (Figure \ref{bv_baikal}). Dans sa partie aval, le bassin versant est principalement recouvert de forêts; la partie amont, plus montagneuse, est recouverte de steppe. La particularité de cette partie de la planète est la présence quasi-exclusive de permafrost, sol dont la température reste égale ou inférieure à 0°C tout au long de l'année voir pendant plusieurs années \citep{tornqvist2014}. \\
Au niveau climatique, le bassin du Selenga est soumis à un climat continental  particulièrement rude avec des hivers froids et secs ($\overline{T}_{janvier}$ = -23.5 °C) et des étés tempérés ($\overline{T}_{juillet}$ = 16.9 °C) \citep{tornqvist2014} \footnote{Dwb dans la classification de Köppen-Geiger \citep{beck2018}}.

\begin{figure}[h!]
\centering
\includegraphics[width=0.8\textwidth]{BV_baikal}
\caption{Bassin versant de l'Angara depuis Irkoutsk.}
\label{bv_baikal}
\end{figure}

\subsubsection*{{\fontfamily{lmss}\selectfont Hydrologie}}
\label{sec:hydrologie_baikal}

Le Selenga est le principal affluent du lac Baïkal et contribue pour 50 à 60 \% aux écoulements entrants dans le lac. L'hydrologie du bassin est saisonnière et dominée par la succession de période de gel en hiver et de dégel en été. Ainsi le principal apport au niveau des masses d'eau provient de la fonte de la neige, apport qui est aujourd'hui altéré par la fonte du permafrost \citep{karlsson2012}. Une dépendance hydrologique très forte existe entre le Selenga et le lac Baïkal puisque 82\% du bassin du lac est recouvert par le bassin du Selenga \citep{nadmitov2015}. \\
Le lac Baïkal alimente et contrôle un unique effluent: l'Angara qui lui-même fait partie du bassin du Yenissei, 5\ieme{} plus long fleuve du monde.

\subsubsection*{{\fontfamily{lmss}\selectfont Intérêt économique et gestion du bassin}}

L'intérêt du lac Baïkal sur l'économie régionale est évidente en ce qui concerne l'approvisionnement en eau douce continue au cours de l'année, la pêche et par l'attrait touristique des lieux. Deux pays se partagent son trait de côte: la Mongolie et la Russie. De plus, la ressource minière du bassin du lac Baïkal est riche  (aluminium, or, tungsten) et participe considérablement au développement économique des régions voisines \citep{brunello2003}. Enfin la Russie et plus particulièrement l'Oblast d'Irkoutsk utilise les eaux du bassin pour refroidir les centrales thermiques et produire de l'électricité à partir de barrages hydroélectriques.\\

Pour faire face à la pression importante introduite par l'homme sur le lac et l'impact écologique qui en résultent, la Russie a établi une commission pour le lac Baïkal en 1993. Cette commission composée de représentants locaux et fédéraux a abouti à la "Loi Baïkal" en 1999 fixant des règles pour une gestion active du bassin en matière de pollutions et d'usages et gérées par l'Agence Fédérale pour la Protection de l'Environnement \citep{garmaeva2001,brunello2006}. Cette loi préserve le biotope exceptionnel du lac Baïkal, conséquence de l'oxygénation de la totalité de la colonne d'eau, protégé depuis 1996 en tant que patrimoine mondial de l'UNESCO \citep{moore2009}.  

\subsection{{\fontfamily{lmss}\selectfont Bassin versant du lac Victoria}}
\label{sec:victoria}
\subsubsection*{{\fontfamily{lmss}\selectfont Morphologie du bassin}}

Deuxième plus grand lac du monde avec 69 500 km$^{2}$, le lac Victoria est le plus grand des lacs africains (Figure \ref{bv_victoria}). Seul Grand Lac africain à ne pas être situé dans une dépression du Grand Rift, son origine est encore discutée et résulterait vraisemblablement d'un inversion de l'écoulement des rivières suite à un soulèvement des régions situées à l'ouest. De par cette localisation particulière, le lac recouvre près de 42\% du bassin versant avec des conséquences sur l'hydrologie régionale. Le bassin d'alimentation du lac est donc relativement petit par rapport au lac en lui-même.\\
Malgré une superficie importante, le lac Victoria, dont la profondeur moyenne est de 40 m et la profondeur maximale de 80 m, contient une masse d'eau très inférieure à celle des lacs voisins comme le Tanganika ou le lac Malawi.\\

\begin{figure}[h!]
\centering
\includegraphics[width=0.8\textwidth]{BV_victoria}
\caption{Bassin versant du lac Victoria depuis Jinja.}
\label{bv_victoria}
\end{figure}

D'un point de vue climatique, le bassin du lac Victoria se trouve dans une zone tropicale \footnote{Af dans la classification de Köppen-Geiger \citep{beck2018}} avec une alternance de périodes humides et de périodes sèches. Malgré cette alternance, la température moyenne de l'air est plutôt uniforme avec des variations saisonnières inférieures à 3°C. Au niveau des précipitations, des disparités existent entre les différentes rives mais elles sont principalement dues au mouvement de la Zone de Convergence Inter-Tropicale \citep[ZCIT,][]{nicholson2017}. On retrouve donc une saison humide bien marquée entre mars et mai puis une deuxième période qui l'est beaucoup moins entre novembre et décembre. Ainsi, les rives Ouest sont généralement plus humides ($\approx$ 2 000 mm/an) que les rives Sud-Ouest ($\approx$ 1 100 m/an) ou Sud-Est ($\approx$ 750 mm/an) \citep{paugy2019}. Une autre particularité est l'intensification de 30\% des cumuls de pluie au-dessus du lac Victoria engendrée par la seule présence du lac \citep{sutcliffe1999}.

\subsubsection*{{\fontfamily{lmss}\selectfont Hydrologie}}
\label{sec:hydrologie_victoria}

Le lac Victoria constitue la partie amont du Nil Blanc et contribue à environ 14\% du débit conjoint Nil Blanc-Nil Bleu à Assouan \citep{crul1995}. Une dizaine d'affluents principaux alimente le lac mais c'est surtout le couple précipitations-évaporation qui contrôle les niveaux d'eaux. L'évaporation estimée du lac Victoria se situe aux alentours de 1 500 mm/an ce qui fait que le bilan de masse du lac est globalement équilibré sur l'année \citep{vanderkelen2018a}.\\
Pour ce qui est de l'hydrologie, le lac Victoria possède donc un intérêt particulier car principalement contrôlé par les facteurs climatiques.

\subsubsection*{{\fontfamily{lmss}\selectfont Intérêt économique}}

Les côtes du lac Victoria sont partagées par trois pays: la Tanzanie, l'Ouganda et le Kenya qui profitent de la riche biodiversité et des ressources minières comme leviers économiques majeurs.\\
Comme la majorité des grands lacs et encore plus les Grands Lacs africains, le lac Victoria apporte une indépendance économique majeure issue de la pêche, de la ressource en eau douce, du transport et du tourisme \citep{crul1995}. Aussi, depuis les années 1990, le bassin est au cœur d'une industrie pétrolière qui profite des ressources du sous-sol.\\
Cependant l'importance majeure du lac Victoria repose sur son niveau d'eau qui régule en grande partie la vie socio-économique du bassin versant du Nil et ce jusqu'en \'Egypte.\\
L'importante biodiversité du lac provient de la variété des zones d'intérêt comme les grandes zones marécageuses littorales ou de la diversité des substrats dans les zones pélagiques \citep{paugy2019}.\\
Enfin, l'intérêt du lac est aussi industriel puisque ses eaux sont aujourd'hui largement anthropisées et notamment pour les besoins en hydroélectricité avec la construction du complexe de Nalubaale initié en 1954 et finalisé en 2000 par un deuxième barrage \citep{kull2006}. 
\subsection{{\fontfamily{lmss}\selectfont Bassin versant du lac Ladoga}}
\label{sec:ladoga}
\subsubsection*{{\fontfamily{lmss}\selectfont Morphologie du bassin}}

La Neva est un fleuve russe long de 74 km et qui draine une surface de 281 000 km$^{2}$. Situé en Russie occidentale, au sein du bouclier scandinave, ce système possède les deux plus grands lacs d'eau douce du continent européen: le lac Ladoga et le lac Onega. Situé au nord-est de Saint-Pétersbourg la superficie du lac Ladoga est de 17 800 km$^{2}$ et son volume de 850 km$^{3}$. En amont de ce lac se trouve le lac Onega dont la superficie est de 9 720 km$^{2}$ pour un volume de 290 km$^{3}$ \citep{filatov2019}. Ces lacs se sont formés en deux temps, tout d'abord la formation du bassin à la suite de mouvements tectoniques et ensuite le remplissage en eau douce lors du retrait glaciaire \citep{malmqvist2009}.\\
Le bassin versant de la Neva est un système hydrologique connecté où chaque compartiment est relié aux autres par un réseau de rivières et de lacs dont l'exutoire se trouve dans le Golfe de Finlande à Saint-Pétersbourg. Ainsi la Neva est issue du lac Ladoga, lui-même connecté au lac Onega par le Svir', au lac Saimaa par la Vuoksa et au lac Ilmen par le Volkhv \citep{rukhovets2010}.\\
Contrairement aux lacs précédents l'environnement proche du lac Ladoga est assez uniforme et composé de plaines recouvertes d'une combinaison de forêts de conifères boréales (à hauteur de 55\%) et de zones humides (à hauteur de 13\%) \citep{malmqvist2009}.\\

\begin{figure}[h!]
\centering
\includegraphics[width=0.8\textwidth]{BV_ladoga}
\caption{Bassin versant de la Neva depuis Kirovsk.}
\label{bv_ladoga}
\end{figure}

En ce qui concerne le climat, la région est à la frontière entre un climat océanique et un climat continental humide \footnote{Dfb dans la classification Köppen-Geiger \citep{beck2018}} caractérisé par des hivers froids et des étés tempérés pour une température moyenne annuelle aux alentours de 2.5 °C. Pour ce qui est des précipitations, elles se répartissent uniformément tout au long de l'année pour un cumul annuel aux alentours de 550 mm dont 70\% sous forme de neige \citep{malmqvist2009}. Ces conditions climatiques sont essentielles pour comprendre le comportement de ces lacs et expliquent notamment la couverture glaciaire partielle du Ladoga et totale de l'Onega chaque hiver.

\subsubsection*{{\fontfamily{lmss}\selectfont Hydrologie}}
\label{sec:hydrologie_ladoga}

Le régime nival de la région, associé au système lacustre particulier assure une alimentation en eau constante de la Neva. Il existe cependant un pic de débit arrivant généralement à la fin du mois de mai. Le couple Ladoga-Onega recouvre près de 30\% du bassin versant du fleuve. C'est pour cela qu'en hydrologie le linéaire Svir'-Neva est généralement considéré comme un ensemble \citep{malmqvist2009}.\\
Au vu des conditions climatiques, le bilan hydrologique est dominé par les affluents et effluents ce qui donne un lac de type fluvial.\\
L'intérêt hydrologique d'étudier ce lac vient de l'importance de simuler correctement les processus hydrologiques de surface et le transfert d'eau mais aussi le fait que le lac Ladoga est lui même dominé par les apports d'un grand lac en amont.

\subsubsection*{{\fontfamily{lmss}\selectfont Intérêt économique}}
Riche d'une biodiversité unique, le bassin de la Neva et notamment ces deux grands lacs en font une destination touristique prisée notamment par les agglomérations proches (4.7 millions d'habitants). Cet environnement unique reste fragile et une partie des berges du lac Ladoga est aujourd'hui protégée par la réserve naturelle de Nizhnesvirsky \citep{malmqvist2009}.\\
La Neva est l'unique ressource d'eau potable alimentant la ville de Saint-Pétersbourg et d'autres villes de Karélie pour environ 1.1 km$^{3}$ d'eau par an \citep{rukhovets2010}.\\

La présence de ce système hydrologique assure à la région un essor économique considérable qui passe par une proportion et une grande diversité d'industries. Au-delà du niveau socio-économique élevé, la pression anthropique est forte et pose des enjeux environnementaux majeurs notamment pour la préservation de la qualité des eaux des lacs \citep{rukhovets2010}

En plus de ces émissaires naturels, la Neva et le lac Ladoga sont connectés à un réseau national de canaux, qui permettent le transport de marchandises vers la Volga, la Mer Noire et la Mer Blanche et contribuent donc à l'indépendance économique de Saint-Pétersbourg \citep{malmqvist2009}. \\
Les lacs sont aussi utilisés pour la production électrique et les deux principales usines hydroélectriques se situent sur le Svir' en amont du lac Ladoga.

\section{{\fontfamily{lmss}\selectfont Les forçages globaux}}
\label{sec:forcing_blog}

\subsection{{\fontfamily{lmss}\selectfont Les précipitations}}
\label{sec:e2O}
Dans son étude \citet{decharme2019} a utilisé et testé différents forçages atmosphériques pour l'évaluation à l'échelle globale du modèle ISBA-CTRIP.\\
Les premiers forçages sont les Princeton Global Forcing \citep[PGF; {\small \url{https://rda.ucar.edu/datasets/ds314.0/}}]{sheffield2006} pour la période 1978-2014. Les cumuls de précipitations horaires PGF sont issus des réanalyses NCEP-NCAR issus d'observations des variables atmosphériques corrigées par des cumuls mensuels observés par le Global Precipitation Climatology Center (GPCC).\\
Le deuxième forçage est issu de réanalyses du projet Earth2Observe (E2O) et plus spécifiquement des réanalyses Tier-2 Water Resources (WRR2). Ces réanalyses proviennent des produits ERA-Interim ({\small \url{https://www.ecmwf.int/en/forecasts/datasets/reanalysis-datasets/era-interim}}) qui sont combinées aux observations mensuelles Multi-Source Weighted-Ensemble Precipitation \citep[MSWEP,]{beck2017}.\\

L'utilisation de plusieurs jeux de forçages permet l'analyse des biais introduits par la réponse des variables de surface et de sub-surface aux conditions atmosphériques. Cependant, \citet{decharme2019} a montré que les forçages E2O avaient de meilleures performances pour l'estimation des intensités de précipitations et permettait notamment une meilleure simulation des débits de rivières. Ces forçages sont disponibles pour la période 1978-2014.\\
Pour ces raisons, le modèle ISBA dans sa version diffusive sera forcé pour la suite de l'étude par les forçages ERA-Interim E2O.

\subsection{{\fontfamily{lmss}\selectfont L'évaporation}}
\label{sec:flake_globe}

Comme présenté dans la section \ref{sec:config_rhone}, les flux de masse au niveau des lacs sont déduits d'une estimation du bilan entre l'évaporation et les précipitations. Les estimations d'évaporation, sur la période  1979-2014, sont issues d'une simulation globale avec FLake similaire à celle de \citet{voldoire2019} avec la configuration proposée par citet{lemoigne2016}. Le coefficient d'extinction est fixé à 0.5 m$^{-1}$ et la profondeur maximale des lacs à 60 $m$.\\ Ainsi, un calcul préliminaire simule un correction des forçages en prenant la différence entre les estimations de précipitations MSWEP et l'évaporation au-dessus du lac. Cela permet de déduire la masse d'eau qui contribue directement au bilan sur le masque de ruissellement du lac (voir section \ref{sec:MLake})
Les estimations d'évaporation sont issues d'une simulation globale avec FLake similaire à celle de \citet{voldoire2019}.\\

\section{{\fontfamily{lmss}\selectfont Les caractéristiques de TRIP}}
\label{sec:trip_globe_carac}

La version de CTRIP utilisée en global repose sur une version préliminaire développé par Simon Munier, chercheur au CNRM. Pour ce faire, l'essentiel des paramétrisations développées sur la France et à l'échelle globale \citep{decharme2010,decharme2012, decharme2019} ont été reprises ici. Afin de garantir la cohérence des réseaux hydrographiques à la résolution 1/12°, seuls certains paramètres comme la pente des rivières ou la largeur ont été recalculés pour s'adapter au changement de résolution.\\
Les figures \ref{globe_w} et \ref{globe_seq} donnent un aperçu du numéro de séquence ainsi que de la largeur des rivières pour les trois bassins d'études. Il est évident que ces bassins n'ont pas les mêmes dimensions que le bassin du Rhône et présentent donc un intérêt particulier pour la validation du modèle.\\
Pour cette évaluation globale, les schémas d'aquifères et de plaines d'inondations n'ont pas été pris en compte. Seul le modèle MLake modifie la dynamique de rivières.

\begin{figure}[h!]
\centering
\includegraphics[width=0.75\textwidth]{globe_w}
\caption{Largeur de rivière pour le bassin de l'Angara (A), le bassin de la Neva (B) et le bassin du Nil Blanc (C).}
\label{globe_w}
\end{figure}

~\\

\begin{figure}[h!]
\centering
\includegraphics[width=0.8\textwidth]{globe_seq}
\caption{Numéro de séquence pour le bassin de l'Angara (A), le bassin de la Neva (B) et le bassin du Nil Blanc (C).}
\label{globe_seq}
\end{figure}
\clearpage


\section{{\fontfamily{lmss}\selectfont Les données de validation}}
\label{sec:donnees_globe}

Comme nous l'avons vu en introduction de cette thèse, les mesures \textit{in situ} sont souvent rares, discontinues et se raréfient ces dernières années \citep{duan2013}. Dans ce contexte, les données issues de la télédétection spatiale revêtent une importance particulière puisqu'elles donnent accès à un suivi fiable, quasi-continu de la majorité de la surface. En outre, la télédétection assure un suivi de la ressource même dans des zones inaccessibles \citep{avisse2017}.\\
Pour autant, certaines variables comme les débits sont difficilement suivies depuis l'espace et l'évaluation reste principalement contrainte par la disponiblité en données de terrain.
\subsection*{{\fontfamily{lmss}\selectfont Données de débits}}
\label{sec:donnees_globe_debits}

L'évaluation des simulations pour les débits repose sur plusieurs bases de données. Pour chaque site d'étude, les séries temporelles de la station choisie doivent couvrir la période d'étude 1978-2014 avec, au minimum, trois ans de données continues sur une période totale de 10 ans et pour un bassin de drainage minimum de 10000 km$^{2}$. Dans le cas où deux stations se trouvent sur la même maille CTRIP, la station choisie est celle qui possède l'aire de drainage la plus grande.\\
Concernant l'Angara, la station choisie se situe à Irkoutsk au niveau du barrage hydroélectrique du même nom et localisé à 66 km de l'exutoire du lac. Les mesures sont issues de la base de données globale du GRDC.\\
Pour la Neva, le site de mesures choisi se trouve à la station de Novosaratovka. Les données de débits sont issues de la base de données ARCTICNET \citep{lammers2001}.\\
Enfin, le Nil Blanc dans sa partie amont n'est présent dans aucune base de données et nous nous sommes confrontés à un manque évident de mesures dans cette partie du monde. Que ce soit dans la base GRDC, plus grande base de données de débits, ou dans des bases plus locales, aucune donnée n'était assez fiable pour le lac Victoria. Les mesures de débits faites au barrage de Nalubaale sont incomplètes du fait du non-respect des règles de gestion (voir en section \ref{sec:discussions_globe}). Une série temporelle du Nil Blanc à Jinja a été reconstruite par \citet{vanderkelen2018a} pour la période 1950-2006. Ces données ont été gracieusement fournies pour l'évaluation de cette étude. Ces données reposent donc sur un mélange de mesures directes et indirectes qui n'ont pas forcément les mêmes méthodes d'échantillonnage.
\clearpage
\subsection*{{\fontfamily{lmss}\selectfont Données sur les variations de hauteurs}}

Contrairement au Léman, dont les hauteurs ont été évaluées sur la base de stations de jaugeage, l'évaluation globale des variations de niveaux de lac s'est appuyée sur la plateforme Hydroweb  \citep[\url{http://hydroweb.theia-land.fr/}]{cretaux2011}. Cette plateforme fournit des mesures altimétriques à la résolution centimétrique pour un peu plus de 1000 fleuves et 230 lacs à travers le monde. Ces données sont généralement accessibles depuis 1993 et possèdent une estimation des erreurs. En plus des données altimétriques, la plateforme propose des observations d'emprise au sol des lacs et estime le volume associé.\\
Pour nos sites d'études, toutes les informations sont accessibles depuis Hydroweb.


\section{{\fontfamily{lmss}\selectfont Intégration des lacs sur les bassins versants}}
\label{sec:eval_globe}

Comme nous l'avons vu pour le bassin du Rhône, le modèle nécessite quelques années de spin-up avant d'atteindre un niveau d'équilibre. Cette durée est variable suivant les lacs mais s'élève généralement autour de quelques années. Compte tenu des observations disponibles, l'évaluation portera sur la période 1983-2014 pour les débits de l'Angara et de la Neva, et sur la période 1983-2006 pour le Nil Blanc. Concernant les niveaux des lacs, les données issues d'Hydroweb imposent une période d'évaluation de 1993 à 2014 pour les trois lacs considérés. \\
Comme pour l'étude locale, l'évaluation et la validation se font en deux étapes. Tout d'abord l'impact de MLake est évalué sur les performances des simulations de CTRIP. Ensuite, le modèle est confronté aux observations décrites dans la section \ref{sec:donnees_globe}.\\

En complément des résultats sur le test de sensibilité du chapitre précédent, un test a été reconduit afin d'évaluer l'influence de la largeur du seuil sur d'autres cas et avec un intervalle de facteur multiplicatif plus étendu.\\
Pour mener à bien le test de sensibilité plusieurs configurations ont été testées. Les caractéristiques de ces simulations sont regroupées dans le tableau \ref{ctrip_config_globe}.

{\renewcommand{\arraystretch}{1.2}
\begin{table}[h!]
 \caption{Configuration des différentes simulations effectuées sur chacun des bassins fluviaux.}
 \label{ctrip_config_globe}
 \begin{tabularx}{\textwidth}{XXX}
 \hline
 Configuration &Forçages &Détails\\
 \hline
  $ctrip\_nolake$&Earth2Observe& \footnotesize{Simulation de référence ISBA-CTRIP sans activation du modèle de lac}\\
    $ctrip\_mlake\_w0.1\_flake$&Earth2Observe, MSWEP&\footnotesize{ISBA-CTRIP-MLake dont la largeur de rivière est multipliée par 0.1}\\
    $ctrip\_mlake\_w0.5\_flake$&Earth2Observe, MSWEP&\footnotesize{ISBA-CTRIP-MLake dont la largeur de rivière est multipliée par 0.5}\\
      $ctrip\_mlake\_w0.7\_flake$&Earth2Observe, MSWEP&\footnotesize{ISBA-CTRIP-MLake dont la largeur de rivière est multipliée par 0.7}\\
  $ctrip\_mlake\_w1\_flake$&Earth2Observe, MSWEP&\footnotesize{ISBA-CTRIP-MLake dont la largeur de rivière est égale à la largeur de la rivière à l'aval}\\
  $ctrip\_mlake\_w1.5\_flake$&Earth2Observe, MSWEP&\footnotesize{ISBA-CTRIP-MLake dont la largeur de rivière est multipliée par 1.5}\\
  $ctrip\_mlake\_w2\_flake$&Earth2Observe, MSWEP&\footnotesize{ISBA-CTRIP-MLake dont la largeur de rivière est multipliée par 2}\\
  $ctrip\_mlake\_w4\_flake$&Earth2Observe, MSWEP&\footnotesize{ISBA-CTRIP-MLake dont la largeur de rivière est multipliée par 4}\\
  $ctrip\_mlake\_w5\_flake$&Earth2Observe, MSWEP&\footnotesize{ISBA-CTRIP-MLake dont la largeur de rivière est multipliée par 5}\\
  \hline
 \end{tabularx}
\end{table}}

\clearpage
\subsection{{\fontfamily{lmss}\selectfont Apport des lacs sur les simulations CTRIP}}
~\\
Les résultats sur les trois bassins de l'évaluation concordent avec les conclusions émises dans le chapitre précédent concernant le Rhône. Dans un souci de lisibilité, la figure \ref{subset_2005_2008} donne un aperçu des débits simulés sur la période 2005-2008, cependant l'analyse est faite sur la période entière 1983-2014.\\

\begin{figure}[h!]
\includegraphics[width=1.\textwidth]{subset_q_2005_2008}
\caption{Chroniques de débits simulés par CTRIP-MLake et des observations issues de la banque de débits GRDC et ARCTICNET sur la période 2005-2008. A) Nil Blanc (Ouganda), B) Angara (Russie), C) Neva (Russie).}
\label{subset_2005_2008}
\end{figure}
\clearpage
L'ajout du bilan de masse des lacs entraîne une diminution de la variabilité et de l'amplitude des débits simulés. L'impact le plus significatif est celui du lac Victoria sur le débit du Nil avec un effet aussi marqué sur le débit moyen que sur la variabilité. Ainsi, la seule introduction de MLake réduit le débit moyen de 37\% et la variabilité de 55\%. Les résultats des simulations pour l'Angara et la Neva sont plutôt similaires à ceux du Rhône avec un débit moyen annuel simulé peu ou pas impacté par la présence des lacs (augmentation de 4\% du module annuel de l'Angara et diminution de 0.1\% de celui de la Neva). Ces effets s'expliquent par la différence de forçages entre les simulations avec ou sans lac. La prise en compte explicite des lacs introduit un terme d'évaporation issu de FLake qui modifie les forçages de surface et donc le volume d'eau effectivement reçu sur la zone considérée.\\



\noindent La réduction des amplitudes est dépendante de la taille du lac et c'est sur le lac Baïkal que l'on observe la baisse la plus importante avec une réduction de 63\%. Bien sûr cela n'est pas seulement la conséquence d'une diminution des pics de débits, mais c'est aussi dû à une augmentation des basses eaux. Pour l'Angara, le quantile $Q_{90}$ diminue de 4046 $m^{3}.s^{-1}$ à une moyenne de 2 633 $m^{3}.s^{-1}$ (intervalle [2 023 $m^{3}.s^{-1}$ - 3 185 $m^{3}.s^{-1}$]) sur la période 1983-2014. Pour le quantile $Q_{10}$, sa valeur croît de 148 $m^{3}.s^{-1}$ à 1 133 $m^{3}.s^{-1}$ (intervalle [651 $m^{3}.s^{-1}$ - 1 590 $m^{3}.s^{-1}$]) sur la même période. Cela se traduit par un lissage de l'hydrogramme et une atténuation de la variabilité haute fréquence. \\
L'effet est similaire sur le bassin de la Neva avec une réduction de la variabilité de 49\% ($\overline{\sigma}_{ctrip\_mlake}$ = 534 m$^{3}$.s$^{-1}$ et $\sigma_{ctrip\_nolake}$ = 1 095 m$^{3}$.s$^{-1}$). Par ailleurs, le quantile $Q_{90}$ de la Neva diminue de 3 973 m$^{3}$.s$^{-1}$ à 2 970 m$^{3}$.s$^{-1}$ (intervalle [2 734 m$^{3}$.s$^{-1}$ - 3 860 m$^{3}$.s$^{-1}$])  alors que le quantile $Q_{10}$ augmente de 1 045 m$^{3}$.s$^{-1}$ à une moyenne de 1 729 m$^{3}$.s$^{-1}$ (intervalle [1 332 m$^{3}$.s$^{-1}$ - 2 264 m$^{3}$.s$^{-1}$]).\\

\noindent Ces résultats indiquent la prédominance de la dynamique des lacs dans le cycle hydrologique local et la contribution de ces grands lacs à leurs effluents. Ainsi, les trois rivières sont les seuls exutoires des lacs et drainent donc l'intégralité des effluents du lac.\\
L'apport de MLake a donc tendance à lisser les hydrogrammes issus des simulations de débits, à réduire les volumes d'eau transférés vers l'aval durant les périodes de hautes eaux et à soutenir les étiages par un apport de masse plus important en période de basses eaux.\\

Enfin, le test de sensibilité préliminaire permet de confirmer la tendance qui se dégageait sur le Rhône. Un échantillonnage plus important a été utilisé afin de prendre en compte une amplitude permettant de comparer le comportement du modèle et du réseau hydrographique dans des situations extrêmes. Lorsque la largeur du seuil augmente, le transfert d'eau est plus rapide et l'hydrogramme se rapproche des simulations de référence. Au contraire, lorsque le seuil est plus étroit l'effet tampon du lac augmente et la variabilité chute. Dans le cas de la configuration avec lac, l'amplitude maximale augmente de 105\% entre la configuration $ctrip\_mlake\_w1\_flake$ et $ctrip\_mlake\_w5\_flake$.

\subsection{{\fontfamily{lmss}\selectfont Validation du modèle CTRIP-MLake}}
\label{sec:validation_globe}


Vu les différences induites par l'introduction du bilan de masse des lacs sur les simulations de CTRIP, il convient de les valider pour estimer les performances du modèle.\\
La validation est menée en deux étapes, d'abord en comparant les débits simulés puis en analysant la pertinence du diagnostic sur les variations de niveau d'eau.

\subsubsection*{{\fontfamily{lmss}\selectfont Débits simulés}}
\begin{figure}[h!]
\includegraphics[width=1.\textwidth]{subplot_q_flake_globe}
\caption{Chroniques de débits simulés par CTRIP-MLake et des observations issues de la banque de débits GRDC et ARCTICNET sur la période 1983-2014. A) Nil Blanc (Ouganda), B) Angara (Russie), C) Neva (Russie).}
\label{subplot_q_globe}
\end{figure}

La figure \ref{subplot_q_globe} informe sur les performances du modèle CTRIP-MLake à simuler les débits. Une tendance se dégage quant à la meilleure simulation des débits par la configuration $\small{ctrip\_mlake\_w0.5}$. L'intérêt d'introduire les lacs assure un meilleur cycle saisonnier pour tous les bassins fluviaux (Figure \ref{seasonal_q_globe}). L'effet de lissage de l'hydrogramme et la réduction de variabilité permettent aux simulations de se rapprocher des débits observés notamment sur les bassins de l'Angara et de la Neva. Contrairement à la simulation de référence dont l'amplitude annuelle est élevée, les simulations CTRIP-MLake reproduisent une variabilité correspondant aux observations.\\

\begin{figure}[h!]
\includegraphics[width=1.\textwidth]{seasonal_q_flake_globe}
\caption{Cycles annuels des débits simulés par CTRIP-MLake et des observations de la banque de débits GRDC et ARCTICNET sur la période 1983-2014. A) Nil Blanc (Ouganda), B) Angara (Russie), C) Neva (Russie).}
\label{seasonal_q_globe}
\end{figure}

Cela se confirme lorsqu'on compare les débits moyens annuels simulés et observés. Dans tous les cas, la réduction des débits (ou l'augmentation dans le cas de l'Angara) tend à rapprocher le débit moyen simulé de celui qui est observé. Les résultats sont les mêmes pour la variabilité qui est réduite et se rapproche des observations. Le cas de la Neva présente les meilleurs résultats avec un ratio de débit moyen annuel de 1.02, celui de l'Angara est de 0.96 et celui du Nil est de 0.84. Le ratio des écarts-types est de 0.93 pour la Neva, 1.21 pour l'Angara et 2.39 pour le Nil Blanc. Ces améliorations des débits simulés proviennent de la capacité de rétention plus importante des lacs par rapport à celle des rivières.
Excepté pour le Nil Blanc, on constate la bonne corrélation du cycle annuel sur les bassins et une amplitude correcte entre les hautes et basses eaux. Cela est confirmé par le cycle saisonnier qui montre une amélioration des simulations de la variabilité interannuelle quand les lacs sont pris en compte.\\
\clearpage
En matière de scores, la figure \ref{plot_scores_globe} montre une faible performance de CTRIP-MLake à reproduire les débits de l'Angara. 

\begin{figure}[h!]
\includegraphics[width=1.\textwidth]{plot_scores_globe}
\caption{Distribution des critères de NSE et KGE pour chaque site suivant le facteur multiplicatif appliqué à la largeur du seuil sur la période 1983-2014. A) Nil Blanc (Ouganda), B) Angara (Russie), C) Neva (Russie).}
\label{plot_scores_globe}
\end{figure}

Sur le bassin de l'Angara, le critère NSE est positif pour trois configurations avec une amélioration maximale de 12.4 points pour la configuration $ctrip\_mlake\_w0.5$ (NSE = 0.2). Le détail sur le score de NSE$_{log}$ pour cette même configuration (0.12) rend compte d'une meilleure représentation des étiages. Même si la configuration $ctrip\_mlake\_w1$ présente des scores moins convaincants pour le critère NSE (-0.11), elle simule de façon correcte les débits puisque son score KGE est élevé (0.47). La comparaison des scores de NSE et KGE pour cette configuration montre l'influence de la corrélation sur la dégradation des critères NSE. Dans tous les cas, les résultats pour le KGE sont meilleurs ($\overline{KGE}=0.19$) ce qui contribue à une amélioration moyenne de 2.21 points. En tout état de cause, le NIC moyen de 0.87 indique l'importance de prendre en compte le lac Baïkal dans le bassin. Le lac Baïkal est glacé de janvier à mai-juin et entouré de permafrost. Cette saisonnalité spécifique est la principale contributrice au cycle hydrologique régional avec des pics de débits causés par la fonte de glace et de neige. Comme le NSE est particulièrement sensible à ces pics et encore plus au ruissellement de fonte saisonnier, cela confirme le choix de plutôt considérer le KGE dans l'analyse. \\

Pour la Neva, les résultats sont similaires avec des scores positifs pour les mêmes configurations que l'Angara. Le score moyen pour le critère NSE est de 0.19 points (amélioration de 3.13 points) et de 0.17 pour le NSE$_{log}$ (amélioration de 3.37). L'apport global de MLake sur ce bassin est nette avec un NIC moyen de 0.56. Les processus de gel et dégel sont similaires pour la zone du lac Ladoga et les conclusions sur le NSE sont donc les mêmes ici. Il convient de privilégier le KGE sur ce bassin. Les hydrogrammes présentent un décalage temporel assez visible avec des pics de débits simulés en avance par rapport aux observations. Ce décalage impacte fortement les scores et notamment le NSE.\\

Le cas du Nil Blanc reste donc un cas particulier. Sur la période de validation 1983-2006, les critères NSE et KGE sont constamment négatifs avec respectivement -16.06 et -2.77. Ces scores révèlent une inadéquation à représenter les débits observés que ce soit pour la variabilité, le phasage temporel ou l'amplitude. Ainsi, même si on observe une réduction notable du module annuel et du coefficient de variation cela n'est pas suffisant pour simuler correctement les débits. Seule la saisonnalité reste bien respectée et notamment sur la période pré-2000 et il est donc impossible de conclure sur un quelconque apport.\\

\subsubsection*{{\fontfamily{lmss}\selectfont Variations du niveau d'eau}}

Les résultats sur les niveaux des lacs sont présentés sur la figure \ref{h_globe_obs} et la figure \ref{h_globe_seasonal}. Les tableaux de scores se trouvent dans la section \ref{chap:annexe_h_globe} de l'annexe \ref{chap:resultats-etude-globale}. \\

\begin{figure}[h!]
\includegraphics[width=1.\textwidth]{subplot_h_new_flake}
\caption{Séries temporelles des variations de niveau d'eau simulées par CTRIP-MLake et des observations issues d'Hydroweb sur la période 1993-2014. A) Lac Victoria (Ouganda), B) lac Baïkal (Russie), C) Lac Ladoga (Russie).}
\label{h_globe_obs}
\end{figure}

~\\

\begin{figure}[h!]
\includegraphics[width=1.\textwidth]{seasonal_h_flake_globe}
\caption{Cycles saisonniers des variations de niveau d'eau simulées par CTRIP-MLake et des observations issues d'Hydroweb sur la période 1993-2014. A) Lac Victoria (Ouganda), B) lac Baïkal (Russie), C) Lac Ladoga (Russie).}
\label{h_globe_seasonal}
\end{figure}

\clearpage

Pour le lac Baïkal, les hauteurs simulées correspondent aux observations avec une cohérence sur la variabilité interannuelle et des amplitudes qui restent proches. Un léger décalage apparaît à partir de 2002 mais dans l'ensemble le cycle et la temporalité sont respectés. La corrélation entre les simulations et les observations est bonne ($\overline{r}=0.75$) et les meilleurs scores sont atteints pour la configuration $ctrip\_mlake\_w05$. Cela est confirmé par des écarts-types similaires ($\overline{\sigma_{s}}=0.28m$ et $\sigma_{o}=0.28m$). Enfin la variabilité relative $\alpha$=1 confirme les performances de MLake à représenter les variations du lac Baïkal tant sur la saisonnalité que sur l'amplitude.

L'analyse du cycle saisonnier montre un décalage temporel de deux mois sur les basses eaux du lac Baïkal. Ce décalage est en partie rattrapé à l'automne au moment des pics de débits mais reste quand même visible. Malgré ce décalage, les pentes des courbes sont quasiment parallèles et montrent une adéquation du modèle à simuler la dynamique du lac Baïkal.\\

Les résultats pour le lac Ladoga sont similaires. En effet, on observe aussi un décalage temporel, ici d'un mois, entre les simulations et les observations. Ce décalage fait que les pics de débits sont atteints précocement dans l'année. \textit{A contrario}, les étiages sont très bien simulés avec une très bonne précision de la baisse automnale des niveaux. Ce décalage a tendance à réduire les performances du modèle notamment en jouant sur la corrélation. Celle-ci est plutôt basse ($\overline{r}$ = 0.37) malgré une allure qui semble concorder. Cette faible corrélation est aussi la conséquence de la sous-estimation des débits dans les années 1994-1995 et la surestimation pour les étiages de 2003. Malgré cela les écarts-types sont satisfaisants ($\overline{\sigma_{s}}=0.23m$ et $\sigma_{o}=0.26m$) pour une variabilité relative de $\alpha$ = 0.88. \\

Les résultats sur le lac Victoria sont différents par rapport aux deux lacs précédents. Sur la figure \ref{h_globe_obs} se distinguent deux périodes: la période pré-2002, où les niveaux d'eau sont plutôt bien représentés que ce soit la variabilité ou la temporalité, puis une période post-2002 où l'on observe un décrochage des niveaux mesurés du lac de plus d'un mètre. Durant les années 2003-2007, le lac semble atteindre un nouvel état d'équilibre et malgré ce décrochage, la variabilité saisonnière des simulations reste corrélée avec le signal observé.\\
La section \ref{sec:victoria} a déjà présenté le cas spécifique du bassin du lac Victoria qui possède une dépendance aux conditions atmosphériques et à l'anthropisation de son effluent. Pour éviter d'introduire des biais provenant des règles de fonctionnement anthropique, l'analyse des simulations se fait, de façon similaire aux débits, sur la période pré-2004.\\
Même si l'exutoire du lac Victoria est anthropisé les simulations représentent bien l'amplitude et la temporalité des variations de niveau du lac avant 2004. La période de hautes eaux du lac Victoria des années 1998-2000 est très bien simulée. La dispersion sur cette période confirme l'analyse visuelle avec une variabilité relative de $\alpha$ = 1.1 ($\overline{\sigma_{s}}$ = 0.37m et $\sigma_{o}$ = 0.35m). En lien avec la représentation correcte de la variabilité, la corrélation est aussi forte sur la période ($\overline{r}$ = 0.83). La figure \ref{h_globe_seasonal} confirme ces deux derniers résultats avec une simulation adéquate de la période de hautes eaux de début d'été et de la période de basses eaux d'octobre. 

\subsubsection*{{\fontfamily{lmss}\selectfont Test de sensibilité}}
Le test de sensibilité a été reconduit ici sur une gamme de valeurs plus grande. L'approche reste 'one at a time' mais s'applique maintenant à une gamme plus large et surtout aux résultats sur les débits et sur les hauteurs.
Les diagrammes de Taylor pour les hauteurs (Figure \ref{h_globe_taylor}) indiquent une sensibilité des simulations au bassin considéré. Ainsi, sur le lac Baïkal, même si le paramètre reste robuste, la configuration $ctrip\_mlake\_w1$ présente de meilleurs résultats. En particulier, la distribution des scores fait apparaître un maximum de performances autour de cette configuration.
À l'opposé, les résultats pour le lac Ladoga sont beaucoup moins nets et mise à part la configuration $ctrip\_mlake\_w01$, toutes les configurations se tiennent dans une gamme de scores proches. On peut toutefois noter une légère amélioration dans le cas des facteurs multiplicatifs $0.7$ et $1$. Les résultats sont similaires sur le lac Victoria avec globalement des scores moins bons mais qui présentent de meilleurs résultats lorsque la largeur du seuil est réduite.\\
Dans l'ensemble, la largeur du seuil n'a qu'une faible influence sur les simulations de niveau d'eau même si cela reste variable suivant les bassins. \\

\begin{figure}[h!]
\includegraphics[width=1.\textwidth]{taylor-h-globe}
\caption{Diagramme de Taylor représentant, pour les différentes configurations du modèle, les performances de CTRIP-MLake à simuler les marnages sur la période 1983-2014 pour le Nil Blanc (A) et l'Angara (B), sur la période 1983-2006 pour la Neva (C).}
\label{h_globe_taylor}
\end{figure}

Les conclusions sont différentes si l'on se focalise sur les simulations de débits (Figure \ref{q_globe_taylor}). Dans ce cas, les performances de CTRIP sont sensibles à la largeur du seuil. Sur les trois bassins, l'introduction du schéma de lac améliore distinctement les performances. La largeur du seuil impacte donc les scores de débits et on remarque qu'une diminution d'un facteur autour de $0.5$ et $0.7$ présente les meilleurs résultats. Dans le cas du lac Victoria, il n'est pas surprenant de retrouver des résultats inexploitables en l'état et des scores réalistes seulement pour la configuration $ctrip\_mlake\_w01$. En effet, les débits observés sont très faibles et ne présentent aucune variabilité saisonnière ce qui traduit à la fois une grande dispersion et une très faible corrélation avec les simulations.

\begin{figure}[h!]
\includegraphics[width=1.\textwidth]{taylor_q_globe}
\caption{Diagramme de Taylor représentant, pour les différentes configurations du modèle, les performances de CTRIP-MLake à simuler les débits sur la période 1983-2014 pour le lac Baïkal (A) et le lac Ladoga (B), sur la période 1983-2006 pour le lac Victoria (C).}
\label{q_globe_taylor}
\end{figure}

En conclusion, le paramètre de largeur du seuil est robuste pour la simulation des hauteurs d'eau mais reste un paramètre important dans le contrôle du débit à l'exutoire. Si l'on se réfère au test de sensibilité, les configurations privilégiées pour une meilleure représentation de ces débits des lacs se trouvent être entre $ctrip\_mlake\_w05$ et $ctrip\_mlake\_w1$. Dans l'ensemble, il convient d'éviter d'augmenter la largeur du seuil au-delà de la valeur prescrite par CTRIP.

\clearpage

\section{{\fontfamily{lmss}\selectfont Discussions}}
\label{sec:discussions_globe}

Les résultats présentés dans la section précédente confirment la capacité du modèle CTRIP-MLake à simuler les débits des principaux bassins fluviaux et des variations de niveaux des principaux lacs. Cela confirme aussi l'apport d'un modèle non-calibré comme MLake pour décrire le bilan de masse des lacs.\\
Cependant les performances du modèle sont limitées pour plusieurs raisons listées dans la suite de la section.

\subsection{{\fontfamily{lmss}\selectfont Anthropisation}}

Les grands bassins fluviaux sont, dans la majorité des cas, anthropisés mais à des degrés différents. Que ce soit pour les besoins de l'agriculture, de l'industrie, du secteur de l'énergie ou de l'approvisionnement en eau potable, les réserves de surface, quand elles existent, sont la ressource la plus accessible. Même si ces réservoirs offrent des services écosystémiques indéniables, l'augmentation de la pression anthropique modifie les variables de surface et de routage comme les ruissellements ou les débits \citep{grill2019, best2019}.\\
Les lacs ne font pas exception à ce constat et la manière dont leurs niveaux varient est, dans de nombreux cas, liée à des évolutions d'origine anthropique \citep{wurtsbaugh2017, woolway2020}.\\

Au vu des résultats, l'Angara est le fleuve qui semble avoir les débits les plus régulés avec des variations très fortes et un seuil du débit minimal aux alentours de 1500 m$^{3}.s^{-1}$. Ce régime altéré est produit par l'effet conjoint des trois barrages construits sur l'Angara (Irkoutsk, Bratsk et Ust'-Illim). Des études ont déjà mis en avant cet effet et montré que l'impact des réservoirs était perceptible assez loin sur le Yenissei \citep{adam2007}. Dans notre cas, c'est le barrage d'Irkoutsk qui modifie le régime du fleuve. Sa construction a, par ailleurs, aussi modifié les conditions du lac Baïkal qui a vu son volume augmenté de 37 km$^{3}$ et son niveau de 0.79 m \citep{sinyukovich2019}. À cela s'ajoute une baisse d'un tiers de l'amplitude des débits \citep{vyruchalkina2004}.\\  
L'objectif premier du lac est de réguler les flux de masse à l'aval durant les pics de crues. Pour autant, les règles de gestion des eaux imposent une régulation saisonnière calée sur le cycle naturel de la rivière.\\
Cette variabilité spécifique explique en grande partie les faibles scores de NSE impactés par la très faible corrélation entre les débits simulés et observés. Ils sont aussi liés à la physique de MLake dont l'objectif est de reproduire le cycle naturel du bilan d'eau des lacs. 

\subsection{{\fontfamily{lmss}\selectfont Baisse historique des niveaux du lac Victoria}}

L'anthropisation des eaux du lac Victoria et de son effluent le Nil Blanc est assez nette tant sur les observations de débits que sur les niveaux du lac. Les modifications anthropiques et la gestion du barrage de Nalubaale sont les raisons principales expliquant les biais entre simulations et observations. \\
Pour comprendre les enjeux de la gestion des eaux du Nil, il est important de décrire succinctement le contexte local. Le barrage d'Owen Falls a été mis en service en 1954 afin de profiter du débit du Nil à cet endroit pour satisfaire aux besoins en électricité. La construction du complexe a été soumis à des règles et notamment à ce qu'on appelle l'"Agreed Curve", un accord de gestion de l'effluent mis en place pour définir le débit minimal garantissant un cycle naturel du lac Victoria. En 2000, le complexe a été agrandi avec la construction d'un deuxième barrage sous l'impulsion de la Banque Mondiale \citep{kull2006}. \\
De nombreuses études font état d'un non-respect de cet accord comme principale raison de la baisse sévère du niveau du lac sur la période 2004-2006 \citep{kull2006, sutcliffe2007, vanderkelen2018a, getirana2020}. Selon ces mêmes études, la part anthropique liée au déclin du lac Victoria serait à hauteur de 55\% tandis que les 45\% restant seraient attribués à la sécheresse historique touchant la région sur la même période. Selon \citet{vanderkelen2018a}, les estimations de précipitations issues des données PERSIANN-CDR font état d'une baisse des cumuls entre 2004 et 2005 de 13\% comparé à la moyenne climatologique.\\

En analysant les forçages E2O, la sécheresse est notable pour les années 2004-2005 et équivaut à une anomalie de lame d'eau de 0.20 mm. En réponse, la baisse moyenne de niveau du lac associée est de 0.39 m (pour un intervalle compris entre 0.25 m et 0.57 m suivant les largeurs de seuil) ce qui est insuffisant pour retrouver le signal observé de -1.04 m.\\
Cette baisse est notamment expliquée, sur cette période, par une hausse significative des lâchés de barrage \citep{getirana2020}. La baisse non-naturelle supplémentaire serait de l'ordre de 0.61m, valeur qui expliquerait dans notre étude le biais entre les deux minimas \citep{sutcliffe2007}.\\

Aucun module de gestion anthropique des eaux n'est, à ce jour, intégré à CTRIP et il n'est donc pas possible de vérifier l'impact du barrage sur nos simulations. Néanmoins MLake simule les variations naturelles du lac Victoria et les résultats permettent de confirmer indirectement que la part du non-respect du cycle naturel a engendré une baisse significative des niveaux de lac entre 0.79 m et 0.47 m sur la période 2004-2005.

\subsection{{\fontfamily{lmss}\selectfont Déphasage temporel des débits des rivières boréales}}

L'anthropisation n'est pas la seule raison des différences entre les simulations et les observations. Elles peuvent aussi provenir de biais issus de la représentation physique des processus dans ISBA-CTRIP. Ainsi le cycle saisonnier du lac Baïkal présente un déphasage systématique de deux mois entre les simulations et les observations. Ce phénomène n'est pas visible sur les débits de l'Angara du fait des manœuvres du barrage à Irkoutsk. Par contre, ce déphasage se retrouve, dans un ordre de grandeur similaire, sur les débits de la Neva dont le pic de débit est atteint précocement dans les simulations.\\

Comme le révèle \citet{decharme2019}, les simulations de débits de rivières pour les régions boréales sont contraintes par le bilan d'énergie de surface et la capacité du modèle à simuler la fonte nivale. Dans notre cas, ISBA-CTRIP résout un unique bilan d'énergie pour la végétation et ne prend donc pas en compte les effets radiatifs de la forêt sur la neige déposée en dessous. Au sein d'une maille, la température issue du bilan d'énergie représente alors celle de tout le continuum sol-végétation-neige et la fraction de sol reçoit une fraction de rayonnement solaire plus importante. Ce biais n'est pas significatif pour nos régions mais a des conséquences importantes pour les zones où la végétation est dense comme les régions scandinaves ou celle du lac Baïkal. Ainsi la partie sud-est du lac Baïkal représente une surface de 200 500 km$^{2}$ exclusivement recouverte de conifères, appelée "forêt de conifères transbaïkal", et soumise à un climat subarctique. Pour le lac Ladoga, la situation est similaire. Dans ces zones, la fraction plus importante de rayonnement solaire reçue par le sol réchauffe plus rapidement la surface du sol, ce qui provoque un dégel de la glace contenue dans le sol plus rapide que la neige qui le recouvre. Il y a donc une inversion du processus de fonte et la fonte nivale, au lieu de contribuer à l'augmentation des débits, s'infiltre et contribue à l'augmentation du stock d'eau dans le sol.\\

La prise en compte d'un bilan d'énergie distinct pour la végétation, la neige et le sol comme proposé par \citet{boone2017} est requise pour décrire de manière précise les interactions entre les variables de surface. Ainsi l'approche proposée dans le modèle MEB (Multi-Energy Balance) propose une représentation explicite du sol et de la canopée prenant aussi en compte une formulation des flux turbulents et une paramétrisation de la litière \citep{napoly2017}. Par rapport à ISBA, le modèle MEB considère un réservoir explicite de neige pour la canopée, ce qui permet une représentation des interactions de flux entre la surface et l'atmosphère sous le couvert forestier et réduit la vitesse du vent sous canopée.\\
Cette nouvelle paramétrisation induit un effet d'ombrage de la canopée sur le sol qui réduit l'amplitude des flux conductifs et le cycle diurne des températures dans le sol \citep{napoly2020}. L'effet principal de cette approche est de corriger l'épaisseur de neige simulé tout comme sa variabilité temporelle. La figure \ref{napoly2020} présente l'effet d'ISBA-MEB sur deux sites d'études au Canada. L'amélioration dans les simulations est notable et assure une cohérence entre le manteau neigeux simulé et observé.

\begin{figure}[h!]
\includegraphics[width=1.\textwidth]{napoly2020}
\caption{Cycle annuel de l'épaisseur des manteaux neigeux simulés et observés pour deux sites d'observations en Saskatchewan (Canada). Source: \citet{napoly2020}.}
\label{napoly2020}
\end{figure}

\begin{figure}[h!]
\includegraphics[width=1.\textwidth]{napoly_2020}
\caption{Cycle annuel de la température du sol entre la surface et 100 cm simulé par ISBA et MEB et observé sur un site d'pbservation en Saskatchewan (Canada). Source: \citet{napoly2020}.}
\label{napoly_2020}
\end{figure}

Comme on peut le voir sur la figure \ref{napoly_2020}, l'effet d'une meilleure simulation du manteau neigeux corrige le cycle annuel de température du sol. ISBA avait une tendance à simuler une couche de sol trop froide en hiver et à provoquer un réchauffement précoce et plus important de cette même couche en été par rapport aux observations. Grâce à la prise en compte d'un bilan distinct, ces gradients sont diminués et on observe une amélioration du cycle saisonnier pour les processus thermiques dans le sol \citep{napoly2020}.\\

Grâce à cette étude, il est possible de proposer des hypothèses sur l'impact de ce nouveau schéma sur l'hydrologie. En effet, la tendance d'ISBA à faire geler le sol de façon brusque et intense atténue les drainages et diminue la disponibilité en eau. Cela expliquerait le décalage temporel observé sur le lac Baïkal lors de la période de basses eaux. À l'inverse, la tendance d'ISBA à réchauffer le sol provoque une intense fonte neigeuse. Cette fonte contribue alors à un pic de débit précoce comme observé sur le bassin de l'Angara et de la Néva. Bien sûr, ces hypothèses doivent être vérifiées par une étude plus précise prenant notamment en compte le couplage en MEB et CTRIP.


\subsection{{\fontfamily{lmss}\selectfont Sensibilité à la largeur du seuil}}

En première approximation, le facteur multiplicatif 1 présente des résultats satisfaisants ce qui confirme que la largeur de la rivière en aval est un paramètre acceptable pour déterminer le débit de déversement. Cependant une étude de sensibilité plus détaillée est nécessaire car la modification de la largeur du seuil permet d'améliorer significativement les performances du modèle MLake. Cela permettra aussi de déterminer, dans le cas où une valeur optimale systématique est trouvée, une paramétrisation pour des lacs pour lesquels aucune donnée n'est disponible.\\
La sensibilité du modèle a été analysée par une méthode simple 'one at a time' permettant d'attribuer toute modification des résultats à cette seule variable.\\
Cette analyse révèle une amélioration globale des performances du modèle lorsque la largeur du seuil est réduite. Cet effet s'explique par une augmentation de l'effet de rétention au niveau du déversement lorsque le seuil est réduit. \\

Comme expliqué dans la section \ref{subsec:param_riv}, la largeur des rivières est calculée sur la base du débit moyen annuel sur le tronçon et appliquée au lac sous la forme d'une fraction de sa circonférence totale. Dans MLake, le lac est modélisé avec une surface assimilée à un cercle équivalent dont l'aire est prescrite par ECOCLIMAP. En considérant une hypsométrie linéaire, il vient que le lac est représenté sous la forme d'un cylindre. Dans ce cas, il n'y a pas d'effet de la profondeur sur la contraction ou l'expansion de l'aire du lac. De plus la circonférence du lac modélisée, sous forme de cercle, est plus petite que la circonférence réelle ce qui explique les meilleures performances dans le cas d'une largeur de seuil réduite. Puisque la largeur du seuil représente une fraction de la circonférence totale, sa réduction sans modification de la largeur augmente synthétiquement le rapport relatif entre les deux longueurs.\\
Une solution pour aborder ce problème repose sur l'accès à des observations, mais les données sur les caractéristiques de lacs sont peu disponibles et limitées notamment à cause des coûts humains et financiers que les campagnes de mesures nécessitent \citep{duan2013}. À cela s'ajoutent les limites de la télédétection pour mesurer certaines variables telles que la charge en eau au dessus du seuil. Il semble, en effet, assez compliqué d'avoir un accès direct à ce type de données en constante évolution sans station de mesures continues. Une autre méthode possible pourrait être d'estimer une largeur générique de seuil basée sur un regroupement des lacs selon leur géomorphologie.\\
Pour répondre à cette problématique, l'étude menée par \citet{bowling2010} propose dans le développement du module de lac de considérer une fraction dynamique de la circonférence totale proportionnelle  au niveau d'eau dans le lac. Cela introduit, par contre, des questions sur la morphologie et l'intérêt d'intégrer une description précise des propriétés géomorphologiques, non abordées à ce stade de l'étude. Pour autant, ces questions sur la morphologie des lacs sont primordiales et nécessitent d'être abordées.\\

La morphologie des lacs est une des principales sources d'incertitudes dans l'étude des propriétés lacustres. Comme nous venons de le voir, elle peut amener à des surestimations sur les différentes variables étudiées car non stationnaires ou difficilement mesurables, comme le débit. Parmi ces caractéristiques, la connaissance de la bathymétrie est sûrement une des clés pour une meilleure compréhension des processus physiques, biologiques et écologiques \citep{blais1995, yao2018}. De plus, la bathymétrie influence le temps de résidence, le marnage et tous les processus de mouvements d'eau comme les seiches \citep{fricker2000,bastviken2004}.\\
Ces questions, et plus particulièrement la problématique de la bathymétrie ont été abordées dans le cadre de cette thèse. L'étude est en cours de validation et des détails sont donnés dans l'annexe \ref{chap:morpho_lac} de cette thèse. L'annexe permet de poser le problème et d'amener des réponses quant au développement d'une bathymétrie lacustre à l'échelle globale.
\clearpage
\section{{\fontfamily{lmss}\selectfont Simulations à l'échelle du globe}}

Grâce aux résultats obtenus dans ces différentes régions, nous avons pu démontrer l'intérêt d'utiliser le modèle de bilan de masse non calibré MLake dans les simulations hydrologiques appliquées à différentes zones du globe. Que ce soit pour des lacs très anthropisés ou dominés par les forçages atmosphériques ou bien par leurs composantes fluviales, MLake améliore les performances de CTRIP dans les simulations de débits et intègre une variable pour le suivi des variations de niveau d'eau. \\
Le seul point de vue local est limitant et ne caractérise que peu la dynamique globale. Cependant, les simulations globales de CTRIP-MLake ne sont pas uniquement dépendantes du modèle MLake mais incluent d'autres schémas devant aussi être validés à 1/12°.\\

Les premières simulations globales CTRIP-MLake reposent sur une version préliminaire du modèle CTRIP à 1/12°, appelé CTRIP-12D, développé par Simon Munier, chercheur au CNRM. En reprenant les travaux existants en global sur la version de CTRIP au 0.5° \citep{decharme2019}, une première maquette de CTRIP-12D a été développée. Les résultats, incluant le modèle MLake, sont présentés ici et feront l'objet d'une future valorisation scientifique.

\subsection{{\fontfamily{lmss}\selectfont Configuration de CTRIP-12D global}}

Dans son application à l'échelle globale, le modèle CTRIP-12D reprend le cadre expérimental utilisé dans la section ci-avant (section \ref{sec:trip_globe_carac}), en cohérence avec l'étude de \citet{decharme2019}. Les processus physiques n'ont pas été modifiés et seules les paramètres ont été recalculés afin de prendre en compte le passage d'une résolution de 0.5° à 1/12°.\\

La simulation globale est effectuée en mode offline et les forçages atmosphériques sont issus des réanalyses Earth2Observe utilisées pour la validation du modèle (section \ref{sec:e2O}). La seule modification apportée concerne la non prise en compte de l'estimation de l'évaporation produite par FLake sur les lacs. Le calcul des variations de stock en eau dépend donc seulement du ruissellement et du drainage sur chaque maille du modèle. Le ruissellement total utilisé pour forcer CTRIP est issu d'une simulation d'ISBA-DF forcée par les réanalyses E2O sur la période 1978-2014.\\ 
Dans ce cadre, le modèle MLake continue d'utiliser les masques de lacs pour l'estimation des ruissellements entrants dans les lacs et l'intégration de ces mêmes lacs dans le réseau (section \ref{sec:masque_lac}) mais ceux-ci s'appliquent exclusivement aux ruissellements et drainage. Dans cette configuration, le couvert principal présent sur chaque maille dans la base ECOCLIMAP est considéré par ISBA qui calcule l'évapotranspiration résultante. Les estimations d'évapotranspiration d'un couvert végétal ou d'un sol nu sont différentes de l'évaporation au-dessus d'un lac mais une comparaison des débits produits par CTRIP-MLake avec les deux types de forçages sur les trois bassins de validation globale amène à des différences acceptables (Figure \ref{seasonal_flake_e20}). Seul le lac Victoria, qui possède un bilan hydrologique fortement dépendant de la composante atmosphérique présente un cycle saisonnier significativement différent. Il est donc possible de dégager une tendance générale de ces simulations sans pour autant amener une validation complète du schéma de lac à l'échelle globale.

\begin{figure}[h!]
\centering
\includegraphics[width=0.6\textwidth]{seasonal_q_flake_e20}
\caption{Cycle saisonnier des débits simulés pour les trois bassins d'études avec et sans la correction des flux par FLake sur la période 1983-2014. A) Nil Blanc. B) Angara. C) Neva.}
\label{seasonal_flake_e20}
\end{figure}
\clearpage
\subsection{{\fontfamily{lmss}\selectfont Résultats préliminaires}}

L'évaluation des simulations globales de CTRIP est effectuée en deux étapes: tout d'abord en analysant les résultats d'une simulation qui prend en compte explicitement le bilan de masse des lacs,  dans un second temps en comparant ces résultats à une simulation de référence sans prise en compte des lacs. En plus des critères retenus pour le choix des stations de mesures à l'échelle globale, donnés en section \ref{sec:donnees_globe_debits}, un second filtrage a été appliqué pour réaliser l'analyse. Ainsi, les stations dont le critère NSE est inférieur à -1 dans les deux simulations sont écartées. Ce choix s'explique par le fait que de tels scores montrent l'incapacité de CTRIP à représenter la dynamique des débits. Dans le cadre d'une évaluation préliminaire portant sur MLake, il est nécessaire de conserver seulement des stations pouvant être analysées de façon représentative. Sur les 9666 stations disponibles dans la base de mesures, 4133 stations ont  donc été écartées pour ne conserver que 5533 stations. \\

\subsubsection*{Configurations}

La comparaison est effectuée entre deux configurations:\\

\begin{itemize}
\item $CTRIP\_12D\_nolake$, une simulation à l'échelle globale, à 1/12°, avec le schéma d'aquifères et celui représentant les plaines d'inondations. Cette configuration a été validée à 0.5° par \citet{decharme2019};\\

\item $CTRIP\_12D\_mLake$ qui contient, en plus des schémas d'aquifères et de plaines d'inondations, le modèle MLake pour les lacs à 1/12°.
\end{itemize}
~\\
Ces simulations sont effectuées en offline et aucune rétroaction sur l'atmosphère n'est prise en compte.

\subsubsection*{Résultats préliminaires}

Les résultats préliminaires présentés dans cette section s'appuient sur les figures \ref{nse_globe}, \ref{kge_globe}, \ref{diff_nse_globe}, \ref{diff_kge_globe}, \ref{ctrip_globe} et \ref{cycle_globe}. Des figures supplémentaires et notamment des cartes régionales sont disponibles dans l'annexe \ref{chap:resultats-etude-globale}. Les scores utilisés restent les mêmes que ceux présentés dans l'annexe \ref{chap:critere-evaluation} sur la base des observations faites à partir des bases de données GRDC, ARCTICNET et de la banque Hydro. \\

Dans l'ensemble, les tendances sur les scores hydrologiques restent similaires aux résultats présentés dans la section \ref{sec:validation_globe} avec des régions qui présentent de nettes améliorations (Figure \ref{nse_globe} et \ref{kge_globe}). Sur les 5533 stations de l'analyse, la simulation des débits est très disparate avec un score moyen pour le NSE de -0.32 et une dispersion de 1.49 (tableau \ref{tab_repartition_12d}). Ces faibles scores sont notamment expliqués par les 25\% de stations dont le score est inférieur à -0.5. À l'inverse de l'Amérique du Nord dont les scores de NSE sont majoritairement dégradés, 44.6­\% des stations présentent un score supérieur à 0.2 dont celles d'Europe, d'Amazonie ou encore d'Asie du sud-est et du Japon. Les étiages sont, dans l'ensemble, correctement représentés puisque 41.3\% des stations ont un NSE$_{log}$ supérieur à 0.2 pour un score moyen global de 0.04.\\
Les scores de KGE présentent, eux, des résultats plus encourageants avec une majorité de stations dont le critère est positif. Même si les scores bruts restent dans une gamme faible, ce sont des régions à grande densité lacustre comme la Scandinavie et le Canada qui présentent les meilleurs scores. Sur l'ensemble des stations, en plus d'une dispersion plus faible (0.32), le score moyen de KGE (0.3) est largement supérieur à celui du NSE. Plus généralement, 84.7\% des stations présentent un KGE significativement élevé et seulement 2.9\% des stations ont un KGE inférieur à -0.5. Dans le même sens, les hydrogrammes simulés et observés sont fortement corrélés  dans la majorité des régions ($\overline{r}$ = 0.54) et seule la zone arctique présente des corrélations plus faibles. Néanmoins, 63.7\% des stations ont une forte corrélation, supérieure à 0.5. Comme nous l'avons déjà vu, le KGE est plus équilibré, dans sa conception, entre ses différents termes et il est donc moins sensible aux extrêmes. Le NSE, quant à lui, est très impacté par la corrélation mais il est aussi sensible aux écoulements très variables (tels que dans les régions arctiques soumises à des conditions de gel-dégel) \citep{gupta2009}. Cela pourrait, en partie, expliquer la différence de scores et les faibles performances du modèle au regard du NSE sur l'Amérique du Nord et en Scandinavie.\\

\begin{figure}
\centering
\includegraphics[width=1.\textwidth]{nse_globe}
\caption{Carte des scores de NSE (A) et de NSE$_{log}$ (B) pour la simulation CTRIP-12D globale sur la période 1978-2014.}
\label{nse_globe}
\end{figure}

~\\

\begin{figure}
\centering
\includegraphics[width=1.\textwidth]{kge_globe}
\caption{Carte des scores de KGE (A) et de corrélation (B) pour la simulation CTRIP-12D globale sur la période 1978-2014.}
\label{kge_globe}
\end{figure}

~\\

\begin{figure}
\centering
\includegraphics[width=1.\textwidth]{nse_diff_globe}
\caption{Carte globale des différences de NSE (A) et de NSE$_{log}$ (B) entre les deux configurations présentant l'impact de l'ajout de MLake sur la période 1978-2014.}
\label{diff_nse_globe}
\end{figure}

~\\

\begin{figure}
\centering
\includegraphics[width=1.\textwidth]{kge_diff_globe}
\caption{Carte globale des différences de KGE et de corrélation entre les deux configurations présentant l'impact de l'ajout de MLake sur la période 1978-2014.}
\label{diff_kge_globe}
\end{figure}

\clearpage

Ces premiers résultats montrent la capacité de CTRIP-MLake à simuler les débits des rivières particulièrement sur les grands bassins fluviaux (Saint-Laurent, Amazone, Mékong) ou les régions dépendantes de l'alimentation des lacs (Nord Canadien, Scandinavie). Cependant, cette première analyse ne permet pas de quantifier précisément l'apport de MLake sur ces résultats et offre seulement un aperçu global des performances du modèle. Pour aller plus loin, il convient de comparer directement les performances de la simulation $CTRIP\_MLake$ de la simulation sans la dynamique des lacs.\\
Cette comparaison montre une contribution positive de MLake sur la simulation des débits des rivières et notamment sur les zones de grande densité lacustre comme l'Amérique du Nord et la Scandinavie (Figure \ref{diff_nse_globe} et\ref{diff_kge_globe}). Ces résultats sur les différences entre la configuration avec et sans lacs confirment l'analyse des scores réalisée précédemment. Ainsi, tous les scores utilisés présentent des améliorations nettes sur ces régions. À l'inverse, certains grands bassins fluviaux comme le Mékong ou des pays comme le Japon, qui avaient déjà des scores élevés, sont peu impactés par l'introduction du bilan de masse des lacs. En définitive, les différences moyennes sont positives de l'ordre de 0.02 pour le KGE, 0.11 pour le NSE et 0.18 pour le NSE$_{log}$ (tableau \ref{tab_repartition_12d}). La figure \ref{ctrip_globe} montre que l'apport de MLake est nettement visible sur les stations qui avaient des scores initialement faibles.\\

\begin{figure}[h!]
\includegraphics[width=1.\textwidth]{distri_score_12d}
\caption{Distribution des scores de NSE (A), NSE$_{log}$ (B) et KGE (C) présentant le pourcentage de stations au dessus de chaque classes de valeurs sur la période 1978-2014.}
\label{ctrip_globe}
\end{figure}

La contribution du modèle de lac peut être quantifiée plus précisément en regardant les scores de NIC. Ceux-ci révèlent un apport positif de MLake sur les simulations de débits (Tableau \ref{ctrip_classe_globe}). Ainsi 45.4\% des stations sont impactées positivement par l'introduction des lacs, la majorité de ces stations (97.8\%) dans un intervalle compris entre 0 et 0.5 pour un score moyen de 0.36. Il est aussi intéressant de noter que 35\% des stations ne sont pas impactées par l'introduction de MLake.\\

Parmi les stations ayant des scores très positifs, quatre stations ont été choisies en exemple (Figure \ref{cycle_globe}). Deux stations de contrôle ont été choisies pour vérifier la cohérence des résultats avec ceux de la section précédente: la station de Irkoutsk sur l'Angara et la station de Novosaratovka sur la Neva. Les deux autres stations ont été choisies car elles sont des exemples significatifs de l'apport de MLake dans le modèle. La première se situe juste en amont des chutes du Niagara au Canada et à l'exutoire, non anthropisé, du lac Ontario. La deuxième se situe sur la rivière Lockhart à l'exutoire, lui aussi non anthropisé, du lac de l'Artillerie au Canada. \\
Ces hydrogrammes confirment l'intérêt de l'ajout du bilan de masse des lacs dans CTRIP puisque les débits simulés par CTRIP-MLake permettent d'obtenir une variabilité et une amplitude plus réalistes. Ainsi l'effet tampon des lacs a pour conséquence un lissage des hydrogrammes et de leurs amplitudes ainsi qu'un décalage du cycle annuel qui permet une amélioration des scores sur ces stations. Cet effet est particulièrement important dans le cas où l'exutoire du lac est naturel comme pour le lac Ontario et celui de l'Artillerie. Dans ces deux cas, l'ajout de MLake améliore significativement les scores de KGE avec des augmentations respectives de 3.36 points et de 2.51 points. Il est à noter que ces résultats sont localement bons mais présentent des scores moins probants sur d'autres stations. L'annexe \ref{chap:resultats-etude-globale} présente certains cas où, malgré le schéma de lacs, les scores sont dégradés. Dans ces configurations, les lacs considérés sont en fait des réservoirs avec une dynamique non naturelle. Sur ces deux exemples présentés, les débits de pointe sont nettement plus élevés et ne respectent pas l'hydrogramme observé. Bien sûr, le non-étalonnage du modèle a un impact négatif sur ce type de situation qui est en partie causée par le manque de représentation des facteurs anthropiques dans CTRIP. À terme, le domaine d'application de MLake ne concernera que les bassins non anthropisés.

\begin{figure}[h!]
\includegraphics[width=1.\textwidth]{hydrographe_ctrip_globe}
\caption{Hydrogrammes des débits simulés par CTRIP-MLake et observés pour quatre bassins sur la période 1978-2014. A) Angara, B) Neva, C) Niagara, D) Lockhart.}
\label{cycle_globe}
\end{figure}

\clearpage
\subsection{{\fontfamily{lmss}\selectfont Discussions}}

Les résultats et les conclusions sur l'impact de MLake à l'échelle globale présentés dans la section \ref{sec:eval_globe} sont confirmés par les scores de la simulation globale préliminaire de CTRIP à 1/12°.\\
La prise en compte du bilan de masse assure une meilleure simulation des débits des grands bassins fluviaux et des zones de grande densité lacustre avec dans l'ensemble une amélioration des performances du modèle. Néanmoins, ces résultats sont à nuancer car cette simulation est limitée par sa paramétrisation et la non prise en compte de certains processus. \\
Ainsi les forçages ne sont pas corrigés pour prendre en compte l'évaporation au-dessus des lacs et CTRIP est uniquement forcé par les variables de surface (ruissellement et drainage) issus d'ISBA. À cela s'ajoute une absence de couverture en glace pour les lacs. Cet effet domine les bilans de flux d'énergie et de masse dans les régions nordiques. Même si dans le schéma ISBA, le gel du sol est simulé, il n'est pas du même ordre de grandeur. Des biais systématiques apparaissent donc dans ces termes et seront, à l'avenir, corrigés par un modèle couplé prenant explicitement en compte les termes d'évaporation au-dessus des lacs. Au-delà de la couverture en glace, les biais sur les flux de chaleur latente se propagent dans la simulation des débits dans les cas où les lacs sont dominés par les composantes atmosphériques.\\
Enfin, il est difficile de quantifier les incertitudes issues des forçages atmosphériques et notamment des précipitations. Cependant, il semble que l'ajout du processus de bilan de masse des lacs compense en partie les biais introduits par les forçages ou encore le manque de représentation de certains processus (notamment anthropiques). Même si le couplage rivière-lac apporte des résultats sensiblement meilleurs, il est nécessaire de coupler ce système avec un modèle atmosphérique afin de prendre en compte de façon plus réaliste les flux de masse et d'énergie qui existent entre la surface et l'atmosphère. Les biais proviennent aussi de l'incapacité du modèle à simuler les débits issus de barrages et plus généralement les manœuvres non-naturelles (Figure \ref{barrage_12d}).\\

Les incertitudes ne sont pas seulement issues des processus physiques et peuvent aussi venir de la gestion du couvert dans le modèle. L'introduction des lacs dans CTRIP repose sur la carte d'occupation des sols ECOCLIMAP. Celle-ci contient environ 15 000 lacs et même si elle est adaptée aux applications de modélisation climatique et atmosphérique, des zones blanches subsistent. À l'inverse, on peut noter que d'autres zones contiennent une information sur la présence de lac qui résulte de fausses détections. C'est notamment le cas au niveau de grands bassins fluviaux comme celui de l'Amazone ou de du Mackenzie. La largeur de ces fleuves est telle que certains tronçons sont identifiés comme étant des lacs et non comme des rivières. Des biais apparaissent alors par la prise en compte d'une dynamique non réaliste (Figure \ref{rivers}). Malgré cela, le déphasage temporel introduit par MLake peut aussi améliorer la simulation des débits, comme par exemple sur l'Amazone, et aider à corriger les biais inhérents à certains processus de CTRIP sur cette région. \\
Sur cet aspect géographique, il est aussi bon de rappeler que la construction du réseau de rivière 1/12° a été réalisée indépendamment d'ECOCLIMAP. C'est pourquoi des incohérences subsistent sur certaines stations où la maille de lac identifiée à 1/12° ne correspond en fait pas au même tronçon de rivière.\\
Enfin, il faut garder à l'esprit que cette base de données ne prend pas en compte la totalité des lacs et qu'en plus certaines régions du monde ne sont pas influencées par ceux-ci. La majorité des stations présentes dans la simulation globale ne voit pas d'amélioration dans les débits simulés justement parce que les processus représentés n'ont pas été modifiés par l'ajout de MLake.\\

\clearpage
\section{{\fontfamily{lmss}\selectfont Conclusion générale et Perspectives}}

Dans ce chapitre, une estimation de l'impact des lacs et de leur paramétrisation sur les simulations de débits a été effectuée d'abord sur quatre sites d'études puis sur une simulation globale à 1/12° prenant en compte 5533 stations de mesures. Ce chapitre montre l'exploitation possible du modèle MLake au niveau global à la résolution 1/12°. Le modèle MLake est désormais disponible pour des simulations hydrologiques globales dans le modèle CTRIP.\\
Au niveau local, la contribution des lacs améliore la simulation des débits notamment en matière de variabilité et de saisonnalité. Cela se traduit par une diminution de l'amplitude des débits notamment pour les régions Scandinaves et du Grand Nord Canadien. L'ajout de l'effet tampon corrige les débits simulés et permet une simulation correcte de la hauteur d'eau des lacs et de leur cycle annuel.\\

Dans le détail, l'effet tampon est particulièrement visible sur les extrêmes de débits avec une réduction des débits de pointe et une augmentation des étiages. Les résultats sont ainsi particulièrement probants sur le bassin de l'Angara et de la Neva avec des critères de Nash-Sutcliffe positifs ($\overline{NSE}_{Angara}=0.2$ et $\overline{NSE}_{Neva}$ = 0.19) et une contribution significative du modèle ($\overline{NIC}_{Angara}$ = 0.87 et $\overline{NIC}_{Neva}$ = 0.56). Seuls les résultats pour le Nil Blanc dénotent une incapacité du modèle à simuler les débits sur ce bassin. Les biais présents dans les simulations sont généralement liés à une anthropisation des bassins et à la gestion non-naturelle des débits à l'exutoire. Pour corriger cela, une thèse est actuellement en cours afin de prendre en compte explicitement les règles de gestion spécifiques des eaux de barrages dans le bilan hydrologique global.\\

En plus de simuler les débits, MLake apporte un diagnostic sur les variations du niveau des lacs. La simulation des ces variations est cohérente sur les trois lacs d'étude avec une performance significative sur le marnage annuel et interannuel. La corrélation sur les sites d'études oscille entre 0.37 et 0.83 pour une variabilité relative comprise entre 0.88 et 1.06. Il apparaît par contre un décalage temporel des extrêmes sur les niveaux du lac Baïkal et du lac Ladoga qui s'explique en partie par la résolution d'un unique bilan d'énergie pour le sol dans ISBA. En effet, le bilan d'énergie du sol ne distingue pas les interactions entre la canopée, le sol nu et la neige et amène alors à l'introduction de biais dans la simulation de la fonte nivale. Pour corriger ce bilan d'énergie, un schéma multi-énergie a été introduit dans ISBA afin de distinguer le bilan de la neige de celui de la canopée. Les premiers résultats sont encourageants et permettent de corriger la durée et l'intensité des processus de gel et de dégel dans le sol. Ce nouveau modèle ISBA-MEB sera testé en couplage avec CTRIP dans des études futures pour quantifier la réponse hydrologique.\\

Le modèle CTRIP-MLake reste en cohérence avec ces résultats dans la simulation préliminaire à échelle globale. Sur 5533 stations, la majorité des améliorations se situe dans les zones de grande densité lacustre comme la Scandinavie ou la partie nord du Canada. Ces résultats sont à nuancer car tous les processus ne sont pas encore intégrés au modèle global et que des incertitudes résiduelles persistent notamment sur certaines paramétrisations d'ISBA ou de CTRIP. Il est toutefois encourageant de remarquer que l'introduction des lacs apporte une réponse homogène et consistante avec les études locales et accentue la nécessité d'un couplage global.\\

Le test de sensibilité à la largeur du seuil montre que les variables hydrologiques sont sensibles à cette largeur. Il s'avère que les configurations possédant un facteur multiplicatif entre 0.5 et 1 présentent les performances les plus importantes tant sur les débits que sur les niveaux d'eau. Il semblerait donc que la largeur de la rivière en aval du lac soit un bon prédicteur pour la largeur du déversoir. En effet, sous les hypothèses de MLake, l'aire du lac est assimilée à celle d'un cercle dont la circonférence est inférieure à la circonférence réelle. La correction appliquée au seuil compense donc cet effet.\\

Pour autant quelques leviers freinent encore ce couplage et notamment la correction des flux d'évaporation des lacs. Ces flux sont considérés, dans cette étude, indépendant de la morphologie du bassin lacustre et seulement dépendant des variables atmosphériques. Pourtant la forme du bassin joue sur la dynamique des variables morphologiques et notamment sur l'aire et la profondeur du lac. Il est évident que des simulations hydrologiques à long terme doivent prendre en compte une composante morphologique pour corriger les flux d'évaporation à l'interface lac-atmosphère. Cette composante dynamique n'est aujourd'hui pas présente dans le modèle et doit passer par le développement à l'échelle globale d'une paramétrisation de la bathymétrie des lacs. 

\cleardoublepage
\chapter*{{\fontfamily{lmss}\selectfont Conclusion et perspectives}}
\label{chap:conclu_persp}

\addstarredchapter{Conclusion générale et Perspectives}
\markboth{\uppercase{Conclusion générale et Perspectives}}{}

L'importance de l'eau sur Terre est indéniable tant pour l'expansion et le maintien de la vie que pour le développement du système socio-économique. Néanmoins, cette ressource est fragile et sa disponibilité est limitée mais surtout inégalement répartie à la surface du globe. Pour rappel, quatre milliards de personnes font face, au moins une fois par an, à des restrictions quantitatives d'eau. Ces disparités sont exacerbées par des altérations externes d'origine anthropique et aussi à cause du dérèglement irréfutable de la dynamique climatique. \\
Le mouvement et le renouvellement de l'eau sur la planète s'effectue sous la forme d'un cycle que tentent de représenter les scientifiques en caractérisant ses différentes composantes ainsi que leurs interdépendances. L'hydrologue, en s'intéressant à la partie continentale du cycle, possède des outils d'observation et de modélisation qui lui permettent d'accéder à une description efficace et réaliste des processus mis en jeu. Ainsi, la modélisation pallie les limites spatio-temporelles de l'observation en proposant une approche simplifiée des processus physiques. C'est dans ce contexte que sont nés les systèmes couplant les modèles de surface aux modèles de routage en rivière, et décrivant respectivement les transferts verticaux et transfert latéraux de flux. \\
Dans ce cycle de l'eau, les lacs constituent une ressource importante, environ 20\% de l'eau douce de surface et surtout très accessible. En dépit de leur abondance et de leur rôle sur les transferts d'eau à l'échelle régionale \citep{bowling2010}, ces réservoirs ne sont que très rarement pris en compte dans les modèles hydrologiques globaux \citep{downing2010}.\\

C'est face à ce constat que le cadre de cette thèse a été défini. En s'appuyant sur les développements effectués au Centre National de Recherches Météorologiques, cette thèse intègre une paramétrisation du bilan de masse dans le système hydrologique global ISBA-CTRIP. Plus particulièrement, les objectifs d'étude de la thèse sont de développer un modèle de bilan de masse MLake pour améliorer la simulation des débits de rivières, d'introduire un diagnostic pour le suivi des niveaux d'eau de lacs et enfin de caractériser les stocks d'eau lacustres par une meilleure représentation de la bathymétrie.\\

Dans un premier temps, le travail s'est appuyé sur une description succincte des outils existants pour observer et modéliser le cycle de l'eau. La mise en contexte a été suivie d'une brève introduction à la limnologie afin d'étayer et de justifier l'intérêt d'une représentation des systèmes lacustres en hydrologie. Enfin l'état des lieux des connaissances actuelles en modélisation des lacs a mis en évidence la prédominance des développements de paramétrisation du bilan d'énergie par rapport à ceux du bilan de masse.\\

Le chapitre \ref{chap:intro} a introduit le cadre théorique dans lequel s'insèrent les outils de modélisation utiles pour les diverses études menées au cours de la thèse présentées dans le chapitre \ref{chap:descriptions}. Ce dernier a mis l'accent sur la plateforme de modélisation de surface SURFEX \citep{masson2000} comportant notamment le modèle de surface ISBA \citep{noilhan1989} ainsi que sur le modèle de routage en rivière CTRIP auquel il est couplé \citep{decharme2019}. C'est à partir de ces outils qu'a été développé le modèle MLake résolvant le bilan de masse pour les lacs à 1/12°.\\
Ce modèle présente l'avantage de se baser sur une paramétrisation non calibrée et s'adaptant à la diversité de lacs présents à la surface du globe. Pour ce faire, le développement s'est appuyé sur l'utilisation conjointe de la carte d'occupation des sols ECOCLIMAP \citep{faroux2013} et de la base de données de profondeurs des lacs GLDB \citep{choulga2019}. L'intérêt d'utiliser ces bases, au-delà de leur disponibilité, est qu'elles sont cohérentes avec le modèle FLake \citep{mironov2008} résolvant le bilan d'énergie pour les lacs.\\
La première étape a consisté à agréger l'information globale de présence des lacs à la résolution kilométrique pour proposer une carte globale de lacs auxquels a été attribuée un identifiant unique et une profondeur moyenne. La deuxième étape a abouti à l'intégration de ces informations dans le réseau de rivière initial de CTRIP et à la correction d'éventuelles erreurs de routage grâce à l'application d'un masque de réseau. Une gestion du partage des flux d'eau issus des forçages est le résultat, dans un troisième temps, de la mise en place d'un masque de ruissellement. Ce masque, complémentaire à celui de réseau, assure la fermeture du bilan et un partage correct entre les flux s'écoulant en rivière et ceux stockés dans les lacs. Tout ce travail préliminaire a permis la construction d'un réseau réaliste et robuste pour la résolution des processus physiques. Ainsi, à chaque pas de temps du modèle MLake, le bilan de masse est effectué par la résolution d'une équation dont la variable pronostique est le stock d'eau dans le bassin.\\

L'évaluation de cette paramétrisation a d'abord été effectuée sur le bassin versant du Rhône. Ce choix a été motivé par la présence d'un réseau de mesures fiables et étendues mais aussi par la disponibilité de forçages à haute résolution issus de la chaîne hydrométérologique opérationnelle SAFRAN-ISBA-MODCOU \citep{lemoigne2020}. Le bassin du Rhône est aussi le seul bassin versant français possédant une assez grande variété de lacs inclus dans un contexte hydrologique complexe et diversifié. Le modèle CTRIP-MLake a fait l'objet d'une évaluation en mode offline, sur la période 1960-2016, avec une correction des forçages au niveau des lacs par une estimation de l'évaporation issue d'une simulation globale FLake. Cette évaluation locale a permis de confirmer l'intérêt d'ajouter la dynamique des lacs pour la représentation des débits de rivières locaux et régionaux. En se basant sur une équation empirique de déversoir rectangulaire à seuil épais, il est maintenant possible de corriger les débits simulés en aval du bassin et d'améliorer significativement les performances de CTRIP. Ces améliorations portent principalement sur un lissage des hydrographes en lien avec une réduction de la variabilité et de l'amplitude. Au-delà de ces améliorations, MLake apporte aussi un diagnostic correct sur les marnages du Léman, validé par une comparaison avec des mesures \textit{in situ}. Néanmoins, ces résultats sont à nuancer dû fait de la prédominance des facteurs anthropiques sur le bassin et notamment de la régulation des eaux du Rhône et de ces affluents principaux.\\

Une deuxième évaluation s'est intéressée, initialement, à trois bassins versants décrivant des conditions hydroclimatiques différentes: le bassin de l'Angara, le bassin de la Neva et enfin le bassin du Nil Blanc. Cette évaluation s'est effectuée en mode offline avec des ruissellements et drainages issus d'une simulation globale d'ISBA forcés par des réanalyses Earth2Observe, corrigés pour les lacs par les évaporations d'une simulation globale où ISBA a été remplacé par FLake. L'application de ce modèle sur les trois bassins versants confirme les résultats de l'évaluation locale sur les débits et les marnages. Seul le cas du Nil Blanc et donc du lac Victoria présente des résultats peu probants. Dans tous les cas, on observe une réduction de la variabilité et de l'amplitude des débits se traduisant par un lissage des hydrographes. Cela met en lumière la représentation correcte de l'effet tampon des lacs par l'écrêtement des débits de pointes et un soutien accru des étiages grâce à l'intégration du module MLake. En outre, le diagnostic sur les marnages est particulièrement performant pour les lacs avec un cycle saisonnier bien respecté. Seul un décalage temporel systématique apparaît sur les cycles du lac Baïkal et du lac Ladoga. L'origine de ce décalage est identifié comme probablement lié à la résolution du bilan d'énergie dans ISBA. Il semblerait que l'utilisation d'un unique bilan pour la canopée et la neige induit une sous-estimation des cumuls de neige et de la durée de l'enneigement. La réduction des caractéristiques isolante de la couche de neige sur les propriétés thermiques du sol entraine alors une fonte précoce et plus intense de la glace du sol associée à une sous-estimation des ruissellements dans les zones de densité forestière élevée \citep{napoly2020}.\\
En suivant cette évaluation, les résultats d'une simulation préliminaire ont confirmé la validité de MLake à l'échelle globale et plus particulièrement sur les zones de grande densité lacustre. Même s'il est préférable de rester prudent vu que tous les processus ne sont pas encore intégrés, cette simulation préliminaire est encourageante par les nettes améliorations qu'elle présente.\\
Le test de sensibilité à la largeur du seuil semble indiquer que la largeur de la rivière en aval du lac est un bon prédicteur de la largeur du seuil. Ces résultats restent quand même contraints par des éléments externes comme la morphologie du lac et notamment sa bathymétrie qui sont à ce jour peu documentées.\\
Des questions persistent néanmoins notamment sur la prescription d'une hypsométrie appliquée à l'échelle globale afin de mieux caractériser la dynamique des lacs et permettre un suivi plus précis des variations de stocks.\\


Les objectifs principaux énoncés au début de ce projet de recherche ont été remplis. Toutefois il reste des incertitudes à éclaircir et des biais à corriger. Les performances limitées du modèle sur certaines zones sont notamment liées aux limites qui émergent de certaines hypothèses de développement. Un des défauts de MLake est la redistribution spatiale des variables. À chaque pas de temps, les variables décrivant le stock et les hauteurs d'eau sont affectées à toutes les cellules de ce lac dans le masque de réseau. Au pas de temps journalier, cette hypothèse n'est pas tellement restrictive pour des lacs de petites dimensions. Par contre, la validité de l'hypothèse sur des lacs plus importants comme le lac Baïkal semble discutable. Dans le cadre du développement d'une morphologie plus précise, il convient de s'intéresser aussi à d'autres formes pour représenter la surface du lac. Une surface elliptique pourrait, par exemple, être testée et comparée à la version actuelle. Enfin la couverture en glace qui peut se former à la surface du lac est prise en compte dans le bilan d'énergie mais pas dans le bilan de masse. Cela introduit des biais sur les calculs de masses et de débits notamment pour les lacs de hautes latitudes recouverts de glace la majeure partie de l'année. En allant plus loin sur la représentation des débits de sortie, il est vrai que l'approche reste encore simple et mériterait un travail détaillé notamment pour la représentation de la géométrie de l'exutoire avec la prise en compte de la pente d'écoulement ou d'une largeur de seuil dynamique (par exemple sous la forme d'une fraction de la circonférence du lac).\\

Au-delà des hypothèses relatives à MLake, certains processus ne sont tout simplement pas encore pris en compte dans le modèle et participent à l'apparition d'incertitudes. Il est évident que le processus majeur à introduire dans CTRIP concerne l'anthropisation. Les activités anthropiques influencent la majorité des fleuves et lacs du monde et les futurs développements doivent se concentrer sur l'introduction des barrages-réservoirs et les prélèvements pour l'irrigation ou l'industrie. Si l'on regarde le Rhône, il est clair que l'influence de l'homme est un facteur majeur dans la régulation des débits et dans la modification des régimes hydrologiques. Au second plan, l'introduction des flux latéraux et verticaux sont indissociables du développement d'un modèle de lac. La contribution des échanges avec les aquifères dans les bassins endoréiques est indéniable mais reste peu documentée. Des incertitudes persistent sur ces échanges et la paramétrisation des flux aquifères-lacs assurerait une meilleure représentation du soutien des eaux des bassins endoréiques par des remontées du système hydrogéologique. Il en est de même pour les échanges latéraux entre le lac et les zones humides dont la saturation caractérise l'état d'humidité des sols.\\
À moyen terme le développement de MLake porte aussi sur un couplage direct avec la plateforme de modélisation SURFEX et plus particulièrement le modèle résolvant le bilan d'énergie pour les lacs, FLake.\\
Il est aussi important de travailler sur l'assimilation de donnée comme celle de la température de surface comme axe majeur pour corriger le bilan d'énergie du lac. Dans cette même optique, le couplage avec un modèle de Système Terre comme CNRM-ESM \citep{seferian2019} ou l'utilisation de CTRIP dans la représentation du cycle du carbone \citep{delire2020} nécessite la prise en compte des sédiments associés à la prédominance de processus hydrodynamiques comme les seiches.\\

Dans le futur, plusieurs pistes d'évolution sont alors à envisager afin d'obtenir un modèle de lac consistant et représentatif des processus clés pour le suivi hydrologique et climatique global. En plus des pistes de réflexions présentées précédemment, d'autres développements doivent être menés sur le modèle hydrologique de surface ISBA-CTRIP. La quantification des apports du nouveau schéma MEB pour la résolution d'un bilan multi-énergie sur les simulations hydrologiques est ainsi primordial. Il est envisageable que la représentation plus réaliste des processus de gel et dégel du sol corrige les décalages temporels qui apparaissent sur les simulations de débits dans les zones boréales. Une autre piste à exploiter est la mise en place d'une carte dynamique d'occupation des sols s'appuyant sur la distinction de zone saturée semi-permanente \citep{pekel2016} et une évolution des bases de données pour renforcer la précision des fractions de couverts. En parallèle du développement d'une hypsométrie pour les lacs, la prescription d'une profondeur dynamique pour FLake est envisageable pour corriger les termes d'évaporation dans le bilan d'énergie. Cette prescription s'appuiera sur le couplage entre MLake et SURFEX avec la prise en compte des rétroactions entre bilan d'énergie et de masse pour les lacs. Bien sûr, l'évolution temporelle de la profondeur moyenne sera associée à l'évolution d'une surface dynamique utile dans SURFEX pour anticiper l'emprise spatiale lacustre. Ce couplage est primordial pour l’utilisation du système dans un environnement de projections climatiques à l’échelle globale. \\ Certaines cartes comme le produit HydroLAKES \citep{messager2016} présentent aujourd'hui un intérêt particulier car elles fournissent des informations à haute résolution et adaptées à l'échelle globale. Dans une perspective plus lointaine d'une descente d'échelle, le modèle devra considérer des processus hydrodynamiques. Parmi ces processus, il est probable que le développement d'un schéma dynamique permette à CTRIP de proposer une modélisation de la propagation d'ondes dans le réseau hydrographique.\\

Dans tous les cas, ce travail de thèse pose les bases de la modélisation lacustre au sein des développements hydrologiques à Météo-France. Bien qu'il reste des pistes d'améliorations, MLake a été intégré dans le modèle global ISBA-CTRIP pour la simulation des débits des grands bassins fluviaux et propose maintenant un suivi des marnages pour les principaux lacs. En suivant les quelques pistes de réflexions proposées, il est tout à fait envisageable d'introduire cette paramétrisation en couplage avec un modèle de climat, comme CNRM-CM, pour les exercices de projections hydrologiques et climatiques à long terme.
\cleardoublepage

\appendix

\cleardoublepage
\mtcaddpart[Annexes]
\part*{Annexes}
\chapter{{\fontfamily{lmss}\selectfont Critères d'évaluation du modèle MLake}}
\label{chap:critere-evaluation}

Le traitement statistique des résultats de simulations quantifie l'apport de nouvelles paramétrisations comme MLake dans les modèles comme CTRIP. Ces traitements sont nécessaires pour jauger les performances et le bénéfice d'un tel modèle sur les simulations de débits au sein des bassins versants choisis et sur les variations de hauteurs d'eau dans les différents lacs étudiés.\\
Les critères d'évaluation utilisés sont des scores relatifs qui caractérisent les performances des différentes simulations au sein d'un même bassin versant sans pour autant permettre une évaluation entre les bassins. 

\section{{\fontfamily{lmss}\selectfont Critères d'évaluation sur les débits}}

Les débits de rivières ont été comparés suivant un jeu de scores couramment utilisés en hydrologie. Le premier de ces scores est le critère de Nash-Sutcliffe \citep[NSE,]{nash1970} défini sur l'intervalle [-$\infty$;1]. Le critère de Nash quantifie l'adéquation entre la variable simulée et la variable observée par une analyse de la somme des erreurs quadratiques tel que: \\

\begin{equation}
NSE = 1 - \frac{\sum\limits_{t=0}^n (q_{s,t}-q_{o,t})^{2}}{\sum\limits_{t=0}^n (q_{o,t}-\overline{q_{o}})^{2}}
\end{equation}
avec $t$ le temps , $n$ le nombre de pas de temps, $q_{s,t}$ le débit simulé dans la rivière au pas de temps $t$, $q_{o,t}$ le débit observé dans la rivière au pas de temps $t$ et $\overline{q}_{o}$ le débit moyen observé sur une période de référence.\\

Le critère de Nash compare les simulations par rapport à un modèle de référence défini par le débit moyen annuel. Une valeur positive de NSE informe sur la capacité du modèle à représenter la dynamique des débits observés, avec une valeur maximale de 1 indiquant une correspondance parfaite en observations et simulations. \'A l'inverse une valeur nulle ou négative indique que la dynamique du modèle de référence est meilleure que le modèle évalué.\\

Même si ce critère reste très utilisé en hydrologie, il présente, néanmoins, des limites structurelles importantes. Le principal inconvénient du critère NSE est sa sensibilité aux valeurs extrêmes. Comme les différences sont calculées en valeurs quadratiques, il suffit qu'une série de débits contienne un nombre important d'extrêmes (pics de crue ou étiages) pour que le NSE devienne rapidement négatif. De la même façon que le coefficient de détermination $r^{2}$, le NSE est très peu sensible aux biais systématiques des modèles lors des périodes d'étiages\citep{krause2005}. Le poids des étiages devient donc négligeable et n'influence que peu les scores \citep{legates1999}. Pour pallier ces inconvénients d'autre critères ont été développés pour étayer les évaluations.\\

Le second critère utilisé découle du critère NSE puisqu'il s'agit du critère de Nash-Sutcliffe logarithmique ($NSE_{log}$). Connaissant le peu de poids des étiages sur le NSE, il est primordial d'équilibrer cet effet. Le $NSE_{log}$ donne ainsi un poids plus important aux faibles débits:

\begin{equation}
NSE_{log}= 1 - \frac{\sum\limits_{t=0}^n(log(q_{s,t})-log(q_{o,t}))^{2}}{\sum\limits_{t=0}^n(log(q_{o,t})-log(\overline{q_{o}}))^{2}}
\end{equation}
\\

Malgré cela, la sensibilité du NSE (et de son logarithme) aux valeurs extrêmes pose des problèmes évidents notamment sur des bassins présentant des débits aussi variables que ceux du Rhône ou du lac Victoria. Par ailleurs, la réponse hétérogène aux valeurs extrêmes n'est pas le seul inconvénient de ce critère qui manque aussi de stabilité au regard du choix de la période de référence choisie. Enfin ce critère est sensible au rendement du bassin, c'est-à-dire la capacité à transférer la pluie cumulée sous forme de débit à l'exutoire \citep{perrin2000}.\\

Pour prévenir ces effets, le Kling-Gupta Efficiency \citep[KGE,][]{gupta2009, kling2012} a été introduit. Ce score provient d'une discrétisation du NSE selon trois composantes: le coefficient de corrélation linéaire $r$, le coefficient de variation $\gamma$ et la variabilité relative $\alpha$. Le score est rééquilibré pour donner plus de poids au biais et à la variabilité au détriment de la corrélation.

\begin{equation}
KGE = 1-\sqrt{(r-1)^{2}+(\gamma-1)^{2}+(\alpha-1)^{2}}
\end{equation}
avec $\alpha = \frac{\frac{\sigma_{s}}{\mu_{s}}}{ \frac{\sigma_{o}}{\mu_{o}}}$. $\sigma_{s}$ et $\sigma_{o}$ respectivement les écarts-types des débits simulés et observés. $\mu_{s}$, $\mu_{o}$ respectivement les moyennes temporelles des débits simulés et observés. \\

Pour autant, que ce soit le NSE ou le KGE, ces scores ne produisent pas d'information directe sur l'amélioration des performances du modèle. Ainsi, une amélioration de 100\% du NSE dans les valeurs négatives n'est pas comparable à la même amélioration en valeurs positives. C'est pour cela que généralement un seuil de 0.5-0.6 défini un critère NSE correct. Il y a un biais significatif sur les interprétations de ces scores qui peut fausser l'appréciation des résultats. La contribution de MLake aux performances de CTRIP a, par conséquent, été évalué suivant le Normalized Information Contribution \citep[NIC,][]{kumar2009}. Ce score indique spécifiquement l'apport d'un modèle par rapport à l'amélioration maximale possible. Dans cette thèse, le NIC a été appliqué au critère NSE.

\begin{equation}
NIC = \frac{NSE_{ctrip-mlake}-NSE_{ctrip-nolake}}{1-NSE_{ctrip-nolake}}
\end{equation}
avec $NSE_{ctrip-mlake}$ le critère de Nash-Sutcliffe pour les simulations CTRIP-MLake et $NSE_{ctrip-nolake}$ le critère de Nash-Sutcliffe de la simulation de référence CTRIP-nolake.\\

Un NIC positif indique que la configuration choisie améliore les résultats par rapport à la configuration de référence tandis qu'une valeur négative indique une dégradation des résultats.

\section{{\fontfamily{lmss}\selectfont Critère d'évaluation sur les hauteurs}}

Pour les variations de cote d'eau des lacs, le traitement statistique s'est concentré sur des analyses simples visant à mesurer l'adéquation du modèle aux observations. Les scores utilisés sont le coefficient de corrélation $r$ et l'écart quadratique moyen (Root Mean Square Deviation en anglais, RMSD):

\begin{equation}
RMSD=\sqrt{\frac{1}{n}\sum\limits_{t=0}^n(H_{s,t}-H_{o,t})^{2}}
\end{equation}
aec $t$ le temps, $n$ le nombre total de pas de temps, $H_{s,t}$ la hauteur de la cote d'eau simulée au-dessus du seuil au pas de temps $t$ et $H_{o,t}$ la hauteur de la cote d'eau observée au-dessus du seuil au pas de temps $t$.

\cleardoublepage
\chapter{{\fontfamily{lmss}\selectfont Résultats détaillés du chapitre: \'Evaluation et validation globale}}
\label{chap:resultats-etude-globale}

Cette annexe détaille les résultats obtenus lors de l'évaluation globale dans un premier temps en présentant les tableaux de résultats des débits pour l'évaluation sur les trois sites d'études choisis. Dans second temps, les résultats sur les variations de niveaux de hauteurs sont présentés. Enfin, des figures supplémentaires sont introduites afin de compléter l'analyse de la simulation globale au 1/12°.

\section{{\fontfamily{lmss}\selectfont Résultats détaillés sur les simulations de débits}}
\label{chap:annexe_q_globe}

La présente section donne les détails des scores et résultats sur les débits simulés par CTRIP-MLake pour les trois sites d'études de l'évaluation globale.

\begin{table}[b!]
\caption{Résultats issus des simulations de débits pour les trois sites d'études sur la période 1974-2018.}
\footnotesize
\begin{tabular*}{\linewidth}{ @{\extracolsep{\fill}} ll *{9}c @{}}
\toprule
Zone d'étude & Scores & \multicolumn{9}{c}{Facteur Multiplicatif}\\
\midrule \midrule \addlinespace \\
 & & 0.1 & 0.5 & 0.7 & 1 & 1.5 & 2 & 4 & 5 & no lake\\
\cmidrule{3-11}
\addlinespace \\
\multirow{9}{*}{\textbf{Victoria}} & RMSD ($m^{3}.s^{-1}$)& 426 & 436 & 508 & 596 & 710 & 681 & 1056 & 1147 & 1688 \\
                         & $\overline{Q}$ ($m^{3}.s^{-1}$)& 700 & 910 & 920 & 925 & 924 & 922 & 916 & 915 & 1534   \\
                         & $\sigma_{s}$ ($m^{3}.s^{-1}$)& 232 & 463 & 543 & 635 & 750 & 841 & 1091 & 1180 & 1667 \\
                         & $Q_{90}$ ($m^{3}.s^{-1}$) & 1997 & 1517 & 1628 & 1722 & 1832 & 1917 & 2229 & 2324 & 3693 \\
                         & $Q_{10}$ ($m^{3}.s^{-1}$) & 377 & 292 & 238 & 175 & 94 & 41 & 0 & 0 & 298  \\
                         & NSE & -4.4 & -4.67 & -6.74 & -9.6 & -14.1 & 18.2 & -32.4 & -38.4 & -84.4\\
                         & NSE$_{log}$ & -11 & -16.45 & -33.1 & -42 & -46 & -56 & -83 & -78 & -30.9 \\
                         & KGE  & -0.09 & -1.1 & -1.6 & -2.1 & -2.9 & -3.5 & -5.15 & -5.74 & -4.55\\
                         & NIC &0.94&0.93&0.91&0.88&0.82&0.78&0.61&0.54&-\\
                         \midrule \addlinespace \\
\multirow{9}{*}{\textbf{Baïkal}} & RMSD ($m^{3}.s^{-1}$) & 451 & 434 & 462 & 513 & 599 & 681 & 927 & 1014 & 1766 \\
                         & $\overline{Q}$ ($m^{3}.s^{-1}$) & 1794 & 1853 & 1853 & 1853 & 1852 & 1851 & 1850 & 1850 & 1754  \\
                         & $\sigma_{s}$ ($m^{3}.s^{-1}$) & 178 & 356 & 418 & 498 & 607 & 695 & 939 & 1021 & 1696 \\
                         & $Q_{90}$ ($m^{3}.s^{-1}$) & 2023 & 2342 & 2438 & 2536 & 2673 & 2783 & 3085 & 3185 & 4046 \\
                         & $Q_{10}$ ($m^{3}.s^{-1}$) & 1590 & 1406 & 1333 & 1234 & 1106 & 1003 & 737 & 651 & 148  \\
                         & NSE & 0.14 & 0.20 & 0.094 & -0.11 & -0.52 & -0.96 & -2.64 & -3.4 & -12.2 \\
                         & NSE$_{log}$ & 0.14 & 0.12 & -0.04 & -0.33 & -0.93 & -1.59 & -4.5 & -5.99 & -36.1 \\
                         & KGE & 0.18 & 0.46 & 0.49 & 0.47 & 0.36 & 0.22 & -0.23 & -0.4 & -2.02  \\
                         & NIC&0.93&0.94&0.93&0.92&0.88&0.85&0.72&0.67\\
                         \midrule \addlinespace \\
\multirow{9}{*}{\textbf{Ladoga}} & RMSD ($m^{3}.s^{-1}$) & 584 & 557 & 598 & 671 & 797 & 878 & 1109 & 1179 & - \\
                         & $\overline{Q}$ ($m^{3}.s^{-1}$) & 2508 & 2539 & 2538 & 2507 & 2537 & 2537 & 2537 & 2537 & 2541  \\
                         & $\sigma_{s}$ ($m^{3}.s^{-1}$) & 178 & 356 & 418 & 498 & 607 & 695 & 939 & 1021 & 1696\\
                         & $Q_{90}$ ($m^{3}.s^{-1}$) & 2734 & 3048 & 3147 & 3248 & 3391 & 3532 & 3772 & 3860 & 3973 \\
                         & $Q_{10}$ ($m^{3}.s^{-1}$) & 2264 & 2001 & 1885 & 1800 & 1608 & 1561 & 1382 & 1332 & 1045  \\
                         & NSE & 0.18 & 0.26 & 0.14 & -0.08 & -0.44 & -0.85 & -1.95 & -2.33 & -2.94 \\
                         & NSE$_{log}$ & 0.14 & 0.23 & 0.15 & -0.03 & -0.35 & -0.71 & -1.78 & -2.16 & -3.2\\
                         & KGE & 0.10 & 0.37 & 0.38 & 0.35 & 0.25 & 0.17 & -0.06 & -0.14 & -0.22  \\
                         & NIC&0.79&0.81&0.78&0.73&0.63&0.53&0.25&0.15\\
\bottomrule
\end{tabular*}
\end{table}

\cleardoublepage


\section{{\fontfamily{lmss}\selectfont Résultats détaillés sur les simulations de de niveau d'eau}}
\label{chap:annexe_h_globe}

La présente section donne les détails des scores et résultats sur le diagnostic des variations du niveau d'eau par CTRIP-MLake pour les trois sites d'études de l'évaluation globale.\\

\begin{table}[h!]
\caption{Résultats issus des simulations de variations du niveau d'eau pour les trois sites d'études à l'échelle globale sur la période 1974-2018.}
\begin{tabular*}{\linewidth}{ @{\extracolsep{\fill}} ll *{8}c @{}}
\toprule
Zone d'étude & Scores & \multicolumn{8}{c}{Facteur Multiplicatif}\\
\midrule \midrule \addlinespace \\
 & & 0.1 & 0.5 & 0.7 & 1 & 1.5 & 2 & 4 & 5\\
\cmidrule{3-10}
\addlinespace \\
\multirow{5}{*}{\textbf{Victoria}} & r & 0.19 & 0.86 & 0.90 & 0.92 & 0.90 & 0.88 & 0.76 & 0.72 \\
                         & RMSD ($m$) & 0.76 & 0.33 & 0.31 & 0.76 & 0.33 & 0.35 & 0.41 & 0.43  \\
                         & $\sigma_{s}$ ($m$) & 1.04 & 0.55 & 0.50 & 0.46 & 0.42 & 0.39 & 0.33 & 0.31  \\
                         & $h_{min}$ ($m$) & -2.20 & -1.51 & -1.41 & -1.31 & -1.20 & -1.13 & -1.01 & -0.98   \\
                         & $h_{max}$ ($m$) & 1.94 & 1.54 & 1.56 & 1.55 & 1.49 & 1.42 & 1.18 & 1.09 \\
                         \midrule \addlinespace \\
\multirow{5}{*}{\textbf{Baïkal}} & r & 0.73 & 0.86 & 0.85 & 0.82 & 0.78 & 0.75 & 0.63 & 0.59\\
                         & RMSD ($m$) & 0.33 & 0.26 & 0.24 & 0.23 & 0.23 & 0.23 & 0.26 & 0.27  \\
                         & $\sigma_{s}$ ($m$) & 0.47 & 0.31 & 0.29 & 0.27 & 0.25 & 0.24 & 0.21 & 0.19  \\
                         & $h_{min}$ ($m$) & -2.07 & -0.67 & -0.61 & -0.55 & -0.49 & -0.44 & -0.34 & -0.31  \\
                         & $h_{max}$ ($m$) & 1.20 & 1.02 & 0.99 & 0.96 & 0.90 & 0.85 & 0.69 & 0.63  \\
                         \midrule \addlinespace \\
\multirow{5}{*}{\textbf{Ladoga}} & r  & 0.78 & 0.58 & 0.5 & 0.42 & 0.23 & 0.27 & 0.12 & 0.07\\
                         & RMSD ($m$) & 0.18 & 0.24 & 0.26 & 0.28 & 0.36 & 0.31 & 0.32 & 0.32  \\
                         & $\sigma_{s}$ ($m$) & 0.34 & 0.27 & 0.26 & 0.24 & 0.22 & 0.20 & 0.16 & 0.14  \\
                         & $h_{min}$ ($m$) & -0.95 & -0.75 & -0.71 & -0.66 & -0.57 & -0.51 & -0.40 & -0.36   \\
                         & $h_{max}$ ($m$) & 0.66 & 0.58 & 0.53 & 0.46 & 0.46 & 0.45 & 0.41 & 0.38  \\
\bottomrule
\end{tabular*}
\end{table}

~\\
\cleardoublepage

\section{{\fontfamily{lmss}\selectfont Résultats supplémentaires CTRIP-12D}}
\label{chap:supplement_ctrip12d} 

Cette section présente les résultats supplémentaires issus de la simulation globale CTRIP-MLake 12D.

\begin{table}[h!]
 \caption{Répartition des 5533 stations par classe de scores hydrologiques pour la simulation CTRIP-MLake 12D}
 \label{tab_repartition_12d}
 \begin{tabularx}{\textwidth}{cXXXX}
 \hline
 & NSE & NSE$_{log}$ & KGE & Corrélation\\
 \hline
  < -0.5 &1374&1019&162&0\\
 $\left[-0.5,-0.2\right[$ &479&854&4.2&2\\
 $\left[-0.2,0\right[$&490&622&451&136\\
 $\left[0.,0.2\right[$&721&760&935&562\\ 
 $\left[0.2,0.5\right[$&1751&1323&2132&1306\\
  > 0.5 &718&955&1618&3527\\
 \hline
 \end{tabularx}
\end{table}

~\\

\begin{table}[h!]
 \caption{Répartition des différences de scores NSE, NSE$_{log}$, KGE et scores de NIC suivant leurs classes de valeurs pour les 5533 stations de la simulation CTRIP globale sur la période 1978-2014.}
 \label{ctrip_classe_globe}
 \begin{tabularx}{\textwidth}{XXXXX}
 \hline
 Classe &NSE&NSE$_{log}$&KGE&NIC\\
 \hline
    $< -1$ &6&0&0&4\\
    $\left]-1;-0.5\right]$ &14&2&2&9\\
    $\left]-0.5;0\right[$ &1072&980&1319&1079\\
    $0$ &2459&2011&1821&1927\\
    $\left]0;0.5\right]$ &1753&2267&2363&2459\\
    $\left]0.5;1\right]$ &129&119&25&55\\
    $1$ &100&157&3&0\\
  \hline
 \end{tabularx}
\end{table}

~\\

\begin{figure}
\centering
\includegraphics[width=1.\textwidth]{barrage_12d}
\caption{Hydrographe des débits simulés par CTRIP-MLake et observés pour des stations à l'aval de barrage sur la période 1978-2014. A) Rivière Chelan, B) Barrage Hoover (USA), C) Volga (Russie).}
\label{barrage_12d}
\end{figure}

~\\

\begin{figure}
\centering
\includegraphics[width=1.\textwidth]{rivers_12d}
\caption{Hydrographe des débits simulés par CTRIP-MLake et observés pour des stations où les scores sont dégradés sur la période 1978-2014. A) Mackenzie (Canada), B) Ob (Russie).}
\label{rivers}
\end{figure}

~\\

\begin{figure}
\includegraphics[width=1.\textwidth]{bienville}
\caption{Hydrographe des débits simulés par CTRIP-MLake et observés pour des stations à l'aval de barrage sur la période 1978-2014. A) Rivière Chelan (USA), B) Barrage de Hoover (USA), C) Ob (Russie), D) Volga.}
\label{bienville}
\end{figure}

~\\

\begin{figure}
\includegraphics[width=1.\textwidth]{tartaka}
\caption{Hydrographe des débits simulés par CTRIP-MLake et observés pour des stations à l'aval de barrage sur la période 1978-2014. A) Rivière Chelan (USA), B) Barrage de Hoover (USA), C) Ob (Russie), D) Volga.}
\label{tartake}
\end{figure}

~\\
\cleardoublepage
\chapter{{\fontfamily{lmss}\selectfont Paramétrisation d'une bathymétrie adaptée à l'échelle globale}}
\label{chap:morpho_lac}
\minitoc

L'eau stockée dans les réservoirs lacustres est non seulement importante pour l'activité humaine et les populations qui vivent aux abords \citep{marsily2018} mais elle conditionne aussi de nombreux processus biologiques, chimiques et écologiques comme la production primaire \citep{fee1996,blais1995}. Parmi les processus qui agissent au niveau des lacs, les variations de niveaux d'eau engendrées par une modification du bilan de masse sont un des objets de cette thèse. Ces variations dépendent évidemment de la morphologie mais  peuvent être aussi altérées par des conditions externes (voir au chapitre \ref{chap:etude-globale}).\\
Le marnage est associé à des processus lacustres primordiaux et interdépendants, ce qui complexifie la simulation des réponses aux forçages climatiques et anthropiques. Ainsi, une baisse anormale du niveau d'un lac, en lien par exemple avec une augmentation du taux d'évaporation, peut engendrer une augmentation de la température de la colonne d'eau et provoquer une réduction de la couverture en glace hivernale \citep{sharma2019}. Cette chaîne de conséquences physiques peut aussi provoquer des modifications plus générales sur les régimes de mélange \citep{shatwell2019,woolway2019}, les périodes de stratification, la distribution verticale d'oxygène et de nutriments, c'est-à-dire une modification profonde et souvent irréversible du cycle biogéochimique \citep{saros2019}.

\section{{\fontfamily{lmss}\selectfont Importance de la morphologie du bassin lacustre}}

La connaissance des caractéristiques morphologiques d'un lac, et plus spécifiquement de son volume, est essentielle en hydrologie pour quantifier la ressource disponible et en estimer les variations. Plus généralement, la morphologie est l'une des caractéristiques principales pour différencier et classer les lacs. Pourtant les études à l'échelle globale sont rares \citep{meybeck1995,ryanzhin2005,messager2016,cael2017} et se focalisent généralement sur une estimation globale de la ressource tout en donnant peu de crédit aux considérations locales et à l'identification des zones à enjeux. Cette connaissance est limitée par la difficulté d'acquisition de données à la fois fiables et précises mais s'explique aussi par le coût humain et financier associé à des techniques comme la levée bathymétrique \citep{hollister2010}. À celà s'ajoute l'absence de techniques de télédétection pour la cartographie des fonds.\\

Deux paramètres sont généralement utiles pour caractériser la morphologie d'un lac: l'aire et la profondeur. Toutefois, ces données sont statiques et donnent une vision de l'état du système à un certain instant. Il est donc difficile de déterminer la distribution verticale des processus lacustres, comme le marnage, à partir de ces données sans ne connaître la dynamique. En outre, le cadre de notre étude ne nécessite pas, au premier ordre, de connaître explicitement le volume absolu mais plutôt de considérer un volume relatif utile pour simuler les flux de masse au sein du continuum rivière-lac-atmosphère. \\
Nous avons vu précédemment que les marnages pour des grands lacs exoréiques restent relativement faibles par rapport à la profondeur totale de leur bassin. C'est évidemment différent pour les lacs endoréiques qui dépendent plus fortement des forçages atmosphériques et sont plus sensibles à des modifications du régime hydrologique \citep{wurtsbaugh2017, pham2020}. La simulation précise de la dynamique peut alors se restreindre aux variations du volumes relatifs même si la validité de cette hypothèse peut être mise en défaut pour des petits lacs.\\

\begin{figure}
\centering
\includegraphics[width=0.45\textwidth]{hypso_baikal}
\caption{Exemple de courbe hypsométrique pour le bassin sud du lac Baïkal. Source: \citet{piccolroaz2013}}
\label{hypso_baikal}
\end{figure}

Une approche pour répondre à la question de la morphologie des lacs peut être abordée en introduisant des courbes hypsométriques dans le modèle (figure \ref{hypso_baikal}). Ce type de courbe décrit l'évolution de l'aire du lac par rapport à sa profondeur et caractérise la forme du bassin. Ces courbes donnent accès à une multitude d'applications comme par exemple le calcul de volumes pour le suivi du stock ou des niveaux d'eau \citep{arsen2014} où le suivi des régimes de mélanges \citep{piccolroaz2013}. Par contre, en prenant le problème dans le sens opposé, la connaissance de la variation de profondeur et de la courbe hypsométrique du lac informe sur la variation d'aire et de volume associée et permet donc de paramétrer plus précisément les flux de masse. Cette approche paraît intéressante dans le cas de lacs dont la dynamique est dominée par les conditions atmosphériques comme le lac Victoria pour lequel l'évaporation dépen quasi-exclusivement de la surface de lac en contact avec l'atmosphère. En outre, dans un domaine comme l'hydrologie continentale où la part d'observations satellitaires est très importante, notamment en altimétrie, l'hypsométrie donne accès à un suivi global et continu de la ressource en eau. Ce chapitre présente des travaux préliminaires qui pourront être utilisés dans le cadre de future missions altimétriques à haute résolution comme par exemple la future mission spatiale SWOT. Cette mission donnera accès à des données de variations d'altitude et de surface pour le suivi du stocks des principaux lacs \citep{biancamaria2016}.\\

Dans l'état actuel du modèle MLake, l'hypsométrie des lacs est linéaire, ce qui revient à représenter le lac sous forme prismatique. Des inconvénients apparaissent alors pour le calcul de l'évaporation et l'estimation précise des flux de masse entre les différents compartiments hydrologiques. Ajouter une hypsométrie implique donc de pouvoir corriger les données de profondeur, paramètre essentiel pour le calcul du bilan d'énergie actuellement prescrite dans le modèle FLake.\\

Ce chapitre présente une méthode de calcul de l'hypsométrie pour les lacs basée sur une hypothèse géométrique et applicable à l'échelle globale. Bien que ces travaux restent en cours de validation, il semble important de les détailler et de mettre en avant les perspectives qu'ils offrent car ils sont complémentaires des développements du modèle de lac MLake dans sa version couplée.

\section{{\fontfamily{lmss}\selectfont Hypsométrie d'un lac pour l'échelle globale: l'approche gaussienne}}

Les méthodes d'estimation du volume d'un lac sont variées et sont issues de bases de données, de méthodes géostatistiques ou de développement théoriques. Parmi ces méthodes, les approches géométriques sont privilégiées dans le cadre d'études extensives où les seules données connues sont la profondeur moyenne et l'aire \citep{li2019,hayashi2000}. Elles présentent, par contre, l'inconvénient de manquer de précision pour décrire localement la distribution et la morphologie des bassins. À l'inverse, les méthodes plus complexes basées sur l'étude de la topographie aux abords du lac \citep{heathcote2015} ou la triangulation des points de mesure \citep{hollister2010} demandent une quantité de données et des coûts de calcul trop importants pour une application à l'échelle globale \citep{oliver2016}. De plus, ces méthodes rendent compte d'une morphologie locale et sont difficilement extrapolables à des échelles plus grandes. \\

L'approche développée ici propose un compromis entre la précision et la complexité d'un modèle qui doit s'adapter aux contraintes de l'échelle globale. Ce modèle hypsométrique se base sur l'hypothèse que la forme du bassin lacustre peut être assimilée à une surface paramétrée selon une équation gaussienne.\\

\noindent Dans cette approche, la forme du lac est déterminée par sa bathymétrie $z(r, \theta)$ représentée sur la figure \ref{lac_gaussien}. 

\begin{figure}[h!]
\centering
\includegraphics[width=0.9\textwidth]{lac_gaussien}
\caption{Représentation de la bathymétrie d'un lac et de ses paramètres sous l'hypothèse de la fonction de Gauss $z(r,\theta)$}
\label{lac_gaussien}
\end{figure}
\clearpage
\noindent L'enjeu est donc de trouver l'hypsométrie correspondante, c'est-à-dire la fonction telle que:
\begin{equation}
A = f(z)
\end{equation}
avec $A$ l'aire du lac à la profondeur $z$ et $f$ une fonction hypsométrique.
\subsubsection*{Hypothèses de départ}

\noindent Pour permettre la résolution du problème, des hypothèses nécessitent d'être posées.\\
La première hypothèse considère que toutes les sections horizontales d'un lac sous forme gaussienne sont équivalentes à des cercles dont l'aire est égale à:

\begin{equation}
\label{hypothese1}
A(z) = \pi r^2(z)
\end{equation}
avec $r(z)$ la fonction inverse de la bathymétrie. \\

\noindent Cette relation doit vérifier les conditions aux limites:

\begin{itemize}
  \item[$\bullet$]  $A(z=0) \: =  \: \pi R^2 \: = \: A_{lake}$\\
  \item[$\bullet$]  $A(z=z_{max}) = 0$
\end{itemize}
avec $R$ le rayon du cercle à la surface du lac et $z_{max}$ la profondeur maximale de la gaussienne. \\

\noindent La deuxième hypothèse définit la propriété de conservation du volume lors de la construction du modèle gaussien. La modification de la forme théorique du lac ne doit pas modifier le stock, notre variable pronostique. Ce changement de variable doit donc être conservatif:

\begin{equation}
\label{hypothese2}
V_{lake} = V_{g}
\end{equation}
avec $V_{lake}$ le volume moyen du lac prescrit initialement et $V_{g}$ le volume d'eau contenu dans le lac gaussien pour $z$ = 0.\\
\clearpage
\noindent Le volume de lac prescrit fait suite à l'utilisation conjointe de la base de donnée ECOCLIMAP et GLDB comme vue dans la section XX. Ce volume est calculé suivant l'équation:

\begin{equation}
V_{lake} = A_{ECO}.z_{moy,GLDB}
\end{equation}
avec $A_{ECO}$ l'aire moyenne du lac issue de la base ECOCLIMAP (m$^{2}$) et $z_{moy,GLDB}$ la profondeur moyenne issue de la base GLDB (m).


\subsubsection*{\'Equation hypsométrique}

\noindent En se basant sur la première hypothèse (Eq.~\ref{hypothese1}), il vient que le problème de l'hypsométrie peut se résoudre en inversant l'équation de la bathymétrie $r(z)$. Cette bathymétrie est définie comme une fonction de Gauss bidimensionnelle régulière en coordonnées cylindriques:

\begin{equation}
\label{eq:bathy_gaussian}
z(r) = z_0 + z_{max} e^{-\frac{r^2}{2\sigma^2}}
\end{equation}
avec $z_0$ définie comme la hauteur entre la surface du lac et les asymptotes horizontales de la fonction gaussienne et $\sigma$ un paramètre de dispersion dépendant de l'aire du lac.\\

\noindent L'équation ainsi définie doit vérifier les conditions aux limites:

\begin{itemize}
	\item[$\bullet$]  $z(0)= z_0 + z_{max} = z_{eq}$
	\item[$\bullet$]  $\lim_{r\to+\infty} z(r) = z_0$
\end{itemize}

~\\
Connaissant l'expression de la bathymétrie, il convient d'inverser cette dernière pour déterminer la relation cherchée:

\begin{equation}
\label{eq:invbathy_gaussian}
r(z) = \sigma\sqrt{-2\ln\left(\frac{z-z_0}{z_{max}}\right)}
\end{equation}

\noindent qui vérifie les conditions aux limites suivantes:

\begin{itemize}
	\item[$\bullet$]  $r(z_{eq})= \sigma\sqrt{-2\ln(1)}=0$
	\item[$\bullet$]  $\lim_{r\to z_0} r(z) = +\infty$
	\item[$\bullet$]  $r(0)= \sigma\sqrt{-2\ln\left(-\dfrac{z_0}{z_{max}}\right)}$
\end{itemize}

~\\
\noindent En remplaçant l'expression de $r(z)$ dans l'équation \ref{hypothese1}, l'aire des sections horizontales d'un lac gaussien régulier est défini comme:

\begin{equation}
\label{eq:hypso_gaussian}
A(z) = -2\pi\sigma^2\ln\left(\frac{z-z_0}{z_{max}}\right)
\end{equation}
avec $z_{max}$ pouvant être remplacée par $z_{eq}-z_0$ pour réduire le nombre de paramètres de l'équation.\\

\noindent Parmi les paramètres de cette équation, trois sont inconnus: 

\begin{itemize}
\item[$\bullet$] $z_{0}$, un paramètre lié à la topographie des berges et difficilement mesurable (un $z_0$ très petit équivaut à une pente des berges très faibles);
\item[$\bullet$] $z_{eq}$, un paramètre lié au mode de la fonction de Gauss;
\item[$\bullet$] $\sigma$, le paramètre de dispersion de la fonction de Gauss.
\end{itemize}
~\\
Le problème est donc surparamétré et aucune solution directe n'est envisageable. Il convient donc de trouver des équations supplémentaires qui permettent d'estimer les paramètres inconnus.\\

\subsubsection*{\underline{{\fontfamily{lmss}\selectfont Première méthode }}}

\noindent La surface de la fonction gaussienne bi-dimensionnelle est définie à $z=0$ (Figure \ref{lac_gaussien}) par:

\begin{equation}
A(z=0) = \pi.(n\sigma)^2
\end{equation}
avec $n$ un facteur multiplicatif à déterminer.\\

\noindent Grâce à l'hypothèse sur l'égalité des aires (Eq. \ref{hypothese1}), le facteur de dispersion gaussien $\sigma$ peut s'écrire:

\begin{equation}
 \sigma = \sqrt{\dfrac{A_{lake}}{\pi.n^{2}}}
\end{equation}

\noindent Cette nouvelle équation, même si elle introduit un nouveau paramètre inconnu $n$, établit une équation pour déterminer $\sigma$. \\

\noindent De plus, en reprenant l'équation de bathymétrie (Eq. \ref{eq:bathy_gaussian}) et les conditions aux limites associées, il est possible d'en déduire les expressions de $z_0$ et $z_{max}$ sous la forme:

\begin{align}
\label{eq:z0}
z_0 = \dfrac{-e^{-\frac{n^2}{2}}}{1-e^{-\frac{n^2}{2}}} \: z_{eq} \\
\bigskip
z_{max} = \frac{1}{1-e^{-\frac{n^2}{2}}} \: z_{eq}
\end{align}
~\\
~\\

\noindent L'introduction de ces nouvelles équations relie des paramètres difficilement mesurables ($z_{0}$ et $z_{max}$) à un des paramètres morphologiques du lac: la profondeur équivalent ($z_{eq}$).\\
Si on utilise la deuxième hypothèse (Eq. \ref{hypothese2}), il est possible d'aller plus loin dans la résolution puisque le volume $V_{g}$ du lac gaussien se calcule en intégrant l'aire de chaque section horizontale sur toutes la profondeur:

\begin{equation}
\label{eq:vol_gene}
V_{g} = \int_{z_{eq}}^0 A(z)dz
\end{equation}

\noindent dans notre cas, en l'exprimant suivant les trois paramètres $\sigma$, $z_{eq}$ et $n$, son équation est:

\begin{equation}
\label{eq:vol_gaussian_n}
V_{g} = 2\pi\sigma^2z_{eq}\left[1 - \dfrac{n^2}{2}\dfrac{e^{\frac{-n^2}{2}}}{1-e^{\frac{-n^2}{2}}}\right]
\end{equation}
~\\

\noindent Cette dernière équation assure la fermeture du problème en exprimant $z_{eq}$ en fonction de paramètres mesurables. En effet, l'hypothèse sur la conservation du volume introduit la mise en équation de $z_{eq}$ tel que:

\begin{equation}
z_{eq}=\dfrac{n^2}{2}\left[1 - \dfrac{n^2}{2}\dfrac{e^{\frac{-n^2}{2}}}{1-e^{\frac{-n^2}{2}}}\right]^{-1}.z_{moy}
\end{equation}
avec $z_{moy}$ la profondeur moyenne du lac calculée comme le rapport du volume du lac $V_{lake}$ par l'aire de surface $A_{lake}$.\\

\noindent De ce fait, l'expression de $n$ peut être approchée par le rapport de la profondeur moyenne par la profondeur maximale qui donne une valeur comparative à la forme du bassin \citep{wetzel2001}. Le rapport pour un lac gaussien est donné par:

\begin{equation}
\dfrac{z_{moy}}{z_{max}} = \frac{2}{n^2}(1-e^{\frac{-n^2}{2}})\left(1 - \dfrac{n^2}{2}\dfrac{e^{\frac{-n^2}{2}}}{1-e^{\frac{-n^2}{2}}}\right)
\end{equation}
~\

\begin{figure}
\includegraphics[width=1.\textwidth]{Vd}
\caption{Figure d'illustation du développement du volume d'un lac dans les trois cas possibles. (a) $V_{d} < 0.33$, lac convexe. (b) $V_{d} = 0.33$, lac conique. (c) $V_{d} > 0.33$, lac concave.}
\label{Vd}
\end{figure}

\noindent Ce rapport peut être relié au paramètre développement du volume $V_{d}$. Ce dernier paramètre exprime le rapport du volume du lac au volume d'un cône dont la surface basale est égale à la surface du lac et dont la hauteur est égale à la profondeur maximale du lac. Il donne une indication sur la forme concave ou convexe du bassin (Figure \ref{Vd}).  Son expression a été définie par \citet{hutchinson1957} tel que: 

\begin{eqnarray}
V_{d}&=& 3.\dfrac{z_{moy}}{z_{eq}} \\
     &=& \frac{6}{n^2} \left(1 - \dfrac{n^2}{2}\dfrac{e^{\frac{-n^2}{2}}}{1-e^{\frac{-n^2}{2}}}\right)
\end{eqnarray}

\noindent Une valeur inférieure à 0.33 correspond à un lac convexe tandis qu'une valeur supérieure à 0.33 correspond à un lac concave. Dans l'étude historique portant sur 100 lacs, la valeur du développement du volume est supérieure, à certaines exceptions près, à 0.33, pour une valeur moyenne autour de 0.467 \citep{neumann1959}. Pour donner un ordre d'idée, le lac Baïkal (Russie) possède un développement du volume de l'ordre de 0.43 alors que le lac Tahoe (USA) est à 0.62. Une étude plus récente portant sur les lacs Suédois confirme ces valeurs avec valeur moyenne de 0.40 \citep{johansson2007}.\\
\citet{johansson2007}, dans son étude, propose aussi une amélioration du paramètre $V_{d}$ en développant un indicateur de développement hypsographique $H_{d}$. Il permet de comparer le volume du plan d'eau à celui d'une forme géométrique dont le volume, la surface et la profondeur maximale sont identiques. Le seul inconvénient est que l'expression de cet indicateur repose sur la connaissance de $V_{d}$ et donc à la fois de la profondeur moyenne et maximale. Dans tous les cas, ces approches permettent de déduire les aires relatives en réponse à une variation de hauteur et inversement.
\clearpage
\subsubsection*{\underline{{\fontfamily{lmss}\selectfont Deuxième méthode}}}

Une autre méthode de résolution du problème consiste à travailler directement sur le paramètre $z_{0}$. Cette approche se base sur la recherche d'un lien entre la topographie des berges caractérisées par $z_{0}$ et la profondeur équivalente du lac $z_{eq}$, considérée comme égale à la profondeur maximale.\\

\noindent Dans cette approche, le volume est exprimé selon:
\begin{equation}
\label{eq:vol_gaussian_z}
V_{g} = -2\pi\sigma^2\left[z_{eq} + z_0 \ln\left(1-\frac{z_{eq}}{z_0}\right)\right]
\end{equation}

\noindent De la même façon que précédemment, la profondeur moyenne issue du rapport entre le volume et l'aire de surface est dépendante de deux paramètres, $z_{eq}$ et $z_0$:

\begin{equation}
z_{moy} = \frac{z_0}{2}\left[1+\dfrac{\dfrac{z_{eq}}{z_0}}{ln(1-\dfrac{z_{eq}}{z_0})}\right]
\end{equation}

\noindent Pour résoudre cette équation, il convient de connaitre soit directement la topographie des berges autour du lac ($z_0$), soit l'aire de deux sections horizontales du lac .\\
Le rapport des aires donne alors:

\begin{equation}
\dfrac{A_{lake}}{A_{z_1}} = \dfrac{ln\left(\dfrac{-z_0}{z_{eq}-z_0}\right)}{ln\left(\dfrac{z_1-z_0}{z_{eq}-z_0}\right)}
\end{equation} 
avec $A_{z_1}$ l'aire de la section horizontale à la profondeur $z_1$.\\

\noindent Dans cette configuration, il s'agit de résoudre l'équation $f(x)=0$ telle que:

\begin{equation}
f(x) = \dfrac{ln \left(\dfrac{-x}{z_{eq}-x}\right)}{ln\left(\dfrac{z_1-x}{z_{eq}-x}\right)} - \dfrac{A_{lake}}{A_{z_1}}
\end{equation}
~\\

\noindent Cette fonction est la composée de fonctions monotones définies sur $\mathrm{I\! R}^{+*}$ qui admet, s'il existe, un unique zéro.\\

\noindent Dans les deux méthodes précédentes, la résolution de l'équation donnant $n$ ou $z_0$, la détermination des autres paramètres $\sigma$ et $z_{eq}$ est directe.\\

\noindent L'intérêt d'une telle méthode par rapport aux méthodes utilisant une forme conique ou sinusoïdale est qu'ici il existe un degré de liberté supplémentaire. Ce degré complexifie les développements mais en contrepartie fournit un paramètre $n$ qui peut être utilisé pour calibrer la courbe hypsométrique théorique afin de minimiser la distance par rapport à la courbe hypsométrique réelle. Après calibration, il est possible d'extrapoler ce paramètre suivant des classes de lacs sur la base de données géomorphologiques, climatiques ou topographique comme cela se fait déjà \citep{koshinsky1970,johansson2007}. 
\section{{\fontfamily{lmss}\selectfont Validation préliminaires}}

À la date de rédaction de ce manuscrit, la validation du modèle gaussien n'a pas été finalisée et il reste encore des pistes à exploiter pour présenter la totalité de l'étude.

\subsection{{\fontfamily{lmss}\selectfont Exemples d'applications}}

Pour montrer l'utilisation possible de cette hypsométrie, deux lacs ont été utilisés en exemple: le lac Tan et le lac Namco. Le choix n'est pas arbitraire et se base sur la disponibilité de données fiables. \\
Pour le lac Tana, les données sont issues de l'étude de \citet{kebedew2020}. Ainsi l'équation hypsométrique ainsi que des données sur la variation de cote de surface sont disponibles. Pour les deux autres lacs, les données sur les niveaux de lac sont issus de la base de données Hydroweb. En complément, l'étude de \citet{li2019} diffuse des hypsométries validées pour les lacs du plateau tibétain.\\

\subsubsection*{{\fontfamily{lmss}\selectfont Lac Tana}}
\noindent La relation empirique qui lie l'aire du lac Tana à la profondeur a été proposée et validée par \citet{kebedew2020} avec un coefficient de détermination de 0.9972. Cette expression s'écrit sous la forme d'un polynôme du troisième degré, tel que:

\begin{equation}
A_{H} = 0.88.(H-1772)^{3}-35.02.(H-1722)^{2}+537.46.(4-1722)-62.4
\end{equation}
avec $H$ l'altitude au-dessus du niveau de la mer (m) et $A_{H}$ l'aire du lac à cette profondeur (km$^{.2}$)

~\\

\noindent Cette étude donne aussi accès aux variations de niveau du lac ainsi qu'aux paramètres de profondeur moyenne, de profondeur maximale et d'aire de surface. Ces paramètres sont regroupés dans le tableau \ref{carac_lacs_hypso}.\\

\begin{table}[h!]
 \caption{Principales caractéristiques du lac Tana et du lac Namco.}
 \label{carac_lacs_hypso}
 \begin{tabularx}{\textwidth}{XXX}
 \hline
  & Tana & Namco \\
 \hline
  Profondeur moyenne (m)&9.7&33\\
  Profondeur maximale (m)&14.8&125\\
  Superficie (km$^{2}$)&3046&1920\\
  Altitude maximale (m)&1787&4725\\
  Altitude minimale (m)&1784&4719.8\\
  Altitude moyenne (m)&1786.53&4723.7\\
  \hline
 \end{tabularx}
\end{table}

\noindent Grâce à ces valeurs, il est possible d'initialiser le calcul des hypsométries sous forme gaussienne, conique et empirique.\\

\noindent La figure \ref{tana} montre que la distribution d'erreur est croissante pour un facteur de dispersion $n$ croissant. \\

\begin{figure}[h!]
\includegraphics[width=1.\textwidth]{tana_rmse}
\caption{Distribution du RMSD entre les hypsométries théoriques (cône, gaussienne) et l'hypsométrie empirique pour le lac Tana.}
\label{tana}
\end{figure}

\noindent Ainsi les erreurs sont globalement plus faibles dans un intervalle de valeurs situés autour de un. En comparant avec les simulations sous forme de cône, il est aussi possible de remarquer que pour des facteurs de dispersion inférieur à 3, les erreurs sont nettement réduites.\\
Les résultats montrent aussi l'intérêt d'utiliser cette méthode puisque dans ce cas, il est possible de choisir la configuration qui correspond le mieux à la morphologie du lac. Dans le cas du lac Tana, sa morphologie particulière (grande superficie mais profondeur moyenne faible) est bien représentée par l'approche gaussienne (un faible facteur de dispersion équivaut à une profondeur maximale faible et donc à une aire de surface étendue).

\subsubsection*{{\fontfamily{lmss}\selectfont Lac Namco}}

\noindent Cette méthode a aussi été appliquée au lac Namco qui possède une morphologie similaire au lac Tana mais dans une région climatique totalement différente. Les données hypsométriques se basent ici sur l'étude de \citet{li2019}. Dans cette étude, les courbes hypsométriques sont issues de l'analyse d'images Landsat. Dans le cas du lac Namco, le coefficient de détermination est de 0.87 et suivant une équation polynomiale du second degré:

\begin{equation}
A_{H} = 2.43.(H-4724.5)^{2}+5.55.(H-4724.5)+1970.1
\end{equation}

Cependant cette étude ne propose pas de données de variations d'altitude du niveau d'eau. Pour ces données, la base Hydroweb a été utilisée. Il a été possible de récupérer les chroniques de variations de niveau représentées sur la figure \ref{namco} sur la période 2005-2020. Les caractéristiques du lac sont regroupées dans le tableau \ref{carac_lacs_hypso}. \\

\begin{figure}[h!]
\includegraphics[width=1.\textwidth]{namco}
\caption{Série temporelle des variations d'altitude de la cote de surface du lac Namco issue de la plateforme Hydroweb sur la période 2005-2020.}
\label{namco}
\end{figure}

\noindent Les résultats sont similaires à ceux observés pour le Tana (Figure \ref{rmse_namco}. Comme la morphologie est similaire (faible profondeur et surface étendue), la gamme de valeur qui possède les erreurs les plus faibles se trouvent pour des facteurs de dispersion inférieur à 3.\\


\begin{figure}[h!]
\includegraphics[width=1.\textwidth]{rmse_namco}
\caption{Distribution du RMSD entre les hypsométries théoriques (cône, gaussienne) et l'hypsométrie empirique pour le lac Namco.}
\label{rmse_namco}
\end{figure}


\section{{\fontfamily{lmss}\selectfont Discussions/Perspectives}}

Ce chapitre propose le développement d'un modèle de description de la bathymétrie des lacs pour une application à l'échelle globale. Par la suite, la description de cette bathymétrie est utilisée pour déterminer des courbes hypsométriques donnant l'évolution de l'aire des sections horizontales du lac selon la profondeur.\\ 
Cette méthode basée sur une hypothèse gaussienne nécessite la connaissance de certains paramètres morphologiques, comme l'aire du lac ou la profondeur moyenne, ainsi que des paramètres liés à la topographie ou à la forme du bassin.\\
L'intérêt d'utiliser une telle approche repose sur la possibilité d'optimiser la courbe hypsométrique théorique avec des courbes réelles afin d'améliorer la représentation de la dynamique verticale des lacs. Le développement de ce genre d'approche s'adresse principalement aux zones de grande densité lacustre généralement très peu instrumentées à défaut des grands lacs dont les bathymétries sont connues. L'autre intérêt est d'apporter un degré de liberté facilitant la calibration sur les observations et permettre ainsi une estimation des variations de profondeur ou d'aire affinée.\\

Les résultats préliminaires démontrent une certaine capacité du modèle à s'adapter à la géométrie des lacs et notamment pour la représentation de la zone superficielle. De plus, le modèle semble s'adapter aux lacs, quelle que soit leur localisation ou leur origine. L'analyse sur deux lacs ne permet pas de tirer des conclusions générales confirme l'intérêt d'analyser cette méthode à une plus grande échelle. L'adaptablilité du modèle est particulièrement importante dans la perspective d'un couplage de MLake avec un modèle d'atmosphère. Les flux d'évaporation sont sensibles à la surface mais aussi à la profondeur du lac. Ainsi, plus le modèle est précis dans sa prescription des paramètres morphologiques et plus les flux en sortie seront réalistes. Pour les mêmes raisons, elle offre un avantage en hydrologie pour un suivi plus précis de la dynamique du marnage et, par conséquent, du calcul des débits d'effluents. La charge en eau au dessus du seuil est un des prédicteurs contrôlant le débit de déversement. Dans ce cas, une représentation réaliste de l'hypsométrie devrait affiner la simulation de cette charge en eau et par conséquent le réalisme des débits en sortie devrait être amélioré.\\

Il reste cependant des étapes essentielles afin de valider entièrement cette approche. Ces étapes portent sur la mise en place d'une méthode de détermination des prédicteurs de calibration, que ce soit le facteur morphologique $n$ ou le facteur topographique $z_0$. Dans tous les cas, cette validation est restreinte par le manque de données de validation à l'échelle globale. Les données hypsométriques sont rares, locales et plus souvent déduites indirectement de paramètres externes qu'issues de levés bathymétriques.\\
Dans cette perspective, la future mission spatiale SWOT comblera en partie ce manque avec l'acquisition et le traitement de données de surface et de hauteurs plusieurs fois par cycle de 21 jours \citep{biancamaria2016}. Au premier ordre, ces données assureront une validation du modèle par la détermination des facteurs de calibration pour ensuite suivre les stocks d'eau des différents lacs. Dans le cas où plusieurs mesures d'aire sont nécessaires pour entraîner le modèle, les données de la mission pourront servir de jeux de données d'entrées pour déterminer la configuration la moins biaisée pour chaque lac. Du fait de la couverture spatiale étendue de la mission, les jeux de données assureront aussi l'alimentation d'un modèle d'apprentissage machine en données quantitatives pour la détermination des variations régionales du paramètre de forme.\\

Néanmoins cette méthode comporte des biais systématiques qui sont difficilement évitables à l'échelle de travail. Ainsi de nombreuses études démontrent l'intérêt d'utiliser la topographie, en association de l'aire, autour du lac comme prédicteurs de la forme du bassin lacustre et notamment de sa profondeur maximale \citep{sobek2011, hollister2010}. Ces études confirment donc l'idée selon laquelle ces paramètres sont utiles pour déduire une forme pertinente de bassin. Ces méthodes restent régionalement efficaces mais localement imprécises \citep{oliver2016}. En effet, seul 36\% à 50\% de la variabilité de la profondeur maximale est expliquée par la topographie alentour, le reste étant dû à des processus non représentés tel que l'érosion ou la sédimentation \citep{hollister2011}. Les modèles incluant les topographies ne corrigent pas ces valeurs et une déviation systématique apparaît amenant à une surestimation des profondeurs. À ces limites s'ajoute l'hypothèse restrictive qui considère que les lacs situés dans une même zone ont une morphologie similaire. \citet{choulga2014} dans son étude sur la zone boréale a montré que cette hypothèse était vérifiée pour prescrire la profondeur moyenne et il reste donc à valider l'hypothèse sur les formes de bassin.\\

Pour la correction des flux évaporatifs, il est aussi possible d'appliquer directement un facteur de pondération, corrélé à la contraction ou l'augmentation de la surface d'eau et aux forçages. Cette solution est proposée pour le modèle WaterGap \citep{muller2020}. À chaque pas de temps, l'aire est recalculée suivant un facteur de réduction $r$ tel que :

\begin{equation}
A = r . A_{max}
\end{equation}
avec $A_{max}$ l'aire maximale du lac issue de la base GLWD \citep{lehner2004} pour les lacs ou la base GRanD pour les réservoirs \citep{lehner2011}.\\

\noindent L'expression de ce facteur étant directement reliée à la variation de stock au cours du pas de temps:

\begin{equation}
r = \left(\dfrac{|V_{t}-V_{max}|}{2V_{max}}\right)^p
\end{equation}
avec $V_t$ le stock du lac au pas de temps $t$, $V_{max}$ la capacité maximale du lac et $p$ l'exposant de réduction égal à 3.32 \citep{muller2014}.\\

Ces méthodes sont dans tous les cas dépendantes des développements annexes à effectuer sur la représentation de la surface. À ce jour, le calcul des bilans dans la plateforme SURFEX est contraint par une couverture du sol issue de la carte ECOCLIMAP restant statique. L'ajout d'une surface dynamique impose donc une réflexion sur la prise en compte de zones humides recouvertes d'eau de façon semi-permanente. Ce travail devrait se baser sur un renouvellement des cartes de couvertures tel que proposé par \citet{pekel2016}. L'introduction de cartes dynamiques devient donc nécessaire pour gérer l'évolution des surfaces en eau. À cela s'ajoute la nécessité de prendre en compte les transferts latéraux entre ces zones humides et les lacs. En effet, dans le cas simple de la mise en place d'un facteur de réduction pour corriger les flux de masse, il est important de s'assurer que le bilan de masse soit fermé. Pour cela, la part de masse qui ne sort pas du lac par évaporation doit être quantifiée autrement dans le réseau. Une paramétrisation des échanges zones humides/lacs pourraient être développée sur la base du ratio entre la charge hydraulique du lac et celle du sol environnant (basé par exemple sur l'humidité des sols).\\
En conclusion, le développement d'une morphologie de lac à l'échelle globale et son introduction dans le modèle ISBA-CTRIP est dépendante de plusieurs développements à intégrer.
\cleardoublepage
%!TEX root = Manuscrit.tex
\chapter{Publication issue de la conférence conjointe IAGLR-EELS}
\noindent Cette annexe présente une publication pour laquelle je suis co-auteur et qui fait suite à la conférence internationale sur les grands lacs organisée conjointement par l'European Large Lakes Symposium et l'International Association for Great Lakes Research à Evian-les-Bains du 23 au 28 septembre 2018.\\
Cette publication est intégrée à une série d'articles scientifiques ayant pour objectif de synthétiser les connaissances actuelles dans des domaines précis afin de faire écho aux conséquences environnementales et sociétales engendrées par l'activité humaine et le changement climatique. \\
Dans cet esprit, un exercice a été fait pour rassembler les enjeux actuels et futurs concernant les grands lacs et pour poser le cadre actuel qui gouverne les changements intrinsèques à ces environnements uniques. Ce papier balaie donc un éventail de connaissance allant du climat, à l'étude des microplastiques en passant par l'écologie ou la microbiologie.

\vfill
\includepdf[scale=1.,pages=-]{article1.pdf}
\vfill

\cleardoublepage
\include{Acronymes}

\bibliography{biblio2}

\end{document}
