\chapter{{\fontfamily{lmss}\selectfont Remerciements}}

\textit{Verba volant, scripta manent.}

Il est d’usage, en préambule d’un travail de thèse, de remercier ceux qui dans la lumière ou dans l’ombre ont contribué à produire ce qui servira d’excellent cale-meuble ou remplacera pour certains leur livre favori. Loin d’être un passage obligatoire, ce moment unique représente l’opportunité d’avoir une pensée pour chacune des personnes qui a marqué ces trois années de travaux.

En premier lieu, je souhaiterai adresser un immense remerciement à mes directeurs de thèse. Tout d’abord aux deux officiels : Aaron A. Boone et Patrick Le Moigne, sans qui je n’aurai pas pris par à l’aventure. Merci de m’avoir fait confiance.
Aaron, it was a real pleasure to share these 3 years with you. Thank you for the scientific excellence that remains one of the qualities that reveals your such a impressive and talented researcher. Thank you also for your humanity, your humor even at the 7.30am coffee break and your support in all my project. I’m still looking at your website for accurate weather forecast.
Patrick, je vais essayer d’être concis et tu sais que ce n’est pas chose facile pour moi. Tout d’abord merci de m’avoir ouvert les yeux sur l’impossibilité d’utiliser la couleur orange et bleu ensemble dans une présentation (enfin dans des proportions trop importantes). Merci d’être qui tu es : vrai, honnête, drôle, rigoureux, attentionné. En fait, il y a trop d’adjectifs… J’aimerai juste t’avouer qu’au concours quand on m’a demandé les 3 qualités qui constituent un bon chef d’équipe, j’ai effectivement penser à : Patrick Le Moigne. Tu es un modèle.
Enfin j’aimerai aussi remercier un directeur officieux : Simon Munier. Je ne sais pas dans quel bureau j’ai passé le plus de temps, le tien ou le mien. Quel meilleur exemple que nos discussions pour expliquer la relativité à un novice. C’était toujours pour 30 secondes qui durait bien plus souvent 2 heures. Merci pour ces discussions scientifiques, musicales, littéraires ou seulement sur les aspects du quotidien. Merci aussi pour ta passion, pour ton soutien (on peut même dire parfois ton courage à me supporter) et tes nombreux conseils. Tu es un exemple de détermination.

J’aimerai aussi remercier Véronique Ducroq, qui pendant 2 ans en tant que cheffe de groupe, a toujours eu un mot gentil et a toujours été à l’écoute.
Merci à Christine Lac, sans jeu de mots, qui est une chercheuse talentueuse et humaine. Merci pour tes conseils, pour ta confiance et ton sourire continuel.

Ensuite débuter un travail de thèse c’est intégrer une équipe. D’abord MOSAYC puis rapidement devenu SURFACE, j’aimerai remercier des personnes qui me sont chères. 
Merci à Delphine, mon Google Python personnel, avec qui j'ai partagé le bureau pendant deux ans. Ce n’est pas mince à faire, entre les coups de folies (Dich Dich) ou les défis mathématiques. Merci pour ta résilience et ta générosité. Je continuerai à pense que le vélo c’est quand même mieux en extérieur. 
Adiou Marie, la serial potière du couloir. C’est quand même un sacré avantage d’avoir une autre personne venant du Sud-Ouest (même si tu restes Béarnaise). Merci pour les conseils et les sacrés tranches de rires. Adishatz
Le duo de l’extrême : Antoine et Stéphane. Parmi tous les bons moments qu’on a passé ensemble je garderai surtout votre bonne humeur implacable. C’est impressionnant d’être aussi optimiste malgré les aléas de la vie.
Enfin je tiens à remercier Sylvie pour ton écoute, Diane pour ta bienveillance et ta capacité à faire la fête, Gaëtan et Malak les petits jeunes qui feront la fierté de l’équipe. Théo et Maxime les inoubliables stagiaires.

Dans la grande famille du CNRM, j’aimerai remercier tous les doctorants que j’ai pu rencontrer. D’abord les anciens, Hélène D., Carole P., Tiphaine S., Quentin R., Quentin F., Adrien N., César S., Léo D., Alexane L.
Une attention particulière à Pierre-Antoine Joulin, mon petit karaté-geek, tu me manques. Mary Borderies pour l’aide, les conseils et les blagues sur les footballeuses, je suis heureux d’avoir fait ta connaissance.
Puis les docteurs ou presques-docteurs Damien S., Zied.S., Clément S., Daniel. S., Martin C., 

Enfin les copains Météo qui m’ont accepté à leur table : Julie C, Marie C, Hélène D, Marine J, Alice L, Jordan V, Timothée P, Lucas G, Benoît T. Et je ne pouvais pas finir sans le groupe de l’extrême, des copains en or : Thomas Burgot (le plus rayonnant des Bretons), Olivier Audoin (le plus Vendéen des punks), Hugo Marchal (le plus chic des Basques), Marc Mandement et Axel Roy (le plus beau en maillot de bain moulant).

Le CNRM c’est aussi une équipe administrative dévouée : Ouria, Anita, Régine et ma chère Martine. Que la vie au CNRM est simple avec vous. 

J’aimerai ponctuer ce tour d’horizon du laboratoire en remerciant tout d’abord CTI pour leur aide et leur soutien. Vous êtes un pilier important, central et irremplaçable. Enfin je voudrai avoir une pensée particulière pour les copains/copines du vélo : Olivier T., Véronique M., Fred S., Jean-Marie D.
A la confluence de ces deux remerciements, je terminerai en remerciant celui qui est sûrement le  atypique et inoubliable des compagnons de route : Serge Blin. La retraite n’est pas une excuse pour qu’on arrête de rouler ensemble, attends toi à ce que je vienne t’embêter, je te rappelles qu’on a quelques cols à grimper ensemble.

Une réussite de thèse, ce sont aussi des personnes externes qui comptent beaucoup pour moi.

Elise, une belle rencontre par delà les étoiles. Qui aurait cru qu'un physicien et une anthropologue construise un récit autour d'un chou. Je revis encore ce retour depuis le centre de détention de Saint-Sulpice. À nos bières du vendredi après-midi, à tes poules et à Locky.

Arnaud, David, mes amis de longues dates. Pas toujours d'accord sur tout mais c'est pour ça qu'on s'aime. Arnaud tu ferais mieux de réfléchir avant d'acheter un SUV, pense plutôt à tous les casques de vélo que tu pourrais t’acheter. Merci pour les soirées intelligentes, je t’aime.
David même si tu as eu un titre de docteur avant moi, tu restes quand même à 6 ans d'études et surtout tu continues à faire un boulot qui terrorise les gens. Merci pour les week-ends intelligents, je t’aime.
Merci à tous les copains de JDA: Cassandre, Axel, Marine, Paul.

Agathe, que de temps à passé depuis la maternelle. Haut comme trois pommes on s'est trouvé et plus jamais lâché. À nos soirées post-journée pédagogique, aux fourmis écrasés entre lames et lamelles, à la technique australienne dans la piscine. Je suis heureux que tu aies trouvé chaussure à ton pied. Thomas tu es un garçon exceptionnel, tu rayonnes et c'est toujours un plaisir que de passer du temps avec toi. 
Une pensée à ma petite Adèle, qui fût quelquefois notre souffre-douleur mais toujours avec un sourire ravageur (courage Nico !).

Parents : Ninou
Pauline